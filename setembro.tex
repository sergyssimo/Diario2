\addchap{Setembro}
\section*{1 \adfflatleafright \ISODayName{1959-09-01}}

Devo embarcar para Paris no mesmo trem das 13h20m. Josée vem me trazer à estação, muito gentilmente. Dou o meu adeus à Bélgica, que deixa, também, grandes recordações.

Dois detalhes não mencionei ainda: o grande orgulho que têm pelo executor do carrilhão da catedral de Mechelen, Gustavo Nees, considerado o melhor do mundo e que, recentemente, deu recitais nos Estados Unidos. Infelizmente, não o pude ouvir. Josée me prometeu enviar um disco mais tarde.

Outra referência é a moda feminina das saias curtas rodadas, na base de farfalhantes anáguas de \textit{nylon} (que aqui se pronuncia ``nilôn'' mesmo). Em Berlim e na Holanda observei também o predomínio desse estilo.

Após quatro horas e meia de viagem estarei em Paris. Passo pela grande cidade mineira de Mons, no sul da Bélgica, e dali a pouco penetro no território francês. O fiscal belga que controla os passaportes exibe uma estrelinha verde na lapela: é esperantista; saúdo-o e ele, muito satisfeito, após o término do serviço, vem bater um bom ``papo''.

Passados Compiegne e Chantilly, entramos, lá pelas 6 horas da tarde, na gare Norte da capital francesa. Alcino veio me esperar na estação, com cinco cartas para mim, enviadas para o endereço do amigo esperantista de Paris, Syla Chaves.

A primeira carta de casa eu recebera há poucos dias em Mechelen, das mãos de Josée, cujo endereço havia dado com a necessária antecedência. Passei quase dois meses sem uma linha do Brasil, a não ser a rápida carta do colega Brasilo, enviada a Varsóvia, falando sobre o assunto de sua revista técnica.

Vamos para o ``Hotel Peyris'', na ``Rue du Conservatoire'', entre Montmartre e Bonne Nouvelle, os dois famosos \textit{boulevards}. Bem central, pois. Ao tomar o táxi, faço questão de escolher um Citroën último tipo, um ID-19 (conhecido no Brasil como ``boca de sapo''), pois estava doido para entrar em um.

O hotel está muito frequentado por portugueses e, mesmo, alguns brasileiros. A razão é que sua dona, tendo vivido algum tempo em Lisboa, fala português e, assim, tornou o seu hotel o preferido dos lusitanos em Paris.

Janto num restaurante ``selfservice'', o Rex, no \textit{boulevard} Bonne Nouvelle. O hotel fica entre este quarteirão e Montmartre. O restaurante é um primor de organização, bom gosto, modernismo, limpeza e modicidade de preço. Come-se muito bem por uns 450 a 600 francos, quando nos restaurantes convencionais a despesa deve atingir o dobro, via de regra.

À noite, vamos à casa do esperantista Syla, que, há alguns anos, trabalha na UNESCO. Mora na ``Rue d’Anteuil''. Alcino, que é ruim de língua, corta uma volta para dizer o nome dessa rua.

Estreio o famoso metrô parisiense, que dá logo um show de organização, amplitude de ramais e clareza nas indicações internas das estações. Recebi, no hotel, um mapinha de bolso e, pronto, fiquei imediatamente senhor de como me locomover seguramente por toda a cidade!

A metrópole francesa não possui bondes e tem pouquíssimas linhas de ônibus, no perímetro central, para ligações complementares. Aliás, esses ônibus, de cor verde, são horrendos, velhos e de linhas arcaicas, os piores que vi até agora em toda a Europa.

Na casa do Syla, Alcino, que recebera uma fita gravada de sua família (era a tal surpresa) põe a dita no aparelho para ser ouvida. Foi uma interessante composição do seu pessoal (esposa e duas filhas), gravada no dia imediato ao ``Dia dos Pais'' deste ano (10 de agosto), com saudações e cantigas, inclusive das meninas do seu internato. Acontece, que todos os familiares revelaram grande emoção nas palavras, entrecortando-as, às vezes, com mal disfarçados soluços, pois, afinal, já estava prestes a completar dois meses de ausência. Resultado: o rapaz ficou inevitavelmente contagiado pelo clima e nós, eu, Syla e Leda, sua esposa, meio desajeitados.

\section*{2 \adfflatleafright \ISODayName{1959-09-02}}
Após o fraco ``dejeuner'' do Hotel, trato, desde logo, do problema do ``visto'' no passaporte para a posterior viagem à Espanha, de volta da Suíça e Itália, que serão nossos próximos rumos.

Eu estava um tanto apreensivo com as afirmativas reiteradas, ouvidas de patrícios espanhóis --- como aconteceu em Varsóvia, por exemplo --- segundo as quais os carimbos moscovitas nos nossos passaportes levariam à recusa de permissão para entrar na Espanha. Na verdade, não havia no passaporte registro de estada na União Soviética. Os russos, sabia e generosamente, davam os ``vistos'' em folha separada, presa por clipe. Depois era só tirar. Mas havia da Polônia, Tchecoslováquia e Alemanha Oriental.

Mas, no Consulado, parece que tudo correrá favoravelmente, pois prometeram o visto para amanhã e até nos cobraram adiantado (2.500 francos).

Saindo dali, vou fazer meu primeiro bordejo por Paris. O tempo, como sempre até agora, está maravilhoso. Tomo o rumo dos ``Champs Elysées'', a avenida mais central, pois de onde estou avisto, no extremo dela, os contornos do Arco do Triunfo, na amplíssima e desafogada praça ``Etoile'' (estrela). No caminho deparo, com agradável surpresa, com uma pequena e bem cuidada praça de nome ``Brèsil'' e não resisto ao desejo de bater uma chapa ao lado da placa.

Os ``Champs Elysées'' dão empolgante perspectiva. Embora a cidade seja praticamente plana, apenas com a pequena elevação de ``Montmartre'' --- onde se situa a magnífica igreja de ``Sacré Coeur de Montmartre'', branquinha, e de estilo oriental mourisco --- o Arco do Triunfo está no topo de suave ascendência do terreno, surgindo, pois, imenso e sobranceiro de onde quer que se olhe. Há, por ali, muita movimentação e a avenida, no seu longo trecho até a ``Place Concorde'', está profusamente embandeirada com os pavilhões da França de dos Estados Unidos. É que o presidente Eisenhower deverá chegar esta manhã e à tarde virá ao túmulo do ``Soldado Desconhecido'', sob o Arco, para alguma solenidade. Até câmeras de TV já estão no local.

Faço mais fotos no cativante cenário e desço os ``Champs Elysées'' vagarosamente, observando o movimento e, especialmente, o comércio, que é de alta classe. As lojas de automóveis Simca, Renault e Peugeot exibem suas últimas criações e os preços são uma punhalada no coração: em torno de 800.000 francos, isto é, Cr\$240.000,00. Isto os de mais classe, como o belo ``Renault Floride'', tipo esporte; o Dauphine custa 500.000 francos, ou seja Cr\$150.000,00. Quando me lembro que paguei, há dois anos Cr\$165.000,00 pelo meu Citroën já com dez anos de uso\ldots

Tomo um café no ``Le Colisé'', para ``provar'' as mesinhas sombreadas no ``trottoir'' (calçada). Pode-se pedir café sem susto. É elegante. Mas é preciso esclarecer ``café noir'' (preto), pois há o sistema mais usual de bebê-lo com creme de leite. O creme vem frio, quase gelado, numa ridícula ``xicrinha'', que mais parece um dedal.

Mais abaixo, o grande e luxuoso cinema ``Marignan'' exibe, há vários meses, o nosso ``Orfeu Negro'', que, no entanto, é anunciado nos programas como franco-italiano. E é verdade. Mas o filme é falado em português e o título está lá com o ``Negro'', na nossa língua. Preço da poltrona: 600 francos (Cr\$180,00).

O cosmopolitismo da cidade é extraordinário, assim como muito grande, imenso, o número de turistas. Creio mesmo que esta é a capital do turismo no mundo. Mas, com tudo isso, a atmosfera é leve a agradável. O tráfego, como em toda a Europa, é tranquilo e silencioso; os veículos com muita tolerância e respeito pelos pedestres, mesmo fora das faixas. Já senti que Paris pede uma permanência mais longa. Vamos ver.

À tardinha, recebo um telefonema no hotel. É do Dr.~Azeredo Coutinho, que representou oficialmente o Brasil no Congresso de Varsóvia. Por coincidência ele está em Paris, tendo chegado no mesmo dia que o Alcino e soube de nossa presença na cidade pelo Syla Chaves. Vamos jantar juntos, os três, no restaurante ``Les Muses'', na ``Place St. Honoré'', perto da Ópera.

Passeando, após a refeição, pela avenida que leva ao maravilhoso teatro e que também tem o nome de Ópera (a avenida), que aqui se pronuncia ``Operrá'', somos abordados por um português gordo e bonachão, proprietário de uma loja de perfumes nas redondezas. Oferece-nos boas vantagens para as compras na sua casa. E por acaso venho a saber pelo homem que os médicos do Instituto do Câncer de São Paulo, Drs. David e Franciosi, nossos companheiros de bordo, reencontrados no Hotel Berlim de Moscou, estiveram na sua loja esta tarde. Repita-se mais uma vez: mundo pequeno\ldots

Prometemos aceitar o convite quando tivermos que comprar perfumes, o que será, naturalmente, inevitável para quem passa por esta cidade.

\section*{3 \adfflatleafright \ISODayName{1959-09-03}}
Vou apanhar o meu passaporte no consulado espanhol.

Nenhum obstáculo, para tranquilidade geral. De lá sigo para visitar a famosa Torre Eiffel, que fica à margem do Rio Sena e junto à Escola Militar.

O metrô resolve o problema da distância com eficiência e rapidez, sem falar no baixo custo. Eis-me aqui ao pé da lendária construção metálica, que, confesso, não me impressionava muito como beleza nem estética, até hoje. Está pintada de um marrom puxado a chocolate. Aliás, sua metade superior eu já havia visto na véspera, das adjacências do Arco do Triunfo.

O parque fronteiro (``Parc du Champ de Mars'') é espaçoso e agradável. Separa longamente a torre da enorme escola militar no sentido perpendicular ao Sena. Sob a torre é grande a aglomeração neste momento, inclusive de carros oficiais. O presidente Eisenhower, ou parte de sua comitiva, deve estar para chegar por aqui.

Subo sozinho ao topo da torre, com uma pausa no que vou chamar de ``primeiro andar''. Alcino, com o espírito de poupança que o caracteriza, apresentou uma desculpa esfarrapada para ficar no chão. Deve ter achado caro o preço: 500 francos (Cr\$150,00) a subida até o alto; até o primeiro andar, 300 francos (Cr\$90,00).

O elevador até o segundo andar é enorme, pois tem capacidade para 70 pessoas. Dali a ascensão é feita em dois elevadores menores, que fazem percurso vertical até o topo, a 317 metros de altura. O panorama da cidade, que se estende à proporção que o elevador ganha altura lentamente, é maravilhoso. Fico um pouco no segundo pavimento antes de galgar a ponta. Diviso, então, pela primeira vez, muita coisa que, até então, só conhecia de fotografia e cinema, como a catedral de ``Notre Dame'', o jardim das ``Tulhérias'', junto ao Palácio do Louvre, hoje museu, o parque descomunal do ``Bois de Bologne'', quase uma floresta dentro da cidade!

A torre passa a merecer minha admiração. Penetrando-se no seu interior pode-se observar o incrível emaranhado de sua estrutura metálica. E recordo-me que Santos Dumont a circundou, em dirigível, há quase sessenta anos atrás!

De regresso, dou um pulo no ``Parc Monceau'', também agradável, assim como que um Campo de Santana maior e melhorado.

À noite, volto a jantar com o Dr. Coutinho, que, amanhã, embarcará para a famosa Cooperativa Esperantista de Grèsillon, um castelo de férias e fim-de-semana, a 4 horas de trem. Acho difícil fazer o mesmo, porque é preciso reserva prévia e, de outro lado, iria perder três dias.

Na mesa ao lado, travamos relações com duas jovens simpáticas, que mal disfarçadamente, se interessavam em descobrir nossa identidade, já que falávamos português e esperanto. Patrícia, a morena de Toronto, Canadá, conhece um pouco de espanhol; ouve nossas informações sobre o esperanto e o congresso havido e demonstra interesse. Tomo o seu endereço, para presenteá-la com um dicionário inglês-esperanto, quando estiver de volta. A companheira é de Nova York e acompanha tudo com ar enigmático. Como ``boa americana'' deve acreditar na hegemonia eterna e impossível do inglês. Enfim, não se definiu. Pena que embarquem amanhã para Londres.

\section*{4 \adfflatleafright \ISODayName{1959-09-04}}
O hotel parece que renova diariamente sua provisão de portugueses. Hoje, para variar, deu entrada uma família de mexicanos. E alguns brasileiros também circulam, desaparecendo de repente. Fui comprar perfumes na loja do português Manoel da Costa, no ``Boulevard Opera''. Depois saí em visita às famosas Galerias Lafayette. São realmente grandes, mas não têm nem de longe o modernismo dos magazines que conheci em Berlim (``Bilka''), Amsterdam e Rotterdam (a mesma organização, ``De Bijenkorf'' --- A Colméia, traduz-se) e em Bruxelas (``Au Grand Marché'' e ``L’Innovation''). Salvo no detalhe de um \textit{hall} sob a cúpula central, que, embora em estilo antigo, é de uma beleza de palácio de Luiz XIV, XV ou mesmo XVI. Parece-me que o salão central da Confeitaria Colombo, no Rio, copia o estilo aqui da Lafayette.

À tarde saí para comprar a encomenda do colega Aristeu, do IRB, no estabelecimento da Citroën (peça de carro). Aproveito e compro também alguma coisa para o meu. Mas trato de providenciar o despacho dessa bagagem para Cadiz, na Espanha, porto de onde embarcaremos de regresso ao Brasil. Levar mais peso na mão, nem pensar. O despacho é feito numa firma vizinha, indicada pelos homens da Citroën.

O programa noturno é o ``Follies Bergères'', teatro de revista, que fica a dois passos do nosso hotel. O Dr. Coutinho, que desta vez aparece para jantar conosco, por ter cancelado a visita a Grèsillon, faz-nos companhia. O teatro está repleto. A fachada é bem inexpressiva, mas o salão de espera é lindíssimo, atapetado e ornamentado ricamente de vermelho.

Na minha opinião, o espetáculo agradou pela variedade do programa (``Follies Legères''), de cerca de três horas, vestimentas e técnica de cenários, além da beleza destes. A interpretação, porém, decepcionou, assim como a coreografia dos bailados, que me pareceu medíocre. Os coadjuvantes se portam como autômatos, não vivendo os papéis; cantores e cantoras vulgares, mas belos físicos femininos, com profusão dos indefectíveis nus. Um número cômico de mímica, este sim, notável, e outro acrobático (grupo de três) também formidável.

É, no conjunto, valeu o espetáculo. No intervalo, por acaso encontramos um casal brasileiro, do Rio; o marido me pareceu de fisionomia conhecida, de altas rodas talvez. Pelo menos, pose não faltava.

\section*{5 \adfflatleafright \ISODayName{1959-09-05}}
Primeiro sábado parisiense. De manhã, minha garganta ``apareceu'' dolorida. A diferença de temperatura entre o dia e a noite é sensível, nesta época do ano, quando o verão já ensaia sua despedida.

Como já estava com vontade de conhecer a Associação Cristã de Moços daqui, já que sou sócio no Rio, procuro localizá-la. Para variar, fica também próxima ao hotel, mas venho a saber que o serviço médico-social está fechado nas férias de verão. Indicam-me o consultório de um médico que trabalha para a Associação e que me examina detidamente, inclusive fazendo-me passar na radioscopia. resultado: ``tout est bien'', apenas uma irritação na faringe. Apresentei a minha carteira e ele nada me cobra, quando me despeço com a receita de uns comprimidos na mão. Ora viva!

Passo uma tarde descansada, observando o movimento na cidade. Em relação aos outros dias, foi, de fato, mais calmo.

Preciso visitar o Louvre, nas Tulherias, a Bastilha, a ``Place Pigalle'', o túmulo de Allan Kardec, no cemitério ``Père Lachaise'' (Dona Leda, esposa de Syla Chaves me informou) e Versailles, que fica a uns 30 minutos. Haja tempo, porque disposição não falta.

À noite, encontramos dois brasileiros na avenida da Ópera. Um, de nome Salvador, residente, e outro, Osmar, de uma firma fornecedora a navios. Conversam muito e nos dão informações úteis. Osmar, sergipano de Aracajú, com trajes e penteado lamentavelmente parisienses, envereda pelo caminho de evidentes lorotas, contando certas aventuras na Suécia.

Marco, em princípio, para amanhã cedo a visita a Versailles.

\section*{6 \adfflatleafright \ISODayName{1959-09-06}}
Uma indisposição de estômago me desanima de acordar tão cedo como pretendia, para ir a Versailles.

E como temos convite do Syla para almoçarmos juntos às 13 horas, não há tempo de ir esta manhã e retornar para o compromisso.

Vou à ``Notre Dame'', a portentosa igreja construída no século XII. Fica numa ilha, aparentemente artificial, do Sena. Bem central. Como é domingo e cerca de 11 horas, o movimento é grande e a missa concorridíssima. O mesmo acontece com a pequena boutique no interior, junto à entrada, que vende \textit{souvenirs} e pequenos objetos sacros.

Apesar das grandes proporções da catedral, Alcino acha-a menor do que a de Gdansk (Dantzig), na Polônia, que eu não conheci. Detalhe: não há bancos fixos no interior e, sim, cadeiras de aspecto modesto, soltas no alinhamento.

De saída, passeio pelas redondezas. Junto à estação do metrô, uma feira de pássaros, com pombos enormes e também os nossos conhecidos ``bicos de lacre''. Na direção da fachada da Notre Dame, encontro o imenso edifício do Palácio da Justiça. À esquerda, ao voltar, vejo a artística Torre de Saint Jacques, ao lado do teatro Sarah Bernhardt, já na margem do Sena, oposta à ilha. O rio corre ali mansamente. Um remador em ``skiff'' passa velozmente. Fazemos umas fotos à margem, com umas meninas americanas.

Encontro Syla e a esposa no Hotel Felix, na rua Molière, onde se hospeda o Dr. Coutinho. Mas, por qualquer motivo, talvez um mal-entendido, ele não nos esperava. Vamos, então, nós quatro, almoçar num restaurante chinês, viajando no carrinho Dauphine de Syla.

Lembro-me da sopa de tubarão que almoçamos em Amsterdam. Pedimos peixe, sopa de ovos e uma espécie de espaguete ``cabelo de anjo'', brotos de bambu (parecendo palmito) e cogumelos chineses (inteiramente diferentes dos europeus). Com arroz, naturalmente. Syla dá a nota pitoresca, comendo com os tais pausinhos, que, por sinal, aqui são de plástico.

Na mesa ao lado, lá para as tantas, começam a falar português em voz alta. É um casal, em que o marido é jornalista da Bahia e está, ha dois anos, rodando pela Europa como correspondente. Mas partirão em breve para o Brasil. O rapaz se chama Matos e, muito viajado, conta coisas elogiosas que encontrou na Alemanha Oriental, Praga e Bucareste. Trocamos impressões sobre Moscou, que ele também diz conhecer bem.

Após o almoço, saímos a passeio até a Cidade Universitária, onde, há uns três meses, foi inaugurada a ``Maison du Brèsil''. O administrador, Sr. Madureira de Pinho, não está, mas a autoridade imediata é velho amigo do Syla, de nome Benito, moreno e baixinho.

A casa é de concepção arquitetônica modernista, adaptada ao clima. O projeto é de Lúcio Costa, mas a construção foi supervisionada pelo próprio Le Corbusier. Há uns tons algo berrantes, para mim, numas fileiras de sacadas. E a bandeira brasileira, hasteada junto ao nome ``Maison du Brèsil'', no jardim, é também futurista: sem a faixa com a inscrição ``ordem e progresso'' e com um tom de verde inteiramente diverso do oficial e verdadeiro.

Percorro as dependências e recebo, em seguida, de Benito, convite para a recepção amanhã à noite, data da nossa independência.

Partindo, apreciamos também, de fora, pois que fechado, o edifício sede da UNESCO, onde Syla trabalha. É de estilo modernista e não pode deixar de causar ótima impressão.

Seguimos, depois, agora também com a companhia de Benito, para uma volta até o ``Bois de Bologne'', o imenso bosque, praticamente dentro da cidade. Agradabilíssimo, se bem que os pisos das calçadas sejam todos de terra fina, bem poeirenta. Syla comenta que as chuvas que aqui caem são fracas e curtas, por isso não provocam grandes lamaçais. A afluência popular é enorme, neste domingo de fim de verão, lembrando a nossa Quinta da Boa Vista, à cunha\footnote{Palavra que entra no verso só para lhe completar a medida; completamente cheio; na medida certa.}. Há dois grandes lagos, curiosamente em planos diferentes. No inferior, que é o maior, pululam barquinhos com remadores, em lentos passeios.

Vamos, ainda no ``Bois'', até a sua decantada ``Cascade'' (Cascata), junto à qual se encontra elegantíssimo restaurante-bar, no bom estilo parisiense de mesas ao ar livre e que também se chama ``La Cascade''. Dona Leda menciona um filme --- parece-me que ``Amor na Tarde'', com Gary Cooper e Audrey Hepburn --- em que várias passagens foram tomadas aqui.

A cascata é parcialmente artificial, isto é, a arcada de pedra de onde a água despenca foi forjada, juntando-se blocos com argamassa; um vão livre lateral permite a entrada na gruta, podendo-se então ficar atrás da queda d’água. Fizemos esse percurso, que é permitido ao público.

Coitados dos europeus! Como precisam se desdobrar para melhorar um pouquinho a falta de graça de suas paisagens! Esta cascata do ``Bois'' pode ser muito interessante para eles, mas para nós a impressão é discreta, ante o que temos em nossa terra, por todo o lado.

Deixando o ``Bois'', Syla nos conduz, de passagem, à Arena. Foi um interessante achado quando se realizavam escavações para obras do metrô: uma autêntica arena romana, dos longínquos tempos da Gália, com anfiteatro, corredores e jaulas para feras e, do lado oposto, para as suas vítimas, talvez, ou outros gladiadores. A arena propriamente tem um diâmetro de 80 metros, pelo meu cálculo.

No jantar, como de costume no ótimo \textit{self-service} Rex, ouvimos vozes ``camonianas'' provindo da mesa vizinha. Impossível resistir. Puxamos conversa logo. Eram quatro lisboetas muito jovens que cumpriam bom período de intercâmbio estudantil de férias, coisa habitual em toda a Europa. Disseram ter estado um mês na Inglaterra e agora, após alguns dias em Paris, preparavam o regresso à pátria.

Ouvi, com muito interesse, as opiniões deles sobre os ingleses e, depois, sobre o sistema educacional vigente em Portugal, que mereceu duras críticas deles próprios.

\section*{7 \adfflatleafright \ISODayName{1959-09-07}}
Passei grande parte da manhã ``trabalhando'' compras e encomendas. Já estou pensando, agora, em despachar minhas duas malas, lotadas e pesadíssimas, para Cadiz, comprando uma de pequeno porte para seguir viagem. Obtive o endereço de um despachante especializado em atender brasileiros: Roger Levy, que, apesar desse nome, é brasileiro de Campinas.

Janto com a antecedência necessária para não me atrasar na recepção da ``Casa do Brasil'', na ``Cité Universitaire''. O Dr. Azeredo Coutinho vem nos acompanhar no jantar, para comparecer também à festividade comemorativa de nossa data nacional. Antes, lá pelas 19 horas, outra festividade teve lugar na Embaixada do Brasil. Resolvi, porém, estar ausente desta, por não poder conciliar a atividade da tarde com a impropriedade da hora deste programa.

O metrô de Paris opera milagres. Em pouco tempo estou saltando na magnífica avenida ao longo da qual se enfileiram os belos pavilhões, de distintos estilos, da Cidade Universitária. Ainda na condução subterrânea, um tipo baixote, com tremendo aspecto de ``pau de arara'', entrou em determinada estação, em companhia de um moreno bem acaboclado. Não podia dar outra: brasileiro do Ceará e representante da revista ``O Cruzeiro'', na França. O moreno, descendente da África francesa, mas bem mesclado, fala um razoável português, aprendido, segundo eles, em apenas três meses de convivência com o patrício. Iam também para a recepção.

Na \textit{maison} reencontro outro participante do festival de Viena e companheiro de viagem no ``Cabo San Vicente'': Luiz Vergueiro, tipo divertidíssimo, que alcançou o primeiro lugar no baile de disfarces de bordo, com a sua impagável fantasia de imperador Nero. Também lá estão os artistas patrícios Jorge Goulart, Carmélia Alves, Nora Ney, Jimmy Lester (marido de Carmélia) e Hélio Mota, este último praticamente desconhecido no Brasil, mas fazendo sucesso como cantor, há vários anos em Paris.

Dali a pouco surge o Syla com duas cartas para mim. Exultei de satisfação, pois ambas eram de colegas do IRB e eu estava ansioso por saber se vinham recebendo minhas cartas e cartões postais. Uma notícia, porém, me entristece: o falecimento da colega Nancy, em consequência de parto mal sucedido. Imaginei a dolorosa realidade: o que se espera como festa tornar-se o mais pungente drama!

Mas o ambiente, em festejo de nossa data magna, é de animação. Música e dança, ``drinks'' e ``bufê''. Com nova agradável surpresa, deparo com duas outras festivalistas conhecidas: Daisy e sua irmã, de São Paulo; contam que passaram quase um mês na Polônia e adoraram. Vão ficar ainda mais tempo na Europa, salvo algum imprevisto --- dizem elas todas satisfeitas.

Não pudemos esperar o show dos cantores brasileiros. A hora vai adiantada e há o problema do metrô, cujo último carro passa por aqui a 0 hora e cinqüenta minutos. A vida noturna parisiense está limitada às casas de espetáculos centrais. No comum das noites, o movimento das ruas é bem fraco. Ontem e anteontem, sábado e domingo, a afluência foi explicavelmente maior. Somente a estupenda avenida dos ``Champs Elysées'' apresenta índice apreciável de movimentação, principalmente porque ali se congregam de preferência os turistas.

Assim, minha impressão é de que a decantada vida noturna parisiense é a mesma das outras capitais: dentro de bares, ``boites'' e cabarés. Mesmo a centralíssima e elegantíssima avenida da ``Operrá'' anda às moscas depois das 21 horas. Sempre passávamos por ali quando jantávamos com o Dr. Azeredo Coutinho, na ``Place St. Honoré''.

Acontece que o Syla acenou com uma carona para a cidade e, assim, o conforto foi maior. Afinal, já não era mais 7 de setembro e a festa ia mesmo acabando.

Boa impressão quanto ao número de brasileiros em Paris, que me pareceu surpreendente. É bem verdade que muitos, como nós, eram turistas em trânsito, mas é bem acentuada, sem dúvida, a quantidade de estudantes, bolsistas e mesmo residentes, em ocupações diversas.

\section*{8 \adfflatleafright \ISODayName{1959-09-08}}
Estou preocupado com um rolo de filme fotográfico que o laboratório declarou extraviado, de minha encomenda de dois rolos, para revelar. Será uma lástima se não aparecer. Procuro me recordar onde foi batido e logo me lembro: ``zoo'' de Berlim, outros trechos na grande cidade alemã e em Amsterdam. A casa, lá nos ``Champs Elysées'', promete encontrar antes de minha partida.

Resolvo procurar o escritório da Panair do Brasil, onde dizem que sempre se encontram jornais brasileiros, mais ou menos recentes. Afinal, há exatamente dois meses e um dia não sei de nada que lá se passa, embora minha curiosidade seja primordialmente relativa às atividades esportivas. Sobre o restante, sei que posso descansar muitos meses ainda, sem sentir falta.

Afinal, localizo na Avenida Montaigne, junto ao Hotel Plaza, o \textit{bureau} da nossa companhia de aviação. Duas atendentes, francesas, falam português com forte sotaque, mas bem compreensível. Fico logo à vontade, numa poltrona, frente a um maço de jornais cariocas, se bem que a mais recente data do dia dois deste mês. Confesso que foi uma satisfação enorme! Até o intragável ``O Globo'' me pareceu delicioso. Por algumas reportagens retrospectivas deslindei a marcha do campeonato carioca: Botafogo e Fluminense na frente; Flamengo e Vasco fazendo feio, atrás.

Súbito, entram dois senhores brasileiros, um deles fazendo piadinhas com as funcionárias. A voz não me parece estranha. Levanto os olhos e descubro que outro não é senão o Dr. João Carlos Vital, primeiro presidente e organizador do IRB, além de ex-prefeito do Rio de Janeiro. Conversamos, pois a minha condição de funcionário do Instituto é suficiente para tomar esta liberdade. Recebo então notícias também do IRB, principalmente quanto ao cancelamento da viagem do atual presidente, que eu acreditava já se encontrar na Europa.

Deixo a Panair em companhia de outro brasileiro, por coincidência também hospedado no meu hotel. É um rapaz de Alagoas, bolsista de estrada de ferro, que já está há três meses aqui e que ainda vai ficar outros três.

Depois do almoço vou conhecer mais de perto o Jardim das Tulhérias, em frente ao antigo Palácio do Louvre, hoje museu, talvez o mais antigo e mais famoso do mundo. Tanto o palácio como o museu são de uma extraordinária beleza. Ali residiram não os Luízes de França, que viveram em Versailles, mas Napoleão III e seus sucessores. A amplitude das alamedas e a perspectiva fronteira da praça da Concórdia, com sua alta coluna (bem ao estilo francês) e, em profundidade, a Avenida dos ``Champs Elysées'', elevando-se até o Arco do Triunfo, dão ao local espetacular panorama a descortinar, bem no centro da cidade.

Impressiona-me a quantidade de pessoas espalhadas por todos os recantos do jardim. Faço boas fotografias. Ao subir para a estação do metrô, vislumbro o monumento a Roger du Nobre, o genial criador deste jardim e de quase todos os outros do gênero em Paris, inclusive Versailles. Um homem que morreu em mil setecentos e noventa e poucos!

Vou, depois, até a Bastille, à margem do Sena. Ali se erguia a famosa cidadela- prisão derrubada na lendária revolução de 1789. No centro da praça há uma alta e belíssima coluna (sempre elas), com inscrição e ornamentação a ouro. Na subida da estação do metrô ainda se encontra uma parte remanescente das paredes de prisão, bem à margem do rio.

Espetáculo curioso foi o de um tipo, na larga calçada da praça. Todo amarrado com grossas correntes, propunha-se a se libertar delas, desde que os circunstantes ``pingassem'' ali no chão uma certa quantidade de moedas. Até para fotografá-lo tive que pagar, pois ele, ao me ver com a máquina apontada, fez o clássico gesto de esfregar de dedos, exigindo o ``royalty''. Afinal, escapuliu mesmo das cadeias, sem as arrebentar, é claro, mas escorregando o corpo para baixo, lentamente, até jogar as correntes nos ombros\ldots

Janto pela última vez com o Dr. Coutinho, que amanhã cedo embarca para Colônia, na Alemanha. E depois, no café reencontrando o patrício Salvador em companhia do português dos perfumes, surpreendo-me agradavelmente com o seu interesse pelo esperanto e expondo longamente assuntos relacionados com o uso atual do idioma internacional.

\section*{9 \adfflatleafright \ISODayName{1959-09-09}}
Pela manhã volto à agradável loja do português para comprar mais perfumes. Soube que a alfândega do Rio não cobra taxação até uma partida de 10 a 15 vidros diferentes de perfume. Faço, assim, mais um estoquezinho, além das encomendas, que não são poucas.

Marco para depois de amanhã o embarque para a Suíça e, logo, reservo a passagem numa agência do ``Wagon, Lits Cook''. Posso, amanhã, visitar calmamente Versailles, como despedida desta grande cidade, talvez a mais decantada em todo o mundo.

Despacho, finalmente, minhas duas pesadas malas para Cadiz, na Espanha, local de embarque para a viagem de volta. O preço foi bem alto, incluindo seguro, mas espero que cheguem direitinho ao destino, com todas as ``moambas'' em perfeita ordem. Comprei uma maleta de tamanho médio, como planejava, para, daqui por diante, viajar confortavelmente, apenas com o mínimo necessário e indispensável.

Recebo um convite, por intermédio do Syla, para participar do jantar dos esperantistas parisienses, promovidos duas vezes por mês, sempre às quartas-feiras. Compareço com grande satisfação. O programa teve lugar no Restaurant Saint Marie, na Rue des Saussieus, no bairro Mirosmenil. Como convidado esteve também presente o presidente da Associação Esperantista da Áustria, em companhia de sua esposa.

Embora faltando alguns elementos de proa, um dos quais por enfermidade, o jantar transcorreu agradável, com muito interesse pelo Brasil e pelo congresso de Varsóvia. Minha viagem à União Soviética despertou, igualmente, curiosidade, tendo sido instado a responder interessantes perguntas sobre a vida em Moscou e o movimento esperantista naquele grande país.

Deixando os amigos, proponho a Alcino uma visita a um dos cabarés da ``Place Pigalle'', em Montmartre. Coisa típica de Paris, tão falada no mundo inteiro. A praça é pequena e inexpressiva. Seu mérito é a existência quase que exclusiva de cabarés em seu derredor. O mais famoso é o ``Le Naturaliste''. Mas o preço nos pareceu exorbitante e o ``Narcise'', bem próximo, mostra-se da mesma qualidade.

O espetáculo, sob o ponto de vista artístico, é decepcionante. Os elementos surgem no pequeno palco e representam seus papéis com quase nenhuma vida. Os cantores e cantoras inteiramente medíocres. Apenas o \textit{streaptease} desperta interesse, pois a afluência é grande e o show longo.

A exploração nas bebidas é alarmante: 1.200 francos por simples copo de cerveja. E as garotas estão lá, matreiras, para se fazerem convidar a\ldots\ serem convidadas a beber por conta dos incautos.

\section*{10 \adfflatleafright \ISODayName{1959-09-10}}
Em véspera de deixar Paris vou à casa fotográfica para saber do meu filme extraviado. E recebo, contentíssimo, a informação de que ele apareceu e com as ampliações feitas, embora não pedidas. Mas como estão excelentes, nada reclamo e fico com elas.

Dou outra passada na Panair, mas os jornais são os mesmos. Logo mais à tarde é que chegarão novos. Então, fica para a Suíça, onde creio que a Panair também serve.

Após o almoço e outro bom papo com o alagoano, no hotel, saio para visitar Versailles. É um pouco afastado da cidade, ou melhor, um município independente. Pode-se ir de trem ou de metrô, neste até a ``Pont de Sèvres'' e dali seguindo de ônibus, o de nº 171. Escolho o segundo meio. Em Sèvres, também junto ao rio Sena, encontra-se pelo menos parte das fábricas dos automóveis Renault, de grandes proporções.

Versailles se apresenta com o palácio atrás de um grande pátio, apenas calçado e com uma enorme estátua equestre ao centro. Deve ser de um dos Luízes, pelos trajes, mas não está identificada. Transpõe-se a parte central do palácio, que é de quatro pavimentos e larguíssimo, ou se contorna e surgem os jardins descomunais, a se perder de vista, lá num lago em plano inferior. É um belíssimo local, como ornamentação floral, sem dúvida, mas a parte ajardinada é relativamente pequena: mais de três quartos se compõem de bosque, em plano mais abaixo. Há uma balaustra separando as duas partes.

``Peruo'' uns guias de grupos turísticos organizados, em visita ao local. Chamam a atenção para o fato de que, na época de seu esplendor, Versailles era o maior palácio do mundo. Um simples olhar é suficiente para se estranhar o exagero daquelas famílias imperiais quanto ao espaço em suas faustosas moradias! Mas lá deveria morar também um imenso séquito de nobres apaniguados e respectivos serviçais.

O jardim e o bosque, profusamente decorados de estátuas em mármore, no tamanho original de seus motivos, deslumbram e repousam a vista e o espírito. O olhar em três direções, excetuada a do próprio palácio, encontra bosques e suaves colinas. Os imperadores e sua malta de parasitas tinham, inegavelmente, bom gosto em excesso! Por isso mesmo o ``bem-bom'' não durou muito.

Das laterais do imenso alpendre, no mesmo plano do palácio --- maior, talvez, que um campo de futebol --- avistam-se abaixo os jardins tropicais; são palmeiras e outras espécies interessantes, todas plantadas em grandes vasos trabalhados. Passado o verão, são recolhidas às estufas, ali na parte inferior do próprio palácio.

Realmente decorativo e imponente, no conjunto. Mas confesso que acreditei encontrar um deslumbramento mais agressivo, algo assim de embasbacar, tal o renome de Versailles. Isso não aconteceu absolutamente. Apenas belo, como tudo o que é quase 100\% natureza. O palácio está transformado em museu. Seus interiores são maravilhosos, pela riqueza da decoração, dizem. Vi apenas um pequeno setor, da entrada. O meu interesse era quase exclusivo pelos jardins. Talvez --- ou com certeza --- por causa da minha velha ojeriza por faustos de famílias reais, de qualquer lugar e de qualquer época\ldots

\section*{11 \adfflatleafright \ISODayName{1959-09-11}}
Vamos embarcar para Zurich às 8 horas da manhã.

O Hotel Peyris tem ambiente lisboeta, como já disse. A senhora proprietária perdera uma nora na véspera e estava um tanto abalada. Por isso, deixou a extração de nossa conta para hoje de manhã e, com a demora, o tempo voa e nada da mulherzinha apresentar o nosso débito.

Olho para o relógio assustado, por causa do horário de partida do trem. Dizem que os trens franceses são desagradavelmente pontuais. É verdade que a estação fica perto, mas, com tamanho atraso, nem que o trem saísse da porta do hotel.

Começo a me convencer de que, pela primeira vez na vida, irei perder uma passagem comprada com antecedência. Afinal, pegamos e pagamos a conta e vamos para a porta da rua esperar o táxi chamado pela portaria. Agora é o táxi que se faz esperar. O gerente e alguns serviçais vêm para o nosso lado ajudar na aflição. Todos de olho nos relógios. ``Cinco minutos ainda, dá bastante tempo, a estação é logo ali\ldots'' --- diz um deles.

Enfim, a gare foi atingida a dois minutos da partida. Disparada pelo pátio, felizmente com a bagagem reduzida a uma simples maleta, além da passagem reservada antes, o susto foi superado.

Enquanto o comboio --- decepcionantemente puxado a carvão --- começa a se mover lentamente, conjeturo sobre Paris. Quanta coisa merecedora dos maiores elogios e quanta também de não fazer inveja a ninguém. Vi os dois lados da cidade, material e socialmente falando. A pobreza é um capítulo doloroso, frisaram muitos com quem conversei (franceses inclusive, como os esperantistas da Associação Mundial dos Sem-Pátria). Vi os camelôs das próprias Avenues Montmartre e Bonne Nouvelle, no coração elegante da cidade, no qual, aliás, nos hospedamos. E nos passeios, pelos bairros mais distantes e pobres, muito também pude observar naquele sentido. Os amigos Syla Chaves, Benito, da Casa do Brasil, e outros afirmaram que certas favelas aqui deixam longe as do Rio em sordidez e miséria, agravadas pelo clima.

O metrô, pela extraordinária ramificação, foi o ponto alto de Paris para mim. Defeito grande, porém: a ausência quase absoluta de escadas rolantes. Sobe-se e desce-se por tremendas escadarias e corredores de ligações nas ``correspondences'', como são chamadas as conexões das várias linhas em uma mesma estação. Para os idosos e doentes isso representa fatalmente imenso sacrifício. O metrô moscovita abusa das escadas rolantes moderníssimas, em fileiras, às vezes, até de quatro. Explicam os franceses que o \textit{subway} de Paris é muito antigo (começou a funcionar em 1901) e vem, assim, de uma época em que ainda não se conheciam as escadas mecânicas. É uma justificativa. O metrô de Moscou foi inaugurado em 1935, ou por aí.

Outro capítulo: o do pessoal esquizofrênico. É muito grande o número de jovens com ridículas e anti-higiênicas barbichas, além de trajes horrorosos em composição, asseio e desleixo. Uma juventude que parece ``borboletear'' pelo mundo, sem saber o que fazer da vida.

Mas o trem se afasta, agora rapidamente. Lamentei não ter tido tempo para mais coisas. Os sucessos de Josephine Baker (aqui ``Baquér) e de Marcel Marceau, o grande mímico, por exemplo. O filme soviético ``Quando passam as cegonhas'', que não consegui localizar.

Volto-me para observar, da janela do trem, o perfil distante da cidade. E esta foi a última vez que vi Paris!

Os trens franceses são famosos pela estabilidade e velocidade. Mas o carvão ainda está muito utilizado. Teremos de fazer baldeação em Basel (ou Basiléia) na fronteira. Passamos Sezane, Vitry-le-François e Chaumont. Sempre que o trem chega, o alto-falante da estação identifica a localidade para os viajantes: ``Ici Chaumont'' (``Aqui Chaumont'') diz ele agora, e acrescenta informação sobre o horário da partida ou outra qualquer de interesse.

Na nossa cabine viaja Jocelyne, linda jovem de 16 anos, mas que aparenta no mínimo 20. Conversamos longamente, com a participação também de um velho francês, homem rude, que embarcou poucas estações atrás. Jocelyne coleciona postais e diz não possuir nenhum do Brasil. Tomo o seu endereço para enviar alguns do Rio e ela esfrega as mãos de contente.

O velhote ``bobeou'' em Vitry, onde devia saltar e rodou mais uma hora até a estação seguinte, grudado na janela com ar desconsolado.

A garota saltou em Chaumont e ficou na plataforma até o trem sair, para nos dar um adeusinho. Deve ter achado fantástico dar de cara com duas raridades sulamericanas neste pacato interior francês. A atmosfera de interior é internacional. Isto aqui bem poderia ser, além da França, qualquer Santa Rita de Jacutinga, lá na minha terra.

Perto da localidade de Langres, tenho a atenção voltada para uma coisa branca, no alto de uma colina. Uma passageira esclarece que é uma capela projetada por Le Corbusier. A forma é singularíssima.

Chegamos à Basiléia às 16 horas e pouco minutos. Bem junto, no território francês, está a cidade de Saint Louis; o trem arranca lentamente uns poucos quilômetros e pára em Basiléia, na ala francesa da gare. Saltamos e seguimos ao longo da plataforma. No meio está a alfândega suíça, que pouco quis saber a nosso respeito.

Faço um lanche, já na Suíça e, portanto, em um bar-restaurante com os dizeres todos em alemão; aliás, o mesmo acontece com os letreiros da estação.

Alcino, ao deixar sua bagagem sem maiores cuidados na plataforma, para fazer câmbio, foi advertido por um passante, que aconselhou atenção e vigilância para com as suas coisas. E fez um gesto, como querendo significar ``senão roubam tudo''. Então também na Suíça acontecem essas ``belezas''? A diferença é grande entre a lenda e a realidade.

Uma hora depois seguíamos viagem, para atingir Zurich 90 minutos mais tarde. Mas esse pequeno trecho me entupiu de admiração. Logo ao largar, a linha férrea toma a margem do rio Reno, que assim, tenho o prazer de conhecer. De outro lado é Alemanha. Estou na confluência de três países.

Ah, os rios famosos da Europa que a gente estuda nas aulas de geografia, nos tempos de colégio! E quantos agora estou conhecendo? Recordo, de memória, os já vistos: o Tejo, o Pó, o Danúbio, o Vístula, o Moscou, o Óder, o Elba e o Sena. Incorporo mais esse famoso rio ao meu roteiro visual. O panorama é cativante, e o cartão de visita da Confederação Helvética não poderia ser de melhor qualidade. O dia --- como sempre! --- está magnífico. As pequenas cidades e mesmo casas de fazenda são interessantíssimas, principalmente pela arborização e floricultura exuberantes. Os rios menores e, mais tarde, o grande Aare, têm águas cristalinas, volteando caprichosamente no meio de belos bosques e, por vezes, densas florestas. Um brinco!

Passamos por Baden, a famosa estação balneária. Foi o tempo bastante para se constatar a classe do lugar, concorridíssimo, com grandes e belos hotéis. A localização sugere maravilhosas paisagens ali bem perto.

Finalmente, saltamos em Zurich, já no entardecer. Vamos para o ``Hotel Mora'', não longe da estação e razoavelmente central. Depois do banho e do jantar, saímos a passear pela cidade, que, apesar de já ser noite, se me afigura belíssima. A rua central chama-se ``Banhof-strasse'' (rua da Estação) e é profusamente tomada pelas lojas, das mais elegantes, e de bancos e companhias de seguro e resseguro. As sedes destes estabelecimentos paracem prédios públicos, de tão graves e imponentes.

Nem bem dava uns passos, ao sair do restaurante, surge um rapazola me dirigindo a palavra em português! É um estudante de família paulista, já há dois anos aqui. Disse ter me identificado como brasileiro pelo bigode e, depois, pelos trajes. É interessante que uma vistosa vitrine da praça fronteira à estação --- portanto, bem central --- ostenta grande cartaz oferecendo, a preços convidativos, uma viajem ao Brasil (``Brasilien'', em alemão).

Outro detalhe que logo ressalta é a quantidade de bandeiras suíças por todos os prédios. E outra bandeira azul-branca, com uma folha de trevo no meio, que venho a saber referir-se a uma exposição de hortofloricultura. O jovem patrício esclarece que o ``bandeirismo'' aqui é permanente, pois os suíços são extremamente patriotas e maníacos por ver o pavilhão nacional por todo lado, sem falar nas flâmulas, para variar.

Ainda com o rapaz, tomo um café num bar superalinhado, moderníssimo de instalações e decoração, e com magnífica orquestra. Foi o programa de encerramento da noite. Quanta coisa nova e diferente. E pensar que hoje de manhã eu ainda estava em Paris!

\section*{12 \adfflatleafright \ISODayName{1959-09-12}}
Faço um turno turístico pela cidade, de ônibus especial, segundo os planos habituais de turismo de todas as grandes ou mais importantes cidades européias. Faço amizade com um hindu muito simpático, falando bem inglês, e nos sentamos juntos no excelente ``pullman''.

À luz do dia, Zurich, como eu presumia, dá um show de beleza. Corremos todos os seus lugares pitorescos, cujo ponto notável é, naturalmente, o estupendo lago, enorme, em torno do qual a cidade se estende. O guia vai enumerando, logo de saída, os edifícios e os monumentos. Impressiona a concentração de estabelecimentos de cultura e institutos de especialização técnica. Coisa sem par, talvez, na Europa. E que maravilhosas sedes, ajardinadas e abundantemente floridas! A tranquilidade quase paradisíaca do lugar convida, realmente, às mais profundas meditações. Imponentes, também, algumas igrejas. Há católicas, protestantes e até uma sinagoga.

Em um local mais elevado acha-se o Clube Dolder, espetacularmente instalado e aparelhado, no bom estilo suíço de conforto e absoluta limpeza. Saltamos do ônibus para visitá-lo. Possui duas piscinas: uma, de uns 60 metros de comprimento, faz onda (mecanicamente); outra, menor, destina-se às crianças.

Terminado o giro, procuro a sede da Panair do Brasil, para passar os olhos em jornais do Rio mais recentes. Mas é tarde. O expediente aos sábados se encerra ao meio-dia.

Nos restaurantes, já havia observado desde a véspera, um detalhe bem característico: as garçonetes (sempre senhoras maduronas na Europa central) trazem para a mesa um leve fogareiro de folha, à vela, bem elegante, para sobre ele depositar as bandejas com as pedidas quentes; assim, o calor da comida é conservado. Creio que esta prática é mais justificada no inverno daqui, que deve ser tremendo.

Após o almoço, vou fazer umas boas fotos à beira do majestoso lago, um pequeno e plácido mar, para todos os efeitos, na vida da cidade. Lanchas muito grandes, para giros turísticos (semelhantes às de Amsterdam), lanchinhas-automóveis, ``cutters'', pedalinhos, tudo isso transita pelas tranquilas águas do ``Zurich See'', como é chamado o lago.

A arborização e ajardinamento da via pública beira-lago são estupendos, mormente pelo colorido esfusiante das flores, espalhadas por toda parte. Ao atravessar a rua, em certa pracinha de maior trânsito, noto no chão, junto ao meio-fio, como que um tapetinho de borracha frisada, aparafusado na calçada, com barra metálica. O pedestre, ao pisá-la faz a transição do sinal em seu favor; se ninguém fizer isso, o sinal fica aberto permanentemente para o trânsito.

Lembro-me de um túnel em Gênova, no meio de cujo interior havia também um sinal de trânsito manipulável pelos pedestres, por meio de um botão na parede. O estranho, nesse caso, é que o sinal interrompia o trânsito dentro de um túnel, até não muito longo. Coisa de italiano\ldots\ Também em Bruxelas há um detalhe desses numa de suas principais vias, a Avenida Adolpho Max.

Faço uma travessia sensacional do lago pelo alto, no serviço de carrinhos suspensos, que têm capacidade para quatro pessoas, no estilo Pão de Açúcar. As fotos que bati lá de cima devem ficar magníficas.

Desde após o almoço verifico que o embandeiramento aumentou barbaramente! A cidade parece um arraial de festa junina. Deve ser por causa do sábado. Será que amanhã vai ``piorar'' mais ainda?

A noite é movimentadíssima. Lamento, porém, que o sol se tenha ido. Esta pérola de cidade, de 700.000 habitantes (maior que Recife) é, sem dúvida, das mais belas da Europa, ou melhor, do mundo. Imagino o panorama que aqui se tem no inverno, com a neve assomando por aquelas altas montanhas circundantes!

Numa banca de jornais vi afixada uma enorme manchete, falando em ``rocket'' soviético; apesar da língua (alemão) posso compreender que se trata do lançamento de outro foguete russo com destino à lua.

Amanhã cedo deveremos partir para Berna, a capital.

\section*{13 \adfflatleafright \ISODayName{1959-09-13}}
Pegamos o trem às 9,30, para uma rápida viagem de hora e meia pelos velozes e magníficos carros elétricos da S.C.C., a primorosa ferrovia suíça. Um reparo, contudo: é a primeira vez que, na Europa, encontro uma segunda classe com bancos de madeira, se bem que confortáveis pela forma anatômica. Talvez porque a viagem seja curta.

Procuramos o Hotel Wächter, por indicação do anuário esperantista, nosso prestimoso companheiro de viagem. Neste hotel se realizam as reuniões dos esperantistas locais, que aqui têm um órgão importante, a Svisa Esperanto Instituto, onde eu devo encontrar correspondência, pois dei o seu endereço à família e a amigos, desde Varsóvia.

Como é domingo, a cidade está algo sem vida, principalmente quanto à atividade comercial. Almoçamos calmamente. Sempre o sistema suíço de fogareiros de lata, todos de cor bege, a conservar o calor dos pratos.

Saio, em seguida, na direção do instituto esperantista, que funciona na residência de Mme. Fischer. É um edifício de apartamentos bem moderno e elegante. Mas ela não está no momento. Deixo um bilhete dizendo que voltaria à noite.

Fico, então, a percorrer a cidade. Tendo apenas 150.000 habitantes, Berna é uma sossegada sede de representações diplomáticas e, naturalmente, do governo helvético. As ruas têm aspecto \textit{sui generis}, principalmente as centrais ``Kramgasse'', ``Martgrasse'', ``Spitalgasse'' e outras paralelas, que conduzem da estação ferroviária à curva do rio. Este é o mesmo Aare, que bordejamos desde a saída de Zurich.

O palácio do Parlamento e a Catedral são os dois grandes monumentos de Berna.

Vou visitar a famosa fossa dos ursos. Este animal é o símbolo da cidade, cujo nome, em um dialeto suíço, quer dizer urso (``bern''). Em todos os brasões e mesmo nos \textit{souvenirs}, lá está a figura do vistoso animal. São três fossas, com quatro enormes ursos em uma, três em outra e a terceira destinada aos filhotes.

A torre do relógio é outro espetáculo típico da cidade. Leio que o relógio, que é astronômico em uma seção inferior, funciona desde 1559! É um espetáculo de rara curiosidade a visão da rua ``Kramgasse'', com suas casas iguais geminadas, suas janelas profusamente floridas, com as galerias sobre as calçadas e a torre do relógio se destacando ao fundo. Também interessantíssimos os monumentos de pequenas figuras coloridas, no alto de finas colunas, no centro de fontes circulares colocadas bem no meio dessas ruas. Sente-se o deslocamento dessa paisagem no tempo, mas é ela conservada sem retoque até os nossos dias. Coisas que só existem em Berna.

Depois do jantar voltei ao instituto esperantista e, então, encontro ``Frau'' Fischer. Ela me entrega duas cartas de casa. Nossa palestra é longa e agradável, pois Mme. Fischer é uma renomada esperantista em seu país. Deixo um exemplar da revista de nossa Cooperativa Cultural do Rio, em retribuição aos folhetos turísticos sobre Berna e a Suíça, editados em esperanto, que ela me oferece. Foi um contato bastante simpático.

A cidade está mais animada à noite. Não consigo ir ao cinema, para ver o ``Diário de Anne Frank'' ou ``Imitação da Vida'', bons filmes em exibição. O sistema de ingresso é o mesmo quase geral na Europa: lugar marcado, preço conforme a localização (quatro ou mais tipos) e entrada só no início das sessões. Este critério tem me afugentado de cinemas nesta viagem.

\section*{14 \adfflatleafright \ISODayName{1959-09-14}}
O companheiro Alcino está ansioso para procurar o Escritório Comercial do Brasil em Berna, cujo chefe é seu amigo, o Dr. Gentil de Castro, conhecido pediatra e ex-vereador, no Rio de Janeiro.

Ele nos recebe com muita amabilidade. O escritório está muito bem instalado em várias salas e tem 5 ou 6 funcionários, dos quais pelos menos 3 brasileiros. Um desses é o cearense Jorge Rabay, muito falante e bonachão; traz da rua a novidade de que o foguete russo (sobre o qual eu lera na véspera) havia atingido a lua!

Dr. Gentil nos convida, pouco depois, para um passeio pela cidade, no seu luxuoso Chevrolet 59. Os pontos pitorescos eu, naturalmente, já conhecia, salvo a piscina pública, que ``encheu as medidas''; sim, porque ela mede seguramente 80 x 80 metros. Se soubesse, na véspera teria ido lá dar uma caída, para matar a saudade. Desde o navio não sei o que é dar umas braçadas.

Dr. Gentil tem outra versão para a mania das bandeiras: tradição legada pela necessidade de fixar pontos de referência coloridos para contrastar com a vastidão branca da neve. Faz sentido.

Despedimo-nos dos patrícios, que nos proporcionaram agradável convívio. Rabay me dá endereços de amigos em Roma e Madrid. Dr. Gentil escapole discretamente e retorna com presentinhos: um isqueiro muito bonito para cada um.

Depois das últimas compras (Alcino se abastece de relógios para quase toda a família), decidimos embarcar às 16 horas para Roma, via Milano, para a dormida, visto como lá chegaremos em torno das 21 horas.

Dr. Gentil e seu auxiliar Elias comparecem simpaticamente ao embarque. A viagem, para mim, é motivo de grande expectativa: irei atravessar os Alpes e avistar os elevadíssimos picos onde a neve é eterna. E, assim, pela primeira vez meus olhos verão neve verdadeira, ao vivo! Escolhi o horário do trem calculando a hora de passar pelos picos, de modo a alcançá-los com luz do dia.

Dura ilusão. Logo na saída o tempo se mostra nebuloso, fechado, com prenúncio de fortes chuvas. Não me conformo. Logo nesta passagem tão ansiosamente sonhada?! Mas é a crua realidade: desaba um ``toró'' respeitabilíssimo.

``Una giornata bruta'', no dizer do senhor italiano que viaja em nossa cabine. O panorama é nenhum por bastante tempo.

Os túneis são sucessivos e intermináveis. O maior, ao contrário do que supunha, não é o de São Gotardo, mas um outro, já no território italiano. O trem mergulha nele por quase trinta minutos.

Quando a tempestade passa, já a elevação não é das maiores e, além disso, a luz do dia entra em colapso rapidamente. Rogo pragas de desolação.

Perco também o pleno panorama do belíssimo lago Maggiore, imenso, superior mesmo, em tamanho, ao de Zurich. Duas estações, no nosso caminho, estão junto ao lago: Stresa e Arona, esta última destino dos nossos companheiros de cabine. As luzes já estão acesas nessas duas cidades e a amplidão e beleza do lago ficam razoavelmente destacadas.

Finalmente chegamos a Milão, pela segunda vez, fechando nosso avantajado circuito. Preciso recorrer ao princípio deste diário para localizar as datas de nossa anterior estada na grande cidade italiana: 25 e 26 de julho, há 50 dias, portanto.

Uma nota inesperada: Milão está praticamente sem vaga nos hotéis. Realizam-se, no momento, três feiras distintas na cidade e a afluência de visitantes é enorme. Cansamos de procurar hotel e então decido prosseguir viagem para Roma, dormindo no trem. Antes já havia eu pensado na hipótese, mas na base do leito (``cuchete''). Não havia, porém, mais leito para o trem das 22h45m.

Com um bom lanche no ótimo bar da maravilhosa estação milanesa, espero calmamente a partida do comboio seguinte, marcada para uma e dez da madrugada. Se estava pensando que iria viajar pouca gente, muito me enganei. A Itália é terra de gente sobrando aos magotes por todo canto, ainda mais numa condução para Roma. Nossa cabine ficou lotada.

Um incidente engraçado na nossa cabine, logo de saída: dois passageiros, um deles abissínio mesclado, certamente de há muito radicado na Itália, quase se atracam por problemas de reserva de lugar. Já havíamos todos interferido com intuitos apaziguadores, quando o caboclo menciona que, ``pensando bem'', conhecia o outro ``não sabia de onde''. Pronto, dali por diante não foi preciso mais que um minuto para surgir a lembrança, seguida de aperto de mão final e a discussão virar palestra amistosa, com os dois empenhados agora em recordar passagens do alegado conhecimento\ldots

Assim é a Itália. Que grande povo! Quando dois italianos se encontram, adeus silêncio. Às vezes um calorzinho de discussão, mas até para insultar a língua é bonita e gostosa de ouvir.

A viagem tornou-se algo pesada pelo aperto na cabine lotada, pois volta e meia o parceiro do lado cochilava e tombava para cima da gente. Mas, de verdade, ninguém conseguiu dormir. O trem, apesar de ``diretíssimo'', parou bastante e o fez mais demoradamente numa certa estação. Espiei para fora e vi qual era: Firenze (Florença), que eu bem que gostaria de conhecer normalmente. Mas já está fora de cogitação, lamentavelmente. O programa agora vai mais disciplinado, pois o dia do regresso, 5 de outubro, está se aproximando.

\section*{15 \adfflatleafright \ISODayName{1959-09-15}}
Chego a Roma, com o atraso inesperado do trem, depois das 10 da manhã. O cansaço me domina arrasadoramente. Apesar da emoção de pisar uma tão famosa cidade, o programa é banho e cama em primeiro lugar.

A estação ferroviária central, de nome ``Termini'', é moderníssima e faz excelente impressão. Um táxi me leva à ``Via Orazio'', endereço da pensão ``La Quiete'', anunciada no anuário esperantista e onde espero também por correspondência. Fica pertíssimo do Vaticano e, conseqüentemente do Castelo de Santo Ângelo, à margem do rio Tevere (Tibre, nos mapas brasileiros).

Havia uma carta de casa e outra de Josée, a amiga belga. Outra foi recusada e devolvida pela proprietária, por ter chegado antes do meu aviso e reserva de hospedagem.

Durmo profundamente até a hora do jantar. Depois saio para um passeio noturno. A cidade não tem, no centro, artérias espaçosas. As antiguidades estão espalhadas por todo lado, em autêntico deslumbramento. No trajeto do táxi, na chegada, já havia conhecido alguns locais interessantes, como a famosa ``Piazza de Espagna''.

Meu passeio começa pelo Vaticano, aqui vizinho. Mesmo à noite a impressão da ``Piazza de San Pietro'' e da basílica é extraordinária. Que majestosa obra de já mais de quatro séculos!

O centro comercial está entre as Vias Corso, Nazionale e Cavour, as ``Piazzas'' Venezzia (onde falava Mussolini), Colona e República. Nesta, que é circular e tem ao centro estupendo chafariz iluminado, há no momento um show para os fregueses do ``Grande Café Itália'', de mesinhas ao ar livre. A orquestra entoa ``Santa Lucia'', que uma loura canta como só esta gente é capaz.

Vou para o outro lado da cidade, na ``Piazza Bologna'', mas o movimento, face ao adiantado da hora, já está mortiço.

Grandes planos para amanhã.

\section*{16 \adfflatleafright \ISODayName{1959-09-16}}
O dia amanheceu esplendoroso, neste excepcional verão europeu de 1959, que, infelizmente, está por uma semana para dar o seu ``até para o ano''. Desde Paris e mais acentuadamente Zurich e Berna observei que grande parte das árvores já se apresenta com a folhagem de um tom amarelo-avermelhado, para mim inédito e de extraordinária beleza.

Saio para um grande giro, já com a planta da cidade no bolso, como de praxe, para a boa orientação com um mínimo de necessidade de auxílio. Começo pelo Castelo de Santo Ângelo, de formato cilíndrico, aqui pertinho. Vejo-o da janela do meu quarto. É imponente e a ponte fronteira, sobre o rio Tevere, uma obra de arte, pelas brancas e belas estátuas sobre as amuradas. Mais bonita ainda é a ponte ``Vitório Emanuele'', a seguir.

Passo no Coliseu (aqui chamado ``Colosseo''), fabulosa marca de quanto chegou a ser esplendoroso o império romano; data do ano 80, mais ou menos. Está bastante destruído, infelizmente, mesmo por fora. Vê-se que os inimigos dos romanos, bárbaros ou outros, se empenharam em depredar deliberadamente o colosso de pedra e tijolos, tão salpicado está de fraturas todo o seu conjunto. Do alto, isto é, da parte mais elevada das arquibancadas, chamada ``Anfiteatro Flavius'', a perspectiva é magnífica: um ``Maracanã'' projetado há quase dois mil anos! Várias inundações do Tevere destruíram a plataforma de sua arena, estando atualmente à mostra as paredes e subdivisões dos subterrâneos.

Ao lado encontra-se o Arco de Titus, do mesmo estilo que o do Triunfo, de Paris, séculos mais novo, lembra logo à idéia. A ``Via di Fori Imperiali'', junto, conduz ao que resta do outrora suntuoso ``Foro Romano''; é o que está mais destruído e que, por certo, mais sofreu as conseqüências das tais enchentes catastróficas. Pedaços de colunas de branco mármore e outros fragmentos quedam-se amontoados, vastamente, pelo conspícuo cenário.

Adiante, o ``Campidoglio'', velha igreja, que sofre restaurações internas, no momento. Às costas desta, caminho para a ``Piazza Venezzia'', com o supergigantesco monumento a Vitório Emanuele II. Creio ser o maior do mundo, dentre os localizados em perímetro urbano. Nem os russos, no seu exagero de grandiosidade neste particular, têm algo que se compare. E a beleza é fantástica: todo branco, de pedra mármore, com as figuras enormes em bronze escuro. Lá em cima, de cada lado, uma biga romana com os respectivos cavalos. No momento acha-se também ali o túmulo do ``Soldato'' desconhecido.

Ao descer, em demanda à igreja de ``San Pietro in Vinculi'' (São Pedro no Cárcere), vislumbro a ``Coluna de Trajano'', toda trabalhada. Dentro há uma escadinha em espiral, que dá acesso ao topo.

A igreja de São Pedro ``in Vinculi'', cuja entrada se faz por um baixo arco na ``Via Cavour'', tem seu ponto máximo de atração na estátua de Moisés, de Miguel Ângelo, em mármore. O espanto pela perfeição e beleza da obra traz logo em seguida uma inevitável emoção. Fico longo tempo a contemplar o soberbo trabalho. Embora a luminosidade não seja a ideal, não posso deixar de fotografá-lo e, para tanto, exigir tudo da minha ``Leica''. Demoro-me também porque a todo momento entra uma turma de turistas e correm logo a ficar na frente do monumento. Por fim, consigo bater a chapa.

Dali, alcançando e descendo a colina do Capitólio, onde fica a Prefeitura, procuro a Fonte de Trevi, a famosa ``fonte dos desejos'', tão popularizada pelo filme americano. A sua localização, porém, decepciona, já que surge apertada por feias e estreitas ruas. Mas a beleza da fonte, que já tem quase 200 anos, é um fato, pela quantidade e variedade de movimento das figuras, esculpidas na pedra. E como tem gente em torno! Sem dúvida, é um dos pontos turísticos mais populares de Roma. Suas claras águas estão salpicadas pelos reflexos das centenas de moedas lançadas pelos seguidores da lenda do desejo, que, assim, poderá cumprir-se.

Almoço no ``Restaurante de Trevi'', ao lado da fonte, ficando a ouvir de lá o ruído gostoso das águas. E mato a saudade de uma boa ``lasagna verde'' e até, ousadamente, faço-a acompanhar de um ``quartino'' de vinho tinto.

O monumento seguinte é o ``Panteon''. Um espetáculo! É o mais conservado dos monumentos romanos, e dos mais antigos, pois data de 27 a.C., construído por Marcus Agripa, diz em sua testada externa. A penetrar no seu interior, que é circular, levo um susto, pois está tudo perfeito. O revestimento é soberbo, de mármore de várias cores, inclusive no piso. A abóbada é gigantesca, com uma abertura central, por onde entra abundante luz.

Um grupo de turistas providencia, pagando 500 liras, a execução de um número de órgão. Tocam ``Ave Maria'', de Gounod. A acústica é impressionante.

Lá dentro está a tumba do pintor Rafael e também da família imperial, a partir do século passado (Umberto II). Um guia de um grupo de franceses faz uma piada com a noiva de Rafael, segundo pude entender de sua rápida fala.

Estoura uma gargalhada, mas como disse, não ``morei'' na parte final, mas não deixei de me enfezar com a falta de respeito do cicerone para com o vulto glorioso de Rafael Sanzio.

O resto da tarde passei a fazer compras de \textit{souvenirs} e algumas encomendas de meu pessoal, em suas cartas (capas de \textit{nylon}). E um par de sapatos, de ``mocasin'', para mim, que também sou filho de Deus. Desde Bruxelas vinha namorando esse tipo de calçado, com magnífico solado em látex.

À noite fui assistir a um filme alemão, no cinema vizinho: ``Stalingrado''. Achei que devia ser curiosa a versão germânica daquele seu colossal fracasso bélico. O cinema, também aqui, é caríssimo: 500 liras (Cr\$105,00). As poltronas são reclináveis e macias. O filme é dublado em italiano e ficou bem agradável de ouvir e ainda entender.

\section*{17 \adfflatleafright \ISODayName{1959-09-17}}
Logo cedo vou ao Vaticano, para visitar e fazer minhas indispensáveis fotografias. Procurarei também uma pessoa da embaixada do Brasil, junto à Santa Sé, a que fui recomendado por um colega, engenheiro do IRB.

Devo confessar que, desde a visão noturna, há dois dias, embora admirado, não encontrei no Vaticano suntuosidade no grau que esperava. A Praça de São Pedro é também menor do que imaginava, se bem que de bela perspectiva. A Basílica, em seu interior tão despojado de mobiliário, é, na verdade um deslumbramento artístico, em decoração e riqueza. Extraordinário que date do século XV!

Visito as salas do Tesouro, riquíssimos apetrechos e vestes do culto católico. Em uma delas, ao fundo, está outro notável trabalho de Miguel Ângelo: ``Pietá'', aparentemente em mármore.

Seria necessário mais que um dia inteiro para correr todas as dependências. Damo-nos por satisfeitos, por hoje.

Procurando o endereço da Embaixada do Brasil, dirijo-me, por acaso, a três seminaristas e, com surpresa, fico sabendo que são brasileiros. A informação recebida é de que a sede de nossa representação fica fora do Vaticano. Vou à secretaria da Guarda Suíça. De passagem vejo a porta que só é aberta nos Anos Santos. Está cimentada desde 1950. Uma turista tenta fotografá- la e é obstada. Proibido fotografar no interior da Basílica.

Um gordo miliciano da Guarda Suíça, em seu bizarro uniforme medieval, não sabe dizer senão o endereço particular da pessoa que procuro. Assim, deixo também de lado a visita à embaixada.

Com o Alcino volto a percorrer vários monumentos já ontem visitados, inclusive a ``Fontana de Trevi'', para colher melhores fotografias. O filme da máquina acaba na hora, porém, e é necessário esperar até as 4 da tarde, pois o comércio está diariamente fechado desde uma até aquela hora. Não sei se para uma boa sesta romana\ldots

Conheço ali, na beira da fonte, uma italianinha da Calábria, bem interessante, que me dá o endereço para fazer correspondência em inglês, língua que está praticando. Chama-se Aurora Perrone e está com a gorda mamãe, à qual me apresenta, bem como o Alcino.

Dali, após almoçar no mesmo restaurante de ontem, ao lado da fonte, saímos para novas compras. O comércio de Roma é o mais fraco das grandes capitais europeias, ressalvado o caso de Varsóvia e Moscou, que têm sistema econômico inteiramente diferente, ou seja, de armazéns do estado, pouco numerosos e de discreta decoração, mas enormes. Faltam em Roma os magazines vistosos, onde se entra mesmo que só para ver. As lojas romanas exigem boas caminhadas para se localizar maior variedade de artigos.

À noite, no hotel, travo relações com um hóspede, italiano de Gênova, que viveu três anos em São Paulo e tem um filho no Rio, em alto cargo na RCA Victor. Como ele possui uma oficina de automóveis, procuro informações sobre o aluguel de carros e, assim, já preparamos terreno para, a partir de Gênova, fazer a Côte d’Azur e Costa Brava (até Barcelona), devidamente motorizados.

Surpresa desagradável: chove copiosamente esta noite. A quinta chuva, porém, e de poucas horas, desde o dia 7 de julho!

\section*{18 \adfflatleafright \ISODayName{1959-09-18}}
Hoje deve ser nosso último dia em Roma, pois já articulei o embarque amanhã cedo para Gênova, com uma parada em Pisa, para conhecer a famosa torre inclinada. O horário dos trens permite. E a visita à lendária torre me exerce um verdadeiro fascínio, desde garoto.

Ocupo-me de manhã com as providências da viagem e depois saio para outras perambulações. Alcino quer comprar um radiozinho transistorizado marca ``Telefunken'', igual ao que adquiri em Berlim. Mas o preço aqui é quase duas vezes mais caro. Aconselho-o a aguardar Tenerife, ou Tanger se houver outra escala lá, onde se encontram maravilhosos aparelhos japoneses por vinte e poucos dólares.

Vale, porém, o prazer de se entrar numa loja e ser atendido pelas italianinhas. Só de ouvi-las falar já é uma delícia\ldots

Os aparelhos eletroeletrônicos me parecem caros na Itália, comparados os seus preços com os de outros países percorridos. Mesmo quando a fabricação é nacional, como no caso do pequeno transistor ``Autovox'', pelo qual nos pediram 36.000 liras, ou seja, Cr\$7.300,00, aproximadamente.

Baratíssima, por exemplo, é a máquina de escrever ``Olivetti'' portátil, modelo ``Letera 22'', cujo preço, em todo o país, é de 42.000 liras (Cr\$8.900,00). Quanto aos automóveis, do mesmo modo que na França ou na Alemanha, o brasileiro engole em seco. Em Roma, como nas outras grandes cidades, praticamente não existem ``máquinas'' (como são chamados os carros na Itália) velhas. Os modelos mais antigos que dois ou três anos são quase todos táxis. A Fiat domina o mercado quase absolutamente com seus variados e magníficos modelos. Um Fiat 1.100 deste ano, novinha, bem confortável, até para 5 lugares, custa Cr\$210.000. Por mais uns Cr\$70.000,00 ou Cr\$80.000,00 compra-se o tipo 1.800, que parece um carro americano, de tão amplo e bonito. É a melhor ``máquina'' italiana, em tamanho e qualidade, principalmente de motor, possante e econômico.

Já a Alfa-Romeo tem suas belas ``Giulletas'' muito mais caras (Cr\$500.000,00), simplesmente pela superpotência do motor, que dá 200~km/h, e o luxo do acabamento. Vai ser o tipo ``J.K.'', a ser lançado em breve no Brasil.

Deixemos de lado os automóveis. Volto a almoçar junto à fonte de ``Trevi'', sempre um prazer. Depois pego o metrô (aqui ``Metropolitana''), junto ao ``Colosseo'' para visitar os trabalhos preparatórios para a Exposição Universal de Roma (E.U.R.), que coincidirá com os jogos olímpicos, no próximo ano.

O metrô romano está engatinhando, pois é novíssimo. O movimento é quase nenhum, naturalmente, neste único tronco ``Estação Ferroviária (Termini) -- Colosseo -- E.U.R.'' As estações são bonitas e espaçosas, mas, pelo menos a estas horas de um dia útil, quase lúgubres. Nem banco têm para se esperar os carros, que correm a cada ``dieci minuti''\ldots

Chegando ao ponto de destino, encontramo-nos, por acaso, com um português culto e distinto: o engenheiro Eurico de Almeida, de Lisboa. Com ele estivemos durante todo o tempo em longo ``bate-papo'', principalmente ouvindo suas impressões sobre as recentes viagens que fizera a Kênia, Angola, Moçambique e Congo Belga.

Os trabalhos da Exposição estão bastante adiantados. A área é imensa e quase todos os pavilhões têm nome de ``palácio''. Penetramos no ``Palácio dos Congressos'', que, naturalmente, destina-se a convenções e conclaves de qualquer gênero, que aqui se poderão realizar. O estilo arquitetônico é o moderno, bastante agradável. A Itália adere, pelo que se vê, à escola que no Brasil já reina absoluta. A estação ``Termini'', em Roma, é o ponto alto dessas realizações até agora.

Todavia, o ``Palácio dos Congressos'' me parece um pouco fechado, com vários recintos sem circulação suficiente de ar; falta-lhe aquela exuberância de ventilação que nós adotamos. Mesmo com justificativa do clima, há de se considerar que, pelo menos durante três meses, o verão sul-europeu pouco ou nada difere do nosso.

Passeamos pelas principais alamedas, observando as outras edificações. Tudo muito novo e agradável, mormente pela abundância de fachadas claras, em geral brancas mesmo. Um obelisco muito artístico, dedicado a Marconi, está sendo ultimado na entrada. Em frente a esta, grandes movimentos de terra, e, do lado oposto, acaba-se o ``Palácio dos Esportes'', tipo nosso Maracanãzinho, muito bonito e todo envidraçado exteriormente.

Voltando ao centro da cidade, vou apanhar, junto à ``Piazza de Espagna'', mais quatro rolos de filmes revelados e cópias de fotos tiradas em Varsóvia, Viena e Veneza, que mandei fazer. Saiu tudo ótimo.

Ao jantar, no hotel, uma grata surpresa: um casal espanhol de Moyá, esperantista, ocupa a mesa ao nosso lado. Estão em trânsito para Florença, onde se inaugura amanhã mais um Congresso Italiano de Esperanto, que, deveras, muito lamento perder. Ainda esta noite deverá chegar outro esperantista, este japonês.

Hoje, durante a ``colazione'' (café da manhã) já havia conversado agradavelmente com duas irlandesas do norte, que visitam Roma e partem amanhã para a Sicília. Disseram que o Brasil é famoso e elogiaram a castanha do Pará.

\section*{19 \adfflatleafright \ISODayName{1959-09-19}}
O expresso saiu às 8h20m para Torino (Turim), via Pisa e Gênova. Minha passagem para Gênova é válida por três dias. Assim, posso ficar em Pisa, que é uma cidade pequena. Chegarei lá às 12h30m, para retomar a viagem às 16h40m. Essas quatro horas serão mais que suficientes para conhecer o que lá existe de interessante.

A viagem agora costeia sempre o mar, estranhamente calmo como uma lagoa, quando ali é oceano aberto. E esta situação deve ser permanente, pois a costa é rasa e baixa e, às vezes, a rodovia (Via Apia) paralela passa a uns 50 metros do mar. As construções e alguma agricultura (pouca, pois a região é agreste) também estão bem junto.

Passamos Civitavecchia, Grosceto e Livorno, esta última um grande porto. A seguir, Pisa. Boa estação. A cidade é maior do que eu pensava. Vamos procurar um esperantista, indicado no anuário, já que o seu endereço é bem central. Mas o homem viajara também para o congresso de Florença.

Almoçamos antes de atingir o local da Campanile (torre), Catedral e Batistério, que estão juntos, no outro lado da cidade. Vamos mesmo a pé, embora exista um moderno serviço de ônibus elétrico --- outra surpresa. No caminho atravessa-se o rio Arno, bem largo e calmo.

A faladíssima torre inclinada surge à nossa frente. Data de 1350, enquanto a enorme igreja e o batistério foram concluídos antes. As três peças têm, naturalmente, o mesmo estilo, belíssimo em suas caprichosas decorações. O mármore branco, bruto ou polido, é o material exclusivo destes imponentes monumentos.

A torre, cilíndrica, deve medir uns 8 metros de diâmetro e, no máximo, 30 metros de altura, pelos meus cálculos a olho. É sólida e pesada. A estreita escada de mármore que conduz ao topo tem os degraus sulcados na parte central pelo atrito das pisadelas. Quantas, em mais de 600 anos? A pouca altura explica a permanência segura, mesmo tombada. Já tenho lido sobre um reforço de concreto na base; mas aqui nada se vê. Em torno, mesmo em declive, está tudo recoberto com maravilhoso mármore branco, decorado com umas barras mais escuras. Subimos até o alto, de onde se tem uma bela vista de toda a cidade. Fotografamos. Os badalos dos sinos estão imobilizados com grossos arames. Devem ter achado melhor mantê-los em silêncio.

A catedral é das maiores que tenho visto. Talvez, na Itália, depois da de São Pedro, do Vaticano, poucas se lhe comparem. O interior é riquíssimo, com parte do teto ornamentado a ouro, em desenho de grande beleza.

Retomo a viagem para Gênova. Como é sábado, o movimento nos trens é enorme. Aliás, já me haviam advertido a respeito. A Itália é, como já disse, terra de multidões itinerantes. Viajamos longo tempo em pé, se bem que agradavelmente, em vista da simpatia dos circunstantes, sempre gentis a falar e a indagar sobre coisas do Brasil.

Em La Spezzia, como a vazante foi grande, sentamo-nos. Na cabine fervia uma discussão política. Ao meu lado um declarado marxista, quarentão, modestamente trajado, pulverizava com boa base os velhos chavões conservadores do opositor, um bem vestido e simpático gorducho, aparentemente ``semi-bem'' (pelo menos viajava de 2ª).

Entendi tudo e foi muito interessante ouvir em italiano esse debate clássico. O respeito mútuo dos debatedores foi impecável.

Por fim, cansaram e as conversas, com a participação dos demais, variaram bastante. Comentei, à parte, com o esquerdista minha recente visita a Moscou (em italiano, ``Mosca''). Ele então elevou a voz, olhando para o opositor --- a esta altura distraído com um ``bambino’ --- fazendo blague: --- Quer dizer que todos lá estão morrendo de fome, não é?\ldots\ A façanha do ``lunik'' estava fresquinha e foi objeto de várias referências.

Chegando, por volta das 22 horas, a Gênova (segunda cidade repetida no roteiro), volto a procurar a ``Esperantista Domo'', para grande surpresa do proprietário, que nos havia acolhido festivamente há dois meses atrás, e recebendo até abraços da idosa camareira, como se fôssemos filhos pródigos retornando ao lar. Ah, só mesmo os italianos!\ldots

\section*{20 \adfflatleafright \ISODayName{1959-09-20}}
O domingo genovês é, compreensivelmente, apático. Minha primeira providência é procurar a organização automobilística do amigo encontrado em Roma, para tratar o carro em que pretendo seguir viagem até Marselha, pelo menos, percorrendo as famosas Rivieras italiana e francesa.

Apesar de alto o preço, topo o negócio. Alcino, a princípio, relutou em vir junto, naturalmente achando caro. Mas quando lhe disse que iria sozinho e que ele me encontrasse em Marselha, procurando nos hotéis até achar, mudou de idéia.

Partimos depois do almoço, com um motorista dirigindo a ``Fiat 1.100''. Ele teria que ir para trazer o carro de volta. Mas foi ilusão nossa acreditar que poderíamos pegar o volante. O movimento na estrada, por ser domingo, é muito grande; além disso, o trajeto é caprichoso, perigoso mesmo e, por fim, o carro me é bastante desconhecido. A ``Fiat'' é do ano, bonita e valente como quê.

Assim, o bom Mauro, que dá uns ares com o ator Richard Widmark, foi nosso motorista particular durante toda a viagem. Realmente, foi um ponto relevante de nosso giro europeu esta travessia notabilíssima, pelas famosas estâncias marítimas do Mediterrâneo. É interessante que, na parte italiana, as localidades estão quase ligadas, como se fossem uma só, de extensão interminável. Vai-se, assim, passando por Varazze, Savona, Imperia, San Remo e Bordighera, até atingir Ventimiglia, a última, já na fronteira francesa.

A mais bonita e de mais classe é, sem dúvida San Remo, onde fizemos uma parada para tomar café. Muito bonita a arborização, com palmeiras daquele tipo baixo, dando o necessário cunho tropical, que tanto excita as gentes do hemisfério norte. San Remo tem suntuoso cassino, que o nosso chofer pronuncia a francesa ``cazinô'', e, diz ainda ele, suplanta o de Monte Carlo. Vamos verificar se isto é verdade daqui a pouco.

A passagem na fronteira foi fastidiosa. Durante 45 minutos nos arrastamos na fila interminável. Culpa de ser domingo, parece evidente. Saímos de Gênova às 16h10m. e, ainda aqui, a noite já vai adiantada, próxima das 21h. Este primeiro turno rendeu pouco, mesmo assim foi ótimo, pois deixou contemplar suficientemente as belas paisagens do caminho.

Resolvo seguir até Mônaco e lá pernoitar. Mauro diz conhecer um bom hotel. E assim fizemos. Logo após Menton, já em território francês, atingimos o bizarro principado. A visão de longe, à noite, é bonita, pois este pequeno país de brincadeira está encarapitado em dois montes e deles escorrega até a beira-mar, uma enseada que lembra a praia da Urca. Damos uma parada ao lado do cassino. É muito menor do que se pensa, mas, apesar do aspecto envelhecido, parece bastante aristocrático. Uns ângulos interiores que vislumbro demonstram grande riqueza de decoração. Movimento razoável de automóveis à porta. Detalhe: o porteiro é negro.

Ali é o monte de nome Carlo; continuando-se a estrada ela toma um declive e estamos à beira do ancoradouro. À frente e à esquerda ergue-se o que completa o principado (``principauté'', como está na placa, ao lado da estrada). O palácio do mandatário tem iluminação indireta nos muros, o que destaca, com belo efeito, as árvores que os circundam.

Está havendo uma festa ao ar livre, na estreita praça que é o fundo da baía. Vamos, porém, diretamente ao ``Hotel du Siècle'' tomar os quartos. A porta está aberta, mas a portaria deserta e tudo às escuras. Fazemos tilintar um tímpano e um senhor, talvez o dono, surge bocejante, de pijama, descabelado para nos receber. Cobra adiantado quando tem ciência de que iremos sair muito cedo amanhã (ele não estaria acordado\ldots) e volta para a cama, deixando tudo escancarado e escuro como antes. Só mesmo em Mônaco!

Voltamos para a rua, a fim de jantar e, depois, espiar a tal festa, que parece já estar no fim. À beira da baía chega um tipo modestamente vestido e visivelmente ``tocado'' (coisa que não supunha acontecer aos monegascos); mira o castelo do príncipe com olhar de peixe frito e diz, no seu francês titubeante, coisas impublicáveis sobre o seu soberano! Boa surpresa essa!

Às 24 horas a iluminação do palácio se apaga. Já estou na sacada do meu quarto e vejo agora a sua negra silhueta, mais parecendo um castelo medieval.

\section*{21 \adfflatleafright \ISODayName{1959-09-21}}
Não foi possível sair tão cedo como desejava. Forcei um regular atraso, pois havia pouca claridade para fotografar e eu não podia deixar de registrar minha passagem por aqui com uns bons flagrantes.

O ``Richard Widmark'', muito boa praça, me leva onde quero. Subimos até a porta do palácio, onde se veem apenas dois guardas e três ridículos canhões de aspecto secular. Fotografo-me, depois, frente à igreja na qual o príncipe se casou com a atriz Grace Kelly, há alguns anos.

Depois partimos. Logo chegamos a Nice, cujo interior é velho e banal, destacando-se, porém, a avenida que beira o mar, das mais lindas que se possa imaginar, pela classe dos palacetes e hotéis e pela arborização central, dividindo as duas mãos do tráfego, onde predominam as belíssimas palmeiras. Localizo o famoso ``Hotel Negresco'', que tanto apareceu no filme ``Ladrão de Casaca'', de Hitchcok.

A praia é de pouca areia e toda assaltada de barraquinhas. Para nós, brasileiros, a dita é do mesmo baixo padrão europeu, que tenho conhecido até agora.

O mesmo com relação a Cannes, mais adiante um pouco, onde fizemos uma parada para visitar o Hotel Esperanto, na avenida Laitré de Lasigny. Em seguida puxamos até Frejus, depois de passar por S. Rafael, onde, segundo o ``Widmark'', Brigitte Bardot acaba de adquirir uma vila. Tomamos um lanche em Frejus e fizemos câmbio, pois não mais possuíamos dinheiro francês.

Mauro é bom de papo. Fala bastante e é engraçado. E demonstra, pelo menos, entender bem o meu italiano, que, naturalmente, melhorou com esta nova temporada na península. Depois de dizer horrores da atriz Silvana Pampanini, conta as proezas dele como corredor de automóvel, em outros tempos. Aliás, desde os primeiros momentos da viagem já havíamos verificado que ele é exímio no volante.

O trecho final da viagem já não é tão interessante. Afastamo-nos um pouco do mar, pegando bela floresta. Resolvemos almoçar em Toulon, para aproveitar a hora e o apetite. Cidade sem nenhum atrativo. Alcino, entretanto, faz o rapaz nos levar até a beira do porto, lugar feíssimo, para bater uma chapa inexpressiva. Então me lembro de que foi neste porto que os franceses afundaram propositalmente sua esquadra de guerra, para que ela não caísse nas mãos dos alemães.

Finalmente, chegamos a Marselha, lá pelas 15h30m. É a terceira cidade repetida no meu trajeto. Alcino sai, algo aflito, a saber de sua bagagem despachada de Paris. Encontra-a e, após andar de um lado para o outro, deixa-a na estação, com despacho para Barcelona.

Marselha, como já comentei, é a primeira cidade europeia que vejo com problema de tráfego. A concentração de veículos na ``Canabière'', sua principal artéria, faz lembrar nossa zona sul, do Rio. À noite, verifiquei que o engarrafamento piora ainda mais.

Cai uma boa chuva, mas passageira. Janto e, depois de umas voltas, vou para o hotel. Um cinema próximo exibe ``Os Dez Mandamentos''. Bem cansado, desisto de ir assistir o longuíssimo espetáculo. Nem estou com vontade de ouvir Moisés falando francês\ldots

\section*{22 \adfflatleafright \ISODayName{1959-09-22}}
O trem saiu às 7h30m da manhã, bem vazio. Isso porque resolvi pegar uma primeira classe, para prová-la e, ainda, com vistas ao percurso espanhol, que não goza de boa fama. Teremos de fazer algumas baldeações, antes e depois da fronteira espanhola.

Não demoro a cruzar o rio Ródano, perto da Arles. Mais um grande rio europeu para o ``catálogo''. Depois passo por Tarascon, e não posso evitar a lembrança de um livro que li nos tempos de ginásio, do escritor francês Alphonse Daudet, cujo título era ``O Tartarin de Tarascon'', pois que a história se passava nesta cidade. História impagável, contando peripécias de uns tais ``caçadores de bonés''. Nunca esqueci esse livro, assim como também nunca imaginei que um dia iria passar por esta cidade.

Nimes é a maior ``ville'' do percurso. E lá para às 11h30m fazemos a primeira baldeação, em Narbonne. Aqui acabou o meu conforto de primeira classe. Só que, na viagem, os demais passageiros calafetaram inteiramente o carro, tornando o ambiente irrespirável. Precisei sair diversas vezes.

O carro seguinte é do tipo motriz, incomparavelmente menos confortável, e só tem duas classes: 2ª e 3ª. Em Port Vendres, outra saída, com passagem pela alfândega francesa. Esperamos longamente e almoçamos na estação. Afinal, partimos para uma viagem de\ldots\ 10 minutos, o necessário para cruzar a fronteira a passar agora pela alfândega espanhola em Portbou. Outra longa espera.

Enfim largamos, agora no comboio espanhol, também só de 2ª e 3ª classes. O trem é de pobreza beneditina, fazendo lembrar até o interior de Minas Gerais há 20 anos atrás! Mas a viagem é agradável, pois fiz relações em Portbou com dois jovens espanhóis que regressam de longo estágio na França, onde trabalharam e aprenderam perfeitamente o francês. No entanto, conversamos em espanhol, que já preciso reengrenar para os dias que se seguirão. Como eles estão na 3ª classe, de rijos bancos de madeira, vou assim mesmo para lá, a fim de conversarmos mais.

Finalmente, perto das 8 horas da noite chego a Barcelona, onde passei quase dois dias na escala do navio, em meados de julho. Bela e movimentada cidade esta, vale repetir. Alojamo-nos bem nas ``Ramblas'', a grande avenida central, que vai mudando de sobrenome a proporção que penetra pelo norte da cidade.

A larga e original calçada central daquela artéria está, como sempre, cheia de gente. Os bancos laterais, a maioria deles alugáveis, também regorgitando de gente.

\section*{23 \adfflatleafright \ISODayName{1959-09-23}}
O Alcino ainda está às voltas com as suas duas pesadíssimas malas despachadas em Marselha. Aqui poderá entregá-las à companhia do navio que iremos tomar em Cadiz, mas que partirá deste porto.

A cidade está sob o efeito das festividades de ``Nuestra Señora De La Merced'', padroeira local. Amanhã será feriado e haverá ``corridas de toros'' até domingo, diariamente. Hoje à noite será realizado um grande jogo de futebol, decisivo, pela Copa da Europa, entre o Barcelona, campeão da España, e o C.N.K.A., de Sofia, Bulgária. Resolvo adiar o embarque para amanhã, a fim de assistir esse jogo e para fazer um descanso maior, antes de partir para Madrid, onde me interessa demorar mais, pois ainda não a conheço.

Faço outro passeio pela cidade, revendo suas belas lojas, especialmente uma nova, ainda completando as instalações, muito bonita e ampla, com parte superior e sub-solo. Ali compro outros ``souvenirs''. Depois vou a um cinema, que exibe o filme ``As Pontes de Toko-Ri'', com William Holden e Grace Kelly. O cinema fica na grande ``Plaza Cataluña''. O filme é, como sempre, magistralmente dublado em espanhol, mas fraquinho de enredo, às vezes ridículo e absurdo. Vale, contudo, como um libelo contra a guerra.

À noite, eis-me no belo estádio ``Nou Camp'', do Barcelona, novo e moderno, que inaugura hoje suas instalações para jogos noturnos, alardeando-a como a mais perfeita do mundo. De fato é magnífica. Quando entrei estava semi-ligada, quase em penumbra. Minutos depois, a luz despontou feericamente e a assistência, numerosíssima, prorrompeu em aplausos.

O estádio é bem grande e confortável, mas comparado com o nosso Maracanã, só tem a vantagem de estar inteiramente concluído\ldots O Barcelona vence fácil, por 6 x 2. O patrício e ex-Flamengo Evaristo atuou otimamente e fez três gols. Os famosos húngaros Kubala e Czibor, atacantes, não me pareceram os fenômenos tão decantados. Surpreendi-me em encontrar no gol do Barcelona o mesmo Ramallets, que defendeu a seleção espanhola na Copa do Mundo de 1950, no Maracanã. Um craque, mas no jogo de hoje ``papou'' um franguinho, em bola chutada de muito longe\ldots

Ao voltar para a cidade, lembro-me da frase que li dentro do bonde, na ida: ``Es prohibida la blasfemia y la palabra soez''\ldots

E me ``acuerdo'' também do preço da ``tribuna'', coberta, que paguei para assistir este jogo: 270 pesetas, isto é, Cr\$620,00!!

\section*{24 \adfflatleafright \ISODayName{1959-09-24}}
Pretendo partir para Madrid hoje, o quanto antes.

Resolvo optar pelo avião, pois a viagem é também longa (750km, distância igual à de Belo Horizonte a Brasília) e o preço da passagem aérea é convidativo, em relação à ferrovia. Mas fico sabendo que não há lugares disponíveis nos aviões da grande companhia espanhola ``Ibéria''. Hoje é dia santo e 5ª feira, talvez por isso o movimento tenha aumentado.

O pior veio depois: também no rápido das 12,50 hs., não há mais lugares. Teremos que esperar os trens da noite. Com isso perderemos o dia e só estaremos em Madrid amanhã de manhã. Surge uma outra solução: automóveis fazendo lotação para a capital e outras cidades intermediárias. Interesso-me por um ``coche'' que sairá para Saragoza, que fica no meio do trajeto, e de lá tentar outra condução, ganhando assim tempo. Depois surge outro carro, este direto para Madrid. Mas a lotação não se completa e nada é feito. Pensando bem, foi melhor assim, pois os riscos desse tipo de viagem seriam preocupantes.

Tive mesmo que aguardar o tal rápido das 19 horas, garantindo uma primeira classe para maior comodidade. São 14 horas de viagem. O agradável desses estirões é o contato com os passageiros da cabine. Temos tempo para longos ``papos'' e ficar sabendo de muita coisa, assim como também falar do Brasil, desfazendo lendas e invencionices.

O risonho espanhol, madurão, que viaja com o filho de 14 anos, conta piadas, algumas delas finamente antifranquistas e começando todas por ``Resulta que\ldots'' Além deles, viaja um casal relativamente jovem. O marido se espanta ao saber que no Rio há 11 clubes de futebol na primeira divisão regional e que quase todos os outros Estados têm os seus campeonatos anuais próprios.

\section*{25 \adfflatleafright \ISODayName{1959-09-25}}
Lá pelas 8 da manhã já estamos no planalto madrilenho de terras bastante estéreis e vegetação anêmica. --- ``Aqui no dá nada'' --- diz o espanhol risonho.

Passamos, pouco depois, pela pequena localidade (``pueblo'') onde nasceu Cervantes, e nos apontam a casa, enorme, que hoje deve ser museu. A 20 kms. de Madrid, passo ao longo de uma base aérea americana, há pouco instalada. Os passageiros estranham a rapidez com que ali brotaram as edificações especiais. Mais adiante um pouco, vêm-se casas em construção (naturalmente para os oficiais).

Às 9 horas chego à estação ``Medio-dia'', na capital espanhola. Tomamos uma pensão muito boa, barata e confortável. E com chuveiro! Daqui para diante, acho que esse acessório tão desprezado pelos europeus não me faltará mais\ldots

Curioso: se o chofer da Riviera se parecia com o Richard Widmark, o dono da pensão é cópia fiel do ator italiano Rosano Brazzi\ldots

Tiro uma soneca de duas horas, para compensar um pouco a noite não dormida. E às 14 horas estou em frente ao ``Ajuntamiento'' (Prefeitura), de acordo com o combinado em carta que escrevi de Roma ao amigo de viagem Bernardo, que insistiu em nos encontrar em Madrid, para mostrar a cidade. Ele viria, com seu carro, de Salamanca, a cujas festas estaria assistindo até o dia 23 passado.

Mas o homem não apareceu. Qualquer coisa deve ter havido com ele ou com a minha carta. Enfim, a obrigação de comparecer, de minha parte, estava cumprida.

Almoço e saio a conhecer a cidade, fora de sua parte antiga, que não está me impressionando bem. A população revela muita pobreza, em sua maioria. Subindo a ``Calle Mayor'' chego à ``Puerta del Sol'', ponto central (pequena praça) da cidade, de onde saem suas principais ruas. Destas destacam-se a ``Calle Alcalá'' e a ``Avenida José Antônio'', magníficas, amplas e de intenso movimento.

Surpreendo-me, então, com a beleza da cidade. Aliás, Madrid foi muito destruída durante a guerra civil de 1936-39 e, em conseqüência, pôde modernizar a parte reconstruída. Já lá se vão vinte anos e quanta coisa se pode realizar durante todo esse tempo? A Avenida José Antônio é conhecida como ``Gran Via'', mas que eles aqui pronunciam ``Gran Bia''. Sim, porque são os espanhóis e não os portugueses que trocam o ``v'' pelo ``b''. Pensei que fosse só na Catalunha (Barcelona), mas na capital sucede o mesmo, assim como na Galícia também. Logo, deve ser geral. Na sobremesa do almoço pedi uvas, que são deliciosas aqui, doces como mel. Incrível como pareça, o garçom custou a entender. Apontei alguém ao lado saboreando um cacho e ele então exclamou: `` Ah, ubas''. E trouxe\ldots

Assim no meu castelhano bem caprichado introduzi o ``b'' no lugar do ``v'', com os melhores resultados.

As praças ``Cibelles'' e ``Canovas del Castillo'', com grandes e belas fontes circulares centrais, são ``sítios'' dos mais bonitos. Da primeira sai, à direita, o ``Paseo del Prado'', onde fica o famoso museu do mesmo nome. Francamente, este ``paseo'' é uma das avenidas mais belas e decorativas do mundo, tenho a certeza. Extraordinariamente larga, com amplos passeios arborizados e mão dupla, é um portento de artéria! Madrid suplantou de muito a expectativa.

Entro no museu (ingresso a 10 pesetas). Infelizmente, terei pouco tempo, pois fechará dentro de 40 minutos, fico sabendo depois de já estar lá dentro. É uma das mais ricas pinacotecas do mundo. As salas de Goya e Velasquez são deslumbrantes. Que fabulosos pintores esses dois! É importante o número de artistas e estudantes, com todo o equipamento, copiando quadros. Contei mais de 10, muito deles com excelentes reproduções, sempre em escalas reduzidas, evidentemente.

Rubens, Van Dyck e Tiziano lá estão também com dezenas de originais notabilíssimos. Confesso que me emocionei diante de ``São Jorge e o Dragão'' e ``A Adoração dos Reis'', de Rubens, de enorme tamanho, colorido e formas estupendas! Compensou o que não pude ver na casa do pintor, em sua terra natal, Antuérpia, quando a visitei em companhia da amiga belga Josée.

É uma tristeza quando soam as 17,30 e somos verdadeiramente enxotados pelos empregados do museu, ansiosos por encerrar o seu expediente. Têm uma pressa furiosa em esvaziar a casa e fechar tudo.

Providencio, desde logo, minha passagem para Lisboa, na empresa ``Wagon Lits'', para viajar em leito individual, no noturno de domingo, depois de amanhã. A surpresa de Barcelona não se repetirá.

Depois, vou conhecer o metrô. É dos mais antigos da Europa, dizem, e deve ser mesmo, a julgar pela aparência dos carros. Há umas 6 ou 7 linhas. O serviço é eficiente, pois os carros correm desabaladamente. Faltam, porém, bancos na estação de espera e nos vagões há apenas uns 12 assentos; a previsão é para caber mesmo muita gente viajando em pé. O preço é único, de 80 centavos, muito barato.

Janto no ``self-service'' Tobogan, na ``Puerta del Sol''. Bonzinho, mas de instalações inferiores ao ``Rex'' de Paris e ao ``Colmar'', de Bruxelas.

A vida noturna, em matéria de gente na rua, é intensa nos ``sítios'' centrais. A exemplo de Barcelona, que, até agora, ganhou todas as cidades neste particular. O espanhol é francamente rueiro.

\section*{26 \adfflatleafright \ISODayName{1959-09-26}}
Convenço-me de que Madrid é, realmente, uma das cidades mais lindas e agradáveis da Europa. Aliás, o Dr. Reynaldo Machado, médico de Lins, no interior paulista e nosso companheiro de viagem marítima, me havia dito isso numa conversa que então mantivemos.

Saio a fazer fotografias nos diversos pontos pitorescos. O comércio das principais ruas é muito bem instalado, com elegantes magazines. Os bancos têm também sedes pomposas, como o de España, o de Viscaya e o de Bilbao. O ``Palácio de las Comunicaciones'', nas ``Cibelles'', é figura obrigatória nos postais. Que imponente edifício!

Na ``Plaza de España'' está o prédio mais alto da Europa, a ``Torre de Cervantes'', isto é, um moderno e novíssimo arranha-céu de 39 andares. Estilo assim do Banco do Estado de São Paulo. Como o europeu é, sabiamente, infenso a tais monstrengos, este ganhou facilmente o título. Creio, porém, que apenas da Europa Ocidental, porque o Palácio da Cultura, de Varsóvia, e a Universidade e o Hotel Leningradskaia, de Moscou, que conheci, todos com torres centrais ponteagudas, suplantam em altura o edifício espanhol. Sucede que tais monumentos orientais estão localizados no centro de largas áreas, o que afasta as consequências maléficas dos arranha-céus enfileirados e abafantes.

Na mesma praça ergue-se o monumento a Cervantes e, uns 10 metros adiante deste, um expressivo bronze dos personagens principais do escritor, Dom Quixote e Sancho Pança.

Depois do almoço, ainda no Tobogan, vou fazer um ``radial'', como é aqui chamado o giro turístico de ônibus pela cidade, com guia e intérprete. Não posso deixar de mencionar que, durante o almoço, fui surpreendido com a transmissão pelos alto-falantes de um antiquíssimo samba de Carmem Miranda, cantado por ela própria: ``Alô, alô, responde''. Levei um susto, pois nunca poderia esperar uma coisa dessa.

Saem dois carros e vou em um só de americanos. Percorremos toda a cidade, durante cerca de três horas, partindo da ``Plaza Oriente'', onde está o belo ex-palácio real, do mesmo nome. Atualmente é utilizado apenas para grandes solenidades, pois o ``generalíssimo'' reside em uma vila fora da cidade. Ainda no palácio visito o museu das antigas carruagens imperiais, de incrível requinte artístico (as carruagens).

Outro palácio mais retirado, o ``Moncloa'', se bem que relativamente pequeno, é o mais luxuoso, limpo e ricamente decorado que já vi.

Um monumento à arte e bom gosto, mas um estranho repositório de suntuosidade em um país de pobreza generalizada.

O túmulo de Goya e os parques ``del Oeste'' e do ``Retiro'' nos são apresentados em seguida. O primeiro parque é enorme, mas em declive, e o último revelou-se o mais lindo de quantos conheci na Europa, superando mesmo o ``Bois de Bologne'' de Paris e o ``Lazienki'' de Varsóvia, até então os que pontificavam.

A cidade Universitária é outro capítulo à parte, ficando agradavelmente isolada, em continuação ao ``Parque del Oeste''. Todas as faculdades estão separadas, em grandes edifícios de estilo comum, de tijolos aparentes avermelhados. Dificilmente outra a suplantará em localização e instalações. Justo orgulho dos madrilenhos.

Passamos, ainda, pela ``Plaza de Toros'' e, mais ao norte, no início da larguíssima ``Avenida del Generalíssimo'', pelo ``Estádio Santiago Bernabeu'', do clube de futebol Real Madrid. Amanhã assistirei espetáculos em todos os dois, pois os ingressos (caríssimos) já estão no bolso.

À noite vou ao cine ``Palace'', ao lado do hotel do mesmo nome, mais para conhecer o seu interior do que pelo filme. Exibe ``Me Enamoré de una Bruja'', com James Stewart e Kim Novak. Creio ser uma refilmagem da anterior película apresentada no Brasil sob o título ``Casei- me com uma feiticeira'', com Fedric March e Veronica Lake. Filmezinho bem tolo e desinteressante. O cinema é muito confortável, embora antigo, destacando-se a forma original do encosto das poltronas: arredondado e bastante inclinado para trás.

\section*{27 \adfflatleafright \ISODayName{1959-09-27}}
Um domingo numa capital é sempre uma ducha fria no movimento. Mas Madrid tem sempre muita gente na rua, em sua parte central.

Providencio, desde cedo, o transporte da bagagem, embora bem reduzida agora, para a estação das ``Delícias'', onde embarcarei esta noite, às 22 horas, para Lisboa. O nome da estação é uma piada, pois trata-se da mais feia e velhusca das três aqui existentes. Assim, fico livre e desembaraçado para aproveitar o dia.

Outros passeios, depois do almoço, até às 4 da tarde, quando pego o metrô para a ``Plaza de Toros Monumental'', cuja parada é a estação ``Ventas''. O movimento já é grande. O ingresso, adquirido ontem custou 60 pesetas, ou seja, um dólar (para mim, Cr\$135,00).

Francamente, um dos pontos altos das touradas, vistas nos cinemas, sempre me pareceu a parte de apresentação, no início, isto é, a música, as vestimentas, os desfiles. Grande decepção! A banda toca baixo e pouco expressivamente; as roupas, salvo a dos toureiros principais, não têm grande coisa a mostrar, e a proteção dos cavalos são velhas, feias e remendadas. Pelo menos no espetáculo de hoje. Os próprios cavalos têm aspecto ``quixotesco''. Também, para o sacrifício não conviria trazer coisa melhor\ldots

O primeiro touro sacrificado me produziu, verdadeiramente, grande choque emocional. Que estúpido e bárbaro espetáculo! Ainda estou a ver o pobre animal, depois de receber a primeira e certeira estocada, rodopiar em estertores, lançando golfadas de sangue pela boca; cambaleou um pouco e caiu fulminado, enquanto a turba urrava selvagemente.

Meu desejo foi sair na mesma hora, mas o íntimo lembrou: ``Bem feito! Quem mandou vir? Agora aguente!'' Assisti mais três tourocídios. O fato curioso foi, lá para as tantas, a entrada de um popular na arena, que, entre correrias e perseguições dos guardas, logrou tourear um lance com a fera fresquinha, recém- entrada. Saiu detido, mas sob frenéticos aplausos.

É como digo: dêem ao espanhol um touro para ele atormentar e estará ali um homem feliz\ldots Como nós brasileiros somos, em imensa maioria, tremendamente sensíveis a essas coisas! Imagino, então, uma tourada no Brasil: ao primeiro arranhão feito no touro, a assistência invadiria a arena e poria o toureiro para correr\ldots

Aqui, o grande ``mestre'' da atualidade é o tal de Antônio. Parece-me que o seu sobrenome é Ordoñez. Entrou todo pomposo, provocou gritinhos histéricos e foi quem executou os três últimos animais. A plateia explodiu com o homem: ``El mejor! Este és bueno!'' --- gritavam ao nosso lado, provavelmente estranhando e desaprovando que eu e o Alcino não aplaudissemos também. À saída, uma apoteose: lançaram chapéus, capas e carteiras de cigarros. Antônio devolvia capas e chapéus --- a honra estava em ele tocá-los --- e os auxiliares embolsavam os cigarros.

Saímos antes do final, pois ainda faltavam dois animais. Fui direto e sozinho para o futebol no Estádio Santiago Bernabeu, só tendo tempo de comer uns ``bocadilos de jamon''(espécie de sanduíche de presunto). Paguei pela tribuna, desta vez, 180 pesetas, ou seja 3 dólares, ou ainda Cr\$415,00! Viva o Maracanã! A entrada mais barata é de 90 pesetas (Cr\$208,00).

Jogaram Real Madrid versus Espanhol, de Barcelona. Dois brasileiros de cada lado: Didi e Canário do Real e Índio (ex-Flamengo) e Décio Recaman no Espanhol. Brilharam todos, menos Recaman, um tanto lento e sumido. Canário recebeu poucas bolas, mas sempre ameaçou o arco contrário com tiros violentos. Didi é mais conhecido como ``El Negro'', mas sem sentido pejorativo, nota- se. É que não há quase pretos na Espanha. Ele realizou grandes e más jogadas, culminando com um gol notável de falta, batida de longa distância. O melhor em campo, porém, foi o outro ``negro'' brasileiro, o Índio, que pela primeira vez jogava em Madrid, já que fora contratado recentemente pelo Espanhol. Ele fez um partidão, atuando quase sozinho no ataque de seu discreto time. Ouvi comentários surpreendidos e fartos elogios ao caboclo, mesmo de torcedores do Real, a maioria absoluta dos presentes.

Dos outros craques internacionais, Di Stéfano e Puskas mostraram a enorme classe, principalmente o argentino, terrível nos arremates e que fez dois gols. Tive que sair ao término do primeiro tempo, para não facilitar com o horário de saída do trem. O Real fizera 4x0 até aquele momento. Não teria mais graça.

Peguei uma táxi ``Citroën'', que muitos há na Espanha, e larguei-me para a ``Delícias'', onde já me esperava o Alcino. Dali a pouco estava comodamente instalado num leito privativo, a caminho da capital portuguesa. Alcino, meio ``mão fechada'', não topou o mesmo programa. Resolveu viajar em poltrona comum.

\section*{28 \adfflatleafright \ISODayName{1959-09-28}}
Dormi regiamente nesta ótima viagem. Às 9 da manhã acordei e esperei a alfândega lusitana, antes de tomar café. Na saída de Madrid a duana espanhola foi atenciosa e displicente com a minha bagagem, que fez questão de nem ver. O que, aliás, não tem sido diferente nos treze ou quatorze cruzamentos de fronteira que já realizei.

Desde que pulei do leito, já dia claro, foi agradável acompanhar o desfilar do panorama do solo português, cuja fronteira cruzamos, muito antes, entre paradas de Valencia de Alcântara, última cidade espanhola, e Castelo de Vide, já em terras lusitanas. Depois ultrapassamos Abrantes e Santarém, a maior localidade da região. As distâncias são curtas entre cidades e povoados, que se sucedem rapidamente. A esta altura já estamos à beira do Tejo, que tem aqui águas barrentas. Em Vila Franca do Xiro, já bem próximo de Lisboa (uns 30km), o rio se lança na grande barra, verdadeira baía, que então, limpa suas águas.

Às 11,30, desembarquei lépido para encontrar na plataforma o Alcino, que já havia cavado uma carona no táxi de um conhecido de viagem.

Revejo, assim, Lisboa, a primeira cidade europeia que conheci, na escala aqui do navio, em 18 de julho, que já parece bem distante.

Fico bem no coração da cidade, na ótima pensão ``Leiriense'', na Rua Assunção. Posso agora sentir mais de perto o bom povo desta terra. Minha primeira preocupação, porém, é tratar de uma condução marítima para Cadiz, no próximo dia 2 de outubro. Isso porque soube que o transporte ferroviário é mortificante, pelas condições de estrada, voltas e baldeações. Corri, praticamente, todas as companhias, no mínimo meia dúzia, visto que suas sedes eram próximas, e nada consegui. Não haveria barco para a época pretendida. A viagem seria de apenas um dia.

A solução caiu do céu. Caiu não, subiu, pois descobri uma linha aérea para Sevilha. Dali a Cadiz serão três horas e pouco de trem. Reservei logo a passagem. Com este único problema resolvido, lancei-me a respirar tranqüilamente a atmosfera lisboeta. Cessaram, finalmente, todas as dificuldades de língua. Estou em casa! Alcino, ``para variar'', não topou o meu esquema de viagem. Deve ter achado caro, outra vez, e então resolveu enfrentar a viagem por via férrea, com toda a dureza já de antemão conhecida.

Lisboa não tem um centro comercial amplo e moderno, como pensava. Da borda do mar --- pois a barra do Tejo ali é, na verdade, o próprio mar, em larga baía --- estendem- se sucessivos largos bem velhuscos e feiosos: Cais do Sodré, Largo do Duque de Terceira, Largo do Campo Santo e Praça do Comércio, esta com a antiga ``porta da cidade'' (um arco com estreita passagem por baixo). Perpendicularmente a estes ``sítios'', saem as ruas centrais do comércio, estreitas e de construções todas antigas: Rua Augusta, Rua da Prata e Rua Áurea, que todos chamam, porém, de Rua do Ouro.

Ao fim das paralelas Áurea e Augusta está o tradicional largo do Rossio, que tem ao lado a Praça da Figueira e, em continuação para o norte, a Praça dos Restauradores, onde começa a larga e lindíssima Avenida da Liberdade.

Esta avenida é a mais bonita da cidade e, sem favor, das mais belas da Europa. Muito larga, com várias pistas e arborizada magnificamente. Tem até, em certo trecho, um longo lago com cisnes!

O comércio, em instalações, é bem fracote, como disse, e os preços nada convidadtivos. O ambiente da cidade, sim, é agradável, muito sossegado e mesmo de atmosfera provinciana. É um povo dócil, sem dúvida, o português, muito ao contrário do espanhol. Um fato ressalta logo aos olhos do observador, tal como na Espanha: a pobreza maciça do povo, em que pese a inegável existência de uma parcela razoavelmente abastada e de outra medianamente situada.

Notei a intensiva procura das loterias (em Portugal, ``lotaria''), impressionantemente populares nesta península, como em toda parte onde campeia a pobreza. É o jeito de procurar desapertar a penúria\ldots

Correndo as companhias de navegação, esta manhã, verifiquei a massa que se acotovelava a cata de passagens para as Américas e, principalmente, para a África'' (Angola e Moçambique). Navio lotados até março --- ouvi dizer --- e estamos em setembro!

À noite fui assistir, no teatro Coliseu, a revista em que trabalha Rose Rondelli, boa companheira de viagem no ``Cabo San Vicente''. Chama-se ``Há Feira no Coliseu'' e faz sucesso há dez semanas. Bom espetáculo, embora bem picante em numerosas charges, o que causa má impressão pela pobreza cômica da pornofonia. Isso, aliás, é uma constante nos teatros de revista. Um tal de Costinha (deve ser praga do nome) é o responsável por essas passagens. Há, porém, quadros bem bolados, engraçadíssimos mesmo. Rose está visivelmente mais gorda, já até passando da conta. Ela não tem grandes recursos, nem vocais, mas como é muito bonitinha e simpática, agrada em cheio.

No conjunto, como espetáculo, ficou bastante aquém das revistas de Walter Pinto, no Rio, principalmente em riqueza de vestuário e avanço técnico na mudança do palco. Muito me divertiu --- mas sem ser, naturalmente, parte do programa --- a presença de uma corista, em número de dança coletiva, que evoluiu repetidamente com um tornozelo envolto numa atadura\ldots

Outra particularidade estranha para um brasileiro: no intervalo vendedores apregoaram em altas vozes suas mercadorias, certamente algo comestível ou cigarros, sei lá. Só sei que não entendi uma única palavra da pregação deles\ldots Faz lembrar o velho Bernard Shaw, ao dizer que a Inglaterra e os Estados Unidos são dois países separados pela\ldots mesma língua! Portugal e Brasil também\ldots

Quando terminou a apresentação, visitei Rose no seu camarim, para cumprimentá-la e oferecer préstimos no Rio já que meu regresso estava se aproximando. Ela diz que ficará até janeiro do próximo ano e que está adorando Lisboa. Por sinal que já está falando com algumas expressões bem lusas, como o frequentíssimo ``pois, pois''\ldots

\section*{29 \adfflatleafright \ISODayName{1959-09-29}}
Estive de manhã na Panair do Brasil, na avenida da Liberdade, para ler alguns jornais do Rio, naquela avidez de notícias do Brasil. Soube de um desastre com um avião da VASP e que o jogador Dequinha, do Flamengo, fraturou uma perna (outro ``desastre'', menor).

À tarde vou dar um passeio pelos bairros mais distantes. Há ônibus de dois andares, de pintura verde-preto (``saia-e-blusa''), muito confortáveis. Os bondes têm o modelo dos antigos de São Paulo, de assento de palhinha. São de cor bege. O ônibus ``duplo'' frequentemente, ao trafegar sob árvores, roça sua coberta nos galhos, provocando um ruído surdo. Um passageiro, atrás de mim,. adverte o vizinho: ``São as ramagens\ldots''

Mais tarde, enquanto aguardo o visto no passaporte pelo Consulado Espanhol, saio em outra direção. Visita ao novo e bonito estádio do Clube Belenenses. É impressionante a qualidade dos gramados europeus --- um tapete de esmeralda!

Dou uma entrada, depois, nos ``Jerônimos'', antigo mosteiro do século XV, em estilo que puxa o mouro, mas tem variações rococó e ficou conhecido como ``manoelino''. Dentro dele estão vários túmulos, em notáveis e artísticos trabalhos em mármore, como o de Camões e Vasco da Gama. Sobre a tumba, a escultura de cada um, em mármore, estendidos na posição mortuária, em trajes reais da época. Outros túmulos, lá dentro, em outros compartimentos: de membros de famílias imperiais, Alexandre Herculano, Guerra Junqueiro, Almeida Garret e dos ex-presidentes Sidônio Paes, que foi assassinado, e General Carmona.

À beira-mar (chamo sempre de mar esta barra do Tejo), próximo dos ``Jerônimos'', está a Torre de Belém, do mesmo estilo do mosteiro, mas de pequenas proporções. Foi construída pouco mais tarde, para comemorar a famosa viagem de Vasco da Gama, pois está erguida no local de onde partiu o grande navegador.

Nos jardins do mosteiro encontro a portuguesinha Maria José, que aparenta uns 20 anos mas diz que tem apenas 14. Conversamos longamente e acho interessante como ela anarquisa o seu país, dizendo-o ``tão pobrezinho'' e ``que não presta pra nada''\ldots É fã da Suíça, por influência do sistema educacional. De fato, em Portugal o ensino é pago em todos os graus, desde o elementar, pois apenas alunos considerados excepcionais (no sentido de superdotados em inteligência) têm gratuidade, mas só no primário.

À noite contrato, para amanhã, um grande giro de carro até o Porto, para conhecer o interior e o norte do país, com suas cidades intermediárias, como Nazaré e Coimbra.

\section*{30 \adfflatleafright \ISODayName{1959-09-30}}
Saímos às 7 da manhã, numa excelente ``Mercedes-Benz'', das grandes, carro muito popular em Portugal, por ser movido a óleo diesel, combustível muitíssimo mais barato que a gasolina. É um carro de praça, de uma parente do dono do nosso hotel. Ao volante, homem de meia-idade, de nome Joaquim de Oliveira, muito simpático e conversador, vai logo mostrando sua franqueza. Qualifica de contristador o quadro da vida portuguesa, com os preços sempre a subir e os salários estagnados.

``A miséria campeia --- diz ele --- principalmente nas províncias, como os senhores irão ver''. Comento que, comparando com a Espanha, a situação em Portugal me parece melhor. Rebate ele: ``Então é porque aqui está mais disfarçado\ldots''

De fato, o país é pobre e pequeno. A terra, nesta região central, pelo menos, é clara e arenosa, pouco fértil. Passamos, de novo, por Vila Franca do Xiro, onde o Tejo se lança na larga barra, e, mais adiante, por Caldas da Rainha. Aqui paramos para quebrar o jejum. Há uma feira na praça principal. Alcino compra um prato de frutas em cerâmica, especialidade do lugar. Seguimos passando por Leiria, que tem um velhíssimo castelo no topo da colina mais elevada. Depois, atingimos Nazaré, cidade balneária e de pescadores. Todos esses vestem roupas de padrão xadrez, como os escoceses, e as mulheres usam sete saias. Porque, ninguém soube explicar.

Fotografamo-nos com eles, quase todos com grandes barretes enfiados na cabeça, cuja ponta (do barrete) fica pendente, no estilo dos anões da Branca de Neve. É também típico do interior este capuz; dentro do bico caído eles guardam, às vezes, dinheiro, cigarros, fósforos, etc. Vi muitos homens envergando-o, ao longo do caminho.

Nazaré como praia é passável. Mas prejudicada pelo número excessivo de embarcações de pesca entulhando a faixa de areia. Engraçada foi a nossa conversa com os homens, que pediram que nós repetíssemos mais de uma vez a nossa fala. E então, com ar de satisfação, comentavam com os parceiros do lado: --- ``Ora, percebe-se tudo\ldots'' Que língua nós falávamos? Português\ldots

Mais além, chegamos a Figueira da Foz, a melhor praia do país, dizem todos. E nós também achamos. Lembra a praia de Icaraí, em Niterói, pois é bem extensa e de larga faixa de areia bem branquinha. O principal hotel, à beira da praia é bem moderno e bonito. Ali deságua o rio Mondego, que banha Coimbra, daí o nome de ``foz''.

Seguindo, surpreendemos outra feira, esta mensal (sempre no último dia do mês) a de Mira, interessantíssima; compra-se e vende-se tudo, notadamente animais, cerâmicas e verduras. Pergunto a um feirante o preço de um bezerro. A resposta é pitoresca: ``treze notas e meia''; tradução: 1.350 escudos.

Chegando próximo à cidade do Porto, segunda maior do país, o panorama natural é bem diverso. Boas matas e profusão de culturas frutíferas. Esta região é o sustentáculo do país pelo visto. Vamos almoçar, enquanto o nosso carro é deixado pelo bom Oliveira em um posto para ``arranjar um furo'', que é como chamam aqui a providência de um pneu arriado. Porto está esparramada sobre as duas elevadas margens do rio Douro. Tem a maioria de suas ruas e praças em declive. Na parte central uma grande ponte de aço, pintada de vermelho, une as margens. Batemos umas chapas com ela no fundo.

O almoço não foi dos mais felizes. Comemos um ``arroz ao polvo'' pesadíssimo. Como nunca tinha provado esse prato, ficou a lição.

Fiz câmbio no banco mais próximo e, após umas voltas, seguimos mais para o norte, com o intuito de encontrar a vila ou povoado de nome Beiriz, terra do avô do Alcino, que muito tempo depois, no Brasil, acrescentou o nome do lugarejo também ao seu sobrenome. Informaram- nos que ficava entre Vila do Conde e Famalicão. Pergunta daqui, pergunta dali, afinal achamos o povoado. Lugar paupérrimo, apenas entregue ao fabrico de tapetes. Um ermo. Depois de várias indagações, com o pessoal local arredio em falar com ``estranhos'', o companheiro localizou uns parentes, ou possíveis parentes, pois distantes no tempo e no espaço, não parecia haver certeza nas referências. Mas ficamos por lá cerca de uma hora, procurando fazer render o convívio e registrar o encontro com algumas fotografias.

De volta ao Porto, tomamos novamente a esplêndida estrada, que é a melhor do país, dentre as que conhecemos. Muito larga, bem pavimentada e com uma notável característica: mão e contramão demarcadas em todas as curvas.

Iniciamos a volta para Lisboa por outra estrada, mais no interior, a fim de passar por Coimbra. São 110 quilômetros do Porto até esta última. Quando lá chegamos, a luz do dia já se fora. Outro furo para ``arranjar'' no pneu atrasou ainda um pouco a chegada. Jantamos e ficamos a passear pela pequena cidade, cuja rua principal, muito estreita, está toda em remodelação.

A avenida à beira do rio Mondego é larga e bonita. A Universidade fica numa parte mais elevada e está fechada. Não a pudemos visitar e, sim, apenas passar na porta. Seria bem interessante um contato com os ``capas pretas'', que serviram de inspiração para a minha versão em esperanto da canção ``Coimbra'', gravada pela grande Dolores Duran.

O curioso, durante o jantar, foi assistir o ajuntamento do populacho na frente do aparelho de televisão do restaurante. A televisão está sendo implantada agora em Portugal e, por isso, a curiosidade é grande.

Já em plena noite, o regresso à Lisboa faz-se mais longo, pois a velocidade do carro necessitou ser reduzida, apesar da confiança que inspira a excelência das estradas e, no nosso caso presente, as qualidades do motorista profissional.

Passando outra vez por Leiria, cujo castelo agora está vistosamente iluminado, a estrada reencontra-se com a mesma da ida. Oliveira vem contando coisas. Impressiono-me com o sistema de férias em Portugal: primeiro ano de serviço, 5 dias de férias; segundo ano, 6 dias, terceiro ano, 7 ou 8 e nesse ritmo vai até o máximo de 15 dias. A assistência social é praticamente inexistente, apesar de haver sindicatos, mas estes possuem reduzida capacidade de ação.

Chegamos a Lisboa a uma hora da manhã. Foi um grande programa este magnífico giro.
