\addchap{Outubro}
\section*{1 \adfflatleafright \ISODayName{1959-10-01}}
O companheiro Alcino saiu bem cedo para sua mortificante viagem de trem até Cádiz\index{Cádiz}, via Sevilha\index{Sevilha}, de onde partiremos de volta para casa.

Como só voarei amanhã, tenho o dia inteiro para me quedar neste descansado ambiente lusitano. Um bom café no bar ``Pic-nic'', no largo do Rossio, tem o seu sabor, principalmente para apreciar a graça e disposição das três ``cachopas'' que ali servem.

Em Lisboa\index{Lisboa} volta-se a ouvir falar em ``tostões'', velha expressão inteiramente esquecida, no Brasil, com a reforma monetária. O escudo se divide em cêntimos, sendo um ``tostão'' a moedinha de cobre de 10 cêntimos, e dois ``tostões'' a de 20 cêntimos, do mesmo tipo. Assim, ao invés de um escudo e quarenta ou cinquenta cêntimos, por exemplo, ouve-se ``quatorze tostões'' ou ``quinze tostões''.

Logo mais à noite um jogo de futebol entre o Benfica e o Rudzi Pravo, de Praga, Tchecoslováquia\index{Tchecoslováquia}. Partida amistosa, mas que está despertando certo interesse porque o quadro tcheco eliminou, há poucos dias, o F.~C. do Porto da Copa da Europa, jogando na própria ``cidade invicta'', como é conhecido o Porto. Ainda não sei se irei.

Passando pela Panair do Brasil, encontro, grudado nos jornais do Rio, o conhecido esportista brasileiro Zoulo Rabelo, que já foi técnico de futebol no Rio e agora é treinador de voleibol. Ficamos a conversar sobre as atualidades esportivas e por seu intermédio fiquei sabendo que o Brasil disputou um campeonato universitário em Turim, com ele dirigindo a equipe, e outro torneio mundial, de grande expressão, em Paris\index{Milão}. Deste quadro fez parte o jogador Arlindo, do Fluminense\index{Fluminense}, que dali a instantes apareceu para se encontrar com Zoulo. Ambos se mostraram impressionados com a capacidade técnica das equipes do bloco socialista, que conquistaram todos os primeiros lugares.

Depois do almoço, resolvo ir ao cinema. Os grandes sucessos são ``Orpheu Negro'' - sempre ele! - no Tivoli e ``Os Dez Mandamentos''. Este último dura quatro horas de projeção. Tempo demasiado para se perder trancado numa sala. Opto pelo ``Al Capone'', no Politeama. Bom trabalho de Rod Steiger, na biografia do célebre gangster italiano que infernizou Chicago e arredores.

Desisto do futebol, preferindo, de noite, gozar ainda o convívio agradável dos lisboetas. Se bem que a vida noturna seja fraca para os que não quiserem se enfiar em casas de espetáculos. Hermínia Silva tem um ``Solar da Hermínia'', na cidade alta. Amália Rodrigues parece que está fora, pois não se ouve falar dela.

No teatro, João Vilaret apresenta ``Pataté'', na Avenida da Liberdade. Não sei que gênero é. Grandes anúncios indicam a próxima chegada de Maria Della Costa para encenar ``Gimba'', em Lisboa\index{Lisboa}.

Mas, tal como as outras cidades europeias, o movimento nos bares é grande, com mesas repletas. Na Avenida da Liberdade há um sobre a calçada central, bem agradável. O grosso da população, porém, recolhe-se cedo.

\section*{2 \adfflatleafright \ISODayName{1959-10-02}}
Tenho a manhã inteira ainda livre, pois só irei embarcar às 12h30m.

Passo no representante da companhia de navegação, que fica na tradicional rua do Alecrim, bem aladeirada, por sinal. Procuro correspondência, eventualmente chegada nestes últimos dias, mas a resposta é negativa.

Não sei se já falei nas obras do ``metrô'' de Lisboa\index{Lisboa}. Ouço dizer que se vai inaugurar em dezembro. De início são apenas duas linhas, parece-me que de uns 8~km cada uma. Estão completando o revestimento das escadas das várias estações.

Mandei copiar nada menos que cinco rolos de negativos, pois verifiquei, pelo primeiro trabalho aprontado, que aqui o serviço fotográfico é esmerado. Ficaram todos excelentes. Mas ainda faltam outros cinco ou mais rolos, para completar a ``documentação''.

Fui mais cedo para o aeroporto, que se chama ``Portela do Sacavém''. Lá, por acaso, encontrei o chefe do escritório Comercial do Brasil em Lisboa\index{Lisboa}, Fanor Cumplido de Santana, que foi do IRB na administração do General Mendonça de Lima. Ele conversava com os senadores Lourival Fontes e Barros Cavalcanti, que estiveram na conferência interparlamentar de Varsóvia\index{Varsóvia}, e me apresentou aos ilustres ``pais da pátria''. Tenho a surpresa de, no mesmo grupo, encontrar a simpática senhora do também senador nordestino Barros de Carvalho, que viajou só no ``Cabo San Vicente\index{Cabo San Vicente}'', integrando a delegação ao festival de Viena\index{Viena}. Agora, com o senador e duas filhas, está a caminho de Londres.

Assisto chegar o meu avião, da companhia espanhola ``Aviaco''. Quando comprei a passagem na agência, indaguei sobre o tipo de avião e me responderam que era um quadrimotor. E agora tenho a confirmação de que é realmente um quadrimotor, só que de pequeno porte, pois tem apenas 14 lugares. Ao embarcarmos, uma jovem presenteia os passageiros com uma garrafinha de vinho do porto.

Saímos pontualmente. Viagem excelente, de hora e um quarto. Magnífica a visão panorâmica da cidade e da barra do Tejo, com suas claras águas. Na margem oposta, o monumento do ``Cristo Rei'', de Almada, inspirado naturalmente no nosso Cristo Redentor do Corcovado. Contudo, a visão do sul de Portugal e do contíguo território espanhol é pouco animadora. Terra bem árida, sempre clara, com não mais que salpicada vegetação anã. Perto de Sevilha\index{Sevilha} melhora bastante.

Chegando ao destino, deixo logo a bagagem na estação ferroviária, pois pretendo estar embarcando às 20h30m para Cádiz\index{Cádiz}, meu glorioso ponto final destas andanças de pouco menos de três meses pelo Velho Mundo. Começo a ficar excitado com a proximidade do regresso. E a respirar aliviado com o término das movimentações de um lugar para o outro. Em Cádiz\index{Cádiz} será só entrar no navio e --- dez dias depois --- Rio de Janeiro!! Que maravilha!!

Há tempo de sobra para conhecer bem Sevilha\index{Sevilha}, velho fortim árabe, dos tempos da dominação desta península pelos mouros, que durou 800 anos. O estilo arquitetônico árabe está em toda parte. A área velha é acanhada, de ruas bem apertadas e tráfego engasgado, em uma só direção. A impressão é de pobreza generalizada do populacho.

O ponto culminante é a visita à catedral ``La Giralda'', concluída lá por 1500. Assombrosamente grande, pode ser comparada, em área interna, a um campo de futebol, com seus 120 metros de comprimento por mais de 60 de largura. Piso todo de mármore e quatro impressionantes fileiras de oito colunas de pedra, de uns 20 metros de altura. Nestas horas é que a gente tem que reconhecer que o pessoal antigo tinha cabeça, espírito empreendedor e\ldots\ tempo para ousar as construções mais audaciosas. Em determinadas salas há uma exposição de tesouros, sob a vigilância de guardas entediados.

Outra edificação famosa se chama ``Santa Maria La Blanca''. Como não entrei, fiquei sem saber se também era igreja. Deve ser. Cheguei à conclusão de que o Vaticano\index{Vaticano} devia ficar na Espanha\index{Espanha} e não na Itália. Os espanhóis deixam os italianos longe em matéria de devoção católica. Lá são corriqueiras as conversas em público ostensivamente anticlericais. Aqui a contrição é impressionante. O capítulo dos nomes de logradouros públicos, então, revela incrível inclinação para denominações de caráter religioso.

A cidade é banhada pelo rio Guadalquivir, tendo algumas pontes bastante pitorescas, junto a uma das quais se ergue a ``Torre de Ouro''. Bela ornamentação é a profusão de palmeiras daquele tipo mais bonito, de meia altura, que povoa os oásis.

Os nomes das ruas são indicados não por placas metálicas, mas por enormes ladrilhos nas paredes, com as letras aplicadas de maneira a formar os nomes. Ficam enormemente destacados.

O local mais agradável e, mesmo, mais bonito da cidade é, sem dúvida, a longa avenida Menendez Pelayo, em toda a sua extensão, e, notadamente, a praça em que termina, chamada ``Plaza de España''. A arborização de palmeiras é belíssima, na praça, que, em um dos lados, se transforma em amplo parque de intensa vegetação. A ``plaza'', em sua parte central, é tomada por uma imensa e imponente construção, no mais puro estilo mouro, de tijolos vermelhos à mostra. O edifício é em semi-círculo, com duas majestosas torres, uma em cada extremidade. É sede de unidades milicianas, lamentavelmente.

Ao lado está armada uma plataforma, à guisa de palco, diante de ampla arquibancada, tudo de madeira. Hoje à noite terá lugar ali o espetáculo de abertura de um festival de música sinfônica e bailados clássicos. Ensaiam no momento um número de balé. Bato uma chapa, por curiosidade.

Mais tarde, depois de jantar e me preparar para o embarque na estação, venho a saber que o ``rápido'' para Cádiz\index{Cádiz} (procedente de Madrid\index{Madrid}) estava com três horas de atraso, devido a um descarrilamento na linha. O Alcino, que eu havia encontrado lá, à tardinha, por acaso, seguiu no trem correio, com o mesmo atraso.

Volto, assim, para a cidade, a fim de apreciar o movimento noturno, que é surpreendentemente grande. O espanhol é de rua mesmo, qualquer que seja a cidade, e esta é a terceira em que piso. O povo é alegre como poucos, apesar da dureza e do desconforto em que vive.

Finalmente, beirando às 24 horas, parti. A viagem tornou-se rápida, pois os cochilos fizeram o tempo voar. As quatro da manhã saltei em Cádiz\index{Cádiz}, sob pavorosa ventania. Alcino teve a gentileza de me esperar na estação e, como já havia tomado acomodações, só fiz acompanhá-lo, louco por uma cama de verdade.

\section*{3 \adfflatleafright \ISODayName{1959-10-03}}
Nenhuma apreensão me assalta quanto ao problema e garantia de embarque depois de amanhã. Isso porque, tendo considerado a grande antecedência com que marquei a reserva para o regresso (em Gênova\index{Gênova}, no dia 24 de julho), resolvi de Roma\index{Roma} escrever à companhia, aqui em Cádiz\index{Cádiz}, confirmando tudo e pedindo resposta para Lisboa\index{Lisboa}. Recebi-a na capital portuguesa, dando as coisas em perfeita ordem.

Só preciso saber das malas, despachadas de Paris\index{Milão}, e da encomenda da peça do Citroën, remetida também de lá. Apressei-me, por isso, em estar na companhia, mesmo porque hoje é sábado e podem fazer meio-expediente. A agência fica na ``Calle Beato Diego de Cádiz\index{Cádiz}'' (vejam que nome!). Recebo correspondência do Brasil, com enorme satisfação. Após algum trabalho, localizo também o volume da peça de automóvel. Mas sobre as malas, nada sabiam. Confio, porém, em que elas virão no navio, desde Barcelona\index{Barcelona}, pois o despacho foi para lá, via Marselha\index{Marselha}.

Trato de preencher as formalidades de embarque, na ``Policía'' e na ``Sanidad Publica''. Depois, troco de hotel, passando para o ``San Francisco'', pois o arranjado pelo Alcino era bem acanhado e desconfortável.

Cádiz\index{Cádiz} é bem menor do que Sevilha\index{Sevilha}, naturalmente, e bem pouco interessante. Parece viver em função de sua qualidade de porto marítimo. A maior parte da cidade é de ruazinhas estreitíssimas, com os nomes também indicados por letreiros garrafais, tal como em Sevilha\index{Sevilha}. Deve ser o sistema de toda a Andaluzia.

A praça principal, junto ao mar, chama-se ``San Juan de Dios'' e dela sai a avenida Ramon Carranza, beirando o porto. Caminhando-se em sentido contrário, atinge-se a parte mais moderna da cidade, onde existe o ``balneário'', que tem uma praia bem boazinha, com grande faixa de areia. Mas o tempo está encoberto e não demora a cair uma chuva bastante forte, que, felizmente, durou pouco.

Depois do jantar, pude sair a passear de novo. Apesar dos pesares, a cidade tem bonde e ônibus elétrico. Giro em um dos primeiros até os bairros mais distantes. Mas todas as linhas vão ter ao balneário.

\section*{4 \adfflatleafright \ISODayName{1959-10-04}}
Apesar de domingo, a companhia de navegação funciona hoje, para a recepção das passagens e outras providências de última hora, já que estamos na véspera do embarque. Tive que passar lá às 10 da manhã.

Cedíssimo, porém, ainda na cama, fui despertado pelo coro simplesmente maravilhoso de uma procissão que passou sob a nossa janela. A melodia era verdadeiramente linda e a disposição das vozes, masculinas e femininas, de magnífico efeito.

O tempo, ainda encoberto, afugenta minha idéia de pegar a praia. Fui até lá, pouco depois do almoço, e a areia tinha desaparecido, com o mar batendo de rijo na amurada. A ocasião se faz propícia parra comprar coisas, pois daqui é só lançá-las a bordo e desembarcar no Rio, confortavelmente, sem o tremendo problema de lugar e peso na bagagem.

O comércio é modestíssimo, mas com grande número de lojas. Venho a saber das excelências de uns conhaques fabricados em uma cidade próxima, com as marcas ``Carlo II'' e ``Caravelas''. Comprei uma garrafa de cada um, para oferecer aos amigos apreciadores. Encontro, também, as pérolas cultivadas que uma das minhas irmãs me pediu por carta. E castanhas, figos e tâmaras baratíssimos. Faço uma lista, para providenciar amanhã cedo, pois o embarque será às 17 horas.

Voltando a pensar nas malas despachadas, achei por bem preparar uma carta dirigida ao Consulado do Brasil, para o caso de não as receber no navio, como esperado. O Consulado tomaria as medidas necessárias para defender meus interesses, pois o que não posso é deixar de embarcar, com ou sem as malas. E amanhã, dependendo da hora em que tiver a definição do assunto, poderei ficar sem tempo para tomar essa providência. Pensando assim, escrevi logo a carta, no hotel, em papel e envelope timbrados dele, e deixei tudo preparado para a eventualidade desfavorável.

A noite de domingo apresenta aspectos bem provincianos, nesta Andaluzia tão tradicional. Povaréu nas ruas e cinemas superlotados.

\section*{5 \adfflatleafright \ISODayName{1959-10-05}}
Dia do regresso! A ansiedade me domina, confesso.

Como disse nas várias cartas que escrevi ontem e hoje, não posso me considerar cansado de viajar. Além da resistência muito grande aqui do ``degas'', o interesse e a variedade permanente dos cenários iam levando bem a movimentação por estas famosas paragens. Mas cansado de estar fora do meu ambiente, isto na verdade estou. Por esse motivo, a alegria de iniciar o caminho de volta tomou conta de mim.

Também o término do problema de transportar bagagem, tratar passagem, tomar trem, procurar hotéis, entender línguas diferentes etc., traz um delicioso alívio! Estou louco para me ver alojado dentro do barco, naquele sossego enorme da vida de bordo nas longas travessias.

Compro, na cidade, as coisas que desejava. E saio direto para o cais, a fim de pôr os olhos no navio e me certificar logo quanto à vinda da bagagem em perfeita ordem. Lá está o imponente ``Cabo San Roque\index{Cabo San Roque}'', branquinho como o irmão gêmeo, já cercado de muito movimento. Após enfrentar algumas dificuldades criadas pelo pessoal de bordo, porque não era hora ainda de entrar no navio, conseguimos ``cantar'' o camareiro da nossa ala, insistindo que só queríamos olhar da porta da cabine se a bagagem estava lá. E estava! Encontro as duas grandes malas direitinho no chão do camarote. Aproveito, então, para fazer embarcar a peça ``Citroën'', que vinha devidamente encaixotada. Contei com a colaboração do despachante que a desembaraçou.

Voltando ao cais, rasguei e joguei fora a carta para o Consulado. E, em seguida, sou agradavelmente surpreendido com a presença de inúmeros ``festivalistas'' da viagem de vinda, que estão também retornando no ``San Roque'': Lucas, Nélson Cimino, Dona Hilda, filha (Vera Lúcia) e sobrinha (Magali), Edith e Amir, as inseparáveis, Drs. David e Franciosi, os médicos do Instituto de Câncer, os irmãos Dirceu e Vera Lúcia Brisola, a pianista Nacife, Armando, Paulo Tone e, a maior ``boa nova'', Yone Cozza, a santista simpaticíssima, além de outros. Todos de São Paulo e Santos. Uns vinte. Justamente o que esperava encontrar a bordo, pois declarou em Paris\index{Milão} na recepção do dia da independência, que viajaria conosco, o Luiz Vergueiro, não está no barco. Foi, assim, imensa a satisfação de verificar que iria ter a companhia do ótimo grupo, já bem conhecido, na travessia de volta.

É bem verdade que esta viagem do Cabo San Roque\index{Cabo San Roque} se tornou particularmente favorável, pois partindo ele de Barcelona\index{Barcelona} e só tocando em Cádiz\index{Cádiz}, não levará emigrantes italianos nem portugueses, que costumam lotar os navios para a América do Sul. Soube que o nosso está, agora, com pouco mais da metade da capacidade total da ``classe econômica''. E somos só dois na nossa cabine.

Depois do último almoço em terra, na Europa, ainda tenho tempo para comprinhas de última hora, inclusive \textit{souvenirs}. Mais conhaque, figos e tâmaras.

Às 17 horas partimos! Ah, poucas vezes me senti tão eufórico em minha vida! O dia está lindo e a satisfação de estar a bordo atinge o auge. As conversas com os antigos companheiros assumem um conteúdo rigorosamente ``internacional''; toda a Europa desfila, palmilhada por um ou outro grupo. Meu recorde não foi, porém, batido por nenhum. Parte dos festivalistas, após o conclave, andou por Moscou\index{Moscou} (encontrei alguns lá) e Cazaquistão; outros receberam convites diversos: Bulgária\index{Bulgária}, Hungria\index{Hungria}, Romênia\index{Romênia}, Iugoslávia\index{Iugoslávia}, Polônia\index{Polônia} e Tchecoslováquia\index{Tchecoslováquia}.

Ninguém deixou, contudo, de estar na Itália, França\index{França} e Espanha\index{Espanha}. Impressões e curiosidades são trocadas quase freneticamente. E o mesmo quanto às compras e juízos sobre preços.

O Cabo San Roque\index{Cabo San Roque}, além de ser irmão gêmeo do San Vicente\index{San Vicente}, como já disse, pareceu-me ainda mais simpático e acolhedor. O pessoal de bordo é extraordinariamente gentil, o quem nem sempre ocorria no outro.

A sensação de felicidade é geral e contagiante. A maioria dos conhecidos engordou e ostenta ótimo aspecto. O mesmo eles acham de nós dois. Algumas, como Yone e Laís, engordaram dez quilos! Só Edith e Amir demonstraram a palidez original. Também são enjoadíssimas para comer. Somos companheiros de mesa e pude testemunhar.

À noite, no baile, vemos que até a orquestra é melhor que a do San Vicente\index{San Vicente}. Realidade ou sugestão, pouco importa\ldots

\section*{6 \adfflatleafright \ISODayName{1959-10-06}}
A piscina não pôde ser estreada logo nesta primeira manhã. O tempo está encoberto, sem sol e venta muito.

Mas o pior me aconteceu ontem à noite. Senti horrível tonteira e estômago meio embrulhado, com tremenda disposição de vômito. Não é possível! --- lamentei-me. Nunca enjoei em minha vida e não iria estrear justamente agora, na segunda viagem, depois de ter vencido a primeira, de 16 dias, com a galhardia que eu já esperava que iria acontecer! Explica-se: quase não jantei, sem fome às 18 horas, pois sendo ontem o primeiro dia, não pude me transferir logo para o turno das 19 horas. Com o estômago praticamente vazio, o balanço do navio, de início, indispõe fortemente.

O amigo Armando correu à sua cabine e voltou com um comprimido (dramamine?), que me fez engolir imediatamente. Por influência do remédio ou não, o caso é que passei a me sentir melhor, e hoje já não sinto absolutamente mais nada.

Mais tarde, no camarote, fazendo revisão da bagagem despachada, dou por falta do perfume ``Miss Dior'' (duas onças!), encomenda de uma tia de São Paulo, e do fotômetro alemão, encomendado por um colega do IRB. Não resta dúvida: roubaram-me, e, na certa, em Paris\index{Milão} mesmo, pois as malas vieram cintadas, com duras fitas metálicas, sem sinal de violação. Portanto, não foram abertas na alfândega espanhola, por se tratar de bagagem em trânsito. Paciência\ldots

O programa da tarde consistiu numa longuíssima audição de discos, na vitrola alemã do Paulo Tone. Ouvimos também os nossos, comprados em Moscou\index{Moscou}. Depois, o programa de bordo constou de um bingo, muito bom para a gente perder quarenta pesetas e ficar chuchando o dedo.

Amanhã bem cedo aportaremos em Tenerife\index{Tenerife}, única escala da viagem. Há grande planos de compras, pois os artigos lá, em matéria de preços, são uma tentação.

Outro baile, à noite, com a moçada muito alegre e disposta.

\section*{7 \adfflatleafright \ISODayName{1959-10-07}}
Nem bem o dia clareava, o movimento já era grande para o desembarque em Tenerife\index{Tenerife}. Às 6h30m pus-me de pé e dali a meia hora, já de café tomado, pulei em terra. Lembrei-me da parada aqui, na vinda, quando o barco acostou às 4 da manhã. Todo o mundo, então, estava louco por terra, após 8 dias maciços de mar alto.

Mando um último postal para casa e mergulho na fascinação das lojas. O rádio portátil japonês que tinha em mira é comprado por 25 dólares, numa loja de hindus. A marca é ``Hitachi''. Depois, encontro, com muita satisfação, um fotômetro igual ao roubado e o jeito é levá-lo, para não decepcionar o colega, ao qual não revelarei o ocorrido. Fico mesmo no prejuízo\ldots\ Outro perfume inglês (``Bond Street''), não encontrado na Europa, lá está na mesma loja, a preço convidativo.

Deixo aqui uma boa parte das minhas sobras em dólares. O ponto alto foi um toca-discos transistorizado, inglês, a pilhas, de um tamanho incrivelmente pequeno. Lá se vão mais 42 dólares\ldots\ Compro dois perfumes espanhóis, muito elogiados, para tentar compensar o ``Miss Dior'' surripiado. Têm os nomes de ``Sortilege'' e ``Amor Brujo''.

Não é preciso percorrer muito a cidade, pois já a conheci bem na vinda. Mas o mercado ``Nuestra Señora da África'' é visita obrigatória. Que maravilhoso sortimento, que agradável sensação de limpeza e bom gosto na arrumação das mercadorias! Nunca vi um que se comparasse com este em apresentação.

Volto correndo para o navio, a fim de alcançar o almoço. Às 14h30m, retomamos viagem. O dia está maravilhoso e a enseada de Santa Cruz de Tenerife\index{Tenerife!Santa Cruz}, com suas montanhas pedregosas e pontiagudas, atrás do concentrado casario, de muitas construções modernas, é um cenário dos mais atraentes. Não demorou muito a desaparecer o último vestígio de terra. Agora, só o maravilhoso Rio de Janeiro!

Estive na cabine a testar os aparelhos comprados --- rádio e toca-discos --- que corresponderam perfeitamente. Tenerife\index{Tenerife} tem uma estação de rádio. O jantar foi melhorado e festivo. Até champanhe serviram. E à noite o baile esteve mais movimentado, com concursos e um ponche servido às 23 horas.

Já atrasamos o relógio 30 minutos no primeiro dia e ontem. Hoje, mais um retardamento de 23 minutos. Temos que recuperar 4 horas até o Rio. E é agradável esse negócio de, à meia-noite, ir-se dormir às onze e meia\ldots

\section*{8 \adfflatleafright \ISODayName{1959-10-08}}
Manhã radiosa. Faço horas para pegar, finalmente, a deliciosa piscina de bordo.

Venho a saber, pelo Lucas, que o navio tem um laboratório fotográfico, entregue a um profissional. Lucas, que esteve no interior da Bulgária\index{Bulgária}, me mostra as ampliações das boas fotos que fez lá. Penso em mandar algumas das minhas ainda em negativo. Mas o problema financeiro faz-me hesitar. Melhor, talvez, deixar para tratar disso no Rio.

Há uns tipos curiosos a bordo. Desta vez os brasileiros são minoria. Os espanhóis dominam francamente. Há, também, muitos libaneses e argentinos. Mas a atenção geral se prende a uma família iraniana, vinda de Marrocos e embarcada em Cádiz\index{Cádiz}, conosco. Parecem ciganos, pelo aspecto quase miserável e demonstrações de falta de higiene. Mas levam um carro e, dizem, vão fazer turismo no Brasil.

Um espanhol, tipo esguio e desengonçado, há três dias anda ``alto'', com expressão de peixe frito no olhar. Bate palmas cadenciadas, tentando marcar o ritmo de uma ``flamenqueria'', mas a voz rouquenha e desafinada não ajuda. Ontem à tarde deu um show na piscina, executando lamentáveis macaquices.

Junto à minha mesa, no refeitório, sentam-se seis padres, dos dez ou mais que viajam. Benzem-se, de pé, antes e depois das refeições. Outro tanto deve ser o número de freiras.

Depois do almoço, um espetáculo sensacional: cruzamos, em pleno oceano, com o Cabo San Vicente\index{Cabo San Vicente}, que, nesse tempo todo, já voltara e estava indo de novo para a Europa. Naturalmente os pilotos se entendem pelo rádio e forçam a aproximação, que foi de menos de 100 metros. Espoucaram foguetes, ambos apitaram freneticamente em saudação e os passageiros e tripulantes, nas amuradas, acenaram lenços e soltaram gritinhos (as mulheres). De farra, gritamos para lá um monte de ``Manolos'' e ``Pepes'', que são a coisa mais batida em matéria de nome espanhol.

Às três horas teve lugar o exercício de salvamento. Todo mundo envergou o ``elegante'' colete salva-vidas, cor de abóbora, e se distribuiu pelos pontos indicados. Coisa rápida, assim mais para constar.

Mais tarde, a animada corrida de ``caballos'', no salão ``mirador''. Pela primeira vez ganhei em um páreo, salvando as 15 pesetas enterradas nas corridas anteriores.

À noite, estreou o cinema, com enorme afluência. Exibiram ``Retorno al Passado'', velho filme com Robert Mitchum e Kirk Douglas, ambos com caras bem novinhas. Sempre a dublagem em espanhol, magnífica.

\section*{9 \adfflatleafright \ISODayName{1959-10-09}}
Dia sem muitas novidades, a não ser a passagem pelas Ilhas Cabo Verde\index{Cabo Verde}, pertencentes a Portugal. Lembro-me de que, na vinda, essa passagem ocorreu durante a noite e não se viu nada. O arquipélago tem unidades bem grandes, pois se as avista demoradamente. As ilhas são baixas e de aspecto árido. Pode ser que lá dentro haja matas mais densas, mas do lado que observo tudo parece por demais ensolarado. Salazar mantém ali o seu ``famoso'' campo de concentração, que dizem chamar-se Tarrafal.

Difícil encontrar um português falando mal de Salazar. Todos com quem conversamos disseram-se seus admiradores. Mas o curioso é que nenhum desses admiradores vivia em Portugal.

Vamos nos aproximando do equador. O sol tem estado terrível, mesmo de manhã cedo. Miro-me no espelho, para avaliar o quanto já ``amorenei''. O patrício Jacques tem uma bola de plástico e com ela fazemos uma ``linha-de-passe'' na beira da piscina, com mil cuidados para não deixar cair no mar. Quando há esse perigo, todos se divertem acompanhando as peripécias do salvamento.

Entre repouso nas amuradas, sonecas na cabine e imensos ``papos'' (de preferência com Yone) nos salões, o dia se foi.

O cinema da noite gorou. Iam levar ``A Vingança do Lobo'', filme espanhol. Parece que a vingança foi outra: fazer a platéia esperar quase uma hora, sem aparecer\ldots

O possante receptor ``Hitachi'', de ondas médias e curtas foi devidamente testado. A posição é algo desfavorável. As estações mais poderosas de Portugal e França\index{França} são captadas, mas pouco interessam. Também há uma emissora de Rabat, no Marrocos, com sinal bem forte, e é a mais próxima de nós no momento. O problema é que ninguém aguenta mais que cinco minutos de cantoria árabe\ldots

\section*{10 \adfflatleafright \ISODayName{1959-10-10}}
O calor do sol aumenta sensivelmente. Amanhã, domingo, cruzaremos a linha divisória dos dois hemisférios.

Talvez até já se possa ouvir um futebolzinho pelo rádio, à tarde.

Piscina e bate-bola de manhã e um renhido ``buraco'' com Dona Hilda, Magali e Amir, para encher a tarde.

À noite, fizeram a eleição dos reis do equador, que irão ``funcionar'' amanhã nos ``bautismos de los neófitos''. Os homens votam para a escolha da rainha e as mulheres para a de Netuno. Venceram uma semi-velhota espanhola, bastante espevitada nos bailes, e um libanês que passa os dias sumido, mas à noite, nas danças, baila sozinho segurando um lenço no ar e fazendo evoluções comprometedoras\ldots

A animação foi grande. Nosso grupo, em minoria (só 25 brasileiros a bordo), não pôde garantir qualquer representação, o que, aliás, não teve nenhuma importância. Nas festas de todas as noites, o bloco brasileiro é o que comanda toda a animação.

O cinema tornou a gorar. Estava anunciada uma comédia alemã. Parece que a aparelhagem de som entrou em pane.

Fizemos umas tentativas com os radinhos e logramos apanhar razoavelmente\ldots\ a ``Hora do Brasil''. Foi a primeira audição brasileira conseguida! A ansiedade por notícias da terrinha estava acumulada. A proximidade progressiva do nosso continente vai favorecer, daqui por diante, as condições de receptividade. Que delícia!

A dormida, que sempre é em horas tardias, hoje passou da conta, apesar de todo o retrocesso de 23 minutos no relógio, à meia-noite. O espanholzinho miúdo, de horrendas costeletas, que diziam contratado pela televisão argentina, resolveu mostrar suas qualidades. Deu uma bela audição, na calada da noite, para um reduzido grupo, em que todos tinham um copo na mão (menos eu).

\section*{11 \adfflatleafright \ISODayName{1959-10-11}}
Manhã de domingo, como todas as outras: maravilhosa.

O calor do sol, está impondo respeito, porém. O jornalzinho de bordo recomenda cautela e avisa que atravessaremos o equador hoje às 14h30m.

Os reis batizaram os atravessadores neófitos. Antes, ajudei a maquilar o séquito real, todo fantasiado na base do papel crepom. A ``cerimônia'' não foi tão divertida como a da viagem de ida, no Cabo San Vicente\index{Cabo San Vicente}, pois os reis estiveram algo apáticos. Para encerramento, os brasileiros voltaram a exibir o mau espetáculo de lançar gente vestida na piscina. Mas, felizmente outra vez, prevaleceu o bom humor e tudo foi levado no espírito de brincadeira.

Quando eu estava no banho, desfrutando o chuveiro espetacular de bordo, ouvi o alto-falante anunciar que, dentro de cinco minutos, passaríamos junto aos rochedos de São Pedro e São Paulo. Aprontei-me bem depressa a tempo de subir justamente quando o barco navegava ao lado dos pequenos grupamentos rochosos soltos no meio do Atlântico. Uma armação de madeira sobre elas faz supor um farol, mas esta hipótese não parece viável. Belo é o espetáculo das aves marinhas, que sobrevoam, em profusão, aqueles pequenos pontos sólidos na imensidão oceânica, onde, naturalmente, fazem seus ninhos.

À tardinha aventurei-me, novamente, a tentar ouvir alguma transmissão radiofônica esportiva do Rio ou de São Paulo, em ondas curtas. Eram 16h45m e logo de saída sintonizei esplendidamente a Rádio Nacional transmitindo do Maracanã\index{Maracanã} o jogo Vasco\index{Vasco}xFluminense\index{Fluminense}, de cuja realização só na hora tomei conhecimento. Mas a onda fugiu logo em seguida e não foi mais possível captá-la. Estações pernambucanas bem possantes interferiram e como a recepção destas era muito boa, acompanhei um jogo de Recife, com os locutores dando periodicamente os resultados do Rio e de São Paulo.

O Dr. David, caricaturista muito hábil, nas horas vagas, andava fixando tipos com o seu ``crayon'' e me incluiu na lista. Saiu ótimo o seu trabalho, de traço simples, no bom estilo.

Na hora do jantar eu já previa aquele espetáculo engraçadíssimo dos passageiros jantando com bizarros chapéus nas cabeças. Coube-me um de toureiro, com ``rabicho'' e tudo. O do Alcino parecia coisa de um potentado oriental. Era a preparação para a festa de ``travessia del ecuador'', logo mais à noite, quando seria apresentado um show, a cargo dos próprios passageiros.

E o programa foi bastante animado, se bem que também de nível algo inferior ao do Cabo San Vicente\index{Cabo San Vicente}. Lembre-se, porém, que temos agora 580 passageiros na nossa classe, contra mais de 800 na ida, que, obviamente, não contava emigrantes. Alcino resolveu gravar os números e ficamos todo o tempo com o aparelho junto aos ``artistas''. Desta vez os brasileiros ficaram ofuscados pela espanholada, por falta de lastro na nossa turma (apenas dois números bailados --- ``Tico-tico no Fubá'', com Laís, e ``Aquarela do Brasil'', com Dirceu e Vera Lúcia Brisola, Newton e Jacques), de regular efeito.

As honras da noite couberam ao tal espanholzinho de costeletas, que se chama David Toledo, e que cantou com enorme classe três esplêndidos números e ainda fez a apresentação de todo o programa.

Voltei mais tarde ao convés com o rádio e então ouvi estupendamente os noticiários finais das rádios Tupi e Jornal do Brasil; que satisfação sentir, assim, a proximidade cada vez maior da terrinha saudosa!

Antes de dormir, relógios 23 minutos para trás.

\section*{12 \adfflatleafright \ISODayName{1959-10-12}}
A natureza me premiou, efetivamente, com um tempo perfeito durante todos esses três meses e tanto de viagem.

Já agora em pleno hemisfério sul e abaixo de Fernando de Noronha, que devemos ter ultrapassado durante esta madrugada, entro no antepenúltimo dia de viagem. Consta que a chegada ao Rio, na quinta-feira, dia 15, ocorrerá bem cedinho.

No bate-bola, ao lado da piscina, esta manhã, aconteceu o pior: a bola escorregou suavemente pela coberta de lona e caiu no mar! A desolação foi completa e a minha maior ainda, pois dos meus pés saiu o toque, leve embora, para cima da coberta fatal\ldots

Já agora em plena costa pernambucana, passávamos a pegar a qualquer hora, com a maior perfeição, as estações radiofônicas de Recife, mesmo em ondas médias. A diferença da hora de bordo, que se vai aos poucos ajustando à hora brasileira legal, é hoje de 66 minutos (a de bordo adiantada).

Ouvimos rádio até enjoar, para matar saudades\ldots\ E note-se que os programinhas não são grande coisa, nos horários diurnos.

À noite, o cinema continuou apresentado sem som, para desagrado geral. O técnico ainda não atinou com o defeito do amplificador, explicam.

Há um clima de crescente excitação na turma brasileira, ante a iminência da chegada.

Nossos radinhos estão agora solicitadíssimos à noite, quando já se pode também ouvir, em onda comum, as emissoras cariocas. Antes de deitar-me, assisto a passagem de outro grande navio, em sentido contrário, feericamente iluminado. Belo espetáculo, na escuridão da imensidade oceânica.

\section*{13 \adfflatleafright \ISODayName{1959-10-13}}
Anunciam para a noite de hoje o ``baile dos disfarces'', tal como o realizado na viagem de ida. O pessoal desde cedo mobilizou-se. O navio facilita tudo, pondo rolos de papel crepom, tintas, colas e outros recursos à disposição de quem quiser. Vamos ter, na certa, outra demonstração de engenho e arte.

Com o tempo ainda fabuloso, faço o programa habitual da piscina. Saí do nascente outono europeu para a primavera tropical!

As estações de rádio do litoral vão nos dando ideia de nossa posição na costa. Isso indiretamente, porque, como disse, a localização precisa do navio, às 12 horas de cada dia, é indicada numa carta afixada no corredor da coberta, que ainda acrescenta distâncias e velocidade horária. Hoje, ao meio-dia, estávamos precisamente na altura da baía Todos os Santos. A distância diária percorrida é, em média, de 465 milhas (837 kms., aproximadamente).

Convém fazer algumas considerações sobre este magnífico barco, irmão gêmeo do Cabo San Vicente\index{Cabo San Vicente}, de que se pode orgulhar a marinha mercante espanhola. As instalações são verdadeiramente primorosas, mercê do ar-condicionado em todas as dependências. A cabine é muito confortável, com camas maciíssimas. Como se dorme bem ao embalo do movimento suave do barco! Os \textit{aseos} e os banheiros são também esplêndidos, os últimos com chuveiros espetaculares.

O desodorante que borrifam nos corredores e outras dependências, tresandando a limão, dá agradável sensação de limpeza. O único reparo prende-se à comida. Apesar de farta, não é muito variada, com abuso do peixe e do sorvete (``helado'') e das ``manzanas'' nas sobremesas. Os temperos, via de regra, não nos agradam. Em compensação, o pessoal de serviço, mormente do refeitório (``comedor''), é de gentileza e simpatia cativantes. O garçom de nossa mesa, Flávio, abusa do direito de ser atencioso e prestativo. Empurra sobremesa dobrada, mesmo quando a primeira já desceu a contragosto.

Voltando ao dia de hoje: durante longo tempo, no salão, fizemos ouvir a gravação da ``festa del paso de ecuador''; e de improviso, no momento, agregamos à fita outros números. É um bom passatempo esse.

Mas o programa marcante foi o baile de fantasias. Brilharam os participantes --- espanhóis, brasileiros, argentinos, chilenos e libaneses. Um grupo de estudantes chilenas, muito interessantes, mas demasiadamente recatadas, afinal deu um ar de sua graça. Não participaram dos ``disfarces'', mas andaram entoando umas canções típicas de seu país, com acompanhamento ao violão por uma delas.

Os patrícios da fuzarca apareceram de piratas e foi um sucesso. Principalmente o Paulo Tone, que é nipo-brasileiro e, está visto, ostentou visíveis indícios de chefe do bando. Por todo o conjunto, o espetáculo foi muito animado e divertido. Não resta dúvida de que a empresa do navio --- Ybarra y Cia. --- se esforça muito para tornar a travessia bem agradável.

Mas não era possível ir dormir sem uma ``sessão'' de rádio lá no ``deck''; noticiário, músicas, notas esportivas, enfim tudo que pudesse me reidentificar com os ``grandes assuntos nacionais''.

\section*{14 \adfflatleafright \ISODayName{1959-10-14}}
Enquanto espaireço das lidas da deliciosa piscina, fico a ouvir rádio: estações capixabas. Nunca soube que Vitória tivesse três emissoras.

O assunto predominante é a hora da chegada ao Rio amanhã. Pensei não haver dúvida de que seria já com o dia amanhecido, para todos sentirem o deslumbramento da Guanabara penetrada desde fora da barra. Todavia, correm rumores de que o barco recebeu ordem de ``pisar'', para chegar o mais cedo possível; consta haver fila na baía para encostar e, assim, quanto maior antecedência na entrada, mais garantia de não atrasar a escala, pois o navio se destina a Buenos Aires e ainda aportará em Santos.

Aliás, o aumento da velocidade está bem sentido. Ele está jogando um pouco mais, em consequência.

Começo a arrumar as coisas, depois do almoço, o que, na verdade, não toma muito tempo. Para levar para casa pode ir tudo de qualquer maneira.

Permutamos endereços com toda a turma brasileira. Do Rio somos apenas três: eu, Alcino e Newton, que passou 10 meses na Europa, por conta de bolsas de estudos. Há dois dias ele perdeu o anel de formatura, deixado no camarote, e está desolado. Não sabe o que fazer, nem mesmo se pode desconfiar do camareiro.

As vitrolinhas super-portáteis, de fabricação inglesa, compradas em Tenerife\index{Tenerife}, estão proliferando a bordo. Os que as viram nas lojas, não resistiram à tentação de trazer uma, que foi o que aconteceu também comigo e com o Alcino.

Fui esticar as pernas numa boa sombrinha lá na coberta da popa. Dia radioso! De olhar perdido na perspectiva azulmarinho do oceano, fico a evocar as passagens mais marcantes deste meu portentoso giro. Relembro os grandes momentos e as indeléveis emoções que experimentei. As passagens maravilhosas dos contatos com os esperantistas\index{esperantistas} em Viena\index{Viena} e, ainda na capital austríaca, a atmosfera do Festival da Juventude\index{Juventude!Festival}: a confraternização, no parque do Palácio ``Schöbruner'', com os suecos, madagasquenhos e ceiloneses; aquele japonês, no bar ``Rondo'', que, exclamando ``Brasil'', veio para a minha mesa me abraçar, empurrar um gole horrendo de seu copo de cerveja e me pregar na gola o distintivo do monumento de Hiroshima; os iugoslavos, nas ruas de Viena\index{Viena}, entoando o vibrante hino do festival no seu idioma. Aquela mesma melodia que, em português, ensaiamos a bordo:

\begin{verse}
\emph{Jovens do vasto mundo,\\
Nós vivemos com o sonho da paz,\\
Nestes anos terríveis\\
Nossa luta unidos nos traz\\
Pelos vales e montes,\\
Devastando horizontes,\\
Jovens do mundo,\\
Juntos marchemos,\\
Em legiões formemos!}
\end{verse}

O Congresso de Varsóvia\index{Varsóvia}! Três mil e tantos esperantistas\index{esperantistas} de 60 países diferentes, conversando em família, sem necessidade de intérpretes. A ``amostra-mirim'' do milagre do esperanto naquela tarde inesquecível na casa de Stanislaw, para também ouvir os discos brasileiros que levei.

A visita a Moscou\index{Moscou}, quase um sonho louco. A atmosfera tranquila da grande cidade, tão acolhedora como qualquer outra. Os vultos impressionantes de Lenin e Stálin, no soturno mausoléu. As meninas da Bulgária\index{Bulgária} e do Cazaquistão, pedindo autógrafos e me cercando como se eu fosse um marciano recém desembarcado.

As imagens se fundem sem sincronia, como \textit{trailers} de vários filmes, correndo sem ordem. Aquela estação na Tchecoslováquia\index{Tchecoslováquia}, tarde da noite, com cervejadas em profusão (a melhor ``pilsen'' do mundo). Berlim\index{Berlim}, a surpresa extraordinária. Os dias agradáveis com a antiga correspondente, afinal conhecida pessoalmente, na Bélgica\index{Belgica@Bélgica}; as festinhas íntimas com aquela boa gente. E a forte emoção naquele clube interiorano, entre Mechelen e Bruxelas\index{Bruxelas}, quando o conjunto musical inesperadamente ``atacou'' a ``Aquarela do Brasil'', só para homenagear dois desconhecidos brasileiros ali presentes!

O esplendor de Zurich\index{Zurich}, a magia de Paris\index{Milão} recepcionando o presidente Eisenhower\index{Eisenhower}, o Vaticano\index{Vaticano}, a lendária Torre de Pisa\index{Pisa!Torre}. A pavorosa tourada em Madrid\index{Madrid!tourada}, a paisagem tropicalizada das Rivieras italiana e francesa, o monagasco a xingar seu príncipe de fancaria!

Do fundo de todas as recordações, ainda tão frescas, ressalta a satisfação inigualável de sentir o Brasil lá fora já bem conhecido, admirado e até invejado. Para isso é indiscutível que muito concorreu a vitória na Copa do Mundo de futebol, no ano passado, na Suécia. A frase da irlandesa, de sorriso ingênuo, em Roma\index{Roma}: ``Your contry is famous!'' E da senhora inglesa no hotel de Bruxelas\index{Bruxelas}: ``Are you brazilians? How exciting!''

A recordação pula de novo para Moscou\index{Moscou}: o rapazola de boné, que me ensinou o caminho para a estação ``Konsomolskaia'', quando eu estava perdido no interior do fantástico metrô moscovita e que, depois, escapuliu lépido e silencioso, para ficar olhando de longe, no alto de uma passarela, como um animalzinho qualquer dos desenhos de Walt Disney. O outro popular, no mesmo metrô, que nos acompanhou até ser interpelado pela nossa cicerone e então esclarecer que queria saber se éramos jogadores de futebol (o dístico ``Brasil'' estava no nosso peito). E a resposta de Laurish\index{Laurish}, a jovem intérprete, depois traduzida para nós: ``Eles são muito mais que isso!''

Dona Bárbara, do Palácio da Juventude\index{Juventude!Varsóvia} de Varsóvia\index{Varsóvia}! Jamais vi alguém capaz de tanto empenho e prestatividade para agradar dois estrangeiros desconhecidos.

Minhas divagações são interrompidas por Vera Lúcia e Magali, as primas, que chegam trazendo a notícia de última hora: entraremos na baía de Guanabara às 3 horas. Lamentamos. Que espetáculo esta estrangeirada vai perder! Mas a explicação para a pressa é aceitável.

O bloco brasileiro dispõe-se a não dormir e, sim, aguardar, palmo a palmo, a chegada. As meninas recolhem contribuições para uma mesa com champanhe, no momento mais festivo.

O jantar foi bastante mais caprichado, na forma do costume, como despedida aos que ``se quedarán en Rio''. A excitação é gradativamente contagiante, pois os ``neófitos'' europeus mostram-se ansiosos para ``mirar la cuidad maravillosa''.

O alto-falante de bordo chamou pelo nome os que desembarcarão no Rio, para as providências de alfândega e passaporte.

Não houve cinema à noite e o baile terminou mais cedo. O San Roque puxa furiosamente seus 22 nós horários.

Reunimo-nos na lateral da coberta, do lado da costa. Dali a pouco um farol longínquo é avistado: Cabo Frio, sem nenhuma dúvida. A euforia induz a antecipar a ``abertura'' da mesa. Convidamos as argentinas, que estavam gostando de assistir o nosso programa, a participar dele. Uma delas, morena perfeita, poderia ser legítima \textit{miss} de seu país nos concursos internacionais.

Havia me esquecido de citar outro brasileiro presente: Ely, que passou 12 anos no Líbano e agora retorna. Também saltará no Rio. Somos, então, quatro a ficar. Os outros 21 são paulistas.

\section*{15 \adfflatleafright \ISODayName{1959-10-15}}
As comemorações entraram no tempo de batucada.

Cerca de duas horas da manhã, alguém vislumbra um clarão lá longe, do lado do continente. Rio a vista! --- gritam. Mas era Niterói, logicamente, que está do lado de cá.

A algazarra duplica. Eu, Alcino, Newton e Ely entoamos ``Cidade Maravilhosa'' (ainda bem que eu sabia a letra dos dois versos). Como o tema não é muito do agrado dos paulistas, percebe-se logo, emendamos, em homenagem a eles, com a ``Cidade do Arranha-Céu'', com acompanhamento geral, obviamente.

Um grupo, achando pouco, pede silêncio e entrega um bastão improvisado para Nacife, a pianista, reger nada menos que o Hino Nacional. A emoção atingiu o auge! Poucos olhos deixaram de ficar marejados.

Como é bom voltar para casa! Para o nosso meio, para o convívio com os nossos familiares, os nossos amigos e as nossas coisas.

Não só os brasileiros estão acordados. Centenas de outros passageiros, ante o brado de ``Rio à vista!'', ganharam as amuradas e soltam exclamações de encantamento ante a paisagem fascinante da entrada da barra, mesmo àquela hora da noite. O barco navega agora lentamente. A noite não está muito clara, o céu mostra-se semitransparente, mas todo o conjunto é maravilhoso. O monumento do Redentor produz funda impressão nos estrangeiros, que o vêm pela primeira vez; realmente, à noite sua magnificência é mais entusiasmante.

Eram 2h30m da manhã.

Retomamos as nossas comemorações musicais até quase às 4 horas. Um aviso de bordo pede aos que vão desembarcar o comparecimento às 6 horas, no salão superior, para as formalidades com a saúde pública e polícia marítima. Resolvo tirar uma soneca durante essa hora e meia.

As providências regulamentares foram rápidas. Enquanto o navio começava a encostar, no armazém 1, às 7 horas, corro a tomar café, para me reanimar um pouco.

Informaram-me, na portaria de bordo (``Contaduria''), que o aviso divulgado pelos agentes da companhia, no Rio, dava a chegada para as 8 horas. Com a antecipação, os familiares não deverão estar ainda presentes na beira do cais. Mas já podemos descer livremente à terra e adiantar o desembarque e desembaraço das bagagens.

Com surpresa, identifico no cais o colega Aristeu, que me encomendou a peça ``Citroën''. Ele havia me garantido que possuía conhecimentos, diretos ou indiretos, junto ao serviço de fiscalização da alfândega e que, com isso, tanto a encomenda dele como toda a minha bagagem iria ``passar'' sem qualquer problema. E isso realmente aconteceu, o que foi ótimo, embora eu não estivesse conduzindo nenhuma ``moamba'' ou excesso de quaisquer mercadorias.

O ponto final está pingado.

Há 101 dias atrás, nesta mesma plataforma, abraçava quase as mesmas pessoas - minha mãe e meus irmãos --- que agora me recebem com boas-vindas e felizes por eu ter voltado são e salvo! Outra surpresa maravilhosa foi deparar com o meu colega do IRB, engenheiro Mário Trindade, o amigo incrível que tomou conta do meu carro durante minha ausência e vem me entregar as chaves e indicar onde ele estava estacionado ali na praça!

Tempo haja para contar impressões e passagens de tudo que vi e vivi.

A começar pela própria história deste diário, que não estava programado, mas que nasceu de uma conversa com novos amigos no primeiro dia de viagem, quando ficou evidenciado que seria uma maneira original de gastar o excesso de tempo disponível, em 24 horas passadas na imensidão do mar, durante 8 dias, até a primeira escala. Depois, ficou o gosto pela coisa e também o desejo de legar a ``obra'' para o futuro, até não sei quando, talvez para somente eu mesmo reler e relembrar as emoções únicas que experimentei\ldots.

Mas a impressão final já está consolidada em meu espírito: inegavelmente aqueles povos, séculos mais antigos que nós, estão bem na dianteira quanto aos problemas elementares do viver coletivo, como os serviços públicos, por exemplo; não há por lá falta d’água nem, praticamente, problema de transporte, de energia elétrica e outros detalhes corriqueiros que, infelizmente, ainda enfrentamos. Acrescente-se que sofreram nos últimos 45 anos, duas guerras mundiais, perfazendo dez anos de destruição e demais flagelos consequentes dessas desgraças.

Mas, a não ser na parte oriental, que desponta para uma grande experiência evolutiva, graças ao impulso racional de uma economia e educação socializadas, os ocidentais não conseguem ocultar uma certa apatia, como a demonstrar que não podem dar mais do que já deram.

Quanto às condições de qualidade de vida, nem se discute que nos trópicos, onde estamos nós, é tudo melhor e mais favorável a 12 meses de inteiro aproveitamento.

Os coitados têm 3 meses de verão e um de meias-estações (fim de primavera e começo de outono), variando um pouco conforme a posição geográfica. Durante o resto do tempo, salvo nas regiões mais ao sul, ficam entocados, a se protegerem da neve e da chuva, quando não da lama e dos ventos furiosos. Aliás, as agradáveis condições tropicais eles as provocam artificialmente nos interiores, com estufas e calefação.

Nossas perspectivas têm outra cor e um alcance tão profundo que, inexplicavelmente, ainda não se tornaram perceptíveis à maioria dos brasileiros e à totalidade dos governantes. Todos os nossos problemas --- todos, sem exceção! --- têm a solução na frente do nariz de qualquer um, por mais leigo que seja em qualquer assunto.

Somos um país que caminha sem ninguém empurrar e, ao contrário, tantos entravando. Imagine-se o que não sucederá, já em um amanhã bem próximo, quando energias nascentes e pujantes surgirem neste nosso cenário maravilhoso e inigualável!

Não troco o meu grande e belo país por nenhum daqueles treze que visitei. Como ressaltaram nossas infinitas vantagens na consideração de todos os aspectos!

Muito pequena é a vontade de tornar a sair, pelo menos por muito tempo.