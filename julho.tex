\addchap{Julho}
\section*{07 \adfflatleafright \ISODayName{1959-07-07}}

A data de hoje há de ficar marcada para sempre em minha vida,  pois estou dando partida a um sonho que, até poucos anos atrás, parecia fantástico demais, para não dizer quase impossível: embarque para a Europa, e de navio, como eu sempre desejei.

A passagem estava primeiramente reservada no transatlântico italiano ``Federico C'', mas uma greve da tripulação prejudicou meu programa. O navio foi deixado parado em um porto africano e o jeito foi escolher outro, para evitar atraso nas escalas do meu estudado roteiro.

O agente de viagem que me deu assistência sugeriu, então, a transferência da passagem para o navio espanhol ``Cabo San Vicente'', que estaria fazendo sua primeira viagem à América do Sul. A princípio fiquei meio desconfiado, mas quando ele me passou um catálogo com descrição e fotos de todas as dependências do barco, concordei na hora. Assim, o atraso com a troca do meio de transporte, em relação à data da partida, foi de apenas dois dias.

Eis o bruto atracado aqui no cais da Praça Mauá, todo branquinho, no meio de intenso movimento de entra-e-sai. Minha mãe e quase todos os meus irmãos vieram ao embarque e também subiram a bordo. O navio procede de Buenos Aires, que é o seu final de linha, e vai nos levar até Gênova, na Itália.

Vou viajar em companhia de um antigo colega dos tempos de ginásio, o Alcino Beiriz, mas que só fui reencontrar no movimento esperantista. Há umas duas semanas ele apareceu no meu trabalho e, dizendo ter ouvido falar de minha viagem, lá na nossa Cooperativa Cultural, decidiu embarcar nas mesmas águas, desde que eu concordasse, é claro.

Meu plano de viagem inclui, até mesmo como ponto principal, a participação no 44º Congresso Universal de Esperanto, a realizar-se em Varsóvia, na Polônia, nos dias 1 a 8 de agosto próximo. Alcino se diz também esperantista, mas tenho cá as minhas dúvidas quanto ao seu real conhecimento do idioma internacional, pois jamais ouvi uma palavra em esperanto por ele pronunciada.

Descansando com o meu pessoal no salão-de-estar do navio, vislumbrei a um canto a conhecida vedete Rose Rondelli em conversa coloquial com Sérgio Porto, o apreciado cronista ``Stanislaw Ponte Preta'', do jornal ``Última Hora''. Mais tarde, ele desapareceu e ela, ao que tudo indica, será passageira.

A partida estava marcada para as 23 horas, mas sofreu enorme atraso. Logicamente, dispensei meus familiares muito antes. Consta que a causa da demora foi um interminável carregamento de café. De fato, vimos os guindastes despejando montes de sacos nos porões do navio.

Finalmente, beirando uma da manhã, o navio se pôs em movimento. A saída da Guanabara, naquela hora, mesmo assim nos permitiu um silencioso e emocionado ``Adeus, Brasil!''

O contato com os companheiros de viagem já se fizera facilitado desde os primeiros momentos da chegada ao cais. Isso porque lá encontramos, embarcada desde Santos, uma delegação brasileira ao Festival Mundial da Juventude, a realizar-se no fim do mês em Viena, na Áustria. Naquelas longas horas de espera, pude verificar que não são poucos, parece que em torno de cem pessoas, esbanjando mocidade, beleza e simpatia.

A primeira refeição, o jantar, por volta das 19 horas, com o navio ainda atracado, deixou logo agradável impressão, principalmente pela amplitude, conforto e beleza do salão. E, também, pelo enorme jarro de vinho tinto colocado na mesa, embora não tocado por mim\ldots Livrei-me, há pouco mais de dois anos, de uma manifestação alérgica de fundo presumivelmente alimentar e, por isso, decidi ficar, por algum tempo, longe de bebidas alcoólicas.

Fomos dormir às duas da manhã, já explicavelmente cansados, ante as peripécias do embarque. Além do antigo colega, travei melhor contato com os outros dois companheiros de cabine, a de nº 201: os semi-portugueses Srs. Noronha e Seixas, ambos já de meia idade e muito simples, gentis e camaradões.

\section*{8 \adfflatleafright \ISODayName{1959-07-08}}
A noite fez-se curta, mas foi bem dormida, pudera! Após uma breve acomodação ao balanço do barco, a cama confortável ``puxou'' rapidamente o sono. Ao amanhecer, entretanto, a disposição já era grande para o que desse e viesse.

O relógio precisou ser adiantado em 23 minutos, para ir se ajustando à hora legal. Depois do café, uma visita à coberta, para sentir a primeira sensação do mar alto, à luz do dia. O tempo ainda não estava muito firme, mas os prenúncios eram animadores e logo em seguida se confirmaram. A sensação, de fato, foi verificar a cor do mar, já longe da costa: azul muito carregado, ou melhor, azul-marinho\ldots

Outra interessante nova: pela fresta de porta do camarote distribuíram-nos, bem cedinho, o jornalzinho de bordo, de nome ``Maripez''. Bem impresso, com seis páginas, em formato tablóide e cheio de informações sobre o programa do dia e matéria ilustrativa geral. Belo e surpreendente trabalho.

De manhã dei um bruto ``fora'': entrei com o Alcino em um reservado de senhoras. Só percebemos na saída\ldots

Boas relações já fizemos com a turma do Festival da Juventude. Após o almoço, batemos longo papo com três paulistas (mãe, filha e sobrinha) e três dos vinte e tantos engenheiros espanhóis que estagiaram rapidamente no Brasil e estão de regresso. Começou o problema da língua: há que se falar espanhol.

Às 15 horas, como anunciado, um ensaio de salvamento em caso de emergência. Vestimos os salvas-vidas (coletes berrantemente abóbora) e seguimos as instruções afixadas atrás da porta da cabine: dirigir-se a um determinado local, quando soasse a sirene de alarme. O nosso ponto de deslocamento era o ``comedor'' (refeitório).

O cafezinho do bar ganhou o apelido de ``café milionário'' pois custa 4 pesetas (10 cruzeiros). É do tipo italiano, ``expresso'', mas a pequena xícara nem cheia vem\ldots

Afinal, este primeiro dia de viagem resultou bem proveitoso para a propaganda do esperanto. Já ficamos bem identificados como congressistas rumo a Varsóvia e isso tem despertado curiosidade e interesse por parte dos patrícios que se destinam ao Festival da Juventude. À tardinha, as três paulistas revelaram aparente disposição de conhecer maiores detalhes sobre o idioma internacional, notadamente a senhora, Dª Hilda, que pediu algum material que pudéssemos fornecer.

No jantar, sentaram-se à nossa mesa dois elementos da delegação ao Festival. Ao meu lado, o jornalista alagoano Jaime Miranda, olhando fixamente meu pequeno distintivo de estrela verde na lapela (insígnia do esperanto), criva-me de perguntas a respeito. Esclareço tudo como posso e ele também se foi, declarando que gostaria de conhecer melhor o assunto.

Às 21 horas, anima-se a festa no salão. A orquestra, no entanto, abusa dos ``paso-dobles'', gênero predileto dos espanhóis, mas assim mesmo dá para dançar. Não estreei ainda. Há excesso de cavalheiros, parece. Passamos depois ao cinema, no convés. Filme ``O Trem das 3,10'', com Glenn Ford e Van Heflin. O barulho da chaminé do navio perturba um pouco o som. A dublagem é em espanhol e impressionantemente perfeita. Mas naquelas condições perde-se muita coisa.

Quase às 24 horas, adiantamos o relógio mais 23 minutos, como se deverá fazer diariamente até o desembarque final.

\section*{09 \adfflatleafright \ISODayName{1959-07-09}}

Noite de profundo e repousante sono. Não encontro de manhã minha escova de dentes, que estava enfiada em um copo. O companheiro Seixas volta do banho e diz que, por engano, a usou. Paciência\ldots

O diário de bordo, o ``Maripez'', é entregue cedinho.

Esqueci-me de dizer que o jantar de ontem esteve magnífico: peru e champagne!

Pretendo estrear a piscina daqui a pouco. É pequena, mas dá para se divertir. O tempo segue maravilhoso. Limpo e de mar incrivelmente azul-marinho\ldots

Às 15 horas fizeram um bingo no salão. Não entrei porque caía de sono e fui tirar uma soneca, mesmo contra os meus hábitos. Dizem que o mar traz sonolência, no início, e parece que é verdade.

Um mapa nas laterais do passadiço mostra a posição do navio diariamente, ao meio-dia. Estamos na costa da Bahia, a 636 milhas do Rio e a 19 nós horários de velocidade.

Um nó corresponde a uma milha marítima, ou seja 1.800 metros, aproximadamente.

Hoje, 9 de julho, é a data nacional argentina. A sua delegação ao Festival de Viena é também numerosa, fora outros cidadãos em viagem com diferentes destinos. Às 18 horas reuniram-se no salão e cantaram o seu hino nacional. Eu não estava lá, soube pelos alto-falantes. Um ``drink'' foi oferecido aos presentes.

O chefe da delegação brasileira, radicado em São Paulo e naturalizado brasileiro, é espanhol. Soube que é proprietário do salão ``Antoine'', famoso cabeleireiro de senhoras da capital paulista. É, ainda, pai do laureado cestabolista Amauri, com quem aliás muito se parece.

Uma volta pela popa, às 19 horas, antes do jantar. Tempo estupendo. À esquerda (costa), a meia altura, um quadro curioso no céu: certa escuridão e uma fatia de lua brilhante, bem junto a fulgurante estrela, talvez a Dalva. Diz o Alcino, com muita propriedade: a bandeira da Turquia\ldots

Na direção da popa, bem visível, o Cruzeiro do Sul. Não há dúvida, estamos seguindo para o norte.

Após o jantar, mais festejo da data argentina. Um ``show'', no salão cheíssimo. Patrícios (deles) vieram da primeira classe para assistir. Recital de piano, excelente, pela mais brilhante executora de bordo, a brasileira Dona Clarisse. Depois, a bonita Nely, que deve ir representar nosso folclore em Viena, cantou ``Granadina'' e a toada gaúcha ``Prenda Minha''. A surpresa maior veio em seguida. O trio de rapazolas intitulado, parece-me, ``Maraiá'', vestido a caráter (chapéu de palha, camisa listrada verde-branco horizontalmente, calça clara) abafa sensacionalmente. Cantam cinco números. Vão desacatar no festival, sem nenhuma dúvida.

A bordo, mais tarde, um deles diz que são do Rio Grande do Norte e estiveram na TV Tupi do Rio e na de São Paulo. Sugiro-lhe que inclua em seu repertório, por muito se identificar com o estilo deles, a toada ``Vendedor de Caranguejo'' e ofereço-me para ensinar a letra.

Depois, finda o ``show''. Novamente o hino argentino, que termina falando em ``augurio murir com gloria''. Em seguida, o hino nacional brasileiro, após palavras inflamadas de agradecimento do representante argentino, uma grave figura.

Breve chuvisco atrasou um pouco o cinema. Filme: ``Mi Hermana Elena'' colorido, com Janet Leigh e Jack Lemmon. Refilmagem de velha comédia, ao que me lembre exibida no Brasil com o título de ``Jejum de Amor'', com Rosalind Russel e Janet Blair. História de duas irmãs pobres do interior que vêm para Nova Iorque morar em meio-porão devassadíssimo.

\section*{10 \adfflatleafright \ISODayName{1959-07-10}}

Manhã esplendorosa, como todas até agora. Escrevo sobre a perna, porque os salões só abrem às 10 horas. Estou na ``cubierta de paseo'', no convés, na parte mais traseira. Vou descer para voltar de short à piscina, que fica aqui junto mesmo.

Os companheiros de cabine são formidáveis. Discretos e educados. Dormem como anjos, sem o menor ruído. Hoje de manhã todos reclamamos contra a água dos chuveiros, que só estava saindo quente, quase fervendo.

Mais tarde, no salão, descubro que o pai do Amauri, o ``Antoine'', é de nacionalidade argentina e não espanhola.

O marido da cantora Nely, Sr. Armando Angelini, distribuiu o jornalzinho do Festival. Marcaram ensaio do hino deste, no salão. Participo da cantoria, que é vibrante. Em seguida, corre um páreo de ``caballos'', jogados com grandes dados e com apostas (25 pesetas) e tudo. É jogado no salão, em tapete próprio, marcado.

Para a noite está prevista a eleição de ``los reyes del ecuador''. Estamos, agora à tarde, nas costas de Alagoas/Pernambuco. O cruzamento do equador dar-se-á amanhã, conforme anunciam.

O cinema, após o jantar: ``La Venganza'', produção espanhola com o ator italiano Raf Valone e Carmen Sevilla. Teve prêmio em Cannes, no festival ao ano passado. Mas é um monótono drama de colhedores de trigo. Salva-se pelo colorido e alguns cenários.

No salão elegem a rainha do equador, uma argentina muito jovem, de pequena estatura. Em seguida o rei Netuno, o brasileiro Simão, movimentada figura da delegação ao Festival. Na mesma hora recebem faixas, coroas e, o rei, longas barbas e cabeleira. Dançam sozinhos a ``valsa da coroação''.

Estão programadas outras coisas para amanhã, às 10 horas na piscina.

\section*{11 \adfflatleafright \ISODayName{1959-07-11}}

Quarto dia de viagem. O diário de bordo, logo cedo, anuncia que cruzaremos o ``ecuador'' aproximadamente às 18 horas. E faz recomendações para que se evite excesso de exposição ao sol, pois a radiação é mais forte.

Marcam para as 10 horas o ``bautismo de los neófitos'', isto é, os que estão passando pela região pela primeira vez e que, nesta viagem, pelas conversas, são numerosíssimos --- maioria do pessoal do festival e muitos outros, inclusive nós dois\ldots

Fui convidado pela comissão de cultura da delegação brasileira ao Festival da Juventude para fazer uma palestra sobre o esperanto, na segunda-feira, às 16 horas. O jornalista Jaime Miranda já havia tocado no assunto antes e agora confirma. As palestras começam amanhã.

Netuno e sua majestática companheira comparecem pomposamente à beira da piscina, para os ``bautismos''. A chamada é feita nominalmente. Soube que só estão chamando os previamente inscritos na véspera, à hora das eleições dos reis, quando estivemos ausentes. Assim escapamos do vexame: um por um, ou uma por uma, frente às majestades, para umas boas pipocadas, na cabeça, de uma mistura branca, parecendo cuscuz, mas vim a saber depois ser sabão com sal. Em seguida, dois ou mais serviçais agarram o batizado e o atiram na piscina.

Dezenas e dezenas de ``neófitos'' cumpriram o ritual. Lá pelas tantas, os circunstantes foram aos reis e\ldots o banho foi geral. Depois a turma gostou do lance e durante mais de meia hora o programa consistiu no lançamento à piscina de pessoas vestidas, com cadeira junto, etc., etc., Felizmente o bom humor prevaleceu sempre. E felizmente também não se lembraram de mim, nem do Alcino e escapamos ilesos. Para isso concorreu naturalmente nossas freqüentes entradas espontâneas na água\ldots

Fotografias hoje. Fiz minhas primeiras O navio subitamente parou, por uns 15 minutos. Algum problema de máquina, talvez. Pouquíssima gente percebeu. Estávamos, porém na popa, fazendo a terceira catequese para o esperanto: Jeanete, jovem e gordinha, com sua mãe polonesa, em viagem para Israel. Como vamos estar na Polônia, ela se interessa. Naquela posição foi fácil perceber a parada do barco, pois desapareceu a esteira branca deixada por ele quando em movimento.

O ensaio do hino da juventude, pelos festivalistas, foi muito animado, pois o cântico é vibrante. A música é uma só, naturalmente, e cada país canta a versão no seu idioma. Participamos do ensaio gostosamente.

Na hora do jantar do primeiro turno (18 horas), que não é o nosso, passando pela porta do refeitório, quase caí para trás: todo o pessoal nas mesas --- e note-se que o salão é enorme --- ostentava fantasiais na cabeça, com chapéus os mais jocosos e extravagantes. Até o capelão envergava seu pitoresco gorro vermelho (!) de papel crepom. Como surpresa, nada mais hilariante.

Na nossa vez, encontramos os chapéus sobre os pratos virados. Eu e o Alcino ganhamos ``elegantes'' bonés, de imensas palas. Valeria a pena fotografar\ldots

Mais tarde, durante o baile com o pessoal fantasiado (pelo menos na cabeça) e o salão lindamente decorado, foi feita a entrega dos ``diplomas'' de travessia do equador aos neófitos.

Esta travessia aconteceu às 18h50m de hoje, quando o navio apitou repetidamente, em saudação. Assim, desde aquela hora já estamos no hemisfério norte.

\section*{12 \adfflatleafright \ISODayName{1959-07-12}}

O dia não amanheceu tão bonito.

Mas está ainda firme e promete melhorar bastante. Há ainda recomendações no jornalzinho ``Maripez'' para continuar o cuidado com o sol equatorial, muito intenso realmente. Mas o vento permanente da coberta alivia sensivelmente, pelo menos para mim.

Ao café, soubemos de um pequeno ``entrevero'' havido ontem à noite no baile. Fora servido um ponche com sanduíches, cerca das 23 horas. A turma do festival, ou melhor, alguns elementos já embalados pelas comemorações de Netuno, parece que andaram se excedendo.

Escrevo do convés --- ``cubierta de paseo'' --- ao lado da piscina. O balanço embora pequeno, combinado com o apoio precário sobre o joelho, sacrifica um pouco a caligrafia.

Aqui o barulho do navio se assemelha ao de um avião. É a pressão da fumaça da chaminé contra o vento ou a pressão atmosférica, segundo creio.

Hoje é o nosso primeiro domingo a bordo. Após o almoço, Creuza, da delegação do festival, faz uma palestra sobre este conclave, suas origens e objetivos. Explica que eles se realizam de dois em dois anos e sempre nos países da chamada ``Cortina de Ferro'', mas o deste ano, em Viena, será a primeira exceção. Muita gente na assistência. Depois, jogamos bingo, a 5 pesetas, no salão.

Seguimos nosso quinto dia de viagem, sem qualquer panorama externo senão o grande oceano, como sempre lindo e como sempre azul-marinho. No primeiro dia, cruzamos com um navio e passamos mais tarde por outro, ambos pequenos.

À noite, voltou o cinema: ``El Hombre de Laramie'', um faroeste com James Stewart. Já o havia visto há algum tempo no Rio, não sei sob que título. Muito boa dublagem, como sempre.

Na lateral da coberta superior, um dos rapazes do Trio Maraiá, o Hirton, ao violão, canta muito bem um sem número de canções, velhas e novas, numa rodinha muito simpática. Soam as 24 horas. Adianto o relógio mais uma vez em 23 minutos e vou dormir, pensando no jogo de hoje, lá no Rio entre Flamengo e Vasco, cujo resultado não sei quando descobrirei.

\section*{13 \adfflatleafright \ISODayName{1959-07-13}}

Como sempre, pela manhã, a piscina foi o ponto alto. Alcino, que não havia caído n’água desde o embarque, hoje se animou. O jornal de bordo marca para amanhã um baile de disfarces. Deverá ser curioso, pelos tipos que há a bordo. O navio fornece fantasias e outros apetrechos.

Às 12 horas o barco já se situava mais perto da costa africana. Passaremos, alta noite de hoje, pelas Ilhas Cabo Verde, pertencentes a Portugal. A chegada a Tenerife, nas ilhas Canárias, está marcada para depois de amanhã, ao anoitecer.

Escrevo no salão, em meio a certo tumulto. Aguarda-se a segunda palestra da série cultural da delegação do festival: médico brasileiro falará sobre o câncer. Talvez a minha seja a próxima, embora a data anteriormente marcada tivesse que ser revista.

Ontem de manhã lavrei um tento: conversando com duas argentinas e uma chilena, não se deram conta de ser eu brasileiro, a não ser quando eu disse! ``Tan bien hablas el castellano'', disseram-me. Imaginem\ldots

Descobri o significado da marca V e A, com as letras entrelaçadas, que formam a insígnia da companhia do navio, a ``Ybarra y Cia.'': ``vascas y andaluzas'', marca da marinha mercante espanhola, não sei de que época.

Observo o quanto trabalha o pessoal de bordo. Nos salões onde servem o café, após as refeições, apenas dois homens atendem, circulando por toda a imensa área. Nas outras horas, esses mesmos dois atendem a quaisquer encomendas, do bar ou não. Indago a um deles sobre o seu horário de trabalho. ``Das 8 às 24 horas!'' --- responde. E de manhã têm que estar a postos para ``limpiar'' as acomodações. Os músicos da orquestra, que são quatro, dão também sua mãozinha em outros serviços durante o dia. Já os vi limpando cinzeiros.

A palestra sobre o câncer alcançou grande interesse. O conferencista, Dr. David Herlich, do Instituto Central do Câncer, de São Paulo, mostrou competência e entusiasmo pela sua especialidade. Pertence também à delegação do Brasil ao festival de Viena e, em seguida, visitará os países socialistas, para maiores estudos.

Após o jantar, vejo no quadro a confirmação de minha exposição sobre o esperanto para depois de amanhã, quarta-feira, dia 15. Amanhã, um deputado estadual paulista, também da delegação ao festival, falará sobre questões de ensino.

O filme desta noite chamou-se ``Sinfonia en Oro''. É a história de um patinador que ficou incógnito como um certo ``Mister X''. Filme alemão, muito desinteressante, exibido no Rio há uns dois meses. Saí no meio e vim para o camarote alinhavar estas notas e dormir mais cedo.

Não sei se já falei sobre a excelência de nossas camas, com beliche e tudo. Espuma de borracha gostosíssima. E com ar condicionado no camarote, como, de resto, em todo o navio.

E assim se encerra nosso sexto dia de viagem. Já se fala em atraso de um dia, até Gênova, devido às escalas. Tudo ótimo!


\section*{14 \adfflatleafright \ISODayName{1959-07-14}}

É impressionante a calma do mar, desde a saída do Rio. Viajantes habituais, como o simpático português de nossa mesa de refeições, Sr. Guilherme, afirmam que nunca viram fato igual.

Às 13 horas, fomos ver a posição do navio no mapa. Passamos pelo arquipélago Cabo Verde durante a madrugada. Nada se viu. Anunciam que chegaremos a Tenerife amanhã, nas primeiras horas do dia. Já é mais freqüente o encontro com outros navios. Ontem, um pequeno pesqueiro foi avistado e identificado como do Senegal. Hoje, após o almoço, outro barco, este de maiores proporções e também de passageiros, navegava na mesma direção que a nossa. Horas depois sumia na imensidão do oceano.

A palestra, à tarde, foi também muito boa. O deputado estadual paulista, professor Solon, ocupou-se do tema educação. Finda a conferência, muitos debates, perguntas e respostas do palestrante. Achei, apenas, o conteúdo um pouco elementar. Lembre-se, porém, que a maioria do auditório era de estrangeiros.

Ao ser apresentado ao orador, para dar início à palestra, o jovem membro da comissão de cultura referiu-se à data de hoje, 14 de julho. E com boa ênfase exaltou-a, falando, inclusive, em queda de opressão, referindo-se, naturalmente, à data marcante da revolução francesa. Assunto delicado, pois nos encontramos neste barco, que é ``território'' da Espanha franquista\ldots É como se falar em corda em casa de enforcado.

O baile à fantasia excedeu a expectativa. Os recursos não eram muitos: cartão, papel crepom, tinta e cola. No entanto, surgiram ``disfarçados'' notáveis. Premiados: o rei Creso, que foi o marido da cantora Nely Angelini, e a ``Rainha do Egito'', a mesma graciosa argentina que já fora eleita rainha do equador.

Antes um pouco, membros da comissão de cultura chamaram-me para trocar idéias e planejar minha palestra de amanhã. Querem evitar que eu aborde o sentido universalista de fraternidade do esperanto, que pode cheirar a ``coisa'' e desagradar os franquistas.

\section*{15 \adfflatleafright \ISODayName{1959-07-10}}

Oito dias ao mar sem sequer avistar a terra de uma ilhota. Na próxima madrugada, finalmente, pisaremos em Tenerife.

A manhã está encoberta, sem sol, mas firme. Faltei à piscina hoje, antes do almoço, para esboçar minha palestra, que, já observei, está despertando certo interesse, pelos comentários de conhecidos. De fato, a propaganda está sendo maior (não de nossa parte) que a das anteriores. Talvez seja pelo assunto. Além do anúncio escrito no salão, o alto-falante de bordo repetiu o convite na hora do almoço. Poucos momentos antes da hora marcada, 15 horas, o Alcino estendeu no salão a bandeira e flâmulas do esperanto, que fazem parte de sua bagagem.

A afluência foi enorme, na verdade maior do que nas demais, e a sugestão de véspera, do simpático Dr. Maurício, da comissão de cultura, proporcionou magnífico efeito: convidou-se representante de cada nacionalidade presente a bordo para compor a mesa; havia portugueses, espanhóis, chilenos, argentinos, italianos (o vice-cônsul de Campinas), suíços, franceses, húngaros, sírios, libaneses, um belga, uma polonesa (mãe de Jeanete) e mais descendentes diretos que representavam Japão, Israel e União Soviética.

Cada um deles, ao chegar à mesa, cumprimentava-me dizendo ``boa tarde'' no seu idioma.

Pegando esse ``gancho'' da multiplicidade de línguas, passei em seguida a falar sobre ``A realidade do esperanto como língua internacional'' --- o tema oficial da palestra. Inicio desfazendo o clássico equívoco de que se esteja pretendendo suprimir todos os idiomas nacionais e substituí-los por um único; apresento um breve esboço histórico, uma demonstração do mecanismo da língua, sua extrema facilidade e a atual situação de penetração em todo o mundo. Recordo o caso da trabalhosa e demorada implantação do sistema métrico francês (também um projeto de unificação), que levou mais de duzentos anos para ser finalmente aceito. E, para não ultrapassar os 40 minutos programados, finalizo, para que a língua fosse ouvida, declamando a versão em esperanto que fiz do hino do festival da juventude, cantado quase todo dia a bordo.

O sucesso, modéstia à parte, foi estrondoso. Havia deixado a máquina fotográfica já preparada com o Alcino, para fixar a cena, com vista a uma futura publicação na revista da nossa Cooperativa Cultural.

Na forma usual, pus-me à disposição dos presentes para esclarecimentos e debates. Nas palestras anteriores, grande parte da assistência já se retirava nessas ocasiões. Surpreendentemente, hoje, ninguém se afastou e o salão permaneceu lotado por mais 30 minutos de informações e respostas, aproveitados por mim, então, para abordar aspectos que, por limitação do tempo inicial, haviam ficado sem referência.

Ao encerrar, um grande grupo se acercou da mesa e levou todo o material de propaganda e folhetos disponíveis. E me arrancam a promessa de, ainda a bordo, até o fim da viagem, ministrar algumas aulas elementares. O chefe da delegação ao festival, o animado Antoine, fala no encerramento, exaltando o assunto objeto da palestra e cita, a propósito, sua experiência de dois anos atrás, em Moscou, quando durante 45 minutos conseguiu ``não entender'' nada do que o motorista de um taxi, com quase obstinação, lhe desejava transmitir. A barreira das línguas!

Não há dúvida de que fizemos, nesta tarde, um servição pelo esperanto, semeando em excelente terreno.

\section*{16 \adfflatleafright \ISODayName{1959-07-16}}

Foi notável a chegada a Tenerife, em sua capital Santa Cruz de Tenerife, nas ilhas Canárias, pertencentes à Espanha. Havíamos ido dormir a uma da manhã. A chegada deu-se às três e a quase totalidade dos passageiros desembarcou imediatamente.

Os oito dias em mar alto assim o exigiam. Eu queria dormir mais, pois o navio iria permanecer neste porto até as 13 horas. Mas às 4,30 tive de seguir o exemplo dos companheiros e, escuro ainda, desembarcar. A cidade estava, porém, cheia de movimento, com o comércio todo aberto! E táxis em profusão no cais. Ainda a bordo somos ``assaltados'' por vendedores de uma infinidade de coisas.

Tenerife é porto livre, isto é, sem alfândega. Há de tudo. Os preços nos assustam, mas no bom sentido. Um cálculo rápido ``peseta-dólar-cruzeiro'' e verificamos que é quase tudo pela metade dos preços no Brasil.

Tomamos café em mesas ao relento, escuro ainda. A limpeza das primeiras ruas e praças avistadas é impressionante. Carros-pipas estão lavando os calçamentos, com homens esfregando vassourões.

Exploram-nos neste café, pão-manteiga e ``toddy'' para quatro pessoas: 48 pesetas, ou seja Cr\$150,00! Não deixamos gorjeta\ldots

Corremos as lojas. Vários negociantes hindus, pois há uma regular colônia deles nas Canárias. Rádios-transístores e relógios suíços e alemães baratíssimos. Mas não me interesso em adquirir nada, a não ser ``souvenirs''. Inconveniente fazer bagagem tão cedo. Ficará para a mesma escala, na viagem de regresso.

Clareia, finalmente, e a cidade se agita com rapidez. Os perfis das casas espanholas, interessantíssimas, se projetam melhor. A pavimentação é impecável e os serviços de transporte exibem ônibus grandes e numerosos, rodando em todas as direções. Nossa surpresa é enorme. Carros de todos os tipos, europeus na quase totalidade, moderníssimos em maioria.

Batemos nossas primeiras fotografias em terra firme, já encontrando, por acaso, os companheiros de camarote, Noronha e Seixas. Vamos ao grande mercado ``Nuestra Señora de África''. A edificação é maravilhosa, ampla, espaçosa e limpíssima. O movimento é muito grande, mas o sortimento é superabundante. Compramos um quilo de uvas pretas: 10 pesetas (Cr\$25,00).

Um ônibus nos leva a Laguna (San Cristobal de Laguna) local de nascimento do jesuíta José de Anchieta, figura importante da nossa história antiga. É outro município, mas a continuidade urbana une praticamente os dois. A casa onde nasceu Anchieta é muito visitada. Mesmo em Santa Cruz vejo um templo de aspecto secular, restaurado e caiadinho de novo e que deve ser do tempo do apóstolo. Está justamente na ``Calle (rua) José de Anchieta''.

De volta, Alcino, em uma loja de livros, revistas e jornais, obtém informações sobre um esperantista de Tenerife, de nome Régulo Perez. O homem da loja, gentilmente liga para ``Juanito'' --- como o chama --- e com ele converso pelo telefone. Marcamos um encontro, mas ele tarda demais. Enquanto isso conversamos com um bando de ``pibes'', na praça. Sabem que o nosso dinheiro se chama ``cruceiro'' e que a capital futura do Brasil é ``Bracília''\ldots

E muitos deles conhecem várias palavras em esperanto, o que nos surpreende. Bom trabalho dos esperantistas canarienses. Deixamos recado e indicação do navio para Perez e voltamos para bordo, a tempo de alcançar o almoço. A largada está marcada para as 13 horas e acontece com pequeno atraso.

Depois soubemos que um senhor com estrela verde na lapela tentou subir a bordo, mas as escadas já estavam içadas.

Afastamo-nos de Tenerife, cujo cais, muito bom, está cheio de navios, na maioria espanhóis mesmo. Fotografo alguns ângulos.

Este primeiro contato com uma população ``extra-brasileira'' espantou-me bastante. Quantas qualidades boas, na parte material e na pessoal! O povo pareceu simples e gentil. Nenhum vendedor mostrou mau humor por não se comprar nada, depois de ver e indagar muito. Numa farmácia de Laguna, entramos quatro só para tomar o peso; era daquelas balanças antigas de correr pesinho. O homem pesou todos nós, atenciosamente, e continuou sorridente aos nos despedirmos sem nada comprar ou deixar gratificação.

O tráfego tão silencioso, sem algazarra de buzinas ou de pedestres, dá grande e benéfico alívio aos ouvidos.

Assim senti Tenerife. Pouco depois, mar alto outra vez, caí na cama para dormir o sono faltado na véspera e compensar as canseiras da escala.

Após o jantar, a delegação brasileira promove novo ``show''. O baixo (cantor) até então inédito a bordo, faz sua estréia em grande estilo. Depois, a bela Nely Angelini, Clarisse Leite, a esplêndida pianista, e Magali, a paulista de sorriso tipo anúncio de creme dental, recita Bilac; Luiz Vergueiro, também paulista, diz um monólogo jocoso. Por fim o fabuloso Trio Maraiá, que abafa como sempre. Estou louco para ver esses três se exibindo em Viena.

A chegada a Lisboa está marcada para depois de amanhã, cedo, às 6 horas, dizem.

\section*{17 \adfflatleafright \ISODayName{1959-07-17}}

Bem recuperados, lamentamos não poder voltar à piscina. O tempo está firme, mas sem sol e venta fortemente na coberta.

Desço para alinhavar estes comentários, pois a vontade de levar o diário até o fim agora é grande. Não sei se sairá.

Hoje é sexta-feira. O jornal de bordo, em seu noticiário telegráfico, nada disse das disputas esportivas do Rio. Ainda não sei o resultado de Flamengo x Vasco, do dia 12.

De agora em diante, as escalas serão diárias: Lisboa, Tanger, Algeciras (junto a Gibraltar), Barcelona (dia e meio), Marselha e Gênova.

Inscrevemo-nos na excursão organizada pela companhia do navio, durante a parada em Lisboa, amanhã, onde devemos aportar às 6 da manhã. O passeio custará 250 pesetas, mas durará 4 horas. O navio ficará até as 3 da tarde.

Nossos dois companheiros de camarote, os excelentes Noronha e Seixas, ficarão em Lisboa. Também os companheiros de mesa, Sr. Guilherme, D.ª Margarida e a filha, Irene, descerão lá.

Realiza-se, agora, outro ``show'' no salão, este justamente oferecido pela delegação brasileira aos portugueses que se ``quederán en su tierra''. Vou para lá. Acabo de escrever longa carta à Cooperativa Cultural dos Esperantistas, dando as boas notícias dos acontecimentos de bordo. Para casa já havia escrito à tarde. E ainda fui obrigado a refundir o noticiário que o Alcino preparou para ``O Jornal'', do Rio.

\section*{18 \adfflatleafright \ISODayName{1959-07-18}}

A chegada à Lisboa foi maravilhosa.

Acordei às 6 da manhã e os nossos dois companheiros já estavam prontos e com tudo arrumado. Avisaram que ainda faltava um pouco para atingir terra. Vesti-me e subi para a coberta lateral, com o binóculo. Amanhecia o dia, ainda cheio de névoa. Dali a pouco uma nesga de terra fez-se visível --- minha primeira visão da Europa!

Mais um pouco e mesmo a olho nu se divisavam contornos à direita e à esquerda; no meio, a barra do rio Tejo, muito larga. Com o binóculo percebo que a aparente torre da direita é o novo monumento a ``Cristo Rei'', em Almada, frente à Lisboa. Vagarosamente vamos vencendo a barra, que me surpreende pelas águas verdes e claras como o próprio mar.

O dia levantou-se exuberante. Atracamos às 9 horas, mais ou menos. Antes fotografei a lendária Torre de Belém, menor do que eu supunha, mas clarinha e bem conservada. Às 10h30m saímos em excursão de três horas e meia, em magnífico ônibus, com cicerone e sistema de som interno.

O guia fala espanhol. Da zona do cais ganhamos o centro, rua Augusta (das principais do comércio), Rossio (com o monumento ao nosso Pedro I, aqui Pedro IV), Avenida da Liberdade, a principal e que impressiona pela largura, pavimentação impecável, arborização e ajardinamento notavelmente florido.

Tráfego fácil e silencioso. Carros modernos de dezenas de tipos, mas europeus. A cidade não é plana; pelo contrário, assenta-se em colinas, em número de sete, esclarece o guia. As subidas lembram São Paulo. A patrícia Laís, paulista, como praticamente toda a delegação ao festival, senta-se ao meu lado e acha a fisionomia daquele trecho final da Avenida da Liberdade parecida com Montevidéu.

Dali vamos parar no Palácio de Queluz, onde nasceu Pedro I, mas que está fechado para recepcionar, hoje ou amanhã, Hailé Selassié, imperador da Abissínia. O palácio é baixo, róseo e está bem descascadinho, esparramando-se amplamente para os lados.

Dali tomamos o rumo de Sintra, fora de Lisboa, mas bem perto, junto à serra do mesmo nome. Pelo caminho, sempre muito bem asfaltado, vejo vegetação amarelada, morros pelados, de terra com saibro, aquedutos (um subterrâneo, com respiradouro somente aparecendo), moinhos em ruínas. Sintra é um pequeno bosque, de tão arborizada. Ali viveu o poeta Lord Byron (o guia aponta a casa), que lhe cantou as delícias.

As ruas, subindo pelas raízes da serra, são estreitíssimas. Paramos à porta de seu monumento máximo: o castelo mourisco, que visitamos cômodo por cômodo. Depois um lanche no acolhedor bar do Hotel Paris. Fotografei o interior do castelo, minha primeira velharia penetrada na Europa, e permutei poses com o Alcino, à frente do mesmo.

Olha-se em torno, e da mata fresquíssima pontificam torres de castelos. Lá no cimo, duas fortificações do tipo medieval, que também fotografei, e torrinhas menores de construções mais particulares.

Pagamos o lanche em pesetas. O rapaz, nosso primeiro contato a sós em Portugal, fez conta, refez e descobrimos que ia se enganando em 100\% contra nós\ldots

Boa caminhada dali a Cascais, uns 20 quilômetros, ganhando-se a costa. Somos apresentados ao cabo de São Roque.

As dunas da praia dos Guinchos cospem areia e o forte vento fá-las guinchar mesmo, ruidosamente. A costa é toda pedregosa, de lavas vulcânicas. O vulcão era em Sintra e inundou toda a região até quase junto a Lisboa.

Cascais é lugar de endinheirados, com majestosas casas à beira-mar. Mora ali o rei Humberto, da Itália, que comprou um palacete, alugou-o e passou-se para um antigo e enorme palácio ao lado. Antes um pouco de Cascais, a ``Garganta do Inferno'', vão aberto na alta pedraria vulcânica, muito profundo, fazendo um arco natural, através do qual o mar penetra um pouco terra a dentro. Bonito.

Cascais tem também um iate-clube. Mais adiante, o famoso Estoril. O conjunto é fabuloso nas cercanias do cassino, cujos jardins são magnificentes, hotéis, termas e balneários. Mas a praia é ridícula, dois palminhos de areia escura e de mar parado. Aliás, as outras todas não vão muito além.

No caminho de volta à Lisboa vêm, em seguida: Carcavelos (a maior e melhor praia), Oeiras e Passo d’Arcos. O mar é verde e limpo. Mas a faixa praieira é estreita e de cor pardacenta. E todas elas espetadas de armações verticais, alinhadas em toda a extensão da areia: as barracas. Aluga-se o pano e cobre-se a armação. Feíssimo!

Rumamos de volta a Lisboa. Fortes maiores e menores, à beira-mar. Um deles, baixo e compacto, foi adaptado para residência estival de Salazar, o governante português. Hoje ele não está --- diz o guia, pois não se vêm guardas à entrada.

Lembro-me de que hoje é sábado e já passa das 13 horas. Aqui deve ter também uma ``semana inglesa''. O tráfego de automóveis é agora intenso. As praias já mostram um bom movimento de banhistas.

A Torre de Belém, outra vez, agora de trás e de terra, vista de pertinho. Outrora ficava isolada no mar, a quase quilômetro e meio da terra. Os aterros fizeram-na, agora, junto da extensa via beira-mar. Caminho de ferro, eletrificado, como todos os outros que vi, passa também junto.

Um desvio à esquerda para rápida visita ao Estádio Nacional, o tal de gramado divinamente franjado claro-escuro. Uns tons de verde alucinantes. Está vazio, é claro. Homens cuidam do ``relvado''. O estádio é quase todo de pedra clara e decorado, em alguns pontos, com paredões em colunas, retangulares ou arredondadas. Batemos chapas e vamo-nos.

Quase ao voltar ao cais, passamos pelo castelo dos Jerônimos, aliás mosteiro. Belíssimo, no estilo chamado ``manoelino''. O tempo não permitiu, porém, visita ou simples parada à porta. Ficará para a volta à Lisboa, no fim desta viagem.

Almoço a bordo, já quase às 14 horas. Ficaram em Lisboa mais de duzentos passageiros. Tentamos um telefonema para um esperantista local. Linha ocupada seguidamente. Como não temos escudos, pago uma revista em pesetas e reembarcamos. Fico sabendo que também desembarcou aqui a vedete Rose Rondelli, que, durante a viagem, foi sempre vista rodeada de cavalheiros, espanhóis e argentinos.

Nota pitoresca: dez passageiros se atrasam e ficam em terra após o desencosto do navio. Vieram para o largo em um rebocador e embarcaram debaixo de estrondosa mas amistosa vaia, comandada pelo Antoine.

Quase às 16 horas, estamos deixando a barra do Tejo, rumo a Algeciras, pequeno porto espanhol pouco antes de Gibraltar. Revemos toda a orla de cada margem do belo e limpo rio. Interessantíssimo que suas águas não se misturam com o mar até a embocadura, em cujo meio há um farol; vê-se nitidamente a linha divisória, em tons diferentes de verde.

Até mais tarde, Lisboa, que tanto nos encantou! Alinhavei de uma penada todas estas minhas primeiras impressões. A Europa já começa a exceder a expectativa. Um detalhe ressalta em Portugal, aliás dois: a vida parece calma e organizada; e os automobilistas são verdadeiramente felizes, com tão maravilhosas ruas, avenidas e estradas.

Última recordação de Lisboa: as mulheres carregadoras de areia do cais do porto. Transportam até 60 quilos, em cestas ou latas sobre a cabeça, caminhando por uma estreita tábua, dos barcos (espécies de pesqueiros com ponteiras fenícias) à margem.

Leio mais tarde, na revista que comprei, intitulada ``Diário Ilustrado'', que Maysa estreará hoje no Cassino Estoril. E no mesmo órgão leio também que não há semana inglesa em Portugal, pois agora é que se está cogitando de introduzi-la.

São nove e meia da noite e ainda está claro. Pouco mais tarde passamos pelo Cabo de São Vicente, a beiradinha sul de Portugal. Agora já está escuro e a luz giratória do seu farol chama a atenção. Devemos atravessar Gibraltar às 5 da manhã e chegar a Algeciras, na Espanha, lá pelas 6. Soubemos com pesar do cancelamento da escala em Tanger, no Marrocos. Era a minha única oportunidade de pisar em solo africano. Será verdadeiro o boato?

\section*{19 \adfflatleafright \ISODayName{1959-07-19}}

Ontem mesmo, à noite, desfazia-se a dúvida, aliás mal entendido, quanto à parada em Tanger, que estava confirmada. Logo cedo, o ``Maripez'' noticiava os horários. Não consegui acordar na hora em que pretendia para assistir a passagem pelo penhasco de Gibraltar, ou seja, a travessia do estreito que marca a entrada no mar Mediterrâneo.

Mas o horário anteriormente falado não fora vencido como previsto. Às 8 e pouco atingíamos o porto espanhol de Algeciras, nossa primeira visão das terras ``franquistas''. É uma bela baía, no estreito, mas antes de sua parte mais apertada, onde se acha o penhasco, com as fortificações inglesas. Do lado oposto, a costa africana.

O estreito não é tanto assim. Na altura de Gibraltar tem uma largura de uns vinte quilômetros. Fora dali, muito mais. Mas durante a travessia a vista domina inteiramente os dois continentes, que tem costas algo elevadas, com profusão de penhascos e montes. Algeciras não dá acesso a grandes vapores. Por isso ficamos ao largo. Desembarcaram os passageiros com destino a Madrid, entre eles os engenheiros espanhóis que estagiaram no Brasil. A cidade é pequena e encarapitada na elevação do terreno. De binóculo vasculho tudo. O ``Hotel Santa Marina'' e a ``Cerâmica Virgem de los Milagros'', com grandes letreiros nas fachadas. Divisamos, na beira-mar, em apertada pracinha, ônibus e bondes.

A baía é realmente bonita. À direita, fechando-a bem, fica Gibraltar, que, no momento, está sob densa névoa em sua parte baixa. Não se distinguem bem as fortificações, mas os vasos de guerra ingleses lá estão abrigados, em bom número.

Duas horas mais tarde, largamos rumo a Tanger, isto é, retrocedendo pelo estreito. Então passamos bem junto de Gibraltar. Face à névoa, deixo de fotografá-lo, preferindo fazê-lo na volta, hoje mesmo, quando passarmos a caminho de Barcelona.

À hora do almoço, Tanger está à vista e causa sensação. Muito grande, com elevações amontoadas de casas estilo árabe, como se fossem uma só; deve ser a parte velha, a muçulmana. À esquerda, enorme e bela praia, a primeira que vejo capaz de rivalizar com as nossas. Ao centro, perfis de altos e modernos edifícios.

Almoçamos, apressados, enquanto o navio encosta e se cumprem as formalidades de praxe, antes do desembarque. Ao subir do refeitório, já encontro o navio encostado. Lá embaixo, uma quase feira livre de vendedores de mil coisas para os passageiros. Tanger também é porto livre, sem alfândega. Avisto os primeiros muçulmanos de trajes típicos, com e sem fez (chapéu rijo, sem aba); e a primeira mulher de longas vestes escuras, com o rosto coberto! Fotografo-os, discretamente, lá de bordo mesmo.

Depois descemos e nos dão duas horas para ir à cidade. Foi o máximo da viagem até agora! Embrenhamo-nos a pé, pelo bairro muçulmano, o lendário Casbah, dos filmes de ``intriga internacional'' e mistérios orientais. Percorro as ruas estreitíssimas, sujas, lojinhas minúsculas, mal-cheirosas, paredes seculares. Um autêntico mundo árabe. Um rapazola, com muita mímica, resolve se candidatar a nosso cicerone.

O primeiro elemento típico, de trajes árabes e montado de lado em minúsculo jumento me encontra de máquina em punho para fotografá-lo, mas --- grande decepção --- ele faz gestos enérgicos de protesto, além dos gritos, e não me animo a bater as chapas. Verifico, então, que não admitem ser fotografados, por motivos de tradição, religião ou dignidade. Há que disfarçar e, para isso, posamos nós próprios, com eles desatentos em segundo plano.

Predomina a pobreza franciscana, com oscilações até a miséria, pois são muitos os sentados aos cantos das quase intransitáveis ruelas. Mulheres de rosto coberto cruzam a todo momento.

Numa loja compro um fez por 50 pesetas. É só regatear que os preços caem verticalmente. Vamos até a parte mais alta. Há um velho castelo de algum lendário sultão, todo decorado no característico estilo mourisco. O pátio interno, infalível em toda edificação árabe, está florido, mas o aspecto de abandono da velharia, castigada pelo tempo, é inevitável. Mais um pouco e atingimos a beira-mar, lá do alto. Há uma amurada, do complexo arquitetônico que é aquilo. Abaixo, na pedra, a uns três metros, um árabe esta curvado, certamente em orações; levanta meio corpo e põe outra vez a testa sobre uma curta esteira. Por prudência (e, em parte, respeito) escuso-me de fotografá-lo.

Descemos pelo interior de um cabaré, àquela hora vazio, naturalmente. Tapetes de esteirinhas e almofadões no chão, em torno das paredes, nas várias saletas. Mesinhas miúdas em outras, que devem ser os bares. Neste mesmo grupo, uma sacada para o mar, bem elevada. Vista magnífica. Batemos várias chapas. E toca a percorrer labirintos de fazer medo.

Juntamo-nos a outro grupo do navio, com um guia local, em trajes típicos, falando bem espanhol. Leva-nos a um canto de um pátio ensolarado, onde dois velhos árabes ostentam cestas do tipo porta-serpentes. Excitamo-nos ante a perspectiva de presenciar um autêntico ``encantamento'', pois duas cobras são postas para fora. Os dois executam uma dança saltitante, acompanhada de cantoria puxando para choramingo, mas nada de flauta nem de encantamento. Pegam as cobras e fazem com que elas mordam seus narizes! E depois a língua! As moças dão gritinhos de susto --- ou de nojo. Fotografamo-nos ao lado deles, como uma bela recordação\ldots

A estranha sensação de Casbah faz-nos desistir de percorrer, em giro de táxi, a cidade moderna. Valia mais a pena ficar por ali mesmo. Há trechos de comércio predominantemente espanhol, como se vê pelos nomes das lojas (lojinhas, algumas de pouco mais de dois metros quadrados) em castelhano, e em árabe por baixo.

Fizemos mais compras, em lojas e no meio da rua, com camelôs. Tudo nos parece baratíssimo e o preço pago é, às vezes, menos da metade do inicialmente pedido.

Três ou quatro companheiros de viagem, uma moça entre eles, entraram inadvertidamente numa mesquita. Foram, por assim dizer, escorraçados e ouviram falar que muita sorte tiveram, pois a falta fora gravíssima, não raro caso de morte. A moça teve ainda a ousadia de bater uma chapa lá dentro!

Disseram eles que o interior é soberbo, arquitetônica e decorativamente, e que não tem nenhum móvel, somente esteiras e almofadas. Passamos na porta dessa mesquita. Alguns fiéis, plantados em vários degraus de escada acima do nível da rua, olharam-nos friamente. Não sei se naquela hora já havia ocorrido o incidente.

Voltamos quase correndo para o navio, pois que em cima da hora marcada para a partida. Um longo silvo soou, enquanto vínhamos a caminho. Junto do vapor a feira livre fervilhava. Compro dois lenços para cabeça, típicos, como ``souvenirs'', e, em seguida, resolvo me interessar por rádio transistor, que os havia maravilhosos. Peço preço. ``Que moneda, señor?'' --- arranha o vendedor. Falo em dólar e ele pede 35; faço um simulado espanto, digo que o navio já vai partir e exibo uma nota de 20, adiantando que era o que possuía. E saio na direção da escada. O coitado por aquele preço não poderia vender mesmo (seriam Cr\$2.700,00), parece-me, pois implorou longamente para chegar aos 25; se oferecesse 23 ou até 22 mesmo eu traria o rádio. Mas a vontade de comprar era nenhuma e ficou tudo por aquilo mesmo. Só fiquei com pena do homem e me arrependi do que fiz. Em Tenerife, na volta, poderia fazer melhor negócio.

Deixamos Tanger e duas horas mais tarde voltamos a cruzar Gibraltar. Por azar meu, passamos mais junto da costa africana. E a bruma é ainda pior que a da manhã. Quando passava um pequeno petroleiro, bato a chapa só para documentar.

À noite senta-se à nossa mesa, para jantar, um casal marroquino, com duas filhas, Anne e Gislene, de uns 13 ou 14 anos. São brancos, como a maioria dos marroquinos, e falam francês. Anne, a mais velha, fala um pouquinho de inglês. É uma tortura mental nossa conversa e, tomado de surpresa, tenho de arrancar de chofre, da memória, todo o meu francês.

O Alcino merece o ``prêmio Nobel'' da ignorância e negação pelas línguas. Não consegue guardar absolutamente nada e se mostra incapaz de aprender sequer uma palavra. Cabeça ruim está ali\ldots Como se não bastasse, insiste em tagarelar em português, sem se dar conta de que ninguém ``pesca'' vintém\ldots Solta uma longa frase. Os marroquinos sorriem amarelo e se voltam para mim, silenciosamente indagadores. Tenho que me virar para traduzir, de algum modo, o que o falastrão botou para fora.

Após o jantar, sentamo-nos outra vez com eles para assistir o ``show'' da delegação brasileira oferecido à Espanha e aos espanhóis que amanhã desembarcarão em Barcelona. Foi o mais fraco de todos. Mas, felizmente, para encerrar, o Trio Maraiá chegou e, como sempre, empolgou.

O cinema do convés caiu verticalmente. Depois das 9,30 da noite o vento lá fora é insuportável.

Amanhã à noite atingiremos Barcelona, onde a parada será a mais prolongada, talvez 36 horas. Assim, o atraso do navio chegará a dois dias e só no dia 23 chegaremos a Gênova, e não no dia 21, como previsto.

\section*{20 \adfflatleafright \ISODayName{1959-07-20}}

Tempo magnífico, como, aliás, desde a saída do Rio, há 13 dias atrás. A costa espanhola está sempre visível à nossa esquerda. É toda ela agreste, pedregosa.

Da piscina venho assistir um pouco o panorama, da amurada. O espanhol ``boa praça'', Sr. Bernardo de Castro, há trinta anos no Brasil, me identifica a cidade que se delineia pendurada na base de um penhasco: Alicante, junto do Cabo de Santo Antônio.

Já passamos pelo Cabo de Palos, de onde partiu Colombo para descobrir a América. Agora nos afastamos mais da costa porque --- diz o Sr. Bernardo --- há muitos recifes dali por diante. Esse espanhol foi o pioneiro no Brasil da fabricação de redes para pesca, e hoje domina todo o mercado nacional. Mora, ou melhor, tem família em Fermoselle, província de Zamora, fronteira com Portugal. Já nos deu o endereço e combinamos um encontro nas festas de Salamanca, quando voltarmos, em setembro.

Com a luz do sol se pondo às 21 horas, acercamo-nos de Barcelona. O crepúsculo estava espetacular, purpurino, cobrindo as colinas que se estendem até a região central da cidade. Mas ainda temos uns 30 minutos de navegação até o cais. Do lado contrário, uma lua monstruosamente cheia, no céu claro ainda para anoitecer. Quase 400 passageiros --- soube depois --- ficarão em Barcelona. O ambiente é festivo, quase de agitação. O alto-falante de bordo toca ruidosamente melodias marciais.

A chegada foi, assim, algo apoteótico. Ao acostar, não sei de onde saíram tantas serpentinas. O barco espanhol, orgulho de sua marinha mercante, completava, no território pátrio, sua primeira viagem! Cais atulhado de gente. Enfim saltamos, lá para as 22h30m horas. As luzes da cidade já haviam dado uma idéia de sua enorme extensão, com seu milhão e meio de habitantes.

Conhecemo-la, assim, primeiro à noite. E teremos o dia inteiro de amanhã para rodar, pois o navio partirá às 22 horas.

A impressão foi magnífica. Logo ao lado do cais a Plaza Colón (Praça Colombo), com o famoso monumento ao descobridor, que está lá em cima encarapitado, apontando para a América. Depois a Ramblas, avenida estreita nos lados, para o tráfego, e com larguíssima calçada central para o ``paseo''. Excelente arborização, bares à margem e, ali mesmo, estações do ``metrô'', que eu nem sabia que existia.

O movimento é intenso, apesar da hora e de se tratar de uma segunda-feira. A avenida é longa. À esquerda, várias ruelas estreitas levam aos trechos mais antigos e típicos. Um popular nos aborda, talvez atraído pela insígnia ``Brasil'' no nosso peito. Recomenda-nos uns ``nightclubs'' e o lado antigo, como curiosidade.

Passeamos até a Plaza Cataluña. No caminho, mais duas estações do ``metrô'': ``Fernando'' e ``Liceo''. Em um bar --- ``Granja Royal'' --- tomamos um bom leite, com excelentes pãezinhos, puxados a pastelaria. Dobramos a dose. E conversamos com o garçom sobre o preço de açúcar e Evaristo, futebolista brasileiro que alcança grande sucesso em Barcelona. O açúcar vem num saquinho de papel com dose certa. O preço do lanche é caro: 56 pesetas (Cr\$180,00, mais ou menos).

Voltamos para o lado da parte antiga, onde há até um bairro chinês. O ambiente ali é pesado, a freqüência suspeita. O ``nightclub'' recomendado era ali, o ``Jardim de Granada''. Senta-se ou no bar, em altos bancos junto a um balcão (45 pesetas), ou nas mesas ao ar livre (90 pesetas). O ``show'' musical é, porém, muito bom, pois tipicamente espanhol, uma vibrante flamenqueria, regada a fartos ``óles''.

Giramos por outras ruas. E falamos (menos Alcino\ldots) bom espanhol com outros tipos, que elogiam nossa linguagem e pronúncia --- curso de 15 dias a bordo\ldots

Uns companheiros de viagem, encontrados por acaso, reclamam contra o pagamento de uma taxa, porque sentaram-se em poltronas de madeira, na Ramblas. Mas tinham direito de ficar até as 6 horas da manhã, informou o funcionário cobrador\ldots

Voltamos bem tarde para o navio, de táxi. É, aqui, uma condução relativamente barata.

\section*{21 \adfflatleafright \ISODayName{1959-07-21}}

Ficaremos o dia inteiro em Barcelona, pois a largada dar-se-á às 22 horas. Logo cedo, tempo firme e mesmo quente (34 graus), caímos na rua.

Vou direto cumprir uma das encomendas recebidas no Rio: adquirir folhas de um álbum filatélico em casa especializada das mais famosas do mundo. Fica na ``Calle Layetana''. Andamos a pé uma barbaridade. Pelo caminho, paramos a todo momento, para ver vitrines e especular preços.

A cidade agrada em cheio, embora toda de construções antigas. Larga, espaçosa, bem arborizada. O tráfego é intenso, mas silencioso, tal como em Tenerife e Lisboa. Lembra São Paulo sob muitos outros aspectos. A cor e tipo dos bondes, por exemplo, alaranjados e vermelhos. Há também ônibus elétricos. E táxis em profusão, quase todos ``Fiat 1400'', que ali se chama ``Seat'' (montados na Espanha). Também ``Citroens'' em quantidade na praça.

Telefonamos para um esperantista, de nome Ayala Gomez. Pede-nos ele que compareçamos ao seu endereço, na Avenida del Generalíssimo Franco (antiga Diagonal). Outra boa caminhada, pois a avenida, aliás magnífica, é também bastante longa.

Gomez é jovem e simpático. Fala um ótimo esperanto, fluente e correto. Apresenta-nos a família, pede-me que escreva impressões no seu ``livro de ouro'', que pertence ao ``Instituto de Esperanto'' local. O livro fora recém-aberto por esperantistas australianos e japoneses.

Depois, excelente programa: saímos no seu carro para conhecer melhor a cidade. O irmão mais moço, Jaime, dirige. Uma chapa é batida em frente à praça de touros, não muito grande nem alta. Vamos, em seguida, ao ``Montjuich'', elevação muito bem ajardinada, de onde se avista quase todo o porto. Outra novidade: o estacionamento do carro é pago --- uma taxa municipal de uma peseta, válida por todo o dia, mesmo que se troque de local.

Visitamos, depois, ``Pueblo Espanhol'', passando, no caminho, pelo parque ``Cidadela'', um jardim cercado tipo Quinta da Boa Vista, no Rio, mas plano e infinitamente melhor tratado e conservado. Dentro do parque há um ``zoo'' e alguns pequenos museus.

O ``Pueblo'' é o ponto alto, a meu ver. Trata-se de uma reprodução perfeita de estilos arquitetônicos e de costumes e artesanatos, inclusive, das diferentes regiões da Espanha, reunido tudo em um agrupamento como se fora um só bloco. Interessantíssimo. A porta de entrada, por exemplo, é reprodução exata de uma existente em Avila: imenso umbral de pedra, como um fortim medieval. Os apressados poderão ali conhecer, em uma hora, todos os aspectos típicos e de costumes do país --- tão diferenciados entre si.

Dentro dos pavilhões funcionam, além dos artesanatos, atividades múltiplas de cada região, inclusive bailados e cantorias. Compro uma xilogravura e um cinzeiro de cristal. E comemos rosquinhas gostosíssimas de um tipo madrilenho.

Antes de ``Pueblo'' havíamos feito uma parada nas portas do antigo palácio real, hoje museu, se não me engano. Majestosa construção, em meio de imponentes jardins. A fonte luminosa não está funcionando presentemente. Seus jardins causaram estupenda impressão, realmente.

Gomez precisa ir trabalhar. Traz-nos de volta às cercanias do porto e nos despedimos frente ao restaurante ``Siete Puertas'' (varanda com sete arcos), onde vamos almoçar, em hora já bem tardia.

O prato que pedi (por recomendação do amigo esperantista) --- ``paella valenciana'' --- deu-me um susto: uma enorme panelada de arroz amarelo de açafrão, bem papa, povoado de tudo quanto é bicho do mar, inclusive mariscos ainda trancados nas conchas\ldots Comi só um pouco e foi uma lástima, pois lá pela noite o estômago dava pinotes, em luta com o contingente marinho!

De regresso ao navio para deixar os volumes, o Alcino caiu no camarote para uma soneca. Volto sozinho à cidade. Fotografo Colombo em sua sugestiva estátua, dou um passeio de metrô, que é veloz e eficiente. Os carros têm os assentos de palhinha, já bem castigados. No mesmo dia se inaugura outra linha nova, para uma estação chamada ``Piscina''. Ando de bonde e a pé, em outro bom estirão, para fotografar a espetacular igreja da ``Sagrada Família'', ainda inacabada; e talvez nunca o seja, pois o arquiteto que a planejou faleceu em 1900 e desde então a obra ficou abandonada.

Volto num taxi ``Citroen'', para o navio. Preço barato da corrida: 13 pesetas. A condução não é problema em Barcelona. Abundante e barata. O metrô custa só 80 cêntimos. Mas não se pode esquecer que o povo ganha pouco, dizem todos. O salário mínimo diário é de 60 pesetas (Cr\$150,00).

Uma decepção: povo feioso, mesmo as moças. A decantada formosura espanhola não habita, positivamente, esta cidade. Raras beldades, chamando até a nossa atenção, pela excepcionalidade. A marroquina, de bordo, embarcada em Tanger, deixa longe toda esta espanholada\ldots

Com grande atraso, o navio zarpou lá pela meia-noite. Anunciam a chegada à Marselha entre 10 e 11 horas de amanhã.

\section*{22 \adfflatleafright \ISODayName{1959-07-22}}

Estamos curiosos para o primeiro contato com a França.

De manhã, após o café, verificamos ainda estarmos longe, sem visão de terra.

Mas às 11, precisamente, vamos entrando lentamente no grande e abrigado porto de Marselha. A cidade está também encravada entre colinas pedregosas, que, até aqui, tem sido a feição permanente da costa mediterrânea europeia.

A cidade é menor que Barcelona, pois acabo de ler no anuário esperantista que a população marselhesa é de 650.000 habitantes. As construções são antigas, despontando, ao longe, poucos prédios de aspecto moderno.

Almoçamos primeiro, para sair em seguida. O navio ficará até as 8 da noite. Há bastante tempo. No porto encontramos o irmão gêmeo de nosso barco: o ``Cabo San Roque'', que iniciou na véspera sua viagem de volta a América do Sul. Os espanhóis de bordo estão estufados de orgulho, vendo reunidas as duas belonaves.

O porto dista 7 quilômetros da cidade, isto é, de sua parte central. E os taxis cobram uma enormidade para este trajeto. Alcino desencontrou-se de mim e sumiu. Resolvo participar de uma excursão promovida pelo navio, em ótimos ônibus. É barata: 800 francos (Cr\$240,00, na base do câmbio em que comprei dólares no Rio, em maio).

Damos umas voltas pelo centro da cidade. Ruas bem acanhadas e de tráfego engasgado. É a primeira cidade que vejo com esse problema. Só veículos franceses, praticamente. O horrendo e pequenino ``Citroen 2 CV'' aos montes, por todo lado. Os do último tipo também.

Logo ao fim do cais antigo, vários clubes náuticos. Não sei se são particulares ou se alugam os barcos. Junto, também, uma linha de ``ferry boat''. Soube, depois, que é grande o movimento turístico de visita ao castelo-prisão de ``Tif'', numa ilha dentro da baía. Ali esteve ficticiamente preso o lendário Conde de Monte Cristo, da famosa obra literária. Não o visitei, porém, vendo-o apenas do alto e de longe.

Subimos ao monte em cujo topo se encontra a majestosa matriz de ``Notre Dame de Garde'' (Nossa Senhora da Guarda), servida também por funicular. Avista-se, lá do alto, toda a cidade, bem grande, a terceira da França, creio, com seus morros vestidos de granito ou de terra bem esbranquiçada. O panorama é magnífico. O interior da igreja está atulhado de miniaturas de navios pendentes do teto. Vem o esclarecimento do guia da excursão: reconhecimento de marinheiros, que tiveram suas súplicas de socorro atendidas, durante terríveis tormentas.

Dali tomamos o rumo da ``village'' Aix-en-Provence, a uns 25 quilômetros, passando por outra pequena localidade denominada Luyane. A ``auto-route'' é espetacular, de mãos de direção separadas e pavimentação maravilhosa.

Lá em Aix-en-Provence estreei o meu francês para cambiar dinheiro. Os bancos naquela hora estavam fechados e foi necessário encontrar um bar que aceitasse dólar, para fazer um pequeno lanche: três copos de leite com bolos de finas fatias, embalados em celofane. Despesa: 390 francos (Cr\$117,00)! E a gorjeta (obrigatória) de, no mínimo, 100 francos\ldots Se se continuar a pensar em cruzeiros, tudo parece fantasticamente caro. A professora Edith e sua amiga Amir, companheiras de viagem, estão comigo.

Voltamos ao navio pelo centro de Marselha outra vez. Não descobri a foz do rio Ródano. Será que não desagua no Mediterrâneo passando por aqui?

Pelo caminho marginamos pequenos campos de trigo, já colhido e amarrado em feixes. Camponeses carregam carroças fisgando os molhes com aqueles garfões. Os trigais (vejo pedaços não colhidos) são de baixa estatura. Nem metade dos nossos que vi no Mato Grosso.

Às 9 da noite, em plena luz do sol, deixamos Marselha. Lembro-me dos sinais da guerra na elevação que conduz à N.~D. de Garde: um tanque está conservado na subida, como monumento, até hoje, no local em que ficou na luta pela reconquista do porto. Junto a este há outra edificação elevada completamente destruída.

Finalmente, amanhã chegaremos a Gênova, lá pelas 9 horas, dizem. Alegria e pesar. Grandes amizades na longa travessia. Impressões que ficarão indeléveis no meu espírito, que durante tanto tempo acalentou o sonho de uma viagem como esta.

A delegação brasileira ao festival de Viena, eterna recordação desta passagem. Seus tipos marcantes e pitorescos, desde seu chefe ``Antoine'' ao casal Reynaldo-Lola Machado, já tão viajados. O Sr. Palma, português de cultura, há muito radicado no Brasil, os quatro alagoanos, a família de São Carlos, o Évio, de Goiás, Simão e Vergueiro, os mais populares, Lúcio Kowarick, filho do homem das casemiras, o mais exótico, em sua extrema magreza e pavorosa barbicha ruiva. As moças tão simpáticas e, em maioria, bonitinhas: duas Veras Lúcias, duas Magalís, Yone Cozza, de Santos, a ``cozza'' mais interessante de bordo\ldots depois de Rose Rondelli, a vedete abafante, de tanta simplicidade. A pianista Clarisse Leite e Dona Laudelina, a acordeonista e professora, de cativante simpatia.

Enfim, os que faltam para completar os 91 da delegação brasileira, sem deixar de mencionar Hirton, Marconi e Bhering, que compõem o fabuloso ``Trio Maraiá''.

Haveremos de encontrá-los em Viena, todos, daqui a alguns dias.

A tripulação com maior contato conosco --- às vezes rudes e sem classe --- mas laboriosos e humildes na exigência de propinas. Nosso bom garçom Luiz e os servidores de café Gabriel e Henrique.

``Cabo San Vicente'', adeus. Foste palco de uma das maiores emoções de minha vida!

\section*{23 \adfflatleafright \ISODayName{1959-07-23}}
Em ambiente de certa agitação a bordo, pelas arrumações finais de tudo, chegamos ao destino: Gênova. Das 9,30 às 11 esperamos para desembarcar. E até as 12 para ganhar a rua.

A cidade, que é mais populosa que Marselha (780.000 habitantes), é, porém, mais apertada, parecendo de menor área. É também cheia de elevações. O ambiente no porto, malgrado a azáfama, era enormemente festivo. O navio atraca bem junto ao alpendre elevado de desembarque.

Fazendo-nos de agregados à delegação brasileira ao festival, tivemos dispensa de revisão de bagagem. Saímos diretamente para a ``Esperanto Domo'', como é conhecida a ``Pensione Giardino''. A rua, em meia escadaria, é pitoresca. Tudo antigo, mas notavelmente limpo e higiênico. Nosso quarto, de onde agora escrevo, é bem grande e confortável.

O proprietário, Sr. Frederico Gianoglio, esperantista, nos recebe cordialmente. Conversamos longamente enquanto almoçamos. Soubemos que outro brasileiro, o Dr. Alberto Flores, da Companhia Siderúrgica Nacional, chegará no dia 3.

Saímos logo a passear. Gênova encanta de imediato. Secular, mas bem conservada. Descemos as Vias Caffaro e Garibaldi até a Piazza de Ferrari, que é circular e de belo chafariz ao centro. Chão de grandes lajes retangulares, impecável como pavimentação. Lojas não muito grandes, mas bem instaladas e modernas.

Penetramos, depois, na igreja de ``San Lorenzo'', imponente. Fotografo, dentro dela, uma bomba inglesa que, durante a guerra, caiu ao lado e não explodiu. Está lá como relíquia.

Dali andamos pela parte velha, até a beira do porto. Lembra Tanger, pela estreiteza das ruelas de um metro e pouco, quase colando altíssimos e velhíssimos edifícios. Zona pobre. De um bar sai, de repente, um homem para falar conosco. Viu nossa insígnia ``Brasil'' na lapela e vem dizer que tem um irmão em Santo André. Oferecemo-nos para levar alguma correspondência, desde que entregue no hotel até amanhã. Convida-nos para um vinho no seu bar, mas só aceitamos... água... Então ele aromatiza a água com menta e aperta no sifão.

O contato com o povo é admirável. Todos simples, prestativos e simpáticos. Os que nos bateram fotografias, a pedido, na rua, os que nos venderam lembrancinhas baratas, os que nos serviram nos bares. Impressionante a polidez, o respeito e a correção em todas as maneiras, mesmo quando especulamos coisas e nada compramos. O Alcino é campeão disso...

Depois do jantar, recebemos a visita de esperantistas locais, naturalmente chamados pelo dono da casa. São todos idosos: o presidente do grupo local ``Unione Esperantista de Gênova'', Sr. Umberto Segrè, o jornalista Giovanni Barni e o Sr. Arnaldo Opisso. O Sr. Barni mostra-nos artigo de sua autoria sobre o ``Ano de Zamenhof'', publicado no jornal ``Il Lavoro'' de 18 do corrente. Conversamos até tarde sobre nossa viagem, problemas do movimento esperantista e outros assuntos.

Saímos para outra volta pelas redondezas, já passadas as 23 horas. O Sr. Segrè, que esteve há pouco em Israel, informa como o movimento esperantista vem ganhando terreno lá. E nos mostra o teatro lírico de Gênova, atingido por bombardeio e fechado desde então.

Mais abaixo, junto à Piazza Dante, aponta-nos a Casa de Colombo, coberta de vegetação; uma parte da fachada desmoronou. Depois a vimos, completa, em postais mais antigos.

Um café-amigo e fraternais despedidas, pois amanhã à noite deixaremos Gênova. Sugerem-nos bom programa de passeios.

\section*{24 \adfflatleafright \ISODayName{1959-07-24}}
De manhã cambiamos dinheiro e tomamos o caminho do cemitério de Gênova, chamado ``Staglieno'', grande atração turística, pelo repositório artístico ali encontrado. Encontramos, no ponto de ``trôlei-bus'', companheiros de viagem: Marianita Morales e seus pais, e vamos juntos fazer o passeio.

Realmente, o ``campo santo'' é impressionante. Tumbas de extraordinário estilo. O mármore superabunda, nas esculturas, colunas, escadas, mesmo externas, telhados, em tudo. Aliás, quanto a telhados, também os das casas, na cidade inteira, são de mármore, em grande maioria. Vimo-los do alto de ``Castelleto'', pertinho de nosso hotel, e onde se vai por um ascensor.

Quando ali estávamos, surgiu um ônibus especial, cheio de jovens venezuelanos, que viajavam para Viena e estavam percorrendo a cidade. Confraternizamos com eles. São 240 ao todo, dizem. Perguntam pelo número de brasileiros no festival, e ficamos em dúvida quanto ao total, pois só sabemos dos 91 a bordo, com outros 60, talvez, vindos por outros meios.

Falamos sobre o congresso de Varsóvia e nos despedimos com alegre ``hasta Viena''!

À noite pegamos o trem para Milão, Será a sexta cidade europeia a conhecer. Lembro-me de que em 6 dias estivemos em 5 cidades diferentes em tudo --- país, língua e costumes: Lisboa, Tanger, Barcelona, Marselha e Gênova.

Encontramos a delegação da Venezuela outra vez, pois viajará no mesmo comboio, que seguirá até a Áustria. Eles estão na plataforma fazendo enorme algazarra, com suas cantorias típicas e seus chapelões de palha. Na maioria, fisionomias de bandidos mexicanos. Passo com as malas no meio deles para embarcar. O dístico de minha lapela os atrai e fazem-me grande e cativante festa.

A viagem, de quase três horas, foi agradabilíssima e marcou nossa estreia em trens europeus. Achamos cara a passagem de 2ª classe: 1.000 liras (Cr\$230,00). Mas quando entramos na cabine ficamos encantados com as acomodações, com quatro assentos frente a frente, como se vê nos filmes. Limpo e magnífico estofado.

O carro está cheio e temos que guardar nossas bagagens em compartimentos separados. Ficamos maravilhados com as atenções de todos a nossa volta. Uma vez acomodados e em marcha, logo os companheiros de cabine puxaram conversa. Nosso dístico brasileiro interessa imediatamente e ensaiam perguntas as mais variadas. Dali a pouco estamos tagarelando pelos cotovelos.

Sabem muito sobre o Brasil de hoje em dia, pelo menos. Desde Gênova, no hotel, já havia verificado isso. Um jovem de 18 anos, da mesa vizinha no refeitório, mencionou logo Mato-Grosso e Brasília. Ao saber que eu já havia estado nos dois lugares, ficou mais interessado e pediu detalhes.

Na cabine está um casal com duas filhas, de uns 20 ou 22 anos, e uma conhecida deles, além de outro senhor. Todos transbordantes de simpatia. Assim tão à vontade, falamos de tudo. O esperanto tinha que entrar em cena, é lógico; todos o conheciam pelo menos de ouvir falar. Agora pedem e recebem mais particularidades. Depois, o Brasil. Folheamos pequeno mapa, em formato de revista, que trago sempre na minha maleta, para conferir os roteiros. Indagam e palpitam sobre o nível de vida, querendo comparações com a Itália. Espantam-se com certos preços no Brasil: gasolina, Cr\$9,60 (na Itália, Cr\$136,00) o litro; cinema, açúcar e outros exemplos. Mas o salário mínimo aqui é de cerca de 60.000 liras, ou seja Cr\$14.500,00, o que ameniza ou até suplanta o desnível.

Nossa viagem já passada e os planos futuros entusiasmam as mocinhas, que exclamam musicalmente ``Che belo!'' Os meios de entendimento foram os mais variados: italiano (eu só, pois o Alcino é duro de aprender; levou dois dias para ``descobrir'' que ``boa noite'' é ``buona sera''), inglês, esperanto, francês e castelhano! Dona Bianca, Maria Grazia e Gabriela, as irmãs, e Renata, a outra. Os homens não se identificaram. O pai trabalha na Esso há 35 anos e arregala os olhos quando sabe de minha licença-prêmio, pois ele só dispõe de seus 15 dias de férias por ano...

Lá pela meia-noite chegamos a Milão, aliás Milano. A viagem à luz do dia teria sido mais interessante. Nenhum panorama tivemos. Quando atravessamos o rio Pó, corri apara a janela, mas só pude divisar um tênue reflexo das águas; o mesmo com outro rio, mais adiante, nas cercanias da cidade de Pavia.

A ``Stazione Centrale'' de Milano é gigantesca e moderna, tendo escadas rolantes e porta que se abre automaticamente quando alguém se aproxima dela (célula foto-elétrica). Ficamos num hotel próximo, o ``Bernina''. Um ligeiro passeio pelas redondezas e cama.

Vamos conhecer Milano amanhã.

\section*{25 \adfflatleafright \ISODayName{1959-07-25}}
Durante a noite ouvi umas serenatas. ``Ma, questa è Itália!\ldots'' Saímos de manhã a bater perna. A impressão é diferente da de Gênova.

A cidade regula com Barcelona: um e meio milhão de habitantes. A modernização é patente. O tipo de calçamento de grandes lajes vistosamente dispostas e unidas por uma argamassa preta (asfalto ou piche) assemelha-se ao de Gênova.

Descemos pela larga ``Via Vittor Pisani'', de duas alamedas, bem arborizada ao centro. Na ``Piazza Della Republica'' ganhamos os ``Giardini Pubblici'', em direção à ``Piazza Oberdan'' nº 1, onde fica situado o ``Circolo Esperantista Milanese''. Encontramos justamente o seu presidente, Sr. Giovanni Tanzi, que nos recebe simpaticamente. A sala (e agregados) é ampla e muito boazinha, comportando até uma mesa de pingue-pongue. Palestramos um pouco, compramos uns livretos e batemos duas chapas com ele. Hoje é sábado e há reunião lá, depois das 16 horas. Prometemos voltar, se possível.

Dali damos um bom estirão até o consulado da Áustria, para visar o passaporte. Lá ocorre uma agradável surpresa: um rapaz da Líbia, também tirando visto, é esperantista. Papeamos um pouco, mas me fugiu a ideia de lhe tomar o nome.

Após, voltamos a passear. Procuramos o famoso Teatro Scala, na ``piazza'' do mesmo nome. O prédio é inexpressivo externamente. No meio da praça, não muito ampla, uma grande estátua, em bronze escuro, de Leonardo da Vinci. batemos as inevitáveis chapas. Uns alemães simpáticos fazem as tomadas das fotografias, para que eu e Alcino apareçamos juntos.

Volto a examinar de perto a fachada do teatro. O anúncio da ópera desta noite é um impresso comum, em papel bem ordinariozinho. Entusiasmo-me, porém, com o conteúdo: será levada à cena a ópera ``Carmen'', com Guisepe Di Stéfano e Gloria Lane! Um sujeito de meia idade se achega e pergunta se desejo ingresso: galeria, 1.700 liras (Cr\$425,00). Nada resolvo, mas coloco o espetáculo no programa de hoje.

Dali, pela imponente galeria ``Vittorio Emanuele'', bem junto, passamos à ``Piazza Duomo'', onde se acha a notabilíssima catedral do mesmo nome, obra estupenda, começada em 1386 e, sob a maestria do arquiteto Mengoni, terminada em 1877. Os prédios laterais são todos baixos, pois antigos; em conseqüência, a catedral, suficientemente isolada, se recorta contra o céu, belissimamente, em seu estilo 100\% gótico.

Devo acrescentar que o tempo prossegue perfeito, entra dia, sai dia.

Um popular, com quem puxamos conversa, leva-nos a visitar, também nas proximidades, uma velhíssima basílica, a de ``San Lorenzo''. Frente a esta uma estátua, também de negro e polido bronze, do imperador Constantino. Mais adiante, ruínas de 16 colunas romanas, do século II. Fotografamo-nos encostados nelas, que nos suportam bem\ldots

Em seguida, um pulo ao local em que se encontra um tanque-canal onde, em 1479, Leonardo da Vinci fez experiências com o primeiro submarino! Velhíssima inscrição na parede de tijolos noticia o fato.

Junto a essa amurada paramos para assistir um espetáculo curioso: um tipo faz o conhecido jogo das três tampinhas rapidamente trocadas de posição, para alguém acertar qual delas tem uma pinta branca por baixo. Alguns apostam, ganham e perdem. Um deles, aproveitando-se de curta (e suspeita) distração do sujeito, faz grosseira mistificação na ficha marcada, deixando-a visivelmente identificada. Parece divertido, mas a insistência do tipo em que participássemos das apostas ``dá a pinta'' de conto do vigário. Ele não poderia ser tão estúpido a ponto de não perceber a adulteração da tampinha premiada. Tratamos de dar o fora\ldots

Milano destoa, parece-me, das cidades repositórios de monumentos que possui a Itália. O esperantista do ``Circolo'' dera-nos uma magnífica planta da cidade e outros prospectos, tudo em esperanto.

De volta ao hotel, pago um bom banho\ldots Sim, 200 liras (Cr\$50,00) por uma ducha! É o regime da terra.

O cansaço é grande, pela noitinha. Atrasamo-nos nas andanças e para alcançar a ópera será preciso correr demais. Com enorme pesar, cancelo o programa.

\section*{26 \adfflatleafright \ISODayName{1959-07-26}}
Resolvi alterar meu itinerário. A viagem projetada para a Suíça iria, agora, nos apertar demasiado, face à data do Congresso em Varsóvia, muito próxima. Estamos a uma semana de sua abertura. E queremos, também, ver um pouco do Festival da Juventude, em Viena. Assim, a Suíça ficará para mais tarde.

Marcamos embarque para hoje às 12 horas, deixando Milano, rumo à Áustria, mas com escala em Veneza, programa imperdível. Além disso, o estirão direto seria muito duro: vinte horas de trem até Viena. Fracionaremos, assim, o percurso em duas etapas.

Viagem também agradável, embora o comboio seja inferior ao que nos trouxe de Gênova. Para não perder o costume, imediata conversação com os companheiros de cabine, que se interessam logo por nós.

Região inteiramente plantada, muito bonita. Paradas em Brescia, Descenzano, Verona, Vicenza e Padova, nas quatro horas de viagem. Junto à segunda, Descenzano, belíssimo lago, do qual, mesmo do trem em movimento, faço algumas fotos.

A chegada à Veneza foi um encantamento. De Mestre, espécie de subúrbio daquela, o trem caminha lentamente, como que por istmo, até a estação, que é bem ampla e agradável. Logo à porta de saída, o canal cheio de gôndolas. Que espetáculo!

Tomamos alojamento em um bom hotelzinho perto, o ``Minerva'', e saímos a passear, depois naturalmente de almoçar na primeira ``tratoria'' (restaurante popular) que apareceu. Uma carta da cidade, obtida em seguida, nos dá todas as indicações.

A cidade está repleta de turistas, compreensivelmente; na maioria, alemães, parece-me. Aliás, verifico que no verão os europeus se agitam extraordinariamente. As estações ferroviárias estão movimentadíssimas. Estudantes e excursionistas por todo lado, estes últimos com uniformes de calças curtas e volumosas mochilas às costas. Mas parecem muito felizes.

Ressalta, de imediato, o ineditismo sem par de Veneza. Ausência absoluta de automóveis, salvo na entrada da cidade, onde uma praça marca o ponto final. Junto ao leito do ferrovia, desde Mestre, uma estupenda rodovia, onde trafegam, inclusive, ônibus elétricos.

Há um grande canal que serpenteia pela cidade, que é a menor, até agora, das que visitamos: 350.000 habitantes. Deste canal saem os laterais, que são as tais ruas aquáticas, estreitas e com muitas pontes para ligar os lados. Essas ruas aquáticas dispõem, naturalmente, de calçada protegidas por grades. O pitoresco é quase indescritível.

Saímos, nesta primeira noite, para o centro da cidade, já pensando na famosa ``Piazza de San Marcos''. Toma-se os ``motoscafos'' --- espécie de lanchões, uns maiores, outros menores --- em vários pontos do grande canal.

Como hoje é domingo, deve haver algum programa especial. Foi um deslumbramento o percurso, mesmo no motor. A iluminação decorativa das margens, em restaurantes, bares e outros edifícios, dá uma impressão grandiosa. Passam, na pouca luz interna do canal-mestre, dezenas de gôndolas, pois à noite é que são mais procuradas. Algumas conduzem cantores, entoando sonoros ``Soles mios'' e ``Santas Lucias''!

A Praça de São Marcos sacudiu nosso peito. Mesmo à noite, a impressão suplantou, pelo dobro, qualquer expectativa. Ela está cercada pelo velho Palácio dos Doges, compridíssimo e extraordinariamente belo. Ao fundo, a basílica, que é de ``derrubar o queixo''. À esquerda desta, fazendo um prolongamento da praça, em ângulo reto como o pátio principal, o Palácio Ducale, outra maravilha. Unindo este ao Prigioni, outro palácio muito menor, a lendária ``Ponte dos Suspiros''.

Há uma multidão na praça, pois a cidade transborda de turistas. Mas o pátio é como que um campo de futebol calçado e sobra espaço. Ao centro uma sinfônica executa uma composição de Agamenon, conforme indica um aviso num cavalete. Na praça lateral, a orquestra normal que anima o bar. Mesas ao ar livre, em número inimaginável, nos três flancos.

Demoramo-nos bastante. Um retratista, trabalhando a carvão, faz perfis notáveis, em dez minutos. O jovem alemãozinho saiu esplêndido e a senhora de horrendas sobrancelhas, a pincel, arqueadas, estava talhada não para um retrato, mas para uma caricatura. Circunstantes amontoam-se em torno do artista, de místicos cabelos longos, que sorri permanentemente.

Imagino a praça amanhã, à luz do dia, com seus milhares de pombos esvoaçando! Antevejo os ângulos para fotografar este lugar da Itália e da Europa que, até agora, foi o que mais fortemente me impressionou. Volto enlevado para o hotel.

\section*{27 \adfflatleafright \ISODayName{1959-07-27}}
Ao café da manhã, que aqui, como desjejum, se chama ``colazione'' (no navio era ``desayuno''), temos a companhia de uma simpática e recatada morena. Puxamos conversa e ela se identifica: austríaca. Depois, aparecem três americanas, com caras de israelitas.

De ``Leica'' em punho, tocamos para o Lido, que fica no lado do mar Adriático de uma das alongadas ilhas que formam a restinga da ampla baía, em cujo centro se situa a ilha com a cidade principal. Ao lado desta, uma filhote, a Murano, onde está a famosa indústria de cristais.

Outras ilhas mais, como a San Giorgio, com sua monumental igreja e o Teatro Verde, ao ar livre. Pretendemos visitá-la amanhã, se houver tempo.

O Lido é moderno como a nossa Copacabana, no que diz respeito a urbanização e comércio. A ``Gran Viale Santa Maria Elisabetta'' é a principal artéria transversal, muito bonita, larga e arborizada. Leva do lado da baía (interno) para o lado do mar aberto.

A praia é extensíssima, mas cheia de puxados perpendiculares de madeira, que a prejudicam bastante em estética; o mar, surpreendentemente, é um copo d’água, enquanto a terra fina, cor de canela, corresponde ao que nós no Brasil chamamos de areia.s Não é pública; paga-se ingresso, paga-se cabine e paga-se barraca.

A avenida que a margeia é também muito bonita (beira-mar é ``lungomare''). Tem o nome de ``Marconi'' para um lado e de ``D’Annunzio'' para o outro. No Lido circulam automóveis. É, evidentemente, o ponto de granfinismo.

De retorno, ficamos na Praça de São Marcos. Está um esplendor! Fotografamos abundantemente. As chapas mais tentadas são as que retratam os pombos comendo na mão do fotografado. Vendem saquinhos de milho para esse fim. Esgoto os ângulos, mas o prazer de contemplar o espetáculo é inesgotável!

O comércio das redondezas é elegante e bem montado. Lojas e ruas têm seus nomes mesclados do dialeto veneziano, que sofreu, parece, influência espanhola. Há algumas ``calles'' na cidade velha. E ``gran vialles'', como no Lido.

Faço, depois, uma boa expedição de postais; escolhê-los é fácil, pois se encontram a cada passo, aos milhares. Todos maravilhosos.

Veneza pede dez dias, para senti-la, correndo principalmente suas diversas ilhas. Não nos é possível. Amanhã ao meio-dia embarcamos com destino a Viena. O bilhete, comprado em Milão, é válido por dois meses. Com ele saltamos aqui e vamos prosseguir viagem.

De regresso ao hotel, no labirinto que é, praticamente, toda a cidade insular, topamos com uma vibrante ``cantata'', entoando magnificamente na quietude do recanto e da própria hora avançada da noite. Ganhamos, rapidamente, a pontezinha para ver a gôndola passar, com meia dúzia de pessoas, o cantor e um acordeonista. Romantismo raro de século XX!

\section*{28 \adfflatleafright \ISODayName{1959-07-28}}
Enfrentaremos, hoje, dez horas de trem, até Viena, onde devemos chegar por volta das nove da noite.

A composição, automotriz austríaca, chamada na Itália de ``diretíssima'', encurtará três horas no horário comum. Deixamos Veneza pesarosos, mas o nosso primordial compromisso --- o Congresso --- precisa ser respeitado. O comboio, muito moderno, elétrico, possui apenas três carros, mas bem grandes. É veloz, por isso algo barulhento. E sacode regularmente. Em absoluto se pode escrever quando em movimento.

Momentos depois, começamos a sair da planície e a ganhar o nordeste italiano, que começa a se revelar montanhoso. Paisagem até um pouco brasileira. A região é cerradamente cultivada com bastante trigo e milho (em italiano ``grano turco''). Mais adiante, uva e campos de feno.

As elevações são pedregosas. Passamos por Portogruaro, Udine, Gemona e Tarvisio, última estação italiana. Antes, nos acercamos bastante da fronteira iugoslava, com seus também pedregosos montes. Vemos a linha divisória, os guarda-fronteiras e a bandeira vermelha, mesma cor dos bonés dos guardas.

Até Tarvisio viajamos com dois jovens interioranos italianos, um deles garoto de seus 12 anos. Já agora percorre o trem a aduana da Itália, pedindo passaportes. Em seguida, funcionários austríacos. Mas não tocam na nossa bagagem. Perguntam se se tem algo não permitido nas malas. E só.

Deduzo isso, embora só falem alemão. Lá pelas 3 e meia da tarde estávamos penetrando território austríaco. O terreno é, agora, bem montanhoso, plantadíssimo e fartamente recoberto de florestas. Não há morro pelado. Densas matas de pinheiros tipo árvore-de-Natal se estendem maravilhosamente. O panorama é lindo.

Villach é a primeira cidade austríaca. Depois Velden, às margens de enorme e belíssimo lago, que o trem contorna. Gente e automóveis em penca, banhistas e lanchas. Bati algumas chapas.

Já sinto saudades da Itália, pois os novos companheiros de cabine são dois túmulos. Nossas insígnias de ``Brasil'' despertam evidente atenção, mas permanecem emudecidos.

Um deles, gordo e calvo, sacou alguns quilos de jornais e revistas e leu quase sete horas a fio. A presilha do colarinho, sob o nó da gravata, que já foi moda no Rio, é um alfinete de fralda dourado, gigantesco. Ele solta-o para afrouxar um pouco.

Enfim, noite já escura, damos com Viena, toda iluminada. Umas duas horas antes, a Áustria dera uma demonstração extraordinária de organização no campo turístico: uma jovem uniformizada correu todo o trem, indagando dos passageiros se queria reservar acomodações, de que espécie, quantas, a que preços etc. Em menos de cinco minutos ela tagarelou conosco em inglês a situação da cidade, seus atrativos, assinalou em uma planta os edifícios principais, os programas culturais da estação etc., etc. e preencheu uma ficha com o meu nome e as acomodações que escolhemos. Adiantou-nos que estava difícil vagas em hotéis, por causa do Festival da Juventude, principalmente.

Deixou o original da ficha comigo, recomendando-nos procurar o ``bureau'' de turismo na estação. E se foi deixando em nossa mão enorme calhamaço informativo sobre Viena, em inglês e em italiano.

Da estação intermediária em que saltou, ela certamente radiografou para Viena e eles já foram tomando as providências necessárias. Quando chegamos foi só pegar a confirmação da reserva e o endereço. E assim fizemos. Um jovem português está ao meu lado no ``bureau'' e, naturalmente, aproveitou para usar um pouco o nosso gostoso idioma.

Um taxi nos deixa na pensão tipo ``Schimpach'', em ``Johann Straussgasse'' nº 34, não muito longe da estação. O quarto é ótimo, espaçoso e as camas convidativas. Não é possível pensar em sair ainda esta noite, pois já estamos próximos das 23 horas e a cidade se mostra incrivelmente ``família'' com as ruas praticamente desertas.

Deito-me relembrando o trabalho das mulheres na Europa: em Portugal, as carregadeiras de areia; antes, aliás, em Tenerife, distribuindo latões de leite, nas ruas; em Barcelona, nos serviços de limpeza, tal como em Marselha, onde o pátio do porto era varrido e cuidado por duas matronas. À saída do trem em Veneza, várias mulheres limparam os vagão. E no correr da viagem, uma delas, por duas vezes, veio varrer o carro com uma vassoura em cuja cabo estavam prosaicamente enfiados, até embaixo, dois rolos de papel higiênico\ldots

\section*{29 \adfflatleafright \ISODayName{1959-07-29}}
Logo cedo partimos para o centro da cidade, com o programa da convenção esperantista do festival em punho.

Havíamos recebido o exemplar de um dos esperantistas genoveses.

A impressão deixada pela cidade é magnífica. Calçamento, serviço de bondes, tráfego incrivelmente silencioso, lojas, enfim tudo. No fim da linha do bonde que nos levou (passagem cobrada por mulher uniformizada), damos com uma passagem subterrânea espetacular. Lá embaixo, o elegantíssimo bar ``Rondo'', ao centro, e ``boxes'' em círculo, em torno. Acesso e retorno por 5 ou 6 escadas rolantes duplas, com ``mão'' e ``contra-mão''.

O programa marca a exibição de filmes em esperanto, no ``Ring-Kino'', às 10 da manhã. Antes um pouco, já estamos lá e fazemos os primeiros contatos com os esperantistas locais. Causamos sensação, como brasileiros --- ``Oh, o Brasil, tão distante, terra belíssima, etc., etc.'' --- exclamam com entusiasmo!

Estão presentes também japoneses, búlgaros e suecos, depois encontramos alemães e um espanhol, que vive na Áustria. Os filmes foram ótimos, notadamente os coloridos realizados na Bulgária e na Hungria.

Acabada a sessão, com a sala do cinema ainda cheia, o presidente da ``Junulara Grupo'' (Grupo da Juventude) da Federação Esperantista Austríaca, nos apresenta publicamente e sou instado a dizer algumas palavras. Os búlgaros impressionam pela personalidade e pelos trajes descuidados (em mangas de camisa); um é engenheiro. Permutam comigo o distintivo esperantista (estrela verde) e deles ainda compro um livro otimamente impresso.

Saímos em animada conversa, para almoçar, não sem antes fazermos algumas fotografias na frente do cinema. Vamos ao ``Wöks'', grande e bem montado restaurante popular. A simpática Hilde quer saber muito sobre nossa viagem e o Brasil, incluindo o movimento esperantista no nosso país, que goza de enorme prestígio, como dos mais progressistas no campo da divulgação do idioma internacional.

Depois Hilde e Olga (outra esperantista) nos guiam pela cidade, para providenciarmos a passagem para Varsóvia, depois de amanhã. Surge o problema do ``visto'' tcheco no passaporte e então temos uma grata surpresa: duas funcionárias do ``Osterreich Verkehrs-Bureau'' --- órgão oficial de passagens --- falam português! Uma, a belíssima Íris, há 13 anos passados morou em São Paulo; a outra, menos atraente, fala português de Portugal, onde também residiu.

Tudo resolvido, saímos para visitar o ``Museu de Esperanto'', no ``Michaeler Burgtor''. Então conhecemos a primeira chuva fora do Brasil. Cai violenta pancada e o atraso nos faz desencontrar dos demais esperantistas. Mas visitamos o museu, após curta espera. Fica ele no topo de um edifício de monumental grupo de palácios, outrora residências imperiais.

A mostra histórica do movimento é excelente, excedendo a expectativa. Vários são os salões ocupados. Faço algumas chapas no interior. Impressiono-me com a riqueza da parte exposta de livros de todos os países e de todos os tempos, bem como de fotografias, esculturas, flâmulas, etc. Riquíssima também a parte de jornais e revistas.

Este grandioso trabalho deve-se, primordialmente, à atividade do eminente esperantista Steiner, que está a testa do museu e é quem nos recebe e mostra tudo.

Ficamos, depois, passeando pelo centro da cidade com Olga, que não é jovem, mas revela notável energia. Hilde precisou ir, mas prometeu comparecer na reunião noturna.

A Ópera, o Burgteatro, a Prefeitura e alguns outros edifícios são imponentes, principalmente o último, de várias torres, que poderiam lembrar uma igreja. Olga acrescenta que o atual prefeito de Viena é esperantista, como o é, também, o filho do anterior presidente da república, figura muito estimada, falecido há quatro anos.

Chegamos quase atrasados para a convenção; um jantar às carreiras, inclusive com a presença de Balaguer, o tal esperantista espanhol que reside em Viena. No restaurante circulam vários festivalistas. Reencontro um venezuelano com quem conversei na estação de Gênova. E uma bela patrícia sua bate um ``papinho'' na nossa mesa. Grandes companheiros, simpáticos e confraternizadores.

O programa da reunião é ótimo: uma orquestra de cordas executa uns dez números de vários países, pontificando o ``Barqueiro do Volga'', ``La Paloma'' e ``Danúbio Azul''. Depois a hora da arte, com cânticos e ``Sketches'' muito jocosos, se bem que apresentados com fortes colorações amadorísticas. Tudo em esperanto. Somente a japonezinha, ou melhor, a mais gentil das duas, canta em seu idioma e a sala explode em aplausos; ela é obrigada a dar ``bis''. Fotografo-a em plena ação, apesar da falta de claridade no recinto.

A frequência a este programa é muito maior que a da sessão de cinema pela manhã. Vemos que o esperanto tem boa base na Áustria.

Voltando a comentar a cidade: grande número de tiroleses anda pelas ruas em seus trajes típicos, com chapéu de pena enfiada, calças curtas de camurça e suspensórios do mesmo material. Espanta como se põem tão à vontade. Homenzarrões de farto ventre e pernas finas a transitar sem a mínima cerimônia, de calças curtas, meias e sapatos pela cidade. Mas ninguém repara. Só nós\ldots

Curiosidade: usa-se a aliança de casado na mão direita e a de noivado na esquerda. Sistema alemão, parece. As mulheres, além de dominarem como condutoras (cobradoras) dos bondes, são a maioria das jornaleiras também. Todas uniformizadas.

O bilhete do bonde é algo complicado. Vale por tempo percorrido ou por quilometragem, não sei bem. O caso é que o mesmo bilhete, exibido em seguida, mesmo em outra linha, ainda é válido.

Esse nosso primeiro dia vienense foi ``puxado''. Não pudemos sequer pensar em visitar os brasileiros companheiros de viagem no ``Cabo San Vicente''. Mas as emoções foram notáveis e imorredouras.

\section*{30 \adfflatleafright \ISODayName{1959-07-30}}
Alcino amanhece meio gripado e resolve ficar descansando. Saio sozinho para o encontro no castelo ``Schönbruner'', antiga residência de verão dos imperadores austríacos. É um dos pontos altos do turismo em Viena e a afluência é enorme.

A caminho estreio o ``metrô'', que não é bem subterrâneo, pois trata-se de uma tranvia (corresponde ao nosso bonde) que viaja em leito exclusivo, de nível bem mais baixo que o da rua e, vez por outra, mergulha por curtos trechos.

Viena parece não ter também problemas de transporte. O centro forma o chamado ``Ring'', um círculo de largas avenidas, dali partindo as numerosas linhas de ônibus e bondes: para um sentido, eles são designados por números; para outro, por indicação alfabética.

Voltemos ao castelo: é um espetáculo! Maior, talvez, que a nossa Quinta da Boa Vista, porém de terreno plano, ajardinado, florido. Monumentos formidáveis, fontes e outros adornos. Os imperadores sabiam gostar de coisas boas\ldots

Os esperantistas, em número de dez, que me acompanhavam, muito senhores da história pátria, naturalmente, vão dando detalhes: o palácio foi construído pela imperatriz Maria Tereza, em 1722, parece-me. Depois foi sendo ornamentado, sob outras influências, notadamente a francesa, quando Napoleão o habitou em 1809. Aqui faleceu seu filho. A ``sopa'' acabou em 1918, quando o império austro-húngaro, derrotado, separou-se e veio a república.

Falo aos companheiros sobre a curta história do império no Brasil, que durou apenas 67 anos. Paramos para descansar junto a uma fonte, onde um alemão, a esposa e o garotote refrescam mãos e pés.

Tagarelamos sobre vários assuntos. Alguns companheiros são espirituosos. Súbito, passo a ser alvo de sucessivas perguntas sobre o Brasil. Para facilitar as explicações que desejo dar, risco no chão de terra um esboço de mapa e durante uns trinta minutos faço comentários sobre a minha querida terra, intercalados, naturalmente, por apartes de um e de outro. A esta altura, até o alemão vem escutar, mesmo sem nada entender; os austríacos traduzem para ele a essência, deduzo eu. A área superior a 8 milhões de quilômetros quadrados causa espanto. O alemão berra a ``novidade'' para a mulher, que está um pouco distante, e lá vem ela também assistir a improvisada conferência do brasileiro.

Falo sobre clima, produtos agrícolas, indústria, costumes nacionais, cidades, portos, analfabetismo, rebanhos e, principalmente, como são hoje as metrópoles Rio de Janeiro e São Paulo. Aliás, conhecem bem São Paulo, como terra de café; depois, sabem muita coisa sobre Brasília, a nova capital em construção, impressões que completo com imenso prazer. A distância de mil quilômetros entre o Rio e a nova capital causa também admiração (para não dizer assombro).

O alemão, que dizia conhecer do Brasil apenas coisas lidas em algumas crônicas, está visivelmente impressionado. As tais crônicas falavam em florestas virgens e indígenas próximo ao Rio! Paralelamente, observo sua admiração também pelo funcionamento do esperanto, que ele diz conhecer vagamente. Maravilha-se como dez ou doze austríacos se entendem tão bem com a ``avis rara'' sul-americana que apareceu por ali. Já quer saber onde se estuda esperanto, quanto tempo leva e outros detalhes.

Depois ainda faz perguntas sobre o Brasil, tais como: quanto tempo tem que trabalhar um operário para poder comprar uma roupa?

Ao nos despedirmos, ele se identifica: Karl Novac, descendente de tchecos; e me surpreende revelando saber que o atual presidente brasileiro é também de origem parcialmente tcheca, mencionando-lhe até o nome! Parece-me, repito, profundamente impressionado com o acontecido.

Ao retornar ao pátio central do castelo, encontro-o juncado de festivalistas, muitos em trajes típicos. Convido um sueco, de roupagens vivamente coloridas, para se fotografar ao meu lado; ele concorda e posamos abraçados, pois este é o clima emocional do certame vienense. Depois, um grupo de 4 ou 5, com representantes de Ghana, Ceilão e Madagascar. Que grandes chapas! Rezo para que saiam boas.

Diviso americanos, russos e diversos de países árabes, como Jordânia, Síria e Arábia Saudita. O rapaz do Ceilão pediu-me uma moeda brasileira; como não tinha, presenteei-lhe com uma cédula de cinco cruzeiros, que ele recebeu satisfeitíssimo\ldots

Quando volto à pensão, verifico que o Alcino havia saído sem me esperar. Aliás, dei-me mal nesta viagem de volta, pois tomei o bonde em sentido contrário e só no fim da linha percebi a ``mancada''. Preocupo-me um pouco com o patrício, que, como já mencionei, nada fala além do desconhecido português.

Almoço sozinho no ``Rondo'' e, com satisfação, vejo aparecer a primeira brasileirinha da delegação ao festival: Yone Cozza, a santista, e a que eu mais queria encontrar mesmo. Almoçamos juntos e ainda com a companhia do rapaz que está com ela, também patrício. Obtenho o endereço do acampamento e prometo aparecer amanhã. É lá pelos lados do Danúbio, que ainda não vimos. Por coincidência, ao deixá-la, esbarro com outro brasileiro e companheiro de bordo: o Dr. Maurício.

Dai a pouco, nova chuvarada. Perco o encontro das três horas com os esperantistas. À noite, também não os achamos no endereço dado. Alcino deve ter se enganado ao anotar. Parece que vamos partir amanhã, sem nos despedirmos dos que ficam. A menos que compareçam ao nosso ``bota-fora''.

Voltamos ao ``Rondo'', lá para as 9 da noite. O lugar é uma atração irresistível e o centro mais procurado, em geral. Havíamos dado outro azar: o cinema em que se anunciava a exibição do filme ``Hungria en llamas'', como programa do festival, às 20 horas, exibe outra coisa. Outro equívoco, esse do programa impresso. E foi um custo nos entendermos com os funcionários do cinema. Diga-se de passagem que, na rua, poucos falam inglês ou francês, ao contrário do que eu supunha. Tem sido verdadeiro sacrifício perguntar alguma coisa, principalmente sobre localização de ruas ou condução. E como são fracos de mímica!

No ``Rondo'' quase que só festivalistas. O vienense, ao que parece, é terrível caseiro. Desaparece das ruas quando a ``noite ainda é uma criança''. Assim, as ruas estão bem desertas e só os turistas se espalham aqui e ali.

Ao nosso lado, no delicioso e elegante bar subterrâneo, um grupo de uns 6 ou 8 japoneses. Vêm o nosso ``Brasil'' no peito; um deles passa para a nossa mesa, com um copo de cerveja na mão. Eu tomava leite. O rapaz rumina um hesitante inglês e me entrega o copo de cerveja, claramente com objetivo confraternizador; resultado: não tenho outro jeito senão bebericar, o que fiz, aliás, com prazer e emoção, tocado pelo clima elegante do festival. Devolvo-lhe o copo, ``papeamos'' um pouco (ele parece meio ``alto'') e ganho, em seguida, um símbolo de Hiroshima, que ele mesmo espeta na minha lapela. Empurra-me outro gole, ainda de permeio com o meu leite, e se vai.

Depois passeamos pela magnífica área de circulação, rodeada de atraentes lojas, com artísticas e curiosas vitrines. Na decoração destas é frequente uma fileirinha de pequenas bandeiras de vários países. E não raro lá está a nossa.

Duas jovens sorriem simpaticamente e, lógico, nos aproximamos: uma local, a outra soviética, louríssima, de nome Nina. Diz que só fala alemão, além do russo, mas depois vejo que solta frases em francês e até em espanhol. Como dizemos que pretendemos aparecer em Moscou, dá-nos sem rebuços o seu endereço, oferecendo-se para mostrar a cidade.

Depois, quando conversava, em inglês, com dois austríacos bem expansivos, passam numerosos soviéticos. Louros e morenos, descuidadíssimos de trajes e cabeleiras. Seus distintivos são bem vistosos e destacados nas lapelas. Gostaria de pedir um. Mas quando nos despedimos dos austríacos, já não mais encontramos os russos.

Falamos também com seis rapazes moreníssimos do Kuwait, que ocupam duas mesas.

Viena está um monumento de internacionalismo. Mas parece que o povo olha reservadamente para os festivalistas, considerando todos comunistas. Um jornal diário, editado em vários idiomas e distribuído gratuitamente, ataca fortemente o festival. Chama-se ``Notícias de Viena''.

\section*{31 \adfflatleafright \ISODayName{1959-07-31}}
Embarque para Varsóvia, finalmente. Fomos buscar nossos passaportes no ``bureau'', com o ``visto'' tcheco, que nos custou 60 ``shilings'' (moeda austríaca). Almoçamos mais cedo, no ``Wöks'', e partimos para a estação. O trem sairia às 12h20m, para chegar, segundo dizem, às 5 da manhã seguinte a Varsóvia --- 17 horas de viagem!

Os esperantistas comparecem em massa à estação. Tomo os endereços de Hilde e Olga, para mandar saudações e fotografias, mais tarde. Foram excelentes companheiras, gentis e prestativas. Aliás, Olga embarca também conosco, para participar do Congresso.

O vagão vai lotado de esperantistas, tanto austríacos como outros, vindos de mais longe: franceses, italianos, iugoslavos e uma sueca, para só falar dos que conhecemos e com os quais conversamos.

Viajamos na cabine com cinco moças naturais da Sardenha, da cidade de Nuoro. São alegres e comunicativas --- pudera, italianas! A viagem foi, assim, agradável desde o início, por esse lado. Mas o tempo esteve fechado e chuvoso. Na fronteira tcheca, ou melhor, numa estação antes, paramos cerca de uma hora, para as formalidades: revisão de passaportes e declaração de moeda e objetos e valor. Mas não foram revistadas quaisquer bagagens. Uma funcionária tcheca, fardada, chama a atenção, pois é belíssima.

A primeira cidade da Tcheco-Slováquia foi Broclav. A estação está embandeirada, com saudações ao festival da juventude. Depois, a caminho. As paradas são rápidas, nem se pode saltar para tomar qualquer coisa; aliás, nada há para comprar, nem café existe. Vejo que facilitei no tocante à alimentação e a fome começa a apertar. Felizmente as esperantistas da Sardenha trouxeram fartura e fazem-nos participar das suas boas comilâncias.

Observo o território tcheco. É profusamente plantado e arborizado. Vejo grandes florestas de pequenos pinheiros, coisa recente, portanto. Milho, algumas vezes, a perder de vista. E bastante trigo e feno (ou coisa parecida).

Já ao anoitecer, o trem fez uma parada mais longa. Saltamos para tentar comer ou beber. No restaurante da estação, repleto de gente tomando cerveja, eu e Alcino chamamos a atenção. Na hora de falar em dinheiro, só queriam a moeda tcheca, coisa que não tínhamos. Por fim, aceitam minha bela prata de 10 ``shilings''. Peço pão, com gestos, e recebo enormes fatias de um de cor escura, e um copo de sifão, ao invés de cerveja, que não aceitei.

Depois, um dos garçons encheu uma grande bandeja com os tais enormes copos de cerveja e correu ao longo do trem, na plataforma, oferecendo a bebida aos passageiros. Se queriam retribuir com cigarros ou outra coisa, ele aceitava.

Enfim, a fronteira polonesa. Nova longa parada. Mas, também aqui, as bagagens não são tocadas. Os funcionários são gentis e sorridentes. Alcino faz furor na cabine, ao contar seus volumosos ``traveller-checks''.

Uma hora de arte, iniciada de há muito na cabine, prossegue. Sou solicitado a apresentar a música popular brasileira e, assim, forçam-me a dar um pequeno ``show'' dos nossos diferentes rítmos, que todos apreciam e (muito gentis) aplaudem.

Finalmente, dia claro já, quase às 6 horas da manhã, o trem entra lentamente na estação de Varsóvia.
