\addchap{Agosto}
\section*{1 \adfflatleafright \ISODayName{1959-08-01}}
A estação é acanhada. Chove fininho. As plataformas estão enfeitadas com bandeiras polonesas e do esperanto. O comitê polonês de recepção e alojamento está a postos e funciona com eficiência. O seu chefe é um senhor baixinho, de meia idade, falando excelentemente o esperanto e determinando tudo com muita objetividade. A organização impressiona muito bem.

Seguimos de ônibus especial para a sede de uma academia estudantil, onde ficaremos alojados. Custará baratíssimo: apenas três dólares por dia, com direito a desjejum e jantar.

O almoço, fora. É que as atividades do congresso não irão permitir que se venha almoçar na academia. Sinto-me cansadíssimo, mal alimentado e sem dormir toda a noite da viagem.

A distribuição do pessoal é demorada. Saímos antes para comer alguma coisa num restaurante em frente. Chove ainda e a praça, que tem forma circular, está toda em reforma. A lameira é terrível.

Finalmente instalados, caio na cama. Alcino, indócil por correspondência, vai ao Palácio da Cultura (sede do congresso), para ver se há alguma. Lá para as 4 da tarde, bem recuperado, almoço no mesmo restaurante fronteiro, cujo sistema é o popular, em que se apanham os pratos e copos num balcão, mas\ldots não fornecem bandejas. Os poloneses parecem-me caladões. Olham com curiosidade, pois o dístico ``Brasil'' não sai da minha lapela.

Alcino traz-me um cartão de Lisboa, do amigo Noronha, que viajou conosco no ``Cabo San Vicente''. Rapaz gentil. Estranho que não haja notícia de casa, nem da Cooperativa, que ficou de enviar revistas.

Saímos para ir à casa de uma senhora, para a qual Alcino traz encomenda de pessoa amiga (um padre holandês de nome Alberto), residente no Rio. Lá encontramos um jovem esperantista e um senhor que fala um pouco de espanhol. A residência, que fica na ``Ulica (rua) Mokotovska'', é medianamente confortável. Quando chegamos, antes de entrar, uma dúzia de crianças, no pátio do edifício, nos cercou animadamente. Crianças\ldots como em toda a parte.

Outra visita aparece: Madame Maria, que fala francês (viveu em Paris, entre as duas guerras). A ``salada'' de línguas é grande, mas com o esperantista presente o entendimento está garantido. A dona da casa, Dona Bárbara, que se desmancha em amabilidades, é alta funcionária do Palácio da Juventude e professora. Convida-nos para, quando pudermos, ir visitar Zelazowa Wola, a vila em que nasceu o compositor Chopin.

À noite, no Palácio da Cultura e Ciência, que me dá espetacular impressão --- imenso e luxuoso --- tem lugar uma \textit{soirée} dançante, no conjunto do programa semi-livre que é o do interconhecimento entre os congressistas, em número de 3.200. Somos a todo momento abordados por esperantistas, para saberem sobre o Brasil. Vejo, então, que os progressos do movimento em nosso país, noticiados com freqüência e destaque pela imprensa esperantista mundial, estão bem presentes na Europa. Desde Barcelona, aliás, eu já havia verificado isso. Agora, aqui, tocam sistematicamente no assunto, com ênfase e simpatia, que recebemos com inocultável orgulho.

Estou atento à minha correspondente belga, Josée Hendrickx, com quem troco cartas em esperanto, para praticar, há cerca de sete anos. Ela informou que participaria do congresso e, assim, iremos nos conhecer pessoalmente. E, de repente, quando eu conversava em um grupo numeroso, eis que alguém me puxa pelo braço e indaga sorrindo: ``Você não é o Othon Baena?'' (em esperanto naturalmente). Foi uma satisfação e uma surpresa. Satisfação por conhecê-la assim ao vivo (não quanto a fisionomia, pois já havíamos trocado um monte de fotografias). E surpresa, porque ela pronunciou meu nome corretamente.

Infelizmente, estou afônico, por causa da garganta afetada, o que se agravou com a cantoria no trem, além de ter que falar alto, quase gritando, aqui na festa, pois o barulho é infernal. E infelizmente, também, ela não está só, mas com um grupo de marmanjos suecos. Dançamos um pouco, mas a pista está por demais concorrida. Com surpresa, observo que o repertório da orquestra é quase só de ``swings'' e ``rocks''. A lourada saltita indócil!

O interior do palácio, apesar dos amplos salões e corredores, é um tanto abafado e sem ventilação.

Mais tarde, Josée se dispõe a sair com o seu grupo para um clube noturno, o ``Krokodilo''. Lamento não poder acompanhar. Estou sem voz e, àquela altura, cansado outra vez para tornar a não dormir durante a noite.

Como já é quase uma hora da manhã, os bondes sumiram. Encontro, no ponto de parada, as mocinhas da Sardenha e outra polonesa, amiguinha delas. Partimos, de braços, a passo rápido, para a academia, que é a nossa casa. Um idoso holandês vem também, mas não aguenta a passada, coitado, e se fica pelo caminho.

\section*{2 \adfflatleafright \ISODayName{1959-08-02}}
O início da sessão solene de abertura do congresso está marcado para as 9h30m. Estou entusiasmado com o Palácio, agora que o conheço à luz do dia. Ele está situado no centro de amplíssima praça, muito bem ajardinada, andando-se um bom pedaço do bonde até a sua entrada, dotada de belas escadarias e colunas.

Da academia até lá, de bonde, gastam-se uns dez minutos e o ponto final de várias linhas é justamente na praça do nosso alojamento. Ganhamos, como participantes do congresso, um passe grátis para transporte em bondes ou ônibus.

O salão anfiteatro da inauguração é um mimo. Senti um arrepio na espinha ao penetrá-lo. Imenso, comporta todos os três mil e tantos congressistas. A abertura é imponente. Orquestra seletíssima executa alguns belos números. Depois corre a cortina do descomunal palco e surge a mesa completa, notavelmente decorada. Presente o Ministro da Educação polonês, que, após a apresentação inicial, pelo presidente da Associação Esperantista polonesa, fala no seu idioma pátrio, com intérprete vertendo para o esperanto. Discursam, em seguida, os representantes oficiais dos diversos governos. Noto que o nome do Brasil é citado duas vezes nos improvisos iniciais, quando se mencionaram nomes de países que se fazem representar.

Nosso representante oficial é o Dr. Azeredo Coutinho, médico de Belo Horizonte, que fiquei conhecendo numa convenção realizada em Macaé, em 1955. Com orações breves, falam os representantes de todos os países, causando especial entusiasmo os do Japão, China, Indonésia (mulher) e República Árabe Unida, este em trajes típicos. A nota de bom humor mais acentuado, em sua fala, é dada pelo idoso representante norte-americano.

Vocifero por não ter trazido a máquina fotográfica e, assim, perder grandes e inesquecíveis flagrantes. Mas na sessão de encerramento espero compensar. Nota negativa da abertura: ao ser executado o hino esperantista, o coro foi fraquíssimo, demonstrando que a maioria não sabia a letra de cor.

As relações amistosas se fazem a cada segundo. Trocam-se saudações, lembranças e endereços. O húngaro Miklós Abraham pede-me a indicação de um correspondente no Rio. Anoto e, em seguida, ele insinua que seja eu próprio. O presidente da Cooperativa Esperantista da Bulgária também faz empenho em manter intercâmbio com a nossa congênere no Rio. Atendo a todos, como possível. E prometo fazer tudo, quando estiver de volta, para que essas propostas se concretizem.

Depois do almoço, saio com a belga Josée, os suecos e o correspondente dela em Varsóvia, Stanislaw, jovem simpático, inteligente e que fala um esperanto de entusiasmar. Passeamos por alguns bairros pitorescos.

À noite tem lugar a representação teatral de ``Pigmaleão'' de Bernard Shaw, por um grupo tcheco, de Praga. A peça é algo arrastada e os cenários muito surrealistas (só alguns móveis, sem fundo de cena) e diversos atores falam um tanto baixo.

Caindo de sono e cansaço (ando me estranhando), saio antes do final.

\section*{3 \adfflatleafright \ISODayName{1959-08-03}}
Resolvi tirar o dia para descansar o máximo, pois o programa do congresso, hoje, favorece. Durmo até um pouco mais tarde e me fico na ``Studenta Domo'' (a Academia, em esperanto), nosso alojamento, até a hora do almoço, escrevendo e tratando de alguns outros afazeres.

Mesmo sem ser o que esperava, a acomodação vai satisfazendo, se se levar em conta o preço baratíssimo. Teríamos bons hotéis também por preço barato, mas a vantagem da academia é estar no meio dos esperantistas de tudo quanto é lugar, dia e noite, pois lotamos os oito andares do alojamento. Mas o prédio não tem banheiro! (esses europeus!). Temos que tomar banho fora, em uma casa especializada. Dona Bárbara, a propósito, nos ofereceu os banheiros do Palácio, mas não quisemos incomodar.

A comida, com os pratos servidos no balcão, é boa e substanciosa, especialmente a refeição da manhã, verdadeiro almoço. O chá, servido em um bruto copo de quase meio litro, é delicioso, de cheiro e gosto. O problema para mim é comer fora e então fazer o pedido, como já aconteceu algumas vezes. A língua polonesa é terrível, não se parece com nenhuma ocidental. Exceto, felizmente, quanto a umas poucas palavras, como ``zupa'' (sopa) e ``makaronen''(macarrão).

Hoje, por exemplo, fomos almoçar num restaurante algo elegante, onde até se senta. Os populares se servem em pé, como já disse, e o principal deles é o \textit{Praha}, mais ou menos próximo ao Palácio. Vimos uma faceta da população mais seleta. Um esperantista local, que outro não é senão o chefe do comitê de recepção na estação, no dia da nossa chegada, acompanhou-nos até o restaurante e fez as pedidas, para nos auxiliar. Chama-se Józef Arszenik e disse ser ferroviário. Quando ele se foi, pois estava com pressa, pedimos água e veio cerveja, que é \textit{pywa} (engano da mocinha sentada à nossa mesa e que, desejando ajudar, complicou). Para vir água precisei mostrar a água da jarra de flores, molhar o dedo e fazer gotejar. Senão eram capazes de trazer alguns cravos.

Ainda a respeito do esperantista nosso acompanhante, causou-me enorme surpresa, para não dizer espanto, quando ainda no nosso ponto de encontro ele começou nossa conversa dizendo que era espírita. Como ele ``descobriu'' que sou espírita, para poder ter sentido sua revelação? Mistério completo.

À noite tem lugar o baile do congresso, logo em seguida ao banquete monumental. Ficamos de fora deste, por dificuldade de inscrição à última hora. A orquestra é melhor que a de sábado. Os brasileiros continuam despertando interesse e simpatia cativantes. ``Do Brasil? de que cidade? --- perguntam todos. Ao respondermos ``Rio de Janeiro'', um ``oh!'' é exclamado, seguido de ``grande e bela cidade!'' É uma satisfação verificar- se que nossa querida pátria já não é o gigante desconhecido de outrora.

Encontro Josée e com o seu grupo fico até o fim, o que ocorre lá pelas três e meia da manhã.

O ponto pitoresco da noitada foi o ``show'' dado pelos búlgaros durante o baile. Com roupas características, apresentaram danças regionais interessantíssimas. Não me lembro de ter visto coisa igual em matéria de ``folclore''. Amanhã farão um programa oficial, às 20 horas, no anfiteatro do Palácio.

O regresso à academia é feito de novo a pé, sem outra alternativa.

\section*{4 \adfflatleafright \ISODayName{1959-08-04}}
Varsóvia, como cidade, já nos proporcionou as observações mínimas para um juízo razoavelmente formado. Considerando-se que ela foi quase totalmente destruída, o que se vê, apenas 14 anos após o fim da guerra, é notável, quase incrível. Mas os sinais do conflito ainda são bem visíveis. Várias ruínas encontram-se aqui e ali. Mesmo em grande terreno fronteiro ao magnificente Palácio da Cultura, um edifício de uns 8 andares só falta desabar com um espirro um pouco mais forte; paredes, tetos e lages jazem intactos desde algum tenebroso bombardeio que os destruiu. É evidente que a intenção é deixar o estrago bem vivo a mostra, como lembrete dos horrores da guerra para as novas gerações.

Do último andar (30º) do Palácio a vista é estupenda, em todas as direções.

Fiz várias fotos. Não constroem edifícios mais altos que de 10 pavimentos (o Palácio é capítulo à parte, pois trata-se de presente dos russos e a inscrição na fachada, ao alto, menciona ``Jozeph Stalina''). O gabarito comum é de 6 a 8 andares, o que dá agradável sensação de desafogo, arejamento e espaço útil.

As duas avenidas centrais --- \textit{Marszalkowska} e \textit{Jerozolimskie} (Jerusalém) --- são larguíssimas. A reconstrução pôde melhorar a urbanização. Lá do alto do Palácio vimos o rio Vístula, com algumas belas pontes. Amanhã vamos passear por lá, com um oficial polonês que fala português e nos abordou na rua. Muito gentil e inteligente. Chama-se Adamski.

O prédio da ``Studenta Domo'', em todos os lados, está crivado de furos de metralha. Em uma rua, deparo com um monte de flores sobre certo modesto monumento de bronze, junto a uma simples parede. Explicam que naquele local os alemães fuzilavam a granel. As flores são, até hoje, renovadas quase diariamente.

Converso com um jovem polonês sobre salários. O médio anda por 1.500 \textit{slotis} (pronuncia-se \textit{zuótis}, a moeda polonesa, que se divide em 100 \textit{groszis}). Os preços de alimentação, transporte e diversão parecem bem baixos. Bonde: 50 \textit{groszis}. Boa refeição nos restaurantes populares: 8 a 10 \textit{slotis}.

Quanto aos bondes, parece que uma suposta convenção intereuropeia estabeleceu um padrão de cor (palpite meu): são estreitos, pintados a ``saia-e-blusa'', vermelho em baixo, creme em cima. Os ônibus têm a mesma coloração.

Almoço a sós com Josée, num restaurante bem elegante, que ela própria escolheu: o ``Kandelabry'', na ``Plac Konstytucji'' (Praça da Constituição). Comida fracota, insuficiente e cara, desta vez. A verdade é que a pedida esteve também a cargo de minha convidada. Valeu pelo longo e agradável ``papo'' com ela.

À noite tem lugar o programa artístico da representação da Bulgária, que agrada em cheio. Começou com uma peça teatral, denominada em esperanto ``Casantoj de Oficoj'' (``Caçadores de Empregos''), do laureado autor búlgaro Ivan Vazov, hoje clássico da literatura daquele país. Foi notável, sob todos os pontos de vista, o espetáculo, com um homogêneo elenco de 12 atores e esplêndido esperanto. Lembre-se que já existe em Sofia o ``Teatro Búlgaro Esperantista'', composto de atores e diretores profissionais.

Em seguida, extraordinária apresentação de danças típicas, folclóricas. A música, se bem que rápida, enfada um pouco, mas a dança é formidável, quase sempre em grupos de quatro, homens ou mulheres. Um primor de harmonia, ritmo e colorido de vestimentas.

Na saída, tempo chuvoso. Apanho um táxi, juntamente com três esperantistas poloneses. Ao chegarmos eles não me deixam entrar na natural ``vaquinha''. --- ``Você é nosso hóspede'', dizem gentilmente.

\section*{5 \adfflatleafright \ISODayName{1959-08-05}}
Resolvo trocar o projetado passeio a Byalistok, cidade natal de Zamenhof, o criador do esperanto, pela visita a Zelazowa Wola, berço de Frederic Chopin, o genial compositor. O motivo se prende à distância e ao fato de não existir mais naquela cidade a casa em que nasceu Zamenhof, destruída que foi durante a guerra. Fica próxima à fronteira russa. Assim a excursão seria por demais longa, cansativa e desinteressante.

Marcamos um encontro com Dona Bárbara no Palácio da Juventude, que é a parte traseira do mesmo Palácio da Cultura, para confirmar nossa aceitação ao convite que nos havia feito: realizar o referido passeio de automóvel, no próximo sábado.

Mas o dia reservado às excursões (três programas diferentes) é amanhã, quando não haverá atividade no congresso.

Depois do almoço, com o búlgaro Dargov, vou ao encontro marcado com Josée e seu grupo, para ouvirmos, na casa de seu correspondente Stanislaw, os discos que eu trouxe para ela de presente: Dolores Viaja'', em que Dolores Duran canta Coimbra'' em esperanto, em versão de minha autoria feita há quatro anos atrás; Uma Noite no Arpege'', de Waldir Calmon (pianista) e Acadêmicos do Salgueiro'', desta escola de samba.

Foi uma tarde memorável, inesquecível! Um milagre que só o esperanto tornaria possível. Éramos oito convidados: eu, Josée (belga), Bertil (sueco), Humphrey Tonkim (inglês), Dieter Betche (alemão de Berlim ocidental), Weber Bernd (alemão de Leipzig, zona oriental), Ewa Foltarz (polonesa de Czestockowa) e Dargov, o búlgaro. O anfitrião mora na zona suburbana, a quase uma hora do centro, de bonde. A casinha é agradável, poucas habitações nas redondezas, onde proliferam abundantes plantações. O prédio tem as paredes externas recobertas por parreiras, já bastante carregadas. Que inédito espetáculo!

Durante horas palestramos animadamente, sem que as sete nacionalidades presentes constituíssem qualquer barreira. Dieter, o berlinense gordinho, mostrou notável bom humor, passando até o brasileiro para trás. O britânico também pontificou. Todos jovens, só o búlgaro parecendo um pouco mais desgastado que eu, talvez. A novíssima esposa de Stanislaw e a senhora mãe deste são também esperantistas e falam muito bem. Participam, pois, das conversações com graça e desembaraço. Apenas Wanda, loura e linda irmã de Stanislaw, é novata em demasia na língua internacional e fala mais com seu estupendo par de olhos azuis, ora dirigidos para o inglês, ora o alemão de Leipzig e\ldots o sul-americano.

O búlgaro levou uma tremenda aguardente, que figurou na magnífica mesa que nos foi servida, com doces, frutas e diversas bebidas. Faço feio neste \textit{métier}, ganhando, então, um copo de perfumado chá quente. Depois, um vinho bem licoroso, feito com as uvas da casa. Aí, sim, ``entornei'' um pouco.

Os discos que levei agradaram em cheio, especialmente o da escola de samba, cujo rítmo sacudiu todo mundo.

Fizemos inúmeras fotos, antes e depois do lanche. Serão das mais gratas desta viagem.

\section*{6 \adfflatleafright \ISODayName{1959-08-06}}
O congresso estará sem atividades, por ser, como disse, o dia reservado para excursões. Como a nossa será privada, ficamos com o dia livre. Alcino foi ontem a Gdánsk (antiga cidade de Dantzig), com uns amigos húngaros.

Fiquei, em parte porque minha garganta continua mal, passando agora a chiar o peito e algum pigarro. Por outro lado, se tivesse ido perderia a reunião maravilhosa na casa de Stanislaw.

Resolvo ir ao médico do Palácio. Jovem e gentilíssimo, fala bem italiano, pois esteve dois anos na Itália, durante a guerra. Fala também um pouco de inglês. Nesses dois idiomas me explico e ele, após breve exame, diagnostica uma branda bronquite. Bonito\ldots O doutor preenche a receita e, entre mesuras, pede-me que volte amanhã. A consulta é grátis e a receita custa só 8 \textit{slotis} numa \textit{apoteka} (farmácia). A medicina aqui é socializada. Paguei o equivalente a 30\% do valor dos medicamentos, por ser estrangeiro. Depois vejo o nome dos preparados: ``pyramidon'' (pílulas), ``isochin'' (comprimidos) e xarope à base de ``gwajakol''; para o nariz, gotas de óleo gomenolado. Como é internacional a medicina! Detalhe: a farmácia não embrulha os remédios. Saio com eles na mão, distribuindo-os depois pelos bolsos.

Almoço, hoje, no mesmo restaurante do primeiro dia, fronteiro à ``Studenta Domo''. Tornei-me admirador da cozinha polonesa, bem substanciosa e leve. Não posso é variar muito, pela dificuldade de pedir. Isto, bem entendido, quando estou sem a companhia de poloneses. Assim, tome ``zupo makaronen'', ``mlemko'' (leite), ``bulzka'' (pão pequeno) e ``punding'', que se pronuncia com acentuação na última sílaba; é a sobremesa fixa.

Meu polonês, fora daí, é de ``tak'' (sim), ``nie'' (não), ``djinkue'' (obrigado), ``dindobre'' (bom dia), ``dobre notse'' (boa noite), ``ulsemer'' (último) --- para dizer o andar ao ascensorista, na academia --- ``woda (água) e pouca coisa mais. Aliás, nestes últimos dias os ascensoristas já mostraram ter aprendido em esperanto os números indicativos dos pavimentos. Naturalmente, de tanto ouvir.

Passo a tarde pondo a correspondência em dia, o que foi bom para me desobrigar devidamente.

No jantar, durante hora e meia ``papeio'' com o amigo húngaro Miklós Abraham, Lídia, a polonesa com cara de brasileira, e outro polonês camaradão, um dos que me pagaram o táxi, há dias. Os assuntos são variados, inclusive discos voadores. Depois, influenciado pela presença do húngaro, levo a conversa para a origem dos ciganos e sua fabulosa música. Miklós confirma a origem egípcia daquela curiosa gente e revela medidas do governo húngaro para amparar os nômades e divulgar melhor a música e dança zíngaras. O resultado, diz ele, em parte falhou na finalidade, porque numerosos ex-ciganos de perambulação não gostaram de perder a vida largada.

Para me poupar da friagem, que à noite é bem sensível, vou dormir mais cedo, depois de ler bastante o livrinho sobre a Polônia que nos foi distribuído, por sinal muito bem impresso e com textos em polonês e esperanto, além de gravuras e mapas históricos.

Esquecia-me de dizer que, antes do jantar, recebi a visita, em meu quarto, de Adamski, o polonês oficial da reserva que fala magnífico português. Está ``seco'' para trocar dólares conosco. Em sua companhia veio um polonês que só fala a própria língua. Embora o brilhante Adamski seja também esperantista, conversamos em português, pois ele quer praticar. Dali a pouco, o parceiro, que por motivo linguístico ficou totalmente fora da conversa, começou a dar um ``show'' de falta de educação, bocejando com enfado e cortando seguidamente nossas falas para propor ao outro irem embora. Adamski não traduziu, mas só podia ser\ldots

\section*{7 \adfflatleafright \ISODayName{1959-08-07}}
Alcino chegou bem cedo e me acordou em péssima hora. Aproveito para lhe mostrar meu esboço de itinerário a partir daqui. Começo a me desinteressar pelo pulo a Moscou, já que até agora não tivemos chance de tomar qualquer providência. Vamos ver hoje à tarde.

A convite de um grupo de amiguinhas de Lídia, saímos para um passeio ao parque mais bonito da cidade. Imenso, deixa longe nossa Quinta da Boa Vista. Fotografamos a valer, pois são quatro máquinas a operar. Além de Lídia estão Renata e Zenóbia (polonesas) e o simpático Wlodzimiers, que, ao lhe perguntarmos, quando saímos, se era polonês (pois falávamos esperanto e não sabíamos), respondeu: ``bedaurinde'' (``in\-fe\-liz\-men\-te'').

Zenóbia, morena, baixinha e balzaquiana, bate chapas aos montes, com sua máquina de 36 poses. As últimas foram colhidas diante do maravilhoso monumento a Chopin, O parque se chama ``Lazienski''.

Detalhe pitoresco: as meninas pagaram nosso lanche, no bar do parque, e depois, na cidade, pagaram também o nosso almoço, pois, no momento, não possuíamos mais dinheiro polonês cambiado. Ficamos de reembolsá-las mais tarde.

Pouco depois, no congresso, assisto à terceira representação teatral, a cargo de outro grupo, este internacional. A peça é a célebre ``Patifarias de Scapain'', de Molière, comédia passada na Itália.

Apesar de algumas boas cenas cômicas, não me agradou; os atores falaram muito depressa e não muito alto, fazendo-nos perder bastante da dialogação. Além disso, o cenário era por demais sumário. Por fim, a própria história é bem fracota, por elementar na pobreza de espírito de certos personagens, para favorecer a comédia. Mas, como teatro esperantista, cumpriu a finalidade. O búlgaro, porém, foi o melhor de todos, sem nenhuma dúvida.

Enfim, recebo a primeira carta: do Brasilo, colega do IRB. Apesar de restrita a assunto de sua revista hidroviária, foi uma satisfação imensa.

Depois, não sei se por canseira de tanto andar e comer fora de hora (e geralmente pouco), sinto estranha sensação de fraqueza, coisa que comigo é muito rara. Fico preocupado, a ruminar dúvidas. A bronquite ainda não cedeu.

Venho jantar na ``Akademicka'' e cancelo o programa da noite, para facilitar a recuperação. Os remédio fizeram algum efeito, pois a expectoração cedeu um pouco, pelo menos.

Amanhã ocorrerá o encerramento do congresso. Uma lástima, mas o tempo passa e a intensidade do programa ajudou a levar tudo de roldão, celeremente.

\section*{8 \adfflatleafright \ISODayName{1959-08-08}}
Saímos cedíssimo para o passeio à Zelazowa Wola, onde nasceu Chopin.

Vamos com Dona Bárbara Horowicz, a extraordinária diretora do Palácio da Juventude, que não sabe mais o que inventar para nos agradar. O seu diretor, Josef Mikotajevicz, também nos acompanha e dirige o carro, que é do Palácio, um ``Warszawa'' (Varsóvia, em polonês), de seis lugares, muito bom. Josef é bem moço (uns 35 anos) e já ocupa cargo importantíssimo. Simples e gentil, como temos vistos poucos. Vem, ainda, o jovem esperantista que conhecemos na primeira visita à Dona Bárbara e que se chama Janusz Bielevicz. Figura preciosa, pois será o intérprete, visto que os nossos amigos só falam sua língua pátria.

A viagem é relativamente rápida: uma hora. Deixando Varsóvia, duas coisas chamam minha atenção: a excelência da estrada e a extensão dos campos de verdura, em torno da cidade. Um autêntico ``cinturão verde'', nome tão repetido no Rio, mas cada vez mais irrealizável, com a proliferação da praga dos loteamentos.

A vila onde nasceu Chopin é verdadeiro solar, de imensa arborização circundante e quietude quase paradisíaca. Casa de rico, como de fato bem aquinhoada era sua família.

Percorremos a casa, não muito espaçosa por dentro, localizada no centro do tranqüilo parque. Móveis e utensílios da época, inclusive dois pianos e um exemplar precursor do clavicórdio. A parte de cima da casa está fechada hoje. São outros cômodos. Faço muitas fotografias, como todo o grupo.

Dona Bárbara culmina nos presenteando com uma lembrança local, na forma de um jogo de papel de carta e envelopes timbrados com o vulto de Chopin e a casa, e ainda nos dá alguns postais.

Voltando a Varsóvia, almoçamos no Palácio, convidados pelos incríveis anfitriões. Antes, dei um pulo no médico, para saber como iam as coisas com o meu renitente mal-estar, felizmente um pouco arrefecido. Ele me tranquiliza, mas me joga uma injeção de penicilina. Menos mal.

Dona Bárbara quer que nos mudemos para o Palácio, pois a nossa hospedagem na ``Studenta Domo'' termina hoje, com o fim do congresso. Relutamos, a princípio, mas aceitamos e logo tomamos possa da chave do compartimento. Será outra coisa, com ``big'' banheiro ao lado! Marcamos a mudança para amanhã cedo.

Em seguida, assentamos providência para a viagem a Moscou. Josef tem conhecidos e se prontifica, pelo telefone, a sondar preços e o preparo de ``visto'' no consulado soviético. Em 15 minutos tudo se esclarece, mas hoje é sábado e só segunda feira poderão ser concretizadas as medidas.

Um louro rapazola, de uns 13 ou 14 anos, é chamado para servir de intérprete agora, pois o esperantista já se fora. O menino fala francês com um engraçado ar de gente grande. Assim, parece certa nossa ida a Moscou, dentro de dois ou três dias.

Depois do jantar, mais de 20h, porém dia claro, vou dar adeus à caravana tcheca, que embarca de regresso, em ônibus próprios. Reconheço o Dr. Pravoslav Sádlo, violoncelista com quem viajei há dias de bonde, sem me lembrar que era o intérprete masculino principal da peça ``Pigmaleão''. Ele também agora me reconhece e vem ao meu encontro de braços abertos. Penitencio-me da gafe e, com os votos de boa viagem, felicito-o pelo brilhante desempenho.

Fico, depois, a bater longo ``papo'', na porta da ``Akademicka'', com o holandês Jacques Tuinder, moço e bem falante, que gosta de saber coisas sobre o Brasil. Satisfaço-o o quanto posso, antes de trocar outras demoradas idéias de ordem geral. Enquanto isso, vejo sair outro ônibus, agora para Cracóvia, onde amanhã terá início o ``Post-Kongreso'', que durará cinco dias. Abraços, felicitações e repetidos ``até Bruxelas'', onde será realizado o Congresso Internacional do próximo ano.

Esses europeus são felizes, podem estar em todas. Eu sepulto qualquer pretensão de poder retornar.

Por falar em Bruxelas, Josée, minha correspondente belga, deixou um bilhete de despedida e agradeceu o pacote de café com que a presenteei. Deixou, porém, os discos para eu mesmo levar. Lá para o fim do mês nos encontraremos de novo, se Deus quiser.

Também a Sra.~Ana Büchler, iugoslava que tem um filho morando em São Paulo, jogou um bilhetinho por debaixo da porta: embarcava às 13h e não teve tempo de preparar a carta que me ofereci para levar.

Adeus, esperantistas dos cinco continentes com quem aqui confraternizei durante oito dias, na miraculosa realização do sonho elevado de Luís Lázaro Zamenhof!

E dou sentidas graças a Deus por ter eu podido realizar este meu sonho de há sete anos, quando me tornei esperantista: participar de um congresso internacional. Foram dias inesquecíveis, de renovadas emoções, quando me senti feliz e honrado em personificar meu querido país na amostra de concórdia universal, aqui vivida por 3.300 cidadãos, e que será, por certo, o mundo de amanhã!

Por isso abracei efusivamente os representantes da Indonésia e do Vietnam do Norte, nas escadas do Palácio esta tarde, vendo neles os maiores exemplos da realidade esperantista. Um espetáculo destes, uma vez por ano que seja, em cada país, vale mais que qualquer arenga diplomática, tão comumente cercada de insinceridade.

Por isso, igualmente recebi com naturalidade a despedida, aqui no quarto, do cracoviano Tadeo Dropiovski, na forma do, para nós estranho, costume eslavo: um beijo na face.

\section*{9 \adfflatleafright \ISODayName{1959-08-09}}
Logo cedo mudamo-nos para o Palácio da Juventude, que é uma parte bem grande do Palácio da Cultura, em atenção a mais um gentilíssimo convite da incrível Dona Bárbara. Ficaremos só dois dias, enquanto se soluciona o problema do visto e das passagens para Moscou, pois hoje é domingo e nada pode ser feito.

Nosso quarto é um tanto improvisado: um cômodo do ambulatório. Mas perfeitamente confortável, com magníficos sanitário e chuveiros bem próximos.

Verificamos, então, que o ambiente no Palácio é extraordinariamente democrático entre diretores e servidores. A simplicidade de todos é notável. E é intenso o movimento de jovens estudantes de outras cidades que vêm passar três ou quatro dias na capital. Vemos chegar ônibus especiais levando e trazendo gente, com os professores ou professoras acompanhando. Ficam todos hospedados no Palácio.

Saímos para o almoço e, na entrada principal do Palácio, encontramos vários esperantistas, ainda na cidade, muitos deles de partida para o ``post-Congresso'' em Cracóvia. O idoso e simpático representante americano, Charles E. Peterson, nos convida para fotografias, chegando-se para o mesmo grupo o amigo húngaro Miklós e seu patrício, o eminente esperantista Julio Baghy, de Budapest. Tiramos, depois, uma pose a sós, com Baghy. Aproveitando o ensejo, o responsável pela filmagem do Congresso e que está fazendo as últimas tomadas, puxa os brasileiros para participarem da sequência, juntamente com os que já lá estavam e mais um bando de iugoslavos.

Não sei por que o cineasta me atribuiu o papel central da cena: ficar à distância e depois chegar, apontar algo imaginário no céu e sairmos todos andando numa certa direção. Após um curto ensaio, a tomada é feita. O filme, diz o encarregado, será de 30 minutos e poderá ser comprado ou alugado por todos os interessados.

À tarde vamos à Rua Zamenhof, onde o criador do esperanto viveu e elaborou seu genial projeto. Infelizmente, a casa é outra, pois a original foi destruída, como 85\% de Varsóvia, durante a guerra. Uma lápide de mármore, bem grande, registra a particularidade, em esperanto e em polonês.

Dali seguimos de táxi para visitar o túmulo de Zamenhof no cemitério israelita. Foi uma luta transmitirmos nosso desejo ao motorista. Enfim, com alguns desenhos e muita mímica, deu resultado.

O túmulo --- postal obrigatório dos meios esperantistas --- está, hoje, ainda enfeitado com as ``corbeilles'' e braçadas de flores depositadas na solenidade da antevéspera, constante do programa oficial do Congresso.

Fazemos as fotografias e, após, rendemos, em curto silêncio, nosso preito e homenagem ao lúcido espírito missionário ali evocado.

À noite, Dona Bárbara nos convida para um teatro ou, se preferíssemos, circo. Optamos por este último, pois haveria de ser mais interessante para quem não conhece o polonês. O pavilhão algo modesto, está à cunha. Chama-se ``Circ Humberto'' e é iugoslavo. A ``função'', longuíssima (quase três horas), é de boa qualidade e variadíssima. Até o humorismo surpreendeu, nas paródias de danças e nos recursos dos palhaços. Mas as vestimentas destes não são tão exuberantes como as dos nossos.

Os números de animais impressionaram pela quantidade deles na arena de uma só vez: 9 ursos polares em um número; depois, 5 leoas, 3 camelos e 2 iaques e, por fim, 4 pôneis!

O pior foi no final, um número acrobático, sem qualquer proteção de rede. Três sustos pavorosos sacudiram a plateia. O sujeito, de cabeça para baixo, atravessou as alturas de um lado a outro, enfiando o pé nas alças que formavam uma estreita escada de cordas, e pareceu desequilibra-se por duas vezes. Não contente, ousou ainda o patife vir até o chão pelo mesmo processo (sempre de cabeça para baixo). Um gelo no coração e nas mãos, que suaram a valer.

\section*{10 \adfflatleafright \ISODayName{1959-08-10}}
Desde cedo, com a gentilíssima Dona Bárbara e o intérprete Bruno, funcionário do Palácio e que fala impecável francês, saímos no carro de serviço, para tratar da viagem a Moscou.

Do consulado soviético, com promessa do ``visto'', partimos para a organização oficial de turismo, ``Orbis''. Aí temos a surpresa de verificar que o preço, para nós estrangeiros, é elevadíssimo: 196 dólares de passagem aérea e 30 dólares diários em Moscou; e com a obrigação de pagar em dólares. Haviam dito anteriormente (informação telefônica) que a passagem poderia ser paga em moeda polonesa; mas há um acerto, de ordem cambial, sloty/rublo, em nosso desfavor.

Diante da exorbitância, resolvemos desistir. Em tais condições não compensaria a visita à capital soviética. Então entra em cena um esperantista, Glambowski, caixa da Associação de Varsóvia, e propõe que a sua entidade pleiteie para nós tratamento idêntico ao dispensado aos poloneses, para aqueles fins turísticos. Ele faz a carta na Associação --- até onde o acompanhamos --- e dali partimos para o Banco Nacional, que precisava autorizar.

Deu certo. Dona Bárbara também apareceu no banco e reforçou nossa causa. Foi um ``abra-te, sésamo''. Voltamos à Orbis e pagamos a excursão, que será de apenas três dias (90 dólares para cada um, isto é, mais de quatro mil cruzeiros por dia!). É que só fazem programa na categoria luxo para não nacionais. Teremos, porém, direito aos melhores hotéis, passeios completos e, até, automóvel e cicerone para acompanhar.

Depois do almoço voltamos ao consulado. Mas o visto só poderá ser dado amanhã. Eles fazem em ``carta turística'', para não ``complicar'' o portador do passaporte. Desse modo, só poderemos viajar depois de amanhã.

Com Bruno e o chofer Jozefo gastamos o resto da tarde a passear pelos locais que ainda não conhecíamos: a ``Cidadela'', velho fortim à margem do Vístula, onde barbaridades forma perpetradas no século passado e na última guerra. Os nazistas destruíram totalmente o interior. Restam apenas as amuradas que o circundam.

Depois vamos ao pé dos monumentos mais famosos de Varsóvia: o rei Sigismundo III, que está no cimo da alta coluna, e o do escritor Mickienvicz. O do astrônomo Copérnico eu já havia visto em outros passeios.

O dia foi praticamente gasto com o \textit{affair} Moscou. Mas valeu pela evidência da notabilíssima gentileza dos poloneses, empenhados em tudo fazer para que realizássemos nosso desejo de viajar à capital soviética.

Antes de dormir, no Palácio, Dona Bárbara serve-nos a tradicional refeição noturna. Bruno traz uma bebida chamada ``spiritus'', à base da vodca polonesa, mais famosa que a russa. Uma bomba! Até os da casa acharam forte a dose. E o Alcino andou virando uns tragos.

\section*{11 \adfflatleafright \ISODayName{1959-08-11}}
Temos a manhã livre. A atividade no palácio é grande, pois amanhã se encerra a temporada de movimentação de estudantes. Arrumam tudo, para fechar o palácio até o fim das férias. Dona Maria, a professora que fala francês, explica-nos que todos os servidores, do diretor aos atendentes e copeiros, irão passar duas semanas em Sopot, perto de Gdansk, a melhor cidade balneária da Polônia.

Saímos, depois, com ela para um giro aos locais do palácio que ainda não conhecemos: a piscina, salões de recreação infantil, pequenos museus de fitobiologia, salas de iniciação com mecânica e algumas outras. Um mundo de primorosa instalação, para os mais variados misteres pedagógicos. Que maravilhoso trabalho!

O ponto alto, porém, é a piscina, embora no momento coberta. Está vazia, pois trabalhadores reparam ladrilhos soltos. Calculei dimensões de 30x20 metros. É toda revestida de mármore, inclusive nas divisões das tribunas reservadas ao público, em toda a sua volta. Belas colunas, também de mármore, ornamentam dois ou três setores, com lindo efeito.

Após o almoço vamos buscar o visto em nossos passaportes no consulado russo. Partindo dali, encontramos de novo Dona Bárbara, que nos acompanha na providência de compra das passagens, para viajarmos amanhã. Desagradável surpresa: só há lugares no avião daqui há cinco dias! Os outros aviões, diários, são soviéticos e nossa concessão é para a ``Lot'', a companhia polonesa. Impossível esperar todo esse tempo, dizemos à Dona Bárbara, que se mostra visivelmente constrangida com o sucedido. Mas, muito diligente, sugere duas soluções imediatas: tentar aproveitar, no primeiro avião de amanhã cedo, alguma vaga, principalmente as duas diplomáticas obrigatoriamente reservadas e nem sempre ocupadas; e, caso falhasse essa tentativa, irmos de trem (18 horas de viagem) e retornarmos de avião.

Aceitamos. Mesmo porque a Orbis não desfaz o contrato turístico. Se desistirmos, perderemos os 90 dólares que cada um despendeu, fora as passagens.

À noite, recebemos ingressos para uma opereta. Chama-se ``Tangolita''. A casa de espetáculos é também modesta, pequena e de poltronas de madeira, embora anatômicas. Mas a cenografia, interpretação e músicas agradam. A iluminação pareceu-me pobre. Nenhum foco das paredes do salão para o palco. Não entendo senão o polonês elementar que aprendi. Mas o negócio é divertido, pois o público ri muito. ``Pesco'' o fio central da história e olhe lá.

Nos intervalos, um casal amigo de Dona Bárbara (não presente) nos é apresentado. O marido, algo vaidosamente, pede-me para dizer o idioma em que poderíamos conversar. Escolho o inglês. Criva-me, então, de perguntas sobre o custo de nossa viagem. Quando se fala, na Europa, em mil dólares ou mais, arregalam os olhos. O homem quer saber coisas como quanto tempo de trabalho preciso para ganhar aquela quantia. ``Embrulho'' como posso.

\section*{12 \adfflatleafright \ISODayName{1959-08-12}}
Madrugamos, para estar às 6h30m no aeroporto, a espera ou em busca de chance para viajar. O responsável pelo serviço, ao que entendemos, promete à Dª Bárbara que tudo fará visando a nossa pretensão.

O avião sai às 9h20m. Mas após longa e cansativa espera a resposta é negativa. Nada de vaga. Moscou está se tornando difícil para nós. A incrível Dona Bárbara não se dá por vencida, contudo. A viagem de trem já havia sido por nós descartada. Ela resolve pegar um táxi e ir ao Banco Nacional, ou não sabemos a que outro lugar, para tentar obter concessão para viajarmos, nas bases mais econômicas, em avião russo, ao invés de só em aparelho polonês. Intimamente, passo a ter certeza de que ela conseguirá.

Ficamos sinceramente comovidos com a quase absurda prestatividade dessa cativante polonesa, alta funcionária do departamento educacional e que tanto se dedica para nos favorecer. O que fizemos para merecer tanto?

``Panowe'' (senhora) Maria fica nos fazendo companhia. Passado algum tempo --- menos de uma hora --- volta a fabulosa mulherzinha. O sorriso com que salta do táxi mostra que está tudo arranjado. Corremos para o motorista, a fim de evitar que ela pague. Era preciso retribuir de algum modo. E ela ainda traz, de contrapeso, um saco de enormes e perfumados pêssegos para nós. É demais!

O avião sai às 14h40m. Como ainda falta muito até chegar a hora, forçamo-las a retornar logo, para não lhes roubar mais o precioso tempo. Que espíritos bondosos e gentis! Agradecemos como pudemos. Já havíamos oferecido à Dona Bárbara meio quilo de café, trazido do Brasil, um chaveiro (do IRB, com o mapa do Brasil) e um cinzeiro de cerâmica. Era o que tínhamos. Cigarros não pudemos dar porque ela não fuma.

Finalmente, alçamos voo no aparelho soviético. Assemelha-se aos ``Convairs'' americanos, mas é mais e alto e mais amplo interiormente. Há certo desperdício de lugar na parte traseira, que deixa grande espaço para os domínios da aeromoça. Embora com 24 lugares, só viajam 15 passageiros.

Voamos 4 horas sem escala. A Rússia, nesta região, é plana como uma mesa. Nenhuma elevação ao alcance da vista. Que impressionante extensão de planura! Admiro o reflorestamento. Durante todo o percurso o solo mostra-se intensamente cultivado e bordado de densas florestas de pinheiros. Perto de Moscou são mais abundantes ainda as grandes florestas.

Durante a viagem somos chamados a adiantar o relógio uma hora, para nos ajustarmos à hora moscovita. Dali a pouco recebemos o lanche de bordo. Perde para os nossos em quantidade e apresentação: um sanduíche, um pão redondo com manteiga, caldo de frutas, biscoitinhos e um copo do indefectível chá bem quente (aliás, saboroso).

Às 17h40m, hora local, descemos na capital soviética. É dia bem claro, pois o sol, nesta época, só se põe lá para depois das 9. O aeroporto é grande em área e pistas, mas a estação de passageiros não está à altura. Nem se vê grande movimento de viajantes. Observo, no campo, uns 20 exemplares do espetacular jato TU-104, alguns silvando em manobras de decolagem.

O desembaraço, surpreendentemente, é fácil e rápido. O rapaz da alfândega é extremamente simpático e bonachão. A responsável pelo controle da vacinação releva a não apresentação do certificado do Alcino, que, cabeça de vento, deixou-o em Varsóvia. Ele já havia perdido o papel polonês de controle de mercadorias e moedas. Foi preciso fazer outro no aeroporto, na hora da partida.

Um recepcionista da ``Inturist'', órgão oficial das atividades turísticas na União Soviética, nos recebe. O aviso sobre a nossa chegada deve ter sido dado pelo rádio enquanto voávamos. Dona Bárbara telefonou para a ``Orbis'' fazendo a comunicação em Varsóvia. O rapaz se aproximou perguntando ``Brasilinski''? (brasileiros, em russo). Espanto-me com os seus trajes, pois parece um mendigo, de tão andrajoso. Ele nos encaminha ao carro que nos espera e dá o nome do hotel em que ficaremos: Hotel Berlim. A elevada cota que pagamos começa a produzir efeito: nenhum outro desembolso teremos de fazer, abrange tudo.

O carro da ``Inturist'' é um enorme ``Zim'', o automóvel de luxo soviético. O aeroporto é distante. Viajamos uns 40 minutos. Ao entrar na cidade, vemos as largas ruas e as grandes construções de blocos de apartamentos, uns 20 conjuntos, pelo menos, de cada lado. Ao longe, pontificando, com uma luz acesa, o edifício que presumo ser o da nova Universidade. O estilo, com a alta torre central não é senão uma brilhante estrela vermelha. A primeira que vejo.

Uma curva e diviso os contornos do Kremlin, com a igreja de São Basílio, a inconfundível no seu estilo bizantino, à frente, na famosa Praça Vermelha. Não sabia que o nosso hotel ficava na área central. Ele é bem antigo, mas muito bom. Recebemos um enorme apartamento, todo atapetado, com sala mobiliada, toda à antiga, amplo quarto e banheiro. Decepção: falta chuveiro, só tem banheira.

Recebemos cupões para 4 refeições diárias e as outras novidades: direito à guia e intérprete, durante 6 horas por dia, e automóvel por 3 horas. Estamos com tudo! Mas o cansaço é grande e já se faz tarde.

Na portaria falam bem inglês e francês. A moça da ``Inturist'', que tem um departamento lá dentro do hotel, domina magnificamente o francês. A simpatia e afabilidade de todos são notáveis.

Combino o programa de amanhã, antes de bater na cama. A temperatura à noite caiu verticalmente.

\section*{13 \adfflatleafright \ISODayName{1959-08-13}}
Amanhece dia lindíssimo. À noite fez frio, como baixa ainda está a temperatura. Pedimos o ``breakfast'' (café da manhã) pelo telefone, falando em inglês. Servimo-nos lautamente e vamos para os primeiros passeios. Apresentam-nos Laurish (que corresponde a Laura), nossa jovem guia e intérprete, que fala espanhol, com pronúncia catalã. É uma simpática russinha de 23 anos, loura e esguia. Seria mais bonita não fossem umas espinhas miúdas salpicando o rosto de pele clara.

Para solver, desde logo, o problema dos vistos para o regresso, vamos inicialmente com Laurish aos consulados da Polônia e da Alemanha Oriental. Neste último, temos a satisfação de receber dispensa do pagamento do visto, em virtude de nossa flâmula ``Brasil'' na lapela fazer-nos passar por participantes do Festival da Juventude de Viena. Dez rublos (um dólar) de economia.

Depois fazemos câmbio na portaria do Hotel Metropol, a fim de termos rublos para alguns pequenos gastos. O câmbio é de um dólar para dez rublos somente para os turistas. Para os nacionais, o dólar é trocado por apenas 4 rublos.

A cidade causa excelente impressão de modernismo de vida, embora os prédios centrais sejam, na maioria, antigos. A própria arquitetura, aqui, é sóbria nas novas construções. As muito velhas são enfeitadíssimas, profusamente ornamentadas com motivos de quase minúsculas dimensões. Parece-me estilo eslavo, pois em Varsóvia impera a mesma linha.

O tráfego é organizado e intenso. Não vemos bondes na parte central, pelo menos; só ônibus elétricos. A limpeza é outro detalhe digno de nota.

No almoço, no hotel, fazemos uma ``farra egípcia'' e temos que pagar 58 rublos além da cota. Alcino empalideceu.

À tarde, reencontramos Laurish para um giro pelo ``metrô''. Foi espetacular! Embarcamos numa estação central. Elas são designadas por um grande ``M'' na entrada. Os carros circulam bem profundos. Uma escada rolante tripla, com uns 50 metros de rampa, leva-nos ao subterrâneo. A estação é bonita, mas não muito grande. Interior todo de mármore, artisticamente decorado.

Passamos bastante tempo viajando nos carros e saltando em várias estações para conhecê-las. Cada uma é decorada ou tem um motivo de homenagem a alguma coisa, a determinadas indústrias, por exemplo, à cultura, às artes etc. Uma maravilha de quase impossível descrição! O chão é de limpeza absoluta, o mesmo sucedendo no interior dos carros, que são confortáveis e de excelente aspecto. Correm de 50 em 50 segundos --- diz Laurish. Discretamente resolvi conferir e vi que era realidade. À noite o intervalo é de 2 minutos. Desse modo, não ocorre superlotação (Moscou tem 8 milhões de habitantes). Em geral, há lugares para se sentar.

Quando o carro parte, uma voz (quase sempre feminina), que deve ser da motorista, diz o nome da próxima estação. A língua é muito suave falada. A escrita é que são elas.

A estação que mais me entusiasmou foi a ``Konsomolskaia'' (juventude), palavra bem conhecida. Um ``show'' estupendo de cor e concepção, reluzindo no mármore com predominâncias ouro e nos lustres belíssimos, pendentes em longas fileiras. Como o filme da máquina acabou, não posso fotografar nada agora. O cartão de visita dos moscovitas ultrapassou qualquer espectativa.

Havíamos telefonado para alguns esperantistas, consultando o anuário da Associação Universal. Eles apareceram no hotel para uma visita amigável: o professor Bokarev, do Instituto de Línguas de Moscou, e Israel Kudriavicky, figuras respeitáveis. Após boa palestra no apartamento, descemos para um lanche no restaurante com eles, quando provamos pela primeira vez o famoso caviar. Gostei, se bem que o paladar fica sacrificado pela mistura com a manteiga. E é muito nutritivo. Mas bastante caro.

Foi curiosa a observação feita pelo professor Bokarev quando telefonou para o hotel avisando que queria nos visitar o quanto antes. Ele justificou dizendo estar ansioso, porque nunca havia visto um brasileiro na vida dele.

Ficamos com os amigos até a hora de partir para assistir o ``Cinerama'', grande novidade mundial do momento: uma projeção cinematográfica em tela descomunal. O carro da ``Inturist'' nos leva lá. O espetáculo foi interessantíssimo, mas não se trata de terceira dimensão, que é apresentada em outro local. A tela é realmente gigantesca e a projeção parte de três focos diferentes, compondo uma só imagem, de enorme amplitude. O filme é todo sobre o país, destacando Moscou, naturalmente. Quanto ao processo, se bem que de magnífico efeito, não é 100\% perfeito. Nota-se sem esforço, inúmeras vezes, os dois pontos de junção das três imagens projetadas, em conjugação que, de vez em quando, se desconexa ligeiramente.

Na saída, lá estava pontualmente o carro da ``Inturist'' nos esperando. A organização desse serviço está nos parecendo primorosa.

\section*{14 \adfflatleafright \ISODayName{1959-08-14}}
Encontramos Laurish de manhã, para a visita à Exposição Permanente de Indústria e Agricultura. É outro ponto destacadíssimo que a cidade oferece. São 78 pavilhões, ocupando imensa área, primorosamente tratada, disposta e ajardinada. É lugar para se passar vários dias. Mas meu tempo é escasso e só três ou quatro horas disporei para a visita; mas elas serão intensamente aproveitadas.

Os pavilhões são soberbos e, entre eles, verdadeiros jardins com algumas fontes e chafarizes de se ``tirar o chapéu''. Fotografo a valer, para fixar para sempre esta visita a este notável lugar.

Percorremos o pavilhão da Academia de Ciências, onde estão reproduções dos três ``sputniks'' enviados ao espaço. É enorme o número de turistas e de visitantes de países amigos, como Coreia do Norte, Vietnam do Norte e mesmo Japão. Laurish diz que os americanos são os que mais têm visitado Moscou nos últimos anos, o que não deixa de ser um dado curioso. De fato, encontramos vários. No nosso hotel há um grupo numeroso.

Ontem na rua tivemos a satisfação de encontrar a família do professor Israel de Castro, que pertenceu à caravana do Festival da Juventude, nosso companheiro de viagem marítima. Com eles, a filha Creusa, que fez a primeira palestra a bordo. Por incrível coincidência, no ``metrô'' topamos logo depois com o deputado Solon, que fez a terceira palestra no ``Cabo San Vicente'', e o outro deputado estadual paulista, de origem japonesa.

Hoje, ao voltar para o almoço, soube da chegada de cinco brasileiros ao nosso hotel: os médicos David e Franciosi, do Instituto de Câncer de São Paulo, também nossos companheiros de bordo, e três festivalistas que viajarão de avião.

Ontem encontramos, ainda, no Consulado da Polônia, outros três patrícios, com quem confraternizamos efusivamente: um jornalista, um industrial de São Paulo e um diretor da Tabajara Filmes.

À tarde, mal engolido o almoço, Laurish nos espera para um giro pelas lojas. Conhecemos o grande armazém central ``GUM'', na Praça Vermelha, e compramos alguns ``souvenirs'', inclusive discos de músicas brasileiras gravadas na União Soviética, dentre elas algumas de Dolores Duran.

Depois, vamos à organização ``VOKS'', ligadas a assuntos culturais. O encarregado da seção brasileira, Sr. Benjamim, fala português muito bem e anota nossos pedidos. O do Alcino, sobre saúde escolar, e o meu, ligado à revista hidroviária do colega Brasilo, do IRB.

Laurish nos sugere um giro mais retirado pela cidade. No magnífico carro ``Volga'', à nossa disposição, corremos pelas margens do rio Moscou e fazemos alto para umas boas chapas, tendo ao fundo o espetacular conjunto do ``Novo Devichy'', que data de 1729.

Depois, para a nova Universidade, impressionante edificação no bom estilo russo. Tem 36 andares e abrange 6 faculdades, com capacidade para 18.000 alunos, sendo 11.000 internos, morando em apartamento individual. Um espetáculo à parte!

Não distante da Universidade, voltamos a beirar o rio Moscou. O local é mais elevado e do lado oposto tem-se magnífica visão panorâmica da cidade. Junto à margem, o enorme estádio Lenin, muito embandeirado. Lá se realizou o festival da juventude de 1957.

Laurish me informa que os relógios da Universidade são os maiores do mundo. Delicadamente contesto, pois, ao que me parece, os da Central do Brasil e mesmo o da Mesbla, no Rio, são bem maiores. Ela não demonstra ter admitido minha observação.

Deixamos a gentil guia à porta de sua casa, quando a conversa girava em torno da tentativa da tomada de Moscou por Napoleão Bonaparte. ``Fué necessario dejálo entrar --- diz ela --- para después\ldots'' E faz um gesto como o de quem está botando um gato para fora, segurando pelo cangote.

Para amanhã combinamos a visita ao Kremlin e ao mausoléu de Lenin e Stalin, na Praça Vermelha. Hoje, entre as 7 e 8 horas da noite (sol de fora), estivemos percorrendo a praça. Antes, demos um pulo ao Museu de Lenin, enorme e visitadíssimo e que se situa em um dos lados da praça, fronteiro à igreja de São Basílio. Logo à entrada, demos com um grupo de ``pioneiros'' (garotos) romenos. Tanta festa nos fizeram que decidimos bater uma fotografia com eles, dentro mesmo do ``hall'' de entrada do museu. Esta prima pelos descomunais quadros a óleo, reproduzindo os episódios mais marcantes da vida do grande líder.

Na praça --- sem qualquer policiamento, apenas um inspetor de tráfego, ao longe (acreditem ou não) --- assistimos a mudança da guarda do mausoléu, às 7 horas. Depois de algumas chapas frente à incrível catedral de São Basílio, começamos a ser abordados por garotos e rapazolas querendo permutar distintivos por outros ou por qualquer coisa.

Dois armênios querem\ldots comprar nossas roupas, camisas, ``pull-overs'', o que puder ser! São morenos e puxam para o tipo árabe, racialmente. Dizem gostar do Brasil e, ao nos despedirmos, apesar de recusarmos qualquer negócio, jogam\ldots beijos! Merecemos tanto?.

Antes de nos recolhermos, batemos pé um pouco mais pelas redondezas. O movimento é regular, mas não indica vida noturna movimentada. Mesmo porque a temperatura está pelos 16º. Comprávamos cigarros (para ``souvenir'', pois nenhum de nós dois fuma) num varejo de rua, quando um cidadão de meia idade e vestes modestas nos aborda, fixando nossa pequena flâmula na lapela. ``Brasil! Jorge Amado!'', diz ele sorrindo. Oferecemos uma caixa, também de uma marca de cigarros moscovita, mais parecendo aquelas embalagens antigas de injeção, e, não satisfeito, pega-nos pelos braços, rua abaixo. Entendo que quer nos pagar alguma bebida ou levar-nos até sua casa, que deve ser perto. Fala animadamente, mas só russo, logo nada entendemos. O homem não se dá por achado. Foi um custo para nos desvencilharmos e, com muita mímica, darmos a entender que não podíamos acompanhá-lo, que a noite estava muito fria e que tínhamos um compromisso, em seguida, no hotel.

\section*{15 \adfflatleafright \ISODayName{1959-08-15}}
Sendo nosso último dia inteiro em Moscou, precisamos aproveitar bem. Após o suculento ``breakfast'' no hotel, servido desta vez no apartamento, descemos para encontrar Laurish, como de costume pontualmente a postos. Com tal refeição matinal, o jeito é almoçar lá para as 3 da tarde.

O programa inicial é o Kremlin. Partimos de carro até a porta de entrada, atrás da Praça Vermelha. Entrada franca, com numerosíssimo cortejo de visitantes. Começamos pelas igrejas, que são 12 dentro da fabulosa fortaleza. Eram privativas dos tzares, para diferentes fins: uma para serviços religiosos comuns, outra para coroações, outra para túmulo etc., etc. A figura de Ivan (que significa João), O Terrível, pontifica.

O palácio residencial é hoje teatro, biblioteca e, em parte, sede administrativa do governo soviético. Outras edificações servem de museus diversos. Impossível ver tudo com vagar. O interior do Kremlin é imenso. Há até tráfego interno, não sei se só de autoridades. Fazemos numerosas fotos, pois não há qualquer proibição nesse sentido.

Adiante, sobre uma base de pedra especial, está o maior sino do mundo, fundido lá por volta de 1785 e que pesa 200 toneladas! Ao ser içado para a torre da igreja, caiu e quebrou um pedaço. O fragmento está ao lado. Por ele pode-se ver a espessura do bruto, que é de mais de um palmo bem esticado (dos meus). Uma inscrição diz que foi colocado naquele pedestal em 1839.

Andando de um lado para outro do pátio, encontramos Mr. Peterson, o esperantista norte-americano que participou do congresso em Varsóvia, juntamente com sua esposa e alguns turistas seus patrícios. Faz-nos enorme festa e filma-nos de novo (já o fizera em Varsóvia). Depois pede que eu diga algumas palavras, em inglês, aos amigos sobre o milagre e a realidade do esperanto, pois eles estão impressionadíssimos com o nosso eloquente entendimento. Cumpro gostosa e (modéstia à parte) felizmente a missão.

Deixando o Kremlin, vamos visitar, por dentro, a catedral de São Basílio, a joia bizantina de Moscou, que data de 1549. Internamente, como é lógico pela idade secular, não impressiona tanto como por fora. Dizem que Ivan mandou cegar o projetista para que ele não pudesse jamais realizar coisa igual e, muito menos, superior. A construção é toda de tijolos, com tetos muito baixos e escadinhas incrivelmente apertadas. Para a época, foi um prodígio de engenharia e concepção, sem dúvida. Como beleza arquitetônica empolgará sempre.

De São Basílio passamos à fila especial de turistas para visitar o mausoléu de Lenin e Stalin, na mesma praça. A visita começa às 13 horas, mas desde as 9 já se espicha enorme fila. Assim é diariamente, diz Laurish. Ficamos, porém, com um numerosíssimo grupo de jovens, na maioria meninas, do Khazakstão longínquo, lá nas fronteiras com a China, e que visitam Moscou.

Foi comovente! O ``Brasil'' no nosso peito e os trajes elegantes (mais uma vez, modéstia à parte) despertaram grande curiosidade. Cercaram-nos como se fôssemos camelôs do largo da Carioca em plena função. Laurish ajudou como pôde para me entender com a moçada, transbordante de simpatia, e com os dois adultos responsáveis pela turma, que totalizava umas 40 pessoas. Meninas lindas, saudáveis, louras sempre de olhos azuis e morenas gritantemente iguais às brasileiras da mesma idade!

Súbito, um garotote me estendeu um bloquinho e um lápis: queria um autógrafo! Pela primeira vez na minha vida dei autógrafos às dezenas, sem um sentido razoável, como aconteceu, por exemplo, no Congresso de Esperanto. Senti-me verdadeiramente comovido, não sei se pelo meu país ali representado ou por outro motivo qualquer, talvez a encantadora simplicidade daquela criançada.

Também os dois professores perguntaram mil coisas. A meninada, em retribuição aos autógrafos, encheu-me a lapela de distintivos: Lenin em várias idades, alegorias aos ``sputniks'', ``C.C.C.P'' e outras. Algumas meninas se achegaram ``arranhando'' o inglês e o francês. Então gostei mais. Queria falar diretamente sem intérprete, apesar da simpatia e paciência desta.

Verifiquei que têm boa base de geografia. Conhecem perfeitamente a posição do Brasil, sua capital (presente e futura) e muitas coisas mais.

Por fim, uma hora passada, a fila começa a andar. Em breve, penetramos no mausoléu. Desce-se uns lances de escada, tudo de mármore escuro. Lá embaixo é gélido, pela própria temperatura e pela sugestão do túmulo, com suas duas urnas de vidro. Entrando-se pelo lado esquerdo, primeiro está Lenin, que ali jaz há 35 anos. Ao lado, Stalin, há 6 anos e meio. Mas a impressão é de que foram postos ali ontem, tal a normalidade de seus aspectos.

Em silêncio respeitoso, todos desfilam lentamente. Não é permitido parar nem fotografar. Stalin mostra sua basta cabeleira bem grisalha, assim como o denso bigode, que marcou época. No último trecho do recinto, em que se desce alguns degraus, um soldado murmura curta frase a quantos passam. Não compreendo a língua, mas só pode ser ``Cuidado com o degrau''.

Deixando o mausoléu, corremos seu lado superior mais elevado e em sua parte trazeira. Ali estão sepultados outros líderes renomados, como os presidentes Kalinin, Zdanov e dezenas de outros. Também o americano John Reed, autor de ``Os Dez Dias que Abalaram o Mundo''.

Acabado o programa, ainda somos assediados pelos khazakstãos, que desejam se fotografar conosco. Andamos um bom pedaço até o declive da parte norte da Praça Vermelha, onde estão os ônibus especiais deles e o seu fotógrafo, que faz várias chapas.

Dali vamos, uma vez mais, aos imensos armazéns ``GUM''. Compramos utilidades (creme de barba, de fabricação chinesa, e sabonetes) e outros ``souvenirs''. Passamos, depois, para outro magazine, fronteiro ao nosso hotel, especializado em artigos para crianças. Causa, também, excelente impressão, pois tem uns quatro pavimentos, dotados desde o térreo de escadas rolantes, e enorme variedade de mercadorias, desde vestuário, material escolar, brinquedos, etc. Compro uma lapiseira e um caderno para continuar este diário.

Já bem tarde, fomos almoçar. Despedimo-nos de Laurish, nossa gentil cicerone. Anoto seu endereço, para escrever e mandar fotografias.

Mais tarde, volto ao ``metrô'', para, além do prazer de contemplá-lo, fazer umas fotografias coloridas, principalmente da fabulosa estação ``Konsomolskaia''. Foi um trabalhão andar sozinho naquele labirinto. Em determinada estação, um popular de boné pergunta-me, com gestos e a palavra ``futebol'', se sou um ``craque da pelota'' do Brasil, olhando para o distintivo no meu peito, como sempre. Outro garotote, de uns 13 ou 14 anos, também de bonézinho, vendo-me pedir informações sobre a direção da estação ``Konsomolskaia'' ofereceu-se, também por gestos, para acompanhar-me até lá. Mal chegando, apontou a plataforma e disparou por uma rampa acima, naquele mundo subterrâneo, para ficar lá do alto me olhando.

Outras fotos a cores faço da Praça Vermelha, na volta, pois a São Basílio bem o merece.

Finalmente após o jantar, no hotel, recebo um telefonema de outro esperantista. Ele tinha estado antes no hotel --- diz então --- com o professor Bokarev e deixara um pequeno algum de postais de Moscou, com dedicatória. Encontrei de fato o presente. Peço que venha ao hotel e, dez minutos depois, lá está ele batendo na porta. Chama-se Nikolai Rytikov e é ator de cinema. Como os outros dois, fala um excelente esperanto e com ele encerro minha inesquecível passagem pela pitoresca capital soviética.

Um detalhe à margem: os automóveis russos. Todos magníficos. Os grandes são amplos e mesmo luxuosos, como o ``Zim'', o ``Volga'' e o ``Tchaika'' (que significa gaivota), que vi na Praça Vermelha e julguei ``Oldsmobile'' de turistas americanos. Os que tivemos à nossa disposição, excetuando o que nos apanhou no aeroporto, eram ``Volga''. Há ainda o ``Pobeda'' (vitória), bem menor e que, sem dúvida, serviu de modelo parra o ``Warzawa'' polonês. Apenas um tipo pequeno: o ``Moskovitz'', que lembra o Opel-Olimpia alemão.

\section*{16 \adfflatleafright \ISODayName{1959-08-16}}
Domingo. Às 7 da manhã a ``Inturist'', sempre brilhando pela eficiência, lá está na porta do hotel para nos levar ao aeroporto.

Às 8h40m voamos, novamente em avião soviético, de regresso a Varsóvia. Um pequeno susto na hora do embarque, pois facilitamos no gastar o tempo e, em cima da hora, foi tremendamente difícil achar o portão de embarque correspondente à nossa viagem. O alto-falante deve ter avisado e até repetido, mas em russo\ldots E as tabuletas indicativas estavam todas no mesmo idioma.

Outro susto --- se bem que rapidíssimo --- ocorreu a bordo, no meio da viagem. Os aparelhos russos têm altímetro à vista dos passageiros. Emergia eu de uma justa soneca, quando vi, pela janelinha, que o aparelho voava baixíssimo e a terra lá embaixo, um campo relvado, aparecia inteiramente inclinada. Ou era o avião que estava inclinado? Olhei para o altímetro: em torno de 200 metros! Aterrissagem forçada? --- pensei. Fui tranquilizado pela aeromoça, sentada ao meu lado, lendo calmamente uma revista. Fazíamos uma escala em Vilna, capital da Lituânia, esclareceu ela. Ignorava o detalhe, pois na viagem de vinda isso não ocorrera.

Lá ficamos hora e meia, inclusive para almoçar, por conta da ``Aeroflot''. Não deu para ver muito, naturalmente, pois o aeroporto é um pouco retirado da cidade.

Só viajam nove passageiros, sendo cinco funcionários poloneses ou russos, que conduzem mala diplomática, ao que tudo indica, pois não se desgrudam dela em nenhum instante.

Quando realçamos voo, passo a divisar melhor a capital lituana, que, vejo no anuário esperantista, tem 163.000 habitantes e está metida em um vale. Muitas chaminés de fábricas indicam desenvolvimento industrial.

Folheando a revista que a aeromoça deixou no assento ao meu lado, e que outra não é senão a conhecidíssima e muito bem impressa ``União Soviética'', em russo, dei com a cara do nosso escritor Euclides da Cunha! Pela identificação da efígie, consegui ler e entender o nome ``Euclides'' na notícia. Nestes últimos dias aprendi um pouco do alfabeto cirílico. Como a aeromoça voltou a ocupar o lugar, comentei com ela o assunto, em francês, e pedi que traduzisse para mim a nota inteira, que não era grande e se referia ao lançamento da grande obra de Euclides, ``Os Sertões'', em russo. A aeromoça, fiquei reparando, é baixinha e tem o tipo perfeito de cearense.

Ao chegarmos a Varsóvia, atrasamos o relógio uma hora. Partimos para a casa de Dona Bárbara, na rua Mokotowska, onde o seu marido, Dr.~Tadeos, nos recebeu gentilmente. Nossa bagagem principal havia ficado lá, pois só levamos pequena bolsa, com o indispensável, para os dias em Moscou.

Fomos logo à ``Orbis'' providenciar a passagem para Berlim, que só foi possível obter para o dia seguinte, com partida às 17h15m. Foi bom, para descansarmos um pouco mais da viagem recém-terminada.

No ``Grand Hotel'', onde se acha também um ``bureau'' da ``Orbis'', tivemos a grata surpresa de reencontrar outros brasileiros do Festival: Dona Laudelina, a professora, Sr.~Palma e esposa, os quatro alagoanos, Dr.~Maurício e Célia, a correspondente de jornal de interior paulista.

\section*{17 \adfflatleafright \ISODayName{1959-08-17}}
Resolvemos logo cedo o problema das passagens para Berlim. Enquanto o Alcino e Dr. Tadeos vão para a cidade tentar obter cópia da gravação, em fita, de partes importantes do Congresso, dou um pulo ao Palácio para cumprir a promessa feita ao doutor que antes me atendera, um portento de atenção e gentileza.

Chama-se Dr. Kazimigrz Wawrzynczak~(!) Examina-me, ainda uma vez, e constata melhoras, mas a bronquitezinha persiste ainda. Des\-pe\-di\-mo-nos e eu anoto seu nome e endereço, para mais tarde saudá-lo do Brasil.

À hora aprazada, partimos no expresso, que, por sinal, procede de Moscou e vai até Berlim. Uma composição moderna, de quatro pesados carros, bem longos.

A viagem foi ótima, com pouca gente no trem. Passamos por Kutno e Poznam, grande centro industrial polonês. Pudemos ainda uma vez, verificar o avanço da agricultura na Polônia. A leste e oeste está tudo plantado, certinho, com alguma abundância de gado mais para o ocidente. Grandes quadriláteros sem cultivo atestam que fazem o rodízio da terra, como prescreve a boa técnica agrícola.

A última estação polonesa, na fronteira alemã, é Stubice. O trem arranca uns poucos quilômetros e estamos em Frankfurt-sur-Oder, na Alemanha Oriental. O rio Oder faz a fronteira. Na estação, onde nos demoramos algum tempo, um bando de moças trajadas com macacões esportivos nos dá atenção, vindo bater papo junto ao trem. Acabam pedindo autógrafos --- outra vez\ldots\ Sabendo-nos do Rio de janeiro, imediatamente ecoa um \textit{oh!} de admiração, o que me pareceu uma constante nas moças europeias. Sugere coisa de sonho, tenho a impressão. E viva a nossa terra!

Lá pelas 24h chegamos a Berlim, mas cumpre atrasar outra hora no relógio. A estação fica na zona oriental da cidade. O movimento é praticamente nenhum. Não se podia fazer câmbio de moeda alemã. Tudo fechado. Pagamos o taxi em dólar batido. Vejo então que a ``sopa'' da Polônia acabou. O dólar aqui vale apenas 4 marcos e 15 \textit{pfenings}.

Como o visto do passaporte é de 24 horas, por estarmos em trânsito, podemos ficar no setor oriental para dormir. Aceitamos o primeiro hotel que acusava vaga: o ``Newa'', que é bem luxuoso e pede 40 marcos só para dormir. Paciência.

\section*{18 \adfflatleafright \ISODayName{1959-08-18}}
Meu problema é passar para o setor ocidental. Não tenho um vintém em dinheiro alemão, que, aliás são dois: marco ocidental e marco oriental. Precisei fazer ``vale'' no café da manhã, no hotel, e ir de táxi a um banco para fazer câmbio; o chofer teve que ficar esperando para receber.

Em outro táxi e por sugestão do simpático alemão da portaria do hotel ``Newa'', vamos para a Steinplatz, no bairro Charlotenburg, no setor inglês de Berlim Ocidental, e que, por sinal, fica bem junto do centro comercial. Hospedamo-nos na ``Pension Schiff'', ótima e barata.

Outra vez a fazer câmbio, agora de marco ocidental. Como as duas estão em choque, o marco oriental é desvalorizado em quatro vezes perante o ocidental. No ``bureau'', dentro do imenso ``hall'' do ``Bahnhof Zoo'', junto ao jardim zoológico de Berlim, temos a surpresa de encontrar outro patrício festivalista e companheiro de bordo: o Luiz Fernando Alayon, de São Paulo. Com ele saímos para almoçar, viajando pelo ``sub-way'' (``metrô'', mas os alemães não admitem usar o nome francês), na zona oriental, onde --- diz Luiz Fernando --- tudo é mais barato, principalmente a comida. Depois, voltamos pelo trem urbano ``elevado'' para variar.

Dois edifícios chamam-me a atenção, à margem do leito do ``elevado'': o da Câmara Municipal, de arrojadíssima concepção modernista, e um de apartamentos, cujo projeto, fico sabendo, é do nosso Oscar Niemeyer.

Ficamos a bater perna a tarde inteira, ciceronizados pelo Alayon, que está na cidade há cinco dias e conta as atrações de Berlim. Verificamos, então, que a cidade, embora com grandes vestígios ainda da guerra, está economicamente recuperada. Lojas enormes e moderníssimas, notadamente a ``Bilka'', na Kufürstendam, uma das principais ruas centrais do comércio. Na praça próxima, que faz vértice na junção daquela bela rua com a Hardenbergstrasse --- a cinco minutos da nossa pensão --- a famosa igreja de Gedächtnis exibe sua negra silhueta do que restou após os bombardeios que sofreu.

Entusiasmo-me pelos preços, mesmo com o marco forte perante o dólar. Mas deixo as compras para amanhã. Diga-se de passagem que um carro ``Taunus'' (Ford alemão), bem grandezinho, ou uma barata esporte ``BMW'', ou, ainda, um ``Volkswagen'' custam, novinhos, respectivamente, 200.000, 200.000 e 168.000 cruzeiros.

Eu havia telefonado, de manhã, para uma esperantista, Mme. Hoffmann, mas à noite desmarquei a visita, por atraso no jantar, cansaço e dificuldade em encontrar o endereço a tempo, já que ficava distante.

\section*{19 \adfflatleafright \ISODayName{1959-08-19}}
Saio para obter o ``visto'' no passaporte no Consulado da Holanda. Surpresa agradável: não era necessário, bastava a apresentação do passaporte. O direito à permanência no país, com esta simples documentação, é de três meses.

Na galeria do ``Bahnhof'' inicio as compras, inclusive algumas encomendas. Culmino adquirindo, numa das magníficas lojas de artigos eletroeletrônicos da avenida Kürfurstendam, um sonhado rádio transistor ``Telefunken''. Custou, aproximadamente, Cr\$4.600,00.

Compro, também, outra mala, de bom tamanho, porque a única que trouxe do Brasil, se bem que grande e extensível, a esta altura está pesando toneladas e amarrotando demais as roupas.

Amanhã irei procurar um rádio para automóvel. Alcino, por sua vez, está disposto a comprar uma câmera ``Rolleiflex'' e um gravador.

Marcamos, também para amanhã, um giro turístico pela cidade, em organização especializada. Custará oito marcos e levará cerca de quatro horas, abrangendo todos os setores.

À noite, verifico que no cinema perto de nossa rua, aliás praça, está programado o filme brasileiro ``O Cangaceiro'', para o dia 2 de setembro.

\section*{20 \adfflatleafright \ISODayName{1959-08-20}}
Às 9H30M estou firme na Kürfurstendam, para o longo giro. São dois ônibus bem grandes e saem lotados. A guia do nosso carro fala tudo em três línguas: alemão, inglês e francês. Esquadrinhamos a cidade inteira, com oportunas paradas. A primeira foi no setor soviético, no magnífico ``Treptower Park'', um imenso jardim público, onde os russos ergueram mausoléus de gigantescas proporções em honra aos seus mortos na guerra. Fica junto ao rio Spree. Ainda nesta zona percorremos a sua parte em melhor estado, com a principal via, que se chama ``Stalinallee'', com quatro quilômetros de extensão! Paramos lá também.

Tanto no parque como na ``allee'' faço fotos coloridas; nesta última, fotografo-me ao lado do monumento erigido a Stalin. Outra chapa, para melhor perspectiva, exige uma pisadela no centro da parte gramada. É o bastante para o guarda, de uma boa distância, apitar furiosamente e partir em nossa direção. Tratamos de voltar depressa para a calçada.

Vejo quarteirões inteiros destruídos, inclusive belos edifícios públicos, monumentos e igrejas. Este setor se reconstrói, porém, sob as características germânicas e alguma influência russa nas edificações. A parte ocidental já apresenta modernizações que lembram as Américas. O prédio do parlamento municipal, como disse, é ousadamente modernista. Não muito longe, uns quarteirões de prédios de apartamentos, estilo zona sul do Rio. O projetado por Oscar Niemeyer exibe suas colunas de sustentação em ``v'', como o do hospital da Sul América, na Gávea.

Passamos junto do Aeroporto de Tempelhof, que é bem dentro da cidade e o maior do mundo, segundo a guia. Sempre pensei que o Rio fosse a única cidade a ter aeroporto na zona urbana.

Na praça onde se encontra a entrada principal do aeroporto ergue-se um monumento de linhas modernas, mas bem inexpressivas na minha opinião: três lâminas de concreto, altíssimas, com leve curvatura. Homenagem à ponte aérea de 1948, realizada pelas potências ocidentais, quando a cidade sofreu o famoso bloqueio dos russos, que não permitiram o seu acesso por terra. Lembre-se que Berlim está dentro da Alemanha Oriental.

Seguindo na direção noroeste, contornamos a vila olímpica, gigantesca, onde se realizaram as Olimpíadas de 1936. Não parece ter sofrido qualquer dano.

Das mais interessantes foi a parada seguinte, junto à Porta de Brandenburg, o mais movimentado local da divisa Leste-Oeste. Não há sinal de tropa, como, de resto, observo em toda a cidade. Apenas guardas de trânsito, bem poucos, por sinal. A passagem de pedestres é inteiramente livre; apenas os veículos são controlados e devem exibir algum documento. É a avenida mais famosa de Berlim, a ``Unter Der Linden'', mas a política, pelo visto, matou-a. Ela é curta na parte oriental: vem da praça Marx-Engels e, atingida a Porta de Brandenburg, passa a chamar-se ``17 --- Juni Strasse''; depois ``Bismark'' e mais além ``Kaiser''. Não há tráfego de coletivos --- ônibus ou bondes --- na parte próxima à divisa; nem construções ou comércio em longo percurso, que é marginado por enorme e belo parque, muito bem arborizado.

O passeio foi magnífico. Um jovem casal suíço, a meu lado, ofereceu-se para mostrar sua terra natal, Zurich, quando por lá passarmos. Gentilezas de ocasião\ldots

Depois do almoço, volto a dar largas ao furor aquisitivo. Encontro o rádio para automóvel que procurava, tipo para Volkswagen. Foi preciso encomendar para receber amanhã. Custará cerca de Cr\$6.500,00, mas é o melhor modelo existente, da marca Blaupunkt. No Brasil, o mais inferior custa Cr\$20.000,00. Que beleza!

Depois, vou cortar cabelo em um salão vizinho. Ainda não vi na Europa cadeiras de barbeiro, ou de cabelereiro, como as brasileiras. Mesmo aqui em Berlim têm aparência modestíssima, com horrenda pia junto à parede. Na vez anterior em que me ``depilei'' no salão do elegante Hotel Polônia, bem no centro de Varsóvia, o arcaismo do equipamento era ainda mais acentuado.

O lado agradável e original: uma morena belíssima faz-me sentar, põe o protetor branco e pede-me para esperar um pouco. Falou tudo em alemão, nada entendi das palavras, mas a mímica não deixava margem a dúvida. Feito isso, voltou para acabar seu trabalho em uma freguesa. Fico esperando, com a impressão de que virá o cabeleireiro masculino, que, entretanto, não vem. Quem vem mesmo me atender é a dita beldade. Então, pela primeira vez na vida uma mulher me corta o cabelo. Lembrei-me de que em Moscou havia visto mulheres trabalhando também no mesmo ofício.

Quando ela acabou e me trouxe o espelho para olhar atrás, estremeci: horrível o arremate na nuca! Cortou à germânica\ldots Mas deixo um marco de gorjeta, para 1,25 de despesa. A coisinha balbuciou um descomunal ``dankish'' (obrigado). Afinal, não é todo dia que uma figura daquelas passa mais de meia hora nos dedilhando o crânio\ldots

\section*{21 \adfflatleafright \ISODayName{1959-08-21}}
Pelo meu roteiro, hoje será o quarto e último dia em Berlim.

Após o café da manhã, na acolhedora salinha da pensão, vou visitar o famoso Jardim Zoológico, que fica a cinco minutos de casa, a pé.

O ``zoo'' é realmente bonito e bem ``sortido'', mas, francamente, o nosso do Rio, no conjunto, está na frente, sem qualquer exagero ou patriotada. Muitos animais tropicais estão localizados em espécie de estufas envidraçadas, bem grandes, como é o caso, por exemplo, dos macacos. Em outra estavam morcegos e ratinhos. Mas nessas estufas o ``cheiro de bicho'' era terrível.

Faço boas fotografias, para documentar bem esta agradável visita, que durou mais de três horas. De um certo trecho bem arborizado, tenho ao fundo a bela fachada do hotel ``Berlim Hilton'', da famosa cadeia americana.

Detalhe curioso: a máquina fotográfica teve de pagar ingresso (50 pfenigs ou meio marco).

Deixando o ``zoo'', fui ver a revelação --- a primeira! --- de três rolos de fotografias, dos oito já tirados (cada um com 36 chapas). Ficaram ótimas, na quase totalidade. Nem sabia quais eram, pois nunca marquei as capinhas, e, então uma variedade de cenários desfilou ante meus olhos: ``Cabo San Vicente'', Lisboa. Viena, Veneza, Varsóvia. Em Amsterdam, próxima escala, já poderei fazer cópias, para remeter algumas à família.

Depois, fui buscar o rádio para o carro. É pequenino e jeitoso. Marca ``Blaupunkt'', modelo ``Bremem'', o tipo mais moderno, afirma o rapaz da loja, pois capta ondas médias e curtas. Compro também a bateria encomendada por um amigo e um binóculo para teatro, que é um mimo.

Jantei sozinho e foi uma satisfação pedir os pratos e ser atendido sem maiores explicações e repetições. O alemão já está funcionando um pouco, pelo menos em algumas pedidas\ldots

E assim me despeço de ti, Berlim, grande capital, que tanto me agradou e surpreendeu, principalmente porque foi onde encontramos o maior número de bonitas garotas. Verdade seja dita, que a maioria da raça fisicamente nada tem de atraente. As garçonetes dos restaurantes são todas senhoras já maduronas. Nas horas mais apertadas, elas trabalham num ritmo de fazer pena. E não vi restaurante algum que não fosse servido exclusivamente por garçonetes. Os homens andam por outras atividades.

O povo encantou pela simpatia, calma e afabilidade. É ``bitte'' para lá, ``bitte'' para cá, palavra essa que quer dizer ``por favor'', a exemplo do ``porsche'' dos poloneses e do ``prego'' dos italianos.

O centro da cidade está todo em obras de imenso vulto, bem no meio das ruas. Deve ficar um primor, quando tudo estiver pronto. Não vi nenhum mendigo, nem bêbado. Em Varsóvia anotei três pedintes de esmolas (sendo um dentro de uma igreja) e uns três empilecados. Em Moscou, dois beberrões, sendo um, lamentavelmente jovem, que, cerca das 10 da noite, foi retirado de dentro de um quiosque de varejo e metido em um taxi.

Amanhã, lá para as 10,30, estarei viajando para o décimo país deste giro e para a décima terceira cidade da rota: Amsterdam, na Holanda.

Indo, há pouco, tomar as providências para viagem, recebi, pela primeira vez na Europa, um tratamento pouco cortês: a funcionária do guichê da estação, uma senhora bem matrona, tudo respondeu com maus bofes.

Logo quando me enchia de entusiasmo pela finura dos alemães\ldots

\section*{22 \adfflatleafright \ISODayName{1959-08-22}}
Parti de Berlim, embarcando na mesma estação ``Bahnhof Zoo'', bem perto da pensão onde ficamos. A viagem foi como quase todas as outras, mesmo as mais longas, muito boa.

Na estação, antes da partida, encontrei um jovem americano, solitário, participante do festival da juventude. Vi o distintivo e puxei conversa. Ele ia de trem até\ldots Londres. O trem, em determinada cidade da Holanda, embarca num navio próprio e atravessa o canal da Mancha. Entre outras coisas o rapaz levava na mão uma ``balalaika'', um tipo de guitarra triangular russa, comprada em Moscou.

Mal embarcamos, reencontramos, dentro do trem, o velho amigo Sr. Palma, o esclarecido português, com a senhora, e Célia, a jornalista do interior paulista, além de outro brasileiro. Iam todos diretamente a Paris.

A passagem campestre alemã é cativante. Na saída de Berlim, junto a canais ou lagos, passamos por Postdam, o famoso local do acordo de rendição da Alemanha vencida, em 1945. Depois, Magdeburg, que tem enorme estação.

Muita pastagem, cerradíssimos campos de cultura --- batata e beterraba. E as casas com os telhados tão exageradamente pontiagudos e inclinados. Em Hannover trocamos de carro. Dali a pouco, cai tremendo aguaceiro. É preciso baldear em outra cidade, porque o trem vai para Rotterdam.

Na fronteira holandesa, as formalidades de praxe, primando todas as autoridades por inexcedível gentileza. A última cidade alemã é Benthein em seguida alcançamos Oldenzaal, primeira localidade holandesa. O espetáculo é magnífico. Parece uma paisinho de brinquedo. Desfila à nossa vista belíssimo grupo residencial, com sacadas de vidro. Coisa inédita para mim, até agora, na Europa. Nas ruas próximas à estação pululam os ciclistas. Jardins floridos também em profusão. A Holanda encanta de saída.

Em Armspoort desembarcamos para nova baldeação. Como estou caindo de fome e dispomos de 20 minutos, procuro o restaurante da estação e deixo Alcino velando pela bagagem. Quando dou os primeiros passos de volta à plataforma, vejo, espantado, o trem, que havia chegado na linha do lado oposto, já desaparecendo na curva final da estação e o Alcino parado no mesmo lado, com apenas uma das muitas malas da nossa bagagem. Então ouço do companheiro a incrível notícia de que tudo o mais havia seguido, por engano, no trem que acabara de partir! Não é que o homem, apesar de eu já ter identificado a plataforma certa da nossa viagem, meteu-se a fazer indagação desnecessariamente sobre a linha do nosso trem e, ainda por cima, falando português, dando em resultado aquela inominável besteira?

Corro para o escritório do chefe da estação e ele, após ouvir o acontecimento, me tranquiliza e diz que resolverá o assunto por telefone. A mala que ``escapou'' felizmente era uma das minhas. No horário previsto, partimos com ela para Amsterdam. Considerando a boa fama do país, confio em que o extravio seja de algum modo solucionado, com o mínimo de prejuízos.

Os trens holandeses surpreendem. São os melhores que encontrei na Europa, até agora.

Em 30 minutos, se tanto, chegamos à capital holandesa, que nunca foi Haia, como erradamente ``aprendemos'' na escola, mas sim Amsterdam.

Mal pulo do vagão, preocupadíssimo com o problema da bagagem, um funcionário aparece procurando os ``brasileiros que extraviaram bagagem''. Falava bom inglês. E nos encaminha para uma sala onde elas estavam, intactas e completas! Não posso ocultar o assombro, pois a presteza da providência e a seriedade daquela gente excederam a qualquer expectativa. Como a maior parte da bagagem desviada era do Alcino e dele também havia sido a mancada, insisto para que ele deixe cair na mão do louro e sorridente funcionário uma bela prata de 10 marcos alemães.

Lá pelas 11 da noite damos entrada no ``Hotel Tourist'', no canal Singel nº 112.

\section*{23 \adfflatleafright \ISODayName{1959-08-23}}
Domingo. Há justamente uma semana partia do Moscou.

Hoje deparo com o encantador ambiente que apresenta a sete vezes centenária Amsterdam.

O movimento é, neste dia da semana, compreensivelmente pequeno, mas o tempo --- nosso prodigioso e permanente aliado --- está firme e convidativo.

A cidade lembra, de imediato, Veneza. E é conhecida mesmo como ``Veneza do Norte'' ou coisa parecida. Os canais são inumeráveis por quase toda a cidade, partindo da profunda baía onde ela se localiza, e o rio Amstel, que dá nome à cidade. ``Dam'', em holandês, quer dizer ``represa'' ou ``dique''.

Passeio durante toda a manhã, a pé e de bonde. Surpresa: as vias públicas não são lá muito limpas e, pelo meu giro, esta é a cidade mais suja, ou melhor, menos limpa de todas as que visitamos até agora. Internamente, porém, o asseio é absoluto.

O tipo humano holandês surpreende. Embora predominando o louro, há grande parcela de morenos e, notadamente, morenas. São lindas, com efeito, as ``dutch-girls'', das mais graciosas que temos visto, em maior percentagem até que as alemães de Berlim.

Os bondes, finalmente, fogem do padrão centro-europeu do vermelho e creme. São cinzentos, estreitos, mas estáveis e velozes. Há um tipo aerodinâmico notável, o mais bonito que até agora vi: inteiriço, com uma articulação sanfonada flexível no meio, para as curvas.

Em toda parte fala-se bem o inglês, finalmente. Assim é fácil conversar e fazer compras e, ainda, pedir pratos nas refeições.

Uma galeria de venda de jornais exibe-os de todos os países europeus ocidentais, praticamente. Dou com os olhos no ``Diário de Notícias'', de Lisboa, de uns três dias atrás, e compro-o incontinenti, para poder lê-lo todo e, talvez, encontrar notícias do Brasil, coisa que não tenho há exatos 48 dias.

Mas qual! Quase nada sobre nossa terrinha. Fico sabendo que Rose Rondelli, nossa companheira de viagem marítima, alcança sucesso em Portugal. E me divirto com os pitorescos anúncios que falam em passar contratos de aluguel de casas com o ``recheio'' destas.

Um programa dos filmes em cartaz em Amsterdam anuncia a exibição há cinco semanas de ``Orfeu Negro'', o primeiro premiado do último festival internacional de Cannes. O coração bate forte e trato de localizar o cinema, que se chama ``Calypso'', para ir nesta mesma noite.

A moeda local é fortíssima, mais que o marco alemão: um dólar igual a 3,78 florins, ou ``goulding'', como se chama aqui no país. Mas tudo parece barato, pois um florim é bom dinheiro. Um almoço de 4 florins é um verdadeiro banquete.

À noite, no cinema --- sessão das 9,30 --- acontece algo interessante. Antes, foi preciso auxílio de um funcionário para se comprar os bilhetes. Os lugares são marcados. como em teatro, e o preço varia conforme a localização na plateia ou nos ``pullmans''. Os nomes estavam em holandês, daí a impossibilidade de se entender o sistema.

Na sala de espera um casal nos olha insistentemente e, não demora muito, o rapaz vem ao nosso encontro. Era brasileiro e ``desconfiava'' que a nossa conversa, embora em tom baixo era em português. Trata-se de patrício Mariano do Prado Valadares, que está em trânsito para Oxford, na Inglaterra, onde cumprirá bolsa de estudos. Depois, por incrível coincidência, venho a saber que a esposa, Therezinha, é prima do meu colega Ernani João Pinheiro, do IRB\ldots Mundo pequeno\ldots

O filme foi uma grande emoção. Senti que o público gostou, Depois, revi meu já saudoso Rio de Janeiro, embora limitado aos panoramas distantes, vistos do morro. Por outro lado, confesso que esperava mais do filme, ante láurea tão elevada que ele conquistou. Interessantíssima, também, a sensação de, pela primeira vez, assistir a um filme falado na nossa língua com legendas em um idioma estranho, o holandês. Quantas falas perderam os assistentes! E como poderia ter o tradutor transmitido aqueles frequentes ditos de gíria carioquíssima? De qualquer modo, a satisfação foi imensa. Apesar das cinco semanas em cartaz, o cinema estava repleto naquela última sessão dominical, e tudo isso representa o Brasil penetrando no Velho Mundo.

\section*{24 \adfflatleafright \ISODayName{1959-08-24}}
Saio cedo para tratar, desde logo, do ``visto'' belga para a próxima viagem, daqui a alguns dias, quando deixarmos a Holanda. Mas no consulado recebo a informação idêntica à da última vez, em Berlim: não é necessário ``visto'', bastando a apresentação do passaporte.

Íamos saindo quando entravam três rapazes falando esperanto! Eram dois iugoslavos (morenos), ciceroneados por um esperantista local (louríssimo), para tratarem também de visto para os dois primeiros entrarem na Bélgica. E o que para brasileiros era dispensado, não o foi para iugoslavos. Política? Eles tiveram de esperar os vistos e ainda pagaram por isso.

Apresentamo-nos, naturalmente, e é geral a satisfação. Haviam estado os três em Varsóvia, durante o Congresso, e disseram ter ``uma idéia'' da fisionomia dos brasileiros por lá\ldots O holandês é K. Ruyg, profissional de marinha mercante, e os ``iugos'' se chamam Ljubisa Jeftic e Radovic Krsta, ambos de Belgrado. Estão de motocicleta, cada um com uma, e demandam a Paris.

Ruyg leva-nos todos até sua casa, que fica próxima. Seus pais, muito gentis e agradáveis, logo nos proporcionam um belo lanche. O apartamento é muito bem mobiliado, o melhor de quantos conheci até agora, na Europa.

Depois, na praça fronteira, fazemos muitas fotografias e os iugoslavos seguem sua jornada. Ruyg faz-nos companhia na visita a outros recantos de sua bela cidade.

Alcino resolve, finalmente, comprar seu decantado gravador de fita, aqui conhecido como ``magnetofon''. O esperantista nos acompanha gentilmente e auxilia na tradução das explicações técnicas dadas pelo vendedor. A loja é a maior da cidade e de chama ``De Bijenkorf'', nome que significa ``a colméia'', em holandês. É muito ampla e magnificamente instalada, no moderno estilo dos grandes magazines europeus e americanos.

Esta noite, mais fatigados, fomos dormir mais cedo.

Combino com Ruyg uma visita amanhã cedo à Casa de Rembrandt, o grande nome da pintura holandesa.

\section*{25 \adfflatleafright \ISODayName{1959-08-25}}
Saio sozinho para ir ter com o bom amigo esperantista.

O local do encontro é a praça Dam, na metade da principal artéria de Amsterdam, que se chama ``Damrak''. Na praça está o Palácio real. E a grande moda nas praças europeias mais famosas é a colombofilia. Essa característica, que parece ter origem na maravilhosa Praça de San Marcos, em Veneza, generalizou-se. Até em Moscou, a Praça Vermelha, seu principal cartão de visita, já está coalhada de pombos.

A ``Dam'', sendo bem pequenina, não comporta um bando muito numeroso, mas tem lá seus pombinhos adejando ruidosamente.

Como chego antes da hora, vou dar uma nova olhadela na ``Colméia'', também próxima. Acabei comprando um barbeador elétrico ``Philips'', para presentear o grande amigo e colega que ficou ``velando'' pelo meu carro lá no Rio.

Vou com Ruyg à casa onde morou Rembrandt e que hoje é o seu museu. Muito interessante pelas reconstituições de suas gravuras em maioria no Museu da cidade, um enorme e belo edifício, do mesmo estilo da Estação Ferroviária.

Depois de outras andanças, fazendo boas fotografias, encontramos Alcino, para juntos almoçarmos. Escolhemos um restaurante chinês, o ``Hong-Kong'', um dos inumeráveis daquela nacionalidade em Amsterdam. Nosso cardápio: sopa de tubarão (as brânquias) e coelho com arroz.

Mais tarde, fazemos um giro de hora e meia em um grande lanchão turístico, envidraçado e aerodinâmico, começando pelos canais da cidade e terminando no grande cais.

Sempre agradável, Ruyg nos convida para irmos à sua casa à noite, assistir um programa de televisão. Não há como recusar.

Ficamos observando o número de indonésios que vivem na cidade. Todos com quem conversei até agora demonstraram repulsa pelo presidente Sukarno, que consideram comunista. Ruyg esteve com seu navio, há um ano, na Indonésia, antiga colônia holandesa. Explica que o país está em bancarrota, porque a prematura expulsão dos holandeses e outros estrangeiros, que formavam a elite produtora, desaparelhou subitamente o seu sistema econômico. Em consequência --- conclui ele --- baixou o nível de vida da população, que gostava de muita coisa provinda do contato com os holandeses.

Os bondes apresentam, também, detalhes curiosos, como, por exemplo, conduzirem caixa coletora de correspondência na parte trazeira, e acionarem um ``pisca-pisca'', como os automóveis, para indicar mudança de direção.

À noite, partimos para a casa de Ruyg. O seu pai é diretor de uma escola do Estado, de grau primário, e inventou um processo original de ensino das primeiras letras, com cartões em baixo relevo. Muito interessante. Alcino demonstra disposição de utilizar o processo em seu educandário no Rio e recebe vários exemplares. O Sr. Cornelius, esse o seu nome, prontamente o autoriza a adotar o sistema, bastando apenas indicar o seu nome e o da escola que dirige (``Erasmus'') nos cartões.

Depois assistimos um filme na televisão, após o jornal falado. Aliás, aproveito para confirmar a excelência da televisão europeia, tanto na qualidade dos programes como na parte técnica. O aparelho nunca perde a segurança e a imutabilidade da imagem, como é tão comum no Rio.

Vi TV num café da estação em Milano, na Itália, e foi um ``show'' de perfeição de reportagem, como de qualidade de imagem e de som. Antes, em Gênova, conheci um processo de projeção amplificada enormemente, lançando a imagem na parede, como cinema. Em Varsóvia, os aparelhos eram grandes com telas ridiculamente pequenas. Mas a técnica era perfeita. Em Moscou, havia no hotel um aparelho de pequeníssima tela, mas asseguram-me que os havia também de 21 polegadas. O esperantista Professor Bokarev, por exemplo, afirmou possuir em sua casa um de grande vídeo.

O filme da casa do holandês era falado em inglês e, felizemente, pude entender razoavelmente. As legendas eram em holandês. Logo de início lembrei-me de já ter visto no Rio aquele filme, há uns dois ou três anos. É uma história em que Gregory Peck recebe uma nota de um milhão de libras para gastar e, mesmo fazendo uma porção de despesas, não consegue trocá-la.

Após seguidas gentilezas da família anfitriã, que nos serve várias vezes vinho, chá, biscoitos e bombons, retiramo-nos tarde da noite.

Outra particularidade: vim beber em Amsterdam o melhor café da Europa, até agora. Surpreendentemente bem feito. E também reencontro o açucareiro, pois o regime geral é o dos tijolinhos de açúcar, via de regra semi-cristalizado e feito de beterraba, bem fraco para adoçar. Ou então o envelopinho de açúcar em pó, também cristalizado.

\section*{26 \adfflatleafright \ISODayName{1959-08-26}}
Preparo-me para deixar Amsterdam esta manhã, rumo a Rotterdam, que fica bem perto, relativamente. Escolho o caminho via Haarlem e Haia, para conhecer, pelo menos de passagem, a famosa ``falsa'' capital, onde o nosso grande Ruy Barbosa pontificou em 1905.

Duas horas depois chegávamos a Rotterdam, o maior porto da Europa e o segundo do mundo, apenas inferior ao de Nova York. No trajeto avisto os primeiros moinhos de vento e os campos de tulipa, dois pontos característicos da tão lendária Holanda. Infelizmente, com a máquina fotográfica fechada na mala, não possa fixar esse singular panorama. Os campos são de um verde claro maravilhoso, com o seu robusto gado pastando, sem cercas de qualquer espécie em derredor, mas muitas valas de irrigação fazendo o mesmo efeito.

Rotterdam dá-me estupenda impressão logo na estação ferroviária, que é moderníssima, inclusive na arquitetura, plena de amplos envidraçados. Como foi tremendamente destruída durante a guerra (os alemães ocuparam o porto), a reconstrução foi geral e obedeceu, agradavelmente, a linha moderna. Uma paisagem atraente e surpreendente na velha e tradicional Holanda.

Deixo as pesadas malas na estação, na seção própria para guardá-las, e, em bolsa de tamanho mediano, fico apenas com o essencial, tal como já havia feito nos quatro dias em Moscou.

Ficamos no pequeno mas confortável ``Hotel Union'', bem central. Ali somos agradavelmente cientificados de que há chuveiros e o uso deste não é extraordinário! Rotterdam é diferente até nisso!

Vamos logo almoçar no restaurante do ``Hotel De Kroon'' (A Corôa), que anuncia na porta ``se habla español''. Logo que nos identificamos como brasileiros e esperantistas, o proprietário coloca em nossa mesa uma bandeirinha do Brasil, toda autografada, e mostra-me o cardápio em esperanto! Agradecemos, algo emocionados, principalmente pela nossa bandeira e pela curiosidade logo despertada nos circunstantes.

Depois vamos à sede da Associação Universal de Esperanto, a organização mundial centralizadora do movimento esperantista e motivo principal de nossa visita a esta bela cidade.

A A.U.E. fica na maravilhosa e acolhedora avenida ``Eendrachts Weg'' nº 7. Ao centro, com largas margens gramadas e poeticamente arborizadas, corre um riacho de águas cristalinas. Crianças e adultos estão sentados bucolicamente na margem gramada. São todos corados, de bochechas incrivelmente rosadas. Neste quadro, estou me sentindo dentro de um cartão postal\ldots

A recepção na A.U.E. é, compreensivelmente, cordial e entusiástica. Os funcionários que nos recebem participaram também do Congresso em Varsóvia. Glauco Pompílio, italiano de Bologna, e a senhorita van Wijngaarden palestram longamente conosco e gravam suas impressões na fita do magnetofone do Alcino. Antes, já o fizera em Amsterdam o bom amigo Ruyg.

Conheço, na Associação Universal, outros esperantistas que chegam em seguida: um senhor holandês, um inglês e um português, Joaquim Calado, radicado há cinco anos em Rotterdam. Eles se oferecem para um passeio à noite conosco.

Partindo dali, vamos girar um pouco lá pela praça fronteira a estação ferroviária, local bem central, pitoresco e movimentado. Antes um pouco, numa paralela à avenida do nosso hotel, e perto do imponente edifício da prefeitura, com sua alta torre, percorro uma galeria de butiques extraordinariamente simpática, com lojas magníficas e todas modernas. É um prazer visitá-las e comprar coisas.

Junto à estação, surpreendo-me com a presença de um marujo brasileiro, pouco mais que um rapazola. Trata-se de um integrante do grupo que se adestra no manejo do porta-aviões ``Minas Gerais'', de nossa marinha, em trabalhos de reforma nos estaleiros da firma ``Verolme'', neste porto. Ele me diz que são uns 60 praças e 30 oficiais. Estão aqui há mais de um ano e vão ficar mais uns 6 ou 8 meses. Dali a pouco aparece um dos oficiais, em trajes civis, e Roberto, o marujo, nos apresenta: Elias Jorge. Convidam-nos para ir ao estaleiro amanhã cedo; há um ônibus especial, que sai às 7,30.

A hora matinal nos assusta. E ficamos sabendo que a marujada tem cartaz com as holandesas. Roberto me apresenta, em seguida, sua ``girl'', que chega de bicicleta. Chama-se Sandra e fala um fluente inglês. Suas duas irmãs vêm chegando. Enquanto Alcino criva o rapaz de dezenas de perguntas inconsequentes (como sempre\ldots), bato um longo papo com as holandesinhaas, ``brotíssimos''. Elas já aprenderam algumas palavras em português e eu sou convidado a lhes ensinar algumas outras. Dizem que adoraram ``Orfeu Negro'', o filme brasileiro no festival de Cannes e que assistimos em Amsterdam, e cantarolam o samba ``Tristeza'', do filme, com pronúncia engraçadíssima.

Depois do jantar, reencontramos os esperantistas Pompílio, Calado e a esposa deste, holandesa e também boa esperantista, para o anunciado giro.

Vamos pela beira do cais conhecer o túnel subterrâneo, que liga as duas margens do rio principal. Maravilhosa obra, que faz recordar o nosso sepultado túnel Rio-Niterói. se bem que este aqui seja muito mais curto: um quilômetro. Desce-se por escada rolante para as pistas destinadas a pedestres e ciclistas; há outra, paralela, mas separada, para automóveis e que é atingida por uma entrada mais distante, à direita.

No cais, feericamente iluminado, está o novo navio ``Rotterdam'', que fará a linha Europa-América do Norte e se prepara para a viagem inaugural. É de cor cinza e verdadeiramente majestoso.

Depois de outros ``bordejos'' e longos ``papos'', entramos no ``nightclub'' ``La Pompadour'', amplo e elegantíssimo. Pedir café é perfeitamente distinto nestas plagas\ldots Há dança, com ótima orquestra alemã, e ``show'' de variedades no próprio salão. Muita gente. Surpreende-me a animação dos holandeses, que diziam ser casmurros e inexpressivos. As garotas mostram-se realmente exuberantes, bem mais que os homens, pelos divertimentos. Em maioria, são bonitas e alegres.

O ``show'' apresenta mágicas rápidas e de bom efeito, dança acrobática de uma alemã sumarissimamente vestida e, como melhor número, uma cena de dança de apaches, engraçadíssima, ao compasso da música ``Esconderijo de Hernando''.

Foi uma bela e agradável noitada.

\section*{27 \adfflatleafright \ISODayName{1959-08-27}}
De manhã faço um passeio turístico, de lanchão (muito grande) da empresa ``Spido'', pelo cais e adjacências. Tenho então, uma ideia do tamanho deste porto, verdadeiramente descomunal. E --- dizem --- que o nosso trajeto só abrange parte dele. Conto navios e estaleiros às dezenas. Apenas do Brasil não vejo nenhum.

O impressionante transatlântico ``Rotterdam'' ia deixando o porto quando passávamos de volta.

Depois almoçamos e voltamos, com o gravador, à Associação Universal de Esperanto, para registrar a voz do eminente esperantista Dr. Ivo Lapena, de nacionalidade iugoslava, secretário da entidade. Infelizmente, ele não pode nos esperar.

Assento providência para embarcar, esta noite, para Bruxelas.

Jantamos, de novo, no ``De Kroon'' e, às 21,28 tomamos o rápido para a Bélgica.

Calado vem ao nosso ``bota-fora'', muito gentilmente, como esperantista e como português.

Os trens holandeses, repito, são os melhores da Europa, até agora, na minha opinião. Estáveis, rápidos, assustadoramente pontuais. Baldeamos em Roosendall, última cidade holandesa, na fronteira. A passagem pela aduana belga é rápida e sem maiores formalidades.

Passamos para o outro comboio e duas horas depois, se tanto, com paradas em Antuérpia e Mechelen, chegamos a Bruxelas. Mechelen é a cidade de Josée Hendrickx, minha antiga correspondente, conhecida pessoalmente em Varsóvia, e que amanhã procurarei no seu trabalho em Bruxelas.

Saltando na estação ``Nord'', vamos, por sugestão do motorista de taxi, para o ``Hotel Botanique'', na Rua St. Lazare. É pertíssimo, e o malandro nos cobrou 70 francos belgas (quase dólar e meio). O indefectível assalto das chegadas noturnas, quando não se conhece as cidades\ldots O engraçado foi o motorista subir conosco, no hotel, até o quarto e demonstrar, calcando os colchões de mola, que era tudo muito confortável\ldots

\section*{28 \adfflatleafright \ISODayName{1959-08-28}}
Tempo magnífico. Sei, de longe, que os belgas são terríveis comilões, e o desjejum no hotel o demonstra de imediato. Um autêntico almoço!

Um simpático casal de ingleses, com duas crianças já bem grandinhas, está na sala de refeições e não disfarça sua curiosidade pela língua em que eu e o Alcino estamos conversando (português, é claro). Por fim, entabulamos conversação com os ingleses, e ao ouvir o meu ``We are brazilians'', a senhora solta um estúpido ``How exciting''! E tome elogio ao café e ao futebol\ldots

O hotel é bem central. Saímos a procurar o local de trabalho de Josée, no edifício da ``Shell''. Antes, por equívoco, procurei na ``Sabena'', a grande companhia belga de aviação, que está instalada em magnífico prédio, de estilo moderno.

Respiro aliviado pelo rompimento da barreira da língua, que, nas últimas semanas, nos sufocou na Alemanha e na Holanda. Aqui é o sonoro francês. Mas a Bélgica, embora de território tão pequeno, tem dois idiomas oficiais, por incrível que pareça: o francês no sul e o flandre, ou flamengo, no norte; este último está calcado no próprio holandês, com algum ``sotaque''. Tudo é escrito nas duas línguas em Bruxelas, inclusive as placas das ruas e praças.

Da rua do Hotel (St. Lazare) ganho a praça Roger, que ostenta na fachada de um dos prédios um termômetro super-gigantesco, em estilo letreiro. As ruas principais, cruzando ali mesmo, são os ``boulevards'' Jardin Botanique e Adolph Max. Uma igreja na praça de Ste. Goudule, mais adiante, tem grande semelhança com a Notre Dame de Paris.

A cidade agrada bastante, logo à primeira vista. Está também situada sobre sete colinas,. como Roma, Lisboa e talvez alguma outra. Por isso as avenidas e ruas têm suas ondulações, com vários mergulhos por baixo de viadutos.

Encontro Josée no ``hall'' do seu edifício, pois como não sabia exatamente o andar nem o nome da firma, seria impraticável localizar. O jeito foi mesmo esperar na saída para o almoço. Ela mostra grande satisfação em me rever e pergunta muito sobre nossas andanças desde Varsóvia. Saímos a percorrer alguns pontos centrais interessantes, como a ``Grand Place'', que pretende competir com a Praça de São Marcos, em Veneza. Ali estão a Prefeitura, que chamam de ``Hôtel de Ville'', nos países de língua francesa, e outras construções do velho estilo. de fachadas negras bordadas a ouro, muito bonitas. Ao centro, no claro deixado pelos automóveis estacionados, alguns mercados de flores, bem pitorescos.

Vamos, também, ao Condenberg, amplo pátio, junto à nova biblioteca, ao que parece, e onde está a estátua equestre do rei Alberto, aquele que visitou o Rio de Janeiro, em 1920.

Mas o tempo de Josée é curto; ela nos leva a um restaurante, para provarmos algo típico belga, que pede ao garçom em segredo, naturalmente, para nos fazer surpresa. É impressionante o número de bares, cafés, confeitarias e restaurantes diversos, todos muito bem instalados. As vitrines de confeitarias são autênticas tentações. Entretanto, a ``grande especialidade'' que ela nos apresenta e que comemos é\ldots bife com fritas\ldots Explodimos de riso e ela, muito desapontada, não sabe o que entender. Dou as explicações necessárias, para consertar a situação. Mas, na verdade, a maneira de preparar as fritas é realmente especial e notável, pois elas vêm leves e nada gordurosas.

Josée volta para o seu trabalho e marcamos encontro para as 18 horas, quando sair. Enquanto isso, parto para uma visita ao local da Grande Exposição Internacional de 1958, já encerrada, é claro, mas que tem muita coisa ainda instalada, como, por exemplo, o gigantesco ``Atomium''. Fica a 30 minutos do centro, de bonde, e à margem noroeste da cidade.

A exposição deve ter sido realmente portentosa, pelas proporções de sua entrada, com as bolas do ``Atomium'' pontificando. Restam, porém, poucos pavilhões. O do Brasil, que recebeu o primeiro prêmio de concepção, obra de Sérgio Bernardes, foi inteiramente desmontado. O da França e alguns mais estão em fase idêntica. O do Congo, antiga colônia belga, jaz completo ainda.

Alcino, na volta, fica no hotel para descansar, enquanto reencontro Josée, para novos ``bordejos''. Vamos ao monumento do Cinquentenário, um arco tipo do Triunfo, de Paris, ladeado, em inteira conjugação, por dois palácios, dando ao conjunto uma forma de ferradura. Em torno, imensos e tranquilos jardins.

Percorremos, depois, as ``rues'' ``Royale'' e ``de la Loi'' e a ``Place des Palais'', um enorme grupamento e uma seqüência de ministérios e palácios, inclusive o real.

Em outra direção, visitamos o Palácio da Justiça, impressionante edificação, sem dúvida a maior da cidade. À sua frente, pela posição elevada, divisa-se bom panorama de grande parte da cidade. Bruxelas me cativa inteiramente, com apenas este primeiro dia.

Jantamos em um restaurante italiano, onde mato as saudades de uma boa lasanha. E dali, ainda vamos encerrar o programa em um ambiente bem acolhedor, do tipo dos nossos ``inferninhos''.

\section*{29 \adfflatleafright \ISODayName{1959-08-29}}
Combino com Josée ir a Mechelen, sua cidade natal, hoje à tarde. Ela estará ocupada quase o dia inteiro com dois esperantistas de Berlim, que vêm conhecer Bruxelas. Marcamos encontro de manhã, para outros passeios, enquanto os alemães não chegam, o que só acontecerá lá para o meio-dia.

Vamos conhecer o monumento ao Soldado Desconhecido, uma coluna muito alta, ladeada, na base, por enormes leões de bronze. Aliás, o leão é o símbolo da Bélgica, Não seio se por influência do Congo, onde tais bichanos devem proliferar.

Corremos mais umas lojas, devendo ser destacada entre elas a de nome ``Inovation'', e nos separamos. Vou com Alcino, para mostrar a ele o grande monumento que é o Palácio do Cinqüentenário. Fazemos boas fotos e, depois, passeamos pelo amplo parque que circunda o monumento.

É tempo de voltar para a estação Norte, reencontrar Josée e seguir para Mechelen. A cidade fica a apenas 15 minutos de Bruxelas, pelos trens diretos. A Bélgica tem uma notável rede ferroviária, a mais densa do mundo em relação à área do país, que corresponde a apenas uma vez e meia o nosso estado de Sergipe. O movimento na ``gare'' é grande, pois sendo sábado, é elevado o número de pessoas que se desloca para outros pontos, notadamente para Ostende, na costa marítima.

Antes um pouco havíamos ido apanhar nossas primeiras fotos copiadas, da ``Rolleyflex'' do Alcino, e mais três rolos meus, revelados. São de fotos tomadas em Moscou, todas saindo magníficas. E outras passagens que reconheço aos poucos: Gênova, Milão, Barcelona, etc., etc.

Em Mechelen, cidade de 65.000 habitantes, conheço a família de Josée. Ela não havia avisado da nossa ida, que, assim, é objeto de certa surpresa; sabiam, apenas, que eu, o já ``veterano'' correspondente brasileira, estava em Bruxelas. Os sobrinhos de Josée, Rita, de oito anos, e Paul, de seis, disseram que já me conheciam pelas fotografias. Fazem-me bastante festa e me tratam com surpreendente intimidade, bem como a mãe., irmã de Josée, de nome Elizabeth. Que língua falamos? A mãe fala inglês e os filhos um francês arrevesado, misturado com um pouco de inglês também\ldots

A casa de Josée é quase em frente à da irmã, na mesma rua, que se chama ``Rue des Chevalliers'', em francês, e ``Ridenstraat'', em flamengo, ambos os nomes na mesma placa. Sou apresentado aos seus país, Louis e Dª Maria; dali a pouco chega o irmão, Emile. Todos gentilíssimos, embora demonstrando acanhamento por não terem podido organizar uma recepção especial --- segreda-me a correspondente. Exceto Josée, é claro, falam apenas a língua da região, a flamenga.

A tradição gastronômica não tardou. Os donos da casa querem preparar coisa boa. Então saímos para conhecer a cidade enquanto a ``bóia'' é caprichada.

Mechelen, cujo nome em francês é Malines, é bem calçada, com ruas estreitas na maioria. Típica cidazedinha de existência várias vezes secular. Muita agricultura, pois é grande fornecedora de hortaliças e frutas para Bruxelas e até para Paris. É ainda famosa pela indústria de móveis.

A catedral me maravilha, embora a torre tenha ficado rasa lá em cima, sem a tradicional ponta gótica. É que a separação da Holanda fez com que os holandeses carregassem, em represália, o material destinado àquele acabamento, há bem perto de uns duzentos anos atrás. Mechelen é também a sede cardinálicia do país. Penetro na imensa igreja, que possui riquíssimos vitrais --- um deles alusivo a São Romualdo, que teve origem irlandesa --- e um fabuloso púlpito esculpido em madeira, dos mais artísticos e impressionantes.

Também o edifício da prefeitura e um prédio acastelado da praça principal, arcaicos no seu estilo centenário, chamam especial atenção.

Voltamos para jantar. Os Hendrickx têm um canário solto dentro de casa e, por isso, mantêm-na totalmente fechada. Quando falo da fama dos ``canários belgas'' no Brasil, eles mostram não entender esse qualificativo; dizem que o pássaro em questão não tem qualquer origem ou especial criação em seu país, pois é proveniente das Ilhas Canárias, daí o nome.

Conversamos bastante à mesa. O velho e sorridente Louis resmunga qualquer coisa, atrás de mim, no seu incrível idioma. Josée traduz: ``Papai diz para vocês comerem, em vez de falar\ldots'' Esses belgas, em matéria de comida\ldots

Depois vamos ouvir os discos que eu trouxe para Josée, nos vizinhos de sua irmã, o jovem casal Edouard e Marie Luise: ``Dolores Viaja'', com ``Coimbra'' em esperanto, em versão de minha autoria, ``Acadêmicos do Salgueiro'' e ``Uma Noite no Arpege'', com o conjunto de Waldir Calmon. Eles ficaram entusiasmadíssimos com o segundo, embora a eletrola não reproduza a contento a fidelidade da gravação. A canção em esperanto causa admiração, mas esta é maior quando mostro o meu nome, como autor da versão, na contra-capa do disco\ldots

Edouard, para não ficar atrás, mostra-me um disco com a versão francesa do samba ``Madureira Chorou''.

Para encerrar esta noite de sábado, Josée sugere um clube no meio do bosque, a dez quilômetros da cidade. Partimos de ônibus, já perto da meia-noite, com o programa de voltarmos com Emile, o irmão (que já estava lá), no seu ``Chevrolet''.

Verifico, então, como o belga entende, ou melhor, pratica a dança. Primeiro: são loucos pelos ritmos americanos trepidantes, como ``swing'', ``rock’nroll'' e ``charleston'', e ainda pelos centro-americanos também sacolejantes, como conga, mambo e ``cha-cha-cha''. Aliás, na Europa inteira eu já observara isso, desde os bailes do Congresso em Varsóvia. Lá os ditos frios suecos, alemães e poloneses se esbaldavam nos ``rocks'' e similares, para minha enorme surpresa.

Uma vez, em Varsóvia, Josée jogou-me no ``fogo'' em um ``swing'' com uma sueca algo desgastada, baixota e com cara de personagem do nosso caricaturista Carlos Estevão\ldots Chamava-se Karil. Foi um ``vexame'' para mim, mas que remédio senão afinar pelo salão inteiro?\ldots

Os belgas, a exemplo da maioria dos holandeses, dançam afastados, como se fossem profissionais. Zombam dos pares normais (para mim), como é o nosso velho estilo. Cada elemento se mexe sozinho, naqueles trejeitos de bailarinos de conjuntos, acompanhando o ritmo lentamente.

O local do clube é agradável, muito íntimo e aconchegante. A ``orquestra'' é formada por três elementos apenas, mas ótimos instrumentistas e suficientemente barulhentos. Súbito, tocam o nosso chorinho ``Apanhei-te Cavaquinho'', algo deturpado, mas bem cadenciado. Levo um bruto susto. Depois sou obrigado a dançar com Josée um cha-cha-cha à distância\ldots

Outra batida forte no coração: alguém, em altas vozes, pede para a orquestra tocar ``Brasil'' (``Aquarela do Brasil'') . Desconfio que Emile interferiu, para nos fazer cortesia. Aí então me esbaldei com vontade. Identificado como brasileiro, deixaram-me só com Josée na pista, para

apreciarem o ``passo genuíno do samba''. Que delícia, aliada à emoção e à saudade, ouvir nestas distâncias a maravilhosa ``Aquarela'' de Ari Barroso, mesmo um tanto puxada para mambo-rumba!

Lá para as duas da manhã, Emile nos leva de carro até Bruxelas. Na saída do bosque ocorre um incidente: um carro entra na curva quase todo na contramão e o nosso o abalroa, embora não muito fortemente. Pensei ouvir calorosos palavreados, mas nada disso aconteceu. Todos saltaram e quase cochicharam suas razões. Mas a menina (quase isso) que vinha com um rapaz no veículo abalroado desanda a lacrimejar. O carro é do pai --- diz ela --- e a avaria deve dar problema\ldots

\section*{30 \adfflatleafright \ISODayName{1959-08-30}}
Dormimos até mais tarde, para compensar a noitada.

Havia prometido aos Hendrickx voltar a Mechelen hoje, domingo, para assistir aos desfiles da festa dos granjeiros e comerciantes locais.

Parto sozinho e chego em cima da hora, pois não foi fácil reencontrar, desacompanhado de gente do lugar, a rua ``des Chevalliers'', lá para dentro da cidade. A população inteira está nas calçadas para assistir os desfiles.

E foram interessantíssimos: carros floridos, maravilhosamente ornamentados, composições com legumes e frutas, um portento de imaginação e confecção. Por fim, uma parte lendária, com figuras agigantadas e a alusão a um certo boneco, disputado, há não sei quantos séculos, entre Mechelen e Antuérpia. Fotografo profusamente, em companhia dos meus ``fãs'' Rita e Paul, e Elizabeth e seu marido Henrik. Estes últimos falam bom inglês e assim nosso entendimento é facilitado.

Josée ficara em Bruxelas, ainda às voltas com os esperantistas alemães, e só pôde aparecer às 6 da tarde. Mais comilanças (estava tardando\ldots) e algumas fotografias com a família toda, para depois descansarmos um pouco vendo um programa de televisão.

A projeção é perfeita, um autêntico cinema, como, aliás, é o padrão normal da televisão em toda a Europa. Deve ser pela melhor qualidade dos aparelhos, combinada com a estabilidade de voltagem na transmissão e na recepção. Liga-se o aparelho e pronto: um cinema! Não se toca em comando algum para ajeitar nada, e isso por horas a fio.

Os garotos, que estão agarradíssimos comigo, vibram com uma aventura dos ``Rangers'' do Texas. Também aqui o bangue-bangue tem primazia. O noticiário que se segue mostra cenas dos Jogos Panamericanos, em Chicago, onde brasileiros também estão disputando. E depois iria entrar um programa pela cadeia da ``Eurovisão'', tomado no Principado de Mônaco.

Prefiro circular lá fora com a boa amiguinha Josée. Vamos ao baile que complementa o encerramento da tal festa de todo último domingo de agosto. Bailam no próprio ``hall'' (gigantesco) do mercado e a animação é grande. O estilo belga, ``desagarrado'', de dançar impera naturalmente.

Há um número com nuance típica, muito interessante. A orquestra toca uma espécie de polca, lenta; de repente, arrefece mais o ritmo e dá um acorde forte. Ao som deste, o cavalheiro ergue a dama no ar, como puder (em geral pelos cotovelos). Depois, é a vez de acontecer o contrário e, por fim, os dois dão um jeito de subir juntos\ldots

A menina só me deixou voltar para Bruxelas no último trem, de uma da madrugada\ldots

\section*{31 \adfflatleafright \ISODayName{1959-08-31}}
Alcino está ``indócil'' para chegar a Paris. De casa lhe disseram que uma ``grande surpresa'' o aguardaria lá. Cismou, então, contra todas as evidências, que sua mulher o esperava na Cidade Luz.

Assim resolveu embarcar hoje, no trem das 13,20, sem aceitar o convite de Josée para irmos passar o dia em Antuérpia. Ela tem o dia livre hoje.

Levo Alcino até a estação, inclusive para ajudar na bagagem, pois o coitado está com duas malas pesando mais que o fardo do destino, e mais três volumes avulsos, de bom tamanho.

Às duas e pouco reencontro Josée e partimos para Antuérpia, a 40 minutos de viagem, pelos magníficos trens belgas. Assim que chegamos, vamos almoçar, pois, pelo adiantado da hora, a fome ``bateu'' forte. No norte do país, isto é, de Bruxelas para cima, só se fala flamengo. As ruas têm as placas escritas apenas nesse idioma, quando em Bruxelas o são nos dois.

Antuérpia é também alegre e agradável. A rua principal tem o nome de ``Keyserlei'', que significa ``rua do Imperador''. É bem movimentada, com boas lojas e\ldots incontáveis restaurantes e confeitarias, pródigos em ``especialidades''.

Subimos ao 24º andar do edifício de um banco, o mais alto da cidade, e descortino uma visão completa do grande porto belga. Bem junto, está a enorme catedral de Nossa Senhora, do mesmo estilo da de Mechelen, porém com a torre concluída.

Na orla portuária, a parte mais velha, naturalmente, as ruas têm nomes de mercadorias: assim, há a ``rua da uva'', a ``rua do açúcar'' e outras muitas.

A ``Praça Brabo'', antiquíssima, tem no centro um curioso monumento-chafariz, sobre um bloco de pedra e sem contorno ou proteção ao rez-do-chão; a água escorre livremente na pavimentação de paralelepípedo, como se fosse malha do meio-fio. A estátua mostra uma figura de feições lendárias, segurando ao alto da cabeça uma enorme mão. Josée me explica: a mão é de um gigante que dominava o porto e escorchava os moradores, exigindo altos tributos. O herói ``Brabo'' acabou com o abuso, libertando a praça, e tal lenda dá nome à cidade: ``Ant'' quer dizer mão e ``werpen'' significa lançada fora. O nome original da cidade é ``Antwerpen''.

A beira do cais, que é todo no rio Schelde e longuíssimo, além de distar uns 30 quilômetros do mar, ostenta largas calçadas elevadas, onde se passeia agradavelmente, enquanto se aprecia a descarga dos navios. Dois cargueiros suecos descarregam abacaxis em conserva, do Havaí.

Há, também, um castelinho medieval a beira d’água, ante o qual nos fotografamos. Outra igreja imponente é a de São Paulo, que tem a particularidade curiosa de ostentar a torre na parte trazeira. Junto, um monumento singelo ao escritor belga Hendrik Conscience.

Infelizmente não posso visitar a Casa de Rubens, o grande pintor natural de Antuérpia. A construção é enorme e fica na ``Rubenstraat'' (rua de Rubens), que é estreita. Acontece que a dita cuja não abre às segunda- feiras. O artista foi homem rico, pois também comerciante e diplomata. Josée se desmancha em lamentações pelo esquecimento do detalhe, e afirma que a Casa tem interior deslumbrante.

Para encerrar a visita, quando a noite chega, vamos a um ``inferninho'' tipo ``Kelner''(porão), como é a tradição centro-europeia. Mesmo em Varsóvia, o melhor era do gênero e se chamava ``Krokodil''. Este aqui se chama ``Papkelner'', que significa ``Porão do Padre''. Pelas paredes vários desenhos e inscrições engraçadíssimas, tudo em flamengo. Uma delas diz (Josée traduz, é claro): ``De todos os barulhos, a música é o mais agradável''\ldots

Hoje não há orquestra, mas o toca-discos comercial, dos tais de injetar fichas, é excelente e com vasto repertório. Apesar de segunda-feira, muita gente moça a se divertir. Havia ``Besame Mucho'' (uma de minhas favoritas) e das nossas nada menos que ``Brasil''(``Aquarela'' de Ari Barroso), ``Samba do Orfeu'', ``Delicado'' e ``Ave Maria no Morro'', embora todos sob execução de orquestras europeias. Ficamos lá até a hora de voltar, no último trem.
